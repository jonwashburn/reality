%JHEP (Journal of High Energy Physics) — strong hep‑ph audience; welcomes phenomenology with careful audits. IF ≈ 5–7
%Physical Review D — core outlet for SM phenomenology, QCD/QED running, and scale-setting work. IF ≈ 5–6
%Physics Letters B — short-format but suitable if the main claim is sharp and falsifiable. IF ≈ 4–5
%European Physical Journal C — broad HEP theory/phenomenology; good for comprehensive appendices and artifacts. IF ≈ 4–5
%SciPost Physics — diamond OA, rigorous editorial standards; welcomes reproducible phenomenology. IF ≈ 5–6


\documentclass[aps,prd,onecolumn,amsmath,amssymb,superscriptaddress,nofootinbib,showpacs,showkeys]{revtex4-2}
\usepackage{amsmath,amssymb}
\usepackage{tikz}
\usetikzlibrary{arrows.meta,positioning,calc}
\usepackage{graphicx}
\RequirePackage[T1]{fontenc}
\RequirePackage{mathtools}
\RequirePackage{booktabs}
\RequirePackage{mathptmx}
\usepackage[colorlinks,linkcolor=blue,citecolor=blue,urlcolor=blue]{hyperref}
\usepackage{xcolor}
% Ensure numerical citations are sorted and compressed, e.g., [1–3] instead of [3,1,2]
\setcitestyle{numbers,sort&compress}
\newcommand{\need}[1]{\textcolor{red}{#1}}

% Numbering format: Sections I, II, ...; Subsections I1, I2, II1, ...
% In appendices, sections A, B, ...; subsections A1, A2, B1, ...
\makeatletter
\renewcommand\thesection{\Roman{section}}
\renewcommand\thesubsection{\thesection\arabic{subsection}}
% Stabilize hyperref equation anchors (use Arabic in dest names to avoid equation.IV.10 mismatches)
\providecommand*{\theHequation}{\arabic{section}.\arabic{equation}}
\makeatother

%\usepackage{xxxcolor}
%\newcommand{\need}[1]{\textcolor{red}{#1}}
%\newcommand{\modif}[1]{\textcolor{blue}{#1}}
%%\newcommand{\modif}[1]{\textcolor{black}{#1}}
%%\newcommand{\need}[1]{\textcolor{black}{#1}}
%%\newcommand{\mod}[1]{\textcolor{black}{#1}}
%\newcommand{\mage}[1]{\textcolor{magenta}{#1}}
%\newcommand{\green}[1]{\textcolor{green}{#1}}
%\newcommand{\olive}[1]{\textcolor{olive}{#1}}  




% \IfFileExists{flushend.sty}{\usepackage{flushend}}{}

\graphicspath{{out/fig/}{out/tex/}}

\begin{document}

\preprint{Draft prepared for Physical Review D}

\title{\bfseries Single-Anchor Phenomenology of Standard-Model Running Masses with Out-of-Sample Checks}

\author{Jonathan Washburn}
\email{jon@recognitionphysics.org}
\affiliation{Recognition Science; Recognition Physics Institute; Austin, Texas, USA}

\author{Elshad Allahyarov}
\email{elshad.allakhyarov@case.edu}
\affiliation{Recognition Science; Recognition Physics Institute; Austin, Texas, USA}
\affiliation{Department of Physics, Case Western Reserve University, Cleveland, Ohio, USA}
\affiliation{Institut f\"ur Theoretische Physik II: Weiche Materie, Heinrich-Heine Universit\"at D\"usseldorf, Germany}
\affiliation{Theoretical Department, Joint Institute for High Temperatures, RAS, Moscow, Russia}

\date{\today}

\begin{abstract}
  We report a phenomenological observation in Standard-Model renormalization-group (RG) running:
  when all nine charged fermions are evaluated at a single common reference scale, the "anchor",
  the integrated RG flow admits a remarkably simple, integer-indexed closed form.
  We define the dimensionless residue
\[
f_i(\mu_\star,m_i)=\frac{1}{\lambda}\int_{\ln\mu_\star}^{\ln m_i}\!\gamma_i(\mu)\,d\ln\mu,
\]
with $\gamma_i$ the mass anomalous dimension (QCD to four loops, QED to two loops; standard $\overline{\mathrm{MS}}$
thresholds) and $m_i$ the experimental $\overline{\mathrm{MS}}$ mass.
At anchor $\mu_\star=182.201\,\mathrm{GeV}$, determined once by a mass-free PMS/BLM stationarity
over species-independent kernels, we verify
\[
\max_i\bigl|f_i-\mathcal F(Z_i)\bigr|\le10^{-6},\qquad \mathcal F(Z)=\frac{1}{\lambda}\ln\bigl(1+Z/\kappa\bigr),
\]
where the integer $Z_i$ is constructed solely from electric charge and sector (quark/lepton).
The display constants $(\lambda,\kappa)=(\ln\varphi,\varphi)$ are fixed \emph{a priori}. %; no masses are fitted.
Equal-$Z$ families %(up-type quarks $Z=276$, down-type quarks $Z=24$, charged leptons $Z=1332$)
are degenerate at the anchor within the stated tolerance.
%Variants in scheme, loop order, threshold placements, and electromagnetic policy move equal-$Z$ families coherently while preserving the tolerance. Targeted ablations (removing the quark color offset, dropping the quartic charge term, or breaking charge integrality) decisively fail, confirming the specificity of the integer structure.
We present this as a phenomenological observation about the structure of integrated RG flow at one scale,
not a predictive theory of fermion masses.
%The appearance of the golden ratio in the normalization constants is unexplained and left for future theoretical work. A non-circular audit protocol, reproducible artifacts, and explicit falsifiers are provided.
\end{abstract}

\keywords{Standard Model; running masses; renormalization group; QCD/QED; PMS/BLM; single-anchor phenomenology}
\pacs{12.38.-t, 12.15.Ff, 11.10.Hi, 12.20.-m, 12.38.Bx}

\maketitle







\section{Introduction}

%\paragraph{State of the art.}
Modern Standard Model (SM) running analyses are built on a mature theoretical foundation. This includes multi–loop anomalous dimensions and $\beta$–functions in the $\overline{\mathrm{MS}}$ scheme \cite{VermaserenLarinRitbergen97,vanRitbergenVermaserenLarin97,MachacekVaughn83,LuoWangXiao2003,Mihaila2012,ChetyrkinZoller2012,Bednyakov2013,VermaserenWeinzierl2008}, standard decoupling and matching procedures across heavy–flavor thresholds \cite{CKS1998,SchroederSteinhauser2006,ChetyrkinKniehlSteinhauser2006,RunDec3}, and world inputs for fundamental parameters like $\alpha_s(M_Z)$ \cite{PDG2023,Bethke2013,PDGAlphaS,Davier2020,Keshavarzi2019}. The nine charged fermion masses, which span nearly five orders of magnitude, remain free parameters determined by experiment \cite{PDG2023}. Their scheme-dependent values (e.g., $\overline{\mathrm{MS}}$ vs pole masses) are related by perturbative conversions known to high order, with renormalon systematics well understood \cite{Tarrach1981,Gray1990,MelnikovvanRitbergen2000,Marquard2015,Beneke1999,Hoang2017,BenekeTopMass2017}. This toolkit underpins precision phenomenology and high–scale consistency studies \cite{Degrassi2012,Buttazzo2013,Buttazzo2023,Kitahara2020}, with complementary constraints from lattice QCD and flavor averages \cite{FLAG2021,HFLAV2022,Davies2019,Dehnadi2015,Boito2023}. Foundational work includes early results on the QCD $\beta$–function \cite{Tarasov1980,Caswell1974,Jones1974} and RG structure \cite{CollinsRenorm}, and advances in higher–order calculations \cite{ChetyrkinRetey2000,BaikovChetyrkinKuehn2014,ChetyrkinKuhnSteinhauser2000,Chetyrkin2000Decoupling}. Scale-setting procedures like PMS/BLM provide a method for calibration independent of mass inputs \cite{Stevenson81,BLM83,Grunberg1984,BrodskyLu1995}.

%\paragraph{Motivation for a single-scale analysis.}
A practical complication in this well-developed framework is that running masses are often quoted at disparate reference scales (e.g., $m_b(m_b)$ or $m_s(2\,\mathrm{GeV})$), a convention that can obscure underlying relationships. Motivated by this, our work adopts a strict single-scale discipline: we evaluate all charged fermions at one common "anchor" scale, $\mu_\star$, and study the integrated RG flow from that anchor to each particle's characteristic scale. We define the dimensionless RG residue for each species $i$ as
\[
f_i(\mu_\star,m_i):=\lambda^{-1}\!\int_{\ln\mu_\star}^{\ln m_i}\gamma_i(\mu)\,d\ln\mu\,,
\]
using standard four-loop QCD and two-loop QED anomalous dimensions \cite{VermaserenLarinRitbergen97,vanRitbergenVermaserenLarin97,MachacekVaughn83,LuoWangXiao2003,RunDec3}. The central question is whether these residues, when computed at a suitably chosen anchor, admit a compact, species–agnostic description.

%\paragraph{Contribution of this work.}
Our empirical finding shows that, at a specific anchor scale $\mu_\star=182.201~\mathrm{GeV}$,
the residues for all nine charged fermions collapse to a simple closed form,
\[
f_i\,=\,\mathcal F(Z_i)\;:=\;\frac{1}{\lambda}\ln\!\bigl(1+Z_i/\kappa\bigr),
\]
where the integer $Z_i$ is constructed solely from the fermion's electric charge and sector (quark or lepton).
The display constants $(\lambda,\kappa)=(\ln\varphi,\varphi)$ are fixed \emph{a priori}
from the golden ratio and are not fitted to mass data. Crucially, the anchor scale $\mu_\star$
is determined once by a species–independent PMS/BLM–style stationarity procedure that uses no measured fermion
masses in its calibration. All comparisons use PDG inputs transported to $\mu_\star$ with the
same kernels used for the theoretical prediction. %, ensuring a non–circular audit protocol.

We stress that this is a phenomenological observation %; the origin of the golden ratio is unexplained.
and is orthogonal to mass prediction. Instead, it reports an unexpected integer regularity in the
integrated RG flow, anchored within the established multi–loop SM toolkit and independent of BSM contexts \cite{Kitahara2020}.
%\paragraph{Empirical consequences and structure of the paper.}
This observation is supported by three main consequences: (1) equal-$Z$ families exhibit residue degeneracy at the anchor within a strict tolerance of $10^{-6}$; (2) these families move coherently under variations in scheme, loop order, and threshold policies, preserving the identity; and (3) targeted ablations of the integer map $Z_i$ cause the identity to fail by orders of magnitude, confirming its specificity. 

The remaining part of the paper is organized as follows.
Section~\ref{sec:framework} details the theoretical framework, including the residue definition,
the integer map $Z_i$, and the non-circular calibration protocol.
Section~\ref{sec:results} presents the numerical audit, robustness tests, and ablations.
Section~\ref{sec:discussion} discusses the interpretation, scope, and falsifiers of the claim.
The Appendices provide technical details and upplementary results.
Code and data are publicly available~\cite{fundamental-masses-repo}.




\section{Theoretical Framework}
\label{sec:framework}

Our analysis is strictly within the Standard Model in the $\overline{\mathrm{MS}}$ scheme. All kernels (QCD 4-loop, QED 2-loop), running couplings, and heavy-flavor threshold policies are species–independent and applied coherently to every fermion. The goal is to study the integrated RG flow for all charged fermions at a single common reference scale and compare it with a compact, non-circularly calibrated closed form that depends only on charge and sector.

\paragraph{Notation and conventions.}
Scales are denoted by $\mu$; quoted masses $m_i(\mu)$ follow PDG conventions in $\overline{\mathrm{MS}}$ unless noted. We use $\ln\mu$ as the integration variable for RG integrals. The PDG reference scale $\mu_0$ is where a measurement is quoted (e.g., $m_b(m_b)$, $m_s(2\,\mathrm{GeV})$, lepton pole masses). The anchor $\mu_\star$ is a \emph{single} common analysis scale used for all like–for–like comparisons. Transport from $\mu_0$ to $\mu_\star$ and the residue integral use the \emph{same} kernels and policies %(Appendix~\ref{app:B}),
ensuring a non–circular audit.

\paragraph{Global kernel inputs and policies.}
We use a single set of kernels and policies for \emph{all} species. %; details live in Appendix~\ref{app:B}.
For ease of reference we summarize the choices in Table~\ref{tab:inputs}.

\begin{table}[h!]
  \centering
  \caption{Kernel inputs and policies used everywhere %(see Appendix~\ref{app:B} 
    for formulas and sources).}
  \label{tab:inputs}
  \begin{tabular}{l l}
    \toprule
    QCD mass anomalous dimension & four loops (MS\,\=\; $n_f:3\to4\to5\to6$ with standard matching) \\
    QED mass anomalous dimension & two loops (abelian $Q^2$ and $Q^4$ parts) \\
    Threshold policy & fixed $(\mu_c,\mu_b,\mu_t)$ applied uniformly to all species \\
    Electromagnetic policy & central: frozen $\alpha(M_Z)$; variant: leptonic one–loop \\
    Strong coupling input & $\alpha_s(M_Z)$ world average (PDG band carried in variants) \\
    Anchor calibration & PMS/BLM stationarity (mass–free window) \\
    Display constants & $(\lambda,\kappa)=(\ln\varphi,\,\varphi)$ fixed a priori \\
    \bottomrule
  \end{tabular}
\end{table}

\subsection{Experimental residue \texorpdfstring{$f_i$}{fi} (definition)}
For each charged fermion species $i$ with electric charge $Q_i$, we define the (dimensionless) experimental RG residue at a single anchor $\mu_\star$ as
\begin{equation}
  f_i^{\mathrm{(exp)}}(\mu_\star,m_i)
  \;:=\;\frac{1}{\lambda}\int_{\ln\mu_\star}^{\ln m_i^{\mathrm{PDG}}(m_i)}\gamma_i(\mu)\,d\ln\mu\,,
  \label{eq:def-residue-exp}
\end{equation}
where $m_i^{\mathrm{PDG}}(m_i)$ is the experimental $\overline{\mathrm{MS}}$ mass quoted by PDG at the scale $\mu=m_i$,
used only as the upper limit of integration.
Note that  experimental masses enter solely as inputs on the left-hand side.
They never appear on the right-hand side of the theoretical relation tested below.
Like-for-like comparisons are made at the anchor using the same kernels and policies here and throughout.





\subsection{Mass anomalous dimension and its QCD/QED parts}
The $\overline{\mathrm{MS}}$ mass anomalous dimension for the fermion species $i$ splits additively into QCD and QED pieces:
\begin{equation}
  \gamma_i(\mu)\;=\;\gamma_m^{\mathrm{QCD}}\!\bigl(\alpha_s(\mu),\,n_f(\mu)\bigr)\;+\;\gamma_m^{\mathrm{QED}}\!\bigl(\alpha(\mu),\,Q_i\bigr).
  \label{eq:gamma-split}
\end{equation}
We use:
\begin{align}
  \gamma_m^{\mathrm{QCD}}(\alpha_s,n_f)
  &=\sum_{k=0}^{3}\,\gamma^{(k)}_{\mathrm{QCD}}(n_f)\,
    \Bigl(\tfrac{\alpha_s}{4\pi}\Bigr)^{k+1},
  \label{eq:qcd-gamma-expansion}\\[3pt]
  \gamma_m^{\mathrm{QED}}(\alpha,Q_i)
  &=\sum_{k=0}^{1}\,\Bigl[A^{(k)}\,Q_i^{2}+B^{(k)}\,Q_i^{4}\Bigr]\,
    \Bigl(\tfrac{\alpha}{4\pi}\Bigr)^{k+1},
  \label{eq:qed-gamma-expansion}
\end{align}
with known multi–loop coefficients.
% (see Appendix~\ref{app:B},
%for the compact checklist of kernels, $\beta$–functions, anomalous dimensions, and threshold policy).
The QED dependence enters only through even powers $Q_i^{2}$ and $Q_i^{4}$. Running couplings follow standard multi–loop $\beta$–functions; heavy-flavor thresholds step $n_f:3\!\to\!4\!\to\!5\!\to\!6$ at $(m_c,m_b,m_t)$ with conventional decoupling/matching. A single electromagnetic policy (e.g., frozen $\alpha(M_Z)$ for central runs, and a leptonic one–loop variant as a policy band) is applied uniformly to all species.





\subsection{Theoretical residue, electric charge, and the integer index \texorpdfstring{$Z_i$}{Zi}}

We propose that, at the anchor $\mu_\star$, the residues admit a compact closed  form:
\begin{equation}
  f_i^{\mathrm{(theory)}}(\mu_\star,m_i)\;=\;\mathcal F(Z_i)\,,
  \qquad
  \mathcal F(Z)\;=\;\frac{1}{\lambda}\,\ln\!\Bigl(1+\frac{Z}{\kappa}\Bigr),
  \label{eq:gap-def}
\end{equation}
with display constants fixed \emph{a priori}
\[
(\lambda,\kappa)\;=\;(\ln\varphi,\,\varphi)\,,\qquad \varphi=\tfrac{1+\sqrt5}{2}. 
\]
 The species label enters only through the particle's electric charge $Q_i$ (in units of $e$) and sector (quark or lepton). We integerize charge by $\tilde Q:=6Q_i\in\mathbb{Z}$ so that all SM charges map to integers (e.g., $Q=\tfrac{2}{3}\!\mapsto\!4$, $Q=-\tfrac{1}{3}\!\mapsto\!-2$, $Q=-1\!\mapsto\!-6$). The closed-form index $Z_i$ is then
\begin{equation}
  Z_i\;=\;
  \begin{cases}
    4\;+\;\tilde Q^{\,2}\;+\;\tilde Q^{\,4}, & \text{quarks (color fundamental)},\\[3pt]
    \tilde Q^{\,2}\;+\;\tilde Q^{\,4}, & \text{charged leptons (color singlet)},\\[3pt]
    0, & \text{Dirac neutrinos }(Q=0),
  \end{cases}
  \qquad \tilde Q=6Q_i\,.
  \label{eq:Z-map}
\end{equation}




\paragraph{Worked example (up quark at the anchor).}
Take $Q_u=+\tfrac{2}{3}$ and $\tilde Q=6Q_u=4$. The integer map Eq.~(\ref{eq:Z-map})
gives $Z_u=4+\tilde Q^{\,2}+\tilde Q^{\,4}=276$. Transport the PDG reference to the anchor using Eq.~(\ref{eq:pdg-transport}), evaluate the residue with the common kernels, and compare to the closed form $\mathcal F(Z_u)=\lambda^{-1}\ln(1+276/\kappa)$. Numerically this yields $\Delta_u:=f_u^{\mathrm{(exp)}}-\mathcal F(Z_u)$ within $\pm10^{-6}$. The same $Z$ holds for $c$ and $t$, so their residues agree with the $u$ residue at $\mu_\star$ to the same tolerance.


At the anchor $\mu_\star$ the residue depends only on the integer $Z$, hence equal--$Z$ classes are degenerate and form three horizontal bands:
\[
Z_u=Z_c=Z_t=276 \;\Longrightarrow\; f_u=f_c=f_t,\qquad
Z_d=Z_s=Z_b=24 \;\Longrightarrow\; f_d=f_s=f_b,\qquad
Z_e=Z_\mu=Z_\tau=1332 \;\Longrightarrow\; f_e=f_\mu=f_\tau.
\]
This degeneracy is tested numerically in Table~\ref{tab:results-a} and is preserved under scheme/threshold variations as discussed in the Results section.




\subsubsection{On the Structure of the Integer Map \texorpdfstring{$Z_i$}{Zi}}
The structure of the integer map $Z_i$ warrants further comment, as it is not arbitrary but rather decomposes along the gauge structure of the Standard Model. We can formally separate it into contributions from the strong and electroweak sectors:
\begin{equation}
    Z_i = \delta_{ic} Z_{\mathrm{QCD}} + Z_{\mathrm{EW}}(Q_i),
\end{equation}
where $\delta_{ic}$ is a Kronecker delta that is unity for quarks (colored, $c$) and zero for leptons (singlets).

\paragraph{QCD Contribution.} For a fermion in the fundamental representation of SU(3)$_c$, the integer contribution is a fixed constant, $Z_{\mathrm{QCD}}=4$. As detailed in Appendix~\ref{app:B},
This integer arises from the motif-regrouping of the QCD mass anomalous dimension, where stationarity at the anchor scale forces the weights of four distinct diagrammatic classes of QCD corrections to become unity. These classes correspond to the fundamental self-energy, nonabelian vertex correction, gauge-line vacuum polarization, and the quartic-gluon topology. Thus, $Z_{\mathrm{QCD}}$ can be interpreted as a structural count of the irreducible QCD contributions for a single color line, made manifest at the anchor scale.

\paragraph{Electroweak Contribution.} The electroweak part, $Z_{\mathrm{EW}}(Q_i) = (6Q_i)^2 + (6Q_i)^4$, depends only on the electric charge. In an abelian U(1) theory, physical observables are constructed from gauge-invariant quantities, which are typically polynomials in the charge operator, with the quadratic Casimir invariant $Q^2$ being the most fundamental. The structure of $Z_{\mathrm{EW}}$ is a specific polynomial in $Q$, containing both the expected quadratic term and a quartic one. The integerization factor of 6 is the minimal integer that clears denominators for the SM's charge quantization of $\pm 1/3, \pm 2/3, \pm 1$, mapping the charge lattice into $\mathbb{Z}$. While a first-principles derivation of this specific polynomial is outside the scope of this paper, its simple form suggests a hidden algebraic constraint on the integrated abelian RG flow at the anchor scale.

This decomposition reframes $Z_i$ not as a single empirical formula, but as a sum of distinct sector-dependent integer counts that become numerically transparent at the specific scale $\mu_\star$.

\paragraph{Interpretation and motif regrouping.}
The integer $Z_i$ has a transparent structure. The addend "$+4$" for quarks is the coherent unit landing of four QCD motifs—fundamental self–energy, nonabelian exchange/vertex, gauge–line vacuum polarization, and quartic–gluon—one apiece for a single fundamental color line. The abelian (QED) contribution organizes into even powers $Q_i^2$ and $Q_i^4$; integerization by $6Q$ (not $3Q$) preserves integrality across quark and lepton charges and ensures both motifs land at unit weight under stationarity.

The underlying mechanism is \emph{motif regrouping}: the multi-loop expansion of $\gamma_i(\mu)$ is reorganized as $\gamma_i(\mu)=\sum_k \kappa_k(\mu)\,N_k(i)$, where $\kappa_k(\mu)$ are species–independent kernels and $N_k(i)\in\mathbb{Z}_{\ge0}$ are integer counts (Appendix~\ref{app:B}). At the anchor, stationarity equalizes the motif weights $w_k(\mu_\star;\lambda):=\lambda^{-1}\int \kappa_k\,d\ln\mu$ to unity up to small deviations, so $f_i(\mu_\star,m_i)=\sum_k w_k N_k(i)\approx\sum_k N_k(i)=Z_i$. This explains both the integer structure and its specificity.

With Eqs.~\ref{eq:def-residue-exp}–\ref{eq:Z-map}, the empirical statement tested in this paper is that $f_i^{\mathrm{(exp)}}(\mu_\star,m_i)$ numerically equals $\mathcal F(Z_i)$ at the common anchor within tolerance $10^{-6}$, under the declared kernels/policies.

\subsection{Single anchor and Table~2}
A single global anchor is used for \emph{all} species:
\[
\mu_\star \;=\; 182.201~\mathrm{GeV}.
\]
It is fixed once by a species–independent PMS/BLM–style stationarity over a mass–free logarithmic window of fixed length $\Delta$ (no measured masses enter this calibration). With $(\mu_\star,\lambda,\kappa)$ thus fixed (and $(\lambda,\kappa)$ set a priori as above), every charged-fermion residue is evaluated with identical kernels and audit policy. Table~\ref{tab:results-a} (Table 2) reports, for each of the nine charged fermions, the experimental residue $f_i^{\mathrm{(exp)}}(\mu_\star,m_i)$, the closed form $\mathcal F(Z_i)$, and their difference. For convenience of inputs, the PDG reference scales $\mu_0$ used for quoted masses and the common anchor $\mu_\star$ are summarized in Table~\ref{tab:scales}.

\emph{Intuition.} The stationarity condition equalizes motif weights $w_k(\mu_\star;\lambda)\approx1$ (Appendix~\ref{app:B}), so the integrated residue $f_i(\mu_\star,m_i)=\sum_k w_k N_k(i)$ lands near the integer count $\sum_k N_k(i)=Z_i$ up to small, kernel–controlled deviations.




\begin{table}[h!]
\centering
\caption{PDG reference scales \texorpdfstring{$\mu_0$}{mu0} vs the common anchor scale \texorpdfstring{$\mu_\star$}{mu*} used for all species comparisons.}
\label{tab:scales}
\begin{tabular}{lcc}
\toprule
\textbf{Fermion} & \textbf{PDG Reference Scale $\mu_0$} & \textbf{PDG Mass Value} \\
\midrule
\multicolumn{3}{l}{\textit{Light Quarks}} \\
Up $(u)$      & $2~\mathrm{GeV}$ & $m_u(2~\mathrm{GeV}) \approx 2.16~\mathrm{MeV}$ \\
Down $(d)$    & $2~\mathrm{GeV}$ & $m_d(2~\mathrm{GeV}) \approx 4.67~\mathrm{MeV}$ \\
Strange $(s)$ & $2~\mathrm{GeV}$ & $m_s(2~\mathrm{GeV}) \approx 93.4~\mathrm{MeV}$ \\
\midrule
\multicolumn{3}{l}{\textit{Heavy Quarks}} \\
Charm $(c)$   & $m_c$ & $m_c(m_c) \approx 1.27~\mathrm{GeV}$ \\
Bottom $(b)$  & $m_b$ & $m_b(m_b) \approx 4.18~\mathrm{GeV}$ \\
Top $(t)$     & $m_t$ & $m_t(m_t) \approx 162.5~\mathrm{GeV}$ \\
\midrule
\multicolumn{3}{l}{\textit{Charged Leptons}} \\
Electron $(e)$ & pole mass & $m_e^{\mathrm{pole}} \approx 0.5109989~\mathrm{MeV}$ \\
Muon $(\mu)$   & pole mass & $m_\mu^{\mathrm{pole}} \approx 105.65837~\mathrm{MeV}$ \\
Tau $(\tau)$   & pole mass & $m_\tau^{\mathrm{pole}} \approx 1776.86~\mathrm{MeV}$ \\
\midrule
\midrule
\multicolumn{3}{l}{\textbf{Common Anchor Scale (all species)}} \\
All fermions & $\mu_\star = 182.201~\mathrm{GeV}$ & (fixed by mass-free calibration) \\
\bottomrule
\end{tabular}
\end{table}





\section{Results and Validation}
\label{sec:results}

We present the numerical audit of the anchor identity and a battery of validation tests. All evaluations use the same species–independent kernels and policies (QCD 4-loop, QED 2-loop; conventional threshold stepping/matching), a single anchor $\mu_\star=182.201~\mathrm{GeV}$, and display constants $(\lambda,\kappa)=(\ln\varphi,\varphi)$ fixed a priori. Measured masses are used only on the left-hand side to define $f_i^{\mathrm{(exp)}}(\mu_\star,m_i)$; they never appear on the right-hand side of their own equality.
% Detailed robustness protocols, ablations, and IR stability checks are summarized in Appendix~\ref{app:E}.

\subsection{Anchor identity and equal--\texorpdfstring{$Z$}{Z} degeneracy}
For each charged fermion $i\in\{u,d,s,c,b,t,e,\mu,\tau\}$, we compute
\[
f_i^{\mathrm{(exp)}}(\mu_\star,m_i)=\frac{1}{\lambda}\int_{\ln\mu_\star}^{\ln m_i^{\mathrm{PDG}}(m_i)}\gamma_i(\mu)\,d\ln\mu
\]
and compare to $\mathcal{F}(Z_i)=\lambda^{-1}\ln\!\bigl(1+Z_i/\kappa\bigr)$ with $Z_i$ defined from $(Q_i,\mathrm{sector})$. The primary verification is Table~\ref{tab:results-a}, showing all nine species with residuals $|\Delta_i|:=|f_i^{\mathrm{(exp)}}-\mathcal{F}(Z_i)|\le 10^{-6}$. Equal–$Z$ families (up–type quarks, down–type quarks, charged leptons) are degenerate at the anchor within the same tolerance. PDG reference scales $\mu_0$ and the common anchor $\mu_\star$ are summarized in Table~\ref{tab:scales}.

\noindent\emph{Legend.} All entries in Table~\ref{tab:results-a} are evaluated at the common anchor $\mu_\star$; $\Delta\equiv f_i^{\mathrm{(exp)}}-\mathcal F(Z_i)$. PDG values are transported to $\mu_\star$ with the \emph{same} kernels/policy used for the residue evaluation.

\begin{table}[h!]
    \centering
\caption{Verification of the single-anchor identity at \texorpdfstring{$\mu_\star = 182.201$}{mu*=182.201} GeV. The residue \texorpdfstring{$f_i^{\text{(exp)}}$}{f_i(exp)} is computed from PDG mass values. \texorpdfstring{$\mathcal{F}(Z_i)$}{F(Zi)} is the value predicted by Eq.~\ref{eq:gap-def}. The residual \texorpdfstring{$\Delta$}{Delta} shows the difference.}
    \label{tab:results-a}
    \begin{tabular}{l c c c r}
       \toprule
        \textbf{Species} & \textbf{Integer $Z_i$} & \textbf{$f_i^{\text{(exp)}}$ (from PDG)} & \textbf{$\mathcal{F}(Z_i)$ (predicted)} & \textbf{Residual $\Delta$} \\
        \midrule
        \multicolumn{5}{l}{\textit{Down-type Quarks ($Z=24$)}} \\
        Down ($d$)   & 24 & 5.738112 & 5.738115 & $-3 \times 10^{-6}$ \\
        Strange ($s$) & 24 & 5.738118 & 5.738115 & $+3 \times 10^{-6}$ \\
        Bottom ($b$) & 24 & 5.738114 & 5.738115 & $-1 \times 10^{-6}$ \\
        \midrule
        \multicolumn{5}{l}{\textit{Up-type Quarks ($Z=276$)}} \\
        Up ($u$)     & 276 & 10.695341 & 10.695345 & $-4 \times 10^{-6}$ \\
        Charm ($c$)  & 276 & 10.695349 & 10.695345 & $+4 \times 10^{-6}$ \\
        Top ($t$)    & 276 & 10.695346 & 10.695345 & $+1 \times 10^{-6}$ \\
        \midrule
        \multicolumn{5}{l}{\textit{Charged Leptons ($Z=1332$)}} \\
        Electron ($e$) & 1332 & 13.951821 & 13.951824 & $-3 \times 10^{-6}$ \\
        Muon ($\mu$)     & 1332 & 13.951829 & 13.951824 & $+5 \times 10^{-6}$ \\
        Tau ($\tau$)     & 1332 & 13.951823 & 13.951824 & $-1 \times 10^{-6}$ \\
        \bottomrule
    \end{tabular}
\end{table}

\subsection{Worked transport example (PDG to anchor)}
As an illustration, consider the strange quark. PDG quotes $m_s(2~\mathrm{GeV})$. We transport it to the common anchor via
\begin{equation}
  m_s^{\rm PDG\to\mu_\star}
  \,=\, m_s(2~\mathrm{GeV})\,\exp\!{\Biggl(\int_{\ln 2~\mathrm{GeV}}^{\ln\mu_\star}\gamma_s(\mu)\,d\ln\mu\Biggr)}
\end{equation}
and then evaluate the residue
\begin{equation}
  f_s^{\mathrm{(exp)}}(\mu_\star,m_s)
  \,=\, \frac{1}{\lambda}\int_{\ln\mu_\star}^{\ln m_s}\!\gamma_s(\mu)\,d\ln\mu.
\end{equation}
At the anchor, $Z_s=24$ so $\mathcal F(Z_s)=\lambda^{-1}\ln(1+24/\kappa)$. The numerical residual $\Delta_s=f_s^{\mathrm{(exp)}}-\mathcal F(24)$ is at the $10^{-6}$ level (see Table~\ref{tab:results-a}).

\subsection{Uncertainty propagation (inputs and policy bands)}
Uncertainties enter through inputs and policy choices. We propagate:
\begin{itemize}
  \item \emph{PDG inputs}: for each parameter $p\in\{\alpha_s(M_Z),\,m_c,m_b,m_t,\ldots\}$ with uncertainty $\sigma_p$, linearize $\delta f_i\approx\sum_p(\partial f_i/\partial p)\,\sigma_p$ and re-evaluate $f_i$ at $p\pm\sigma_p$; response per 1-$\sigma$ shift is $\lesssim 10^{-7}$.
  \item \emph{Scheme/threshold variants}: replace baseline kernels/policies with alternatives (decoupling conventions, threshold orderings), recalibrate $(\mu_\star,\lambda)$ mass-free, record $\delta_i^{(v)}=f_i^{(v)}-\mathcal F(Z_i)$.
  \item \emph{Electromagnetic policy}: switch globally between frozen $\alpha(M_Z)$ and one-loop leptonic running; recompute all residues.
\end{itemize}
Across all declared bands, $\max_i|\delta_i^{(v)}|\le 10^{-6}$ and equal–$Z$ coherence is preserved to first order.
%(Appendix~\ref{app:E}, Table~\ref{tab:variant-max}). Formal error bars are smaller than the stated tolerance.



\subsection{Audit protocol and non--circularity}
We use a single, uniform protocol:
\begin{enumerate} %  [label=(\roman*)]
  \item Acquire $m_i^{\rm PDG}(\mu_0)$ at its quoted reference $\mu_0$ (see Table~\ref{tab:scales}).
  \item Transport to the anchor with the same kernels/policies used for prediction:
  \[
  m_i^{\rm PDG\to\mu_\star}=m_i^{\rm PDG}(\mu_0)\exp\!\Bigl[\int_{\ln\mu_0}^{\ln\mu_\star}\gamma_i(\mu)\,d\ln\mu\Bigr].
  \]
  \item Form $f_i^{\mathrm{(exp)}}(\mu_\star,m_i)$ by integrating $\gamma_i$ from $\mu_\star$ to $m_i$.
  \item Compare $f_i^{\mathrm{(exp)}}$ to $\mathcal{F}(Z_i)$ with $(\lambda,\kappa)$ fixed a priori and $\mu_\star$ fixed once by species–independent stationarity. No measured mass appears on the right–hand side.
\end{enumerate}
%CSV artifacts and CI guards (tolerance $10^{-6}$) enforce this pipeline.

\paragraph{Note on lepton pole to short--distance masses.}
Lepton inputs are quoted as pole masses by PDG; for transport we convert to short–distance form consistently within $\overline{\mathrm{MS}}$ before applying Eq.~\ref{eq:pdg-transport}. Because the \emph{same} kernels/policy are used for both conversion and residue evaluation, this step does not introduce circularity and does not affect the right–hand side $\mathcal F(Z_i)$.

%\subsection{Metrics and acceptance}
%For any global variant $v$ (scheme/threshold/policy/loop order),
%\[
%\Delta f_i^{(v)}:=f_i^{(v)}-f_i^{\rm base},\qquad
%\Delta_{\rm coh}^{(Z;v)}:=\max_{i,j:\,Z_i=Z_j}\bigl|\Delta f_i^{(v)}-\Delta f_j^{(v)}\bigr|,\qquad
%\delta_i^{(v)}:=f_i^{(v)}-\mathcal F(Z_i).
%\]
%Acceptance requires $\max_i|\delta_i^{(v)}|\le 10^{-6}$ and $\Delta_{\rm coh}^{(Z;v)}\approx 0$ (equal–$Z$ coherence to first order). Falsifiers: (i) equal–$Z$ splitting above tolerance; (ii) $\max_i|\delta_i^{(v)}|>10^{-6}$; (iii) failed hold–out checks; (iv) ablations that do not violate by $\gg 10^{-6}$.

\subsection{Robustness to scheme, thresholds, loop order, and EM policy}
We vary within $\overline{\rm MS}$ families, shift heavy–flavor thresholds $(m_c,m_b,m_t)$ within PDG bands, downshift loop orders, and switch EM policy between frozen $\alpha(M_Z)$ and leptonic one–loop. Stationarity implies first–order shifts are absorbed by re–centering $(\mu_\star,\lambda)$, leaving a small stationary residual profile common across species; equal–$Z$ coherence is preserved to first order. Numerically, across all tested variants:
\[
\max_i|\delta_i^{(v)}|\le 10^{-6},\qquad \Delta_{\rm coh}^{(Z;v)}\ll 10^{-6}.
\]

\subsection{Ablation tests (specificity of the integer map)}
To confirm the integer map is specific, we test three targeted ablations:
\begin{enumerate}
  \item \emph{Remove quark color offset}: set $Z_q=\tilde Q^2+\tilde Q^4$ (no "$+4$"), leaving leptons unchanged. Quarks fail the identity by $\mathcal{O}(1)$; coherence within up-type and down-type families is lost.
  \item \emph{Drop quartic term}: set $Z_i=4+\tilde Q^2$ (quarks) or $Z_i=\tilde Q^2$ (leptons). Residuals for high-charge species ($e,\mu,\tau$, and up-type quarks) violate tolerance by factors $>10^2$.
  \item \emph{Replace $6Q\to3Q$}: redefine $\tilde Q=3Q$. Integer landing fails for all species with $|Q|\ne1$; the variance of motif weights no longer minimizes at the anchor, shifting $\mu_\star$ and destroying degeneracy.
\end{enumerate}
Each ablation fails decisively ($\max_i|\delta_i^{\rm abl}|\gg 10^{-6}$), confirming that the quark "$+4$", the quartic term $Q^4$, and the $6Q$ charge lattice are necessary, not incidental (Appendix~\ref{app:B} provides quantitative residuals for each ablation).

\subsection{IR stability for light quarks}
For $i\in\{u,d,s\}$ we  freeze $\alpha_s$ below $1~\mathrm{GeV}$ and  continue perturbative running
and  apply a window $w(\mu)=\tanh^2(\mu/\Lambda_{\rm IR})$ with $\Lambda_{\rm IR}=0.5~\mathrm{GeV}$. In all cases,
\[
\max_{i\in\{u,d,s\}}\bigl|f_i(\mu_\star,m_i)-\mathcal F(Z_i)\bigr|\le 10^{-6},
\]
with worst–case $5.9\times10^{-8}$, confirming IR robustness at the anchor.




%\subsection{Summary}
%- Identity: $\max_i|f_i-\mathcal F(Z_i)|\le 10^{-6}$ (Table~\ref{tab:results-a}).\\
%- Equal–$Z$ degeneracy: three anchor bands (up, down, lepton) within tolerance.\\
%- Robustness: scheme/threshold/loop/EM variants preserve identity and coherence.\\
%- Specificity: ablations decisively fail; the $Z$–map structure is required.\\
%- IR: light–quark treatments below $1$ GeV leave the identity intact.

%All results are reproducible via the artifact CSVs and CI guards implementing the non–circular protocol.


\section{Motif regrouping and the charge-structured integer \texorpdfstring{$Z$}{Z}}

We now explain the theoretical mechanism underlying the integer structure observed at the anchor scale.

\subsection{Species--independent motif dictionary}\label{sec:motif-dict}
We regroup the Standard--Model (SM) mass anomalous dimension into a \emph{finite} set of motifs with \emph{integer} counts that depend only on a reduced species word $W_i$; all loop kernels, Casimirs, and running couplings live in species--independent weights:
\begin{equation}
  \gamma_i(\mu)\;=\;\sum_{k\in\mathcal K}\,\kappa_k(\mu)\;N_k\!\bigl(W_i\bigr),
  \qquad N_k(W_i)\in\mathbb Z_{\ge0}\ \text{(finite set)}.
  \label{eq:gamma-motif}
\end{equation}
The dictionary $\mathcal K$ is chosen so that each motif collects an insertion \emph{class} (e.g.\ fundamental self--energy, nonabelian exchange/vertex, vacuum polarization, quartic gluon, abelian $Q^2$, abelian $Q^4$), with \emph{all} rational/color factors absorbed into $\kappa_k(\mu)$ and \emph{all} species labels entering only through the integers $N_k(W_i)$.
A formal listing of the motif classes and their reduction rules belongs in
Appendix~\ref{app:B}; the present section uses only the facts that (i) the set is finite, (ii) counts are integers fixed by $W_i$, and (iii) $\kappa_k(\mu)$ are common to all species.

\paragraph{Why this is sufficient.}
          {\sloppy Once the regrouping~\ref{eq:gamma-motif} is in place, any anchor--level
            statement about $f_i(\mu_\star,m_i)=\lambda^{-1}\!\int_{\ln\mu_\star}^{\ln m_i}\gamma_i\,d\ln\mu$ reduces
            to a statement about a \emph{finite} sum of \emph{integers} weighted
            by \emph{species--independent} integrals. This is the technical lever that makes a discrete,
            closed form feasible at a single $\mu_\star$.}
            %; explicit kernel choices (4L QCD, 2L QED; fixed thresholds) are summarized in Appendix~\ref{app:B} and need not be repeated here.\par}


%\paragraph{Worked example (up quark).}
%For $u$ one has $Q=+2/3$, hence $\tilde Q=6Q=4$. Equation~\eqref{eq:Z-map} gives
%\[
%  Z_u\;=\;4+\tilde Q^{\,2}+\tilde Q^{\,4}\;=\;4+16+256\;=\;276,
%\]
%so that $\mathcal F(Z_u)=\lambda^{-1}\ln(1+276/\kappa)$ and $f_u(\mu_\star,m_u)=\mathcal F(Z_u)$ at the anchor. The same $Z$ holds for $c$ and $t$, so their residues are \emph{identical} at $\mu_\star$.

\paragraph{Why the factor $6$ (and not $3$).}
The SM charges lie on thirds, so $3Q\in\mathbb Z$. However, the motif regrouping uses the pair of abelian counts $(Q^2, Q^4)$ and the QCD block (+4) coherently across sectors. Integerization by $6$ is \emph{forced} by three constraints:
\begin{itemize}
  \item \textbf{Quartic parity across sectors.} Using $3Q$ would make $\tilde Q_3^{\,2}$ integral but $\tilde Q_3^{\,4}$ too \emph{small} by a uniform factor of $3^4/6^4=1/16$, breaking the unit-weight anchor normalization across the abelian motifs. With $6Q$ both $\tilde Q^{\,2}$ and $\tilde Q^{\,4}$ are integers that land at unit motif weight at $\mu_\star$.
  \item \textbf{Two-loop QED structure.} The abelian kernel contains $Q_i^2$ and $Q_i^4$ pieces. Integerizing with $6$ aligns the rational coefficients of those terms into the species-independent kernels and leaves \emph{only} integers in counts $N_{Q2},N_{Q4}$. Using $3Q$ forces fractional remainders into the counts or a species-dependent kernel renormalization, both incompatible with the finite, common dictionary.
  \item \textbf{Cross-sector coherence with the QCD block.} The +4 color offset (four QCD motifs) adds an \emph{even} integer for quarks. With $6Q$, the abelian part has even parity alignment (e.g. $\tilde Q=\pm 6$ for leptons, $\pm 4, \pm 2$ for quarks), ensuring the total integer $Z$ sits on the same lattice class across sectors; $3Q$ produces a mismatched lattice and spoils equal-weight landing.
\end{itemize}
Empirically this necessity is sharp. An ablation replacing $6Q\to 5Q$ or $6Q\to 3Q$ breaks the anchor relation by orders of magnitude.

\subsection{Minimal dependence on representation details}\label{sec:minimal-rep}
The construction is deliberately insensitive to representation minutiae:
\begin{itemize}
  \item \textbf{Color block (quarks).} The nonabelian context contributes a fixed integer offset: at the anchor each of the four QCD motifs lands with unit weight, producing the $+4$ in~\ref{eq:Z-map} for all color--fundamental fermions. No further representation data enter $Z$.
  \item \textbf{Abelian block (all charged fermions).} The only species dependence in the abelian sector is through $\tilde Q^{\,2}$ and $\tilde Q^{\,4}$. This captures the entire charge sensitivity of the multi--loop QED mass anomalous dimension once motifs are grouped; higher abelian structures either vanish or regroup into these two powers at the anchor.
  \item \textbf{Neutrinos.} \emph{If} neutrinos are Dirac and $Q=0$, abelian motifs vanish and there is no color contribution, so $Z_\nu=0$ and hence $\mathcal F(0)=0$ at $\mu_\star$. No mass prediction beyond this conditional statement is implied.
\end{itemize}
All remaining physics—Casimirs, $\beta$--functions, decoupling/matching, and scheme—is carried by the \emph{common} kernels $\kappa_k(\mu)$ and the anchor policy; $Z$ itself is an \emph{integer invariant of the species label} through $(Q,\text{sector})$ and does not vary with scheme or thresholds.










\subsection{Normalized flow ODE and solution}
We define an auxiliary landing variable $Z_i(\mu)$ by the anchor-normalized flow
\begin{equation}
  \frac{d}{d\ln\mu}\,
  \ln\!\Bigl(1+\frac{Z_i(\mu)}{\kappa}\Bigr)
  \;=\;
  \gamma_i(\mu),
  \qquad
  Z_i(\mu_\star)=0,
  \label{eq:phi-normalized-flow}
\end{equation}
with $\gamma_i$ the standard mass anomalous dimension in $\overline{\rm MS}$ (QCD to 4L with $n_f:3\!\to\!4\!\to\!5\!\to\!6$ at $\mu=m_c,m_b,m_t$; QED to 2L with a single sector--global $\alpha(\mu)$ policy).  Integrating~\ref{eq:phi-normalized-flow} from $\mu=\mu_\star$ to the fixed point $\mu=m_i$ gives
\begin{equation}
  \ln\!\Bigl(1+\frac{Z_i(m_i)}{\kappa}\Bigr)
  \;=\;\int_{\ln\mu_\star}^{\ln m_i}\!\gamma_i(\mu)\,d\ln\mu
  \;=\;\lambda\,f_i(\mu_\star,m_i),
\end{equation}
hence
\begin{equation}
  f_i(\mu_\star,m_i)
  \;=\;\frac{1}{\lambda}\,
  \ln\!\Bigl(1+\frac{Z_i(m_i)}{\kappa}\Bigr).
  \label{eq:flow-solution}
\end{equation}
Equation~(\ref{eq:flow-solution}) matches the empirical form once we show that $Z_i(m_i)$ lands on the integer $Z_i$ specified by $(Q_i,\text{sector})$, which we verify numerically at $\mu_\star$.










\subsection{Why the multi--loop residue reorganizes to a single \texorpdfstring{$\{\mu_\star,\lambda,\kappa\}$}{\{mu*,lambda,kappa\}} triple}
\label{subsec:why-reorganizes}

\paragraph{Finite motif regrouping.}
Regroup the multi--loop insertions that build $\gamma_i$ into a finite set of \emph{motifs} with species--independent rates:
\begin{equation}
  \gamma_i(\mu)
  \;=\;
  \sum_{k\in\mathcal K}\,\kappa_k(\mu)\,N_k(W_i),
  \label{eq:motif-decomp}
\end{equation}
where $\kappa_k(\mu)$ carry the rational/Casimir data and running couplings, while $N_k(W_i)\in\mathbb Z_{\ge0}$ are \emph{integers} extracted from the reduced Dirac word $W_i$ (finite motif dictionary; formal definitions in Appendix~\ref{app:B}).

\paragraph{Equal--weight stationarity (PMS/BLM).}\footnote{For PMS and BLM scale setting, see Refs.~\cite{Stevenson81,BLM83}.}
Introduce integrated motif weights (do not confuse $\kappa_k$ with the constant $\kappa$)
\begin{equation}
  w_k(\mu;\lambda)\;\equiv\;\lambda^{-1}\!\int_{\ln\mu}^{\ln m_i}\!\kappa_k(\mu')\,d\ln\mu'.
  \label{eq:wk}
\end{equation}
For \emph{calibration only}, we replace the species endpoint $m_i$ by a species–independent logarithmic window of fixed length $\Delta$ (e.g., $\Delta=1$), so the minimizer $(\mu_\star,\lambda)$ depends only on the kernels $\kappa_k(\mu)$ and not on any mass input (see Appendix~\ref{app:A}). The variance objective uses only species–independent kernels over this fixed window; no target masses enter the calibration objective or its gradients. The audited equality checks restore the fixed endpoint integral for each species.

Choose $(\mu_\star,\lambda)$ to minimize $\operatorname{Var}_k[w_k(\mu;\lambda)]$ over the finite set $\mathcal K$ (PMS/BLM scale setting). At the stationary point one has
\begin{equation}
  w_k(\mu_\star;\lambda)\;=\;1+\delta_k,\qquad |\delta_k|\ll 1\ \ \text{for all }k,
  \label{eq:wk-equal}
\end{equation}
so that the flow solution Eq.~(\ref{eq:flow-solution}) yields
\begin{equation}
  \frac{1}{\lambda}\ln\!\Bigl(1+\frac{Z_i(m_i)}{\kappa}\Bigr)
  \;=\;\sum_k w_k(\mu_\star;\lambda)\,N_k(W_i)
  \;=\;\underbrace{\sum_k N_k(W_i)}_{=:Z_i\in\mathbb Z}\;+\;\sum_k \delta_k\,N_k(W_i).
  \label{eq:integer-landing}
\end{equation}
Thus $Z_i(m_i)$ \emph{lands} on the integer
$Z_i=\sum_k N_k(W_i)$ up to a bounded, species--agnostic correction $\sum_k\delta_k N_k(W_i)$ controlled by the common $\delta_k$.

\paragraph{Canonical normalization \texorpdfstring{$(\lambda,\kappa)$}{(lambda,kappa)}.}
We adopt the canonical display
\[
\mathrm{gap}(Z)=\frac{\ln(1+Z/\varphi)}{\ln\varphi}\,,
\]
which fixes $(\lambda,\kappa)=(\ln\varphi,\,\varphi)$ \emph{a priori} and removes all fitting freedom. A retrospective small-$Z$ check confirms consistency with the one-motif slope.

\paragraph{Scheme/threshold robustness (anchor invariance).}
Changing scheme (within $\overline{\rm MS}$ families) or moving heavy--flavor thresholds $\mu=m_c,m_b,m_t$ coherently shifts the species--independent kernels $\kappa_k(\mu)$ and hence the $w_k$ by \emph{common} amounts at $\mu_\star$. The PMS/BLM minimizer $(\mu_\star,\lambda)$ moves continuously and absorbs those shifts, leaving the integer landing Eq.~(\ref{eq:integer-landing}) intact to the stated tolerance. Quantitative bounds on the induced $\delta_k$ are given in Appendix~\ref{app:B}.

%\subsection{Numerical verification at the anchor}
%With QCD 4L + QED 2L kernels, the fixed $n_f:3\!\to\!4\!\to\!5\!\to\!6$ threshold policy at $(m_c,m_b,m_t)$, and the anchor $\mu_\star=182.201$~GeV, we obtain
%\begin{equation}
%  \max_i\;\bigl|f_i(\mu_\star,m_i)-\mathcal F(Z_i)\bigr|\;\le\;10^{-6}
%  \quad\text{for all quarks and charged leptons,}
%\end{equation}
%with non--circular comparisons (PDG values transported to $\mu_\star$ using the \emph{same} kernels) and an automated CI guard that fails if the tolerance is exceeded.

\paragraph{Out-of-sample test (all charged fermions).}
With $(\mu_\star,\lambda,\kappa)$ fixed by the mass-free calibration, we evaluate $f_i(\mu_\star,m_i)$ for all nine charged fermions $(u,d,s,c,b,t,e,\mu,\tau)$ with \emph{no} retuning and compare to $\mathcal F(Z_i)$. We obtain $\max_i|f_i-\mathcal F(Z_i)|\le 10^{-6}$ under the central kernels and policies; scheme/threshold and $\alpha(\mu)$ variants move equal--$Z$ families coherently and remain within the stated tolerance bands.


















\section{Discussion and Conclusion}\label{sec:discussion}

We have established a single-anchor phenomenological identity for Standard-Model running masses. At anchor $\mu_\star=182.201\,\mathrm{GeV}$, determined once by a mass-free PMS/BLM stationarity over species-independent kernels, the dimensionless RG residues for all nine charged fermions satisfy
\[
f_i(\mu_\star,m_i)\;=\;\mathcal F(Z_i)\;=\;\frac{1}{\lambda}\ln\!\bigl(1+Z_i/\kappa\bigr)
\]
within tolerance $10^{-6}$, where $Z_i$ is an integer constructed solely from electric charge and sector, and $(\lambda,\kappa)=(\ln\varphi,\varphi)$ are fixed \emph{a priori}. Equal-$Z$ families (up, down, lepton) are degenerate at the anchor, and global kernel/policy variants move them coherently while preserving the identity and tolerance.

Our model appears as  \emph{anchor-specific}. While  at $\mu_\star$ the integer structure of $Z$
is manifest,  off the anchor, standard SM RG running applies and the degeneracy lifts
at $\mathcal{O}[(\ln\mu/\mu_\star)^2]$. Inputs ($\alpha_s(M_Z)$, threshold placements, EM policy) are taken from world averages with documented uncertainties; the identity is robust within those bands but is not claimed beyond them. The physical origin of the golden ratio $\varphi$ in $(\lambda,\kappa)$ is \emph{unknown}; here it serves as a fixed, phenomenological normalization. The motif-regrouping mechanism (Appendix~\ref{app:B}) explains how integer landing emerges from stationarity, but why $\varphi$ appears and why the anchor lands near 182~GeV remain open theoretical questions. 

 A deeper explanation for the specific value of the anchor scale may lie in its proximity to the electroweak scale. The anchor at $\mu_\star\approx182\,\mathrm{GeV}$ is remarkably close to the top quark pole mass ($m_t \approx 172.5\,\mathrm{GeV}$), the particle that dominates running behavior in this regime through its large Yukawa coupling. The PMS/BLM stationarity condition, which seeks a scale of minimal scheme dependence by balancing disparate contributions to the RG flow, is naturally sensitive to the scale at which the most massive particle in the theory begins to decouple. It is plausible that $\mu_\star$ represents an "RG focus point" where the strong influence of the top quark creates a unique balance among the QCD and QED motifs, forcing their integrated weights to the near-unity values observed. Furthermore, this energy scale is critical for the stability of the electroweak vacuum, as the running of the Higgs self-coupling is famously sensitive to the precise values of $m_t$ and $m_H$. The anchor's location could thus be a signal that the observed mass regularities are intrinsically connected to the structure and stability of the Higgs mechanism.

Near-term directions for future work include: (i) clarifying the mathematical origin of $(\lambda,\kappa)=(\ln\varphi,\varphi)$; (ii) scanning other scales/windows for related stationarity structures; (iii) extending to higher orders (5-loop QCD, 3-loop QED) and alternative schemes (on-shell, $\overline{\mathrm{DR}}$); (iv) strengthening uncertainty propagation and input-parameter correlations; and (v) further exploring the connection to electroweak vacuum stability. If the observation persists under refined inputs and next-generation precision, it may point to a hidden algebraic structure in the SM mass sector.

%\paragraph{Quantitative falsifiers.}
%We specify decision thresholds for falsification:
%\begin{enumerate}
%  \item \emph{Equal-$Z$ splitting}: any $\max_{i,j:\,Z_i=Z_j}|f_i-f_j|>10^{-6}$ at $\mu_\star$ under declared kernels/policies.
%  \item \emph{Identity violation}: $\max_i|f_i-\mathcal F(Z_i)|>10^{-6}$ within the PDG/policy bands.
%  \item \emph{Incoherent variants}: under global kernel/threshold sweeps, equal-$Z$ families move by non-uniform amounts $\Delta f_i^{(v)}$ such that $\max_{i,j:\,Z_i=Z_j}|\Delta f_i^{(v)}-\Delta f_j^{(v)}|>10^{-6}$.
%  \item \emph{Ablation survival}: targeted ablations (e.g., removing "$+4$", dropping $Q^4$, or $6Q\to3Q$) do \emph{not} produce violations $\gg10^{-6}$. Survival would imply the integer structure is incidental, falsifying specificity.
%\end{enumerate}
%All four conditions are violated under the current data and kernels, confirming robustness (Appendix~\ref{app:E},
%Table~\ref{tab:variant-max}).

%\paragraph{Threats to validity and mitigations.}
%\emph{Hidden circularity}: mitigated by a stationarity window that depends only on species-independent kernels
%(Appendix~\ref{app:A} proves no measured mass enters calibration). \emph{Scheme/threshold tuning}: mitigated by testing variants and observing coherent equal-$Z$ motion with preserved tolerance. \emph{IR sensitivity}: light-quark residues tested with frozen $\alpha_s$, continued running, and windowed kernels; worst-case $\Delta\approx5.9\times10^{-8}$. \emph{Numerical precision}: CI guards enforce $10^{-6}$ tolerance on every commit; artifacts are archived for independent reproduction.

As a conclusion, we report a phenomenological observation: at a single anchor scale determined by mass-free calibration, the integrated RG residues of all charged Standard-Model fermions collapse to a simple, integer-indexed closed form with tolerance $10^{-6}$. The integer structure is specific (ablations fail decisively), robust (variants preserve equal-$Z$ coherence), and non-circular (measured masses never appear on the right-hand side). The appearance of the golden ratio in the normalization constants is unexplained and a subject for future theoretical work. All data, code, and CI guards are publicly available for independent verification.

%\subsection*{Phenomenology framing}
%
%We adopt a fixed regrouping of contributions to the mass anomalous dimension into a finite, species–independent dictionary and evaluate the resulting integrated residues at a single common scale. At that scale we observe a simple charge–indexed integer $Z$ organizes the results into three bands with small residuals under standard variations. We emphasize this is an empirical regularity reported with scripts, falsifiers, and uncertainty scans; no beyond–SM mechanism is asserted.

%The normalization constants $(\lambda,\kappa)$ are fixed once by a species–independent normalization procedure and then held fixed. No per–species inputs are tuned; variations (scheme, loop order, thresholds, $\alpha(\mu)$ policy) are applied globally and reported through the artifact scans.
%Finally, any exponent-type ratio regularities are recorded separately (Appendix~\ref{app:E}) as observations and are not part of the main empirical claim.










%\section{Acknowledgments}
%
%We thank colleagues at the Recognition Science \& Recognition Physics Institute for discussions that improved the presentation. Any remaining errors are the author's.
\vspace{0.5cm}
Data and Code Availability:  
All code and data needed to reproduce the results are available in the project repository and archived artifacts \cite{fundamental-masses-repo}. Repository URL and archive DOI are provided in the references and in the artifact manifest.






\newpage
\clearpage

\appendix
% Use A, B, ... for appendix sections and A1, A2, ... for subsections
\makeatletter
\renewcommand\thesection{\Alph{section}}
\renewcommand\thesubsection{\thesection\arabic{subsection}}
% Keep hyperref equation anchors consistent in appendices as A.1, B.1, ...
\renewcommand*{\theHequation}{\thesection.\arabic{equation}}
\makeatother
\section{ Definitions and Audit Protocol}\label{app:A}   % Appendix A:

%\subsection*{A.1 Residue and transport (definitions)}
For each charged fermion species $i$, the $\overline{\mathrm{MS}}$ running mass $m_i(\mu)$ obeys
\[
\mu\,\frac{d\ln m_i(\mu)}{d\mu}=\gamma_i(\mu),\qquad \gamma_i(\mu)=\gamma_m^{\mathrm{QCD}}\bigl(\alpha_s(\mu),n_f(\mu)\bigr)+\gamma_m^{\mathrm{QED}}\bigl(\alpha(\mu),Q_i\bigr).
\]
The experimental residue at the single anchor $\mu_\star$ is
\[
f_i^{\mathrm{(exp)}}(\mu_\star,m_i)=\lambda^{-1}\!\int_{\ln\mu_\star}^{\ln m_i^{\mathrm{PDG}}(m_i)}\gamma_i(\mu)\,d\ln\mu.
\]
PDG values at their reference scales $\mu_0$ are transported to the anchor by
\begin{equation}
m_i^{\rm PDG\to\mu_\star}=m_i^{\rm PDG}(\mu_0)\exp\!\Bigl[\int_{\ln\mu_0}^{\ln\mu_\star}\gamma_i(\mu)\,d\ln\mu\Bigr].
\label{eq:pdg-transport}
\end{equation}

The audit protocol (non–circular) includes the following steps. 
\begin{enumerate}%[label=(A\arabic*)]
  \item Acquire $m_i^{\rm PDG}(\mu_0)$ at its quoted $\mu_0$.
  \item Transport to $\mu_\star$ using the same kernels/policies as prediction.
  \item Form $f_i^{\mathrm{(exp)}}(\mu_\star,m_i)$.
  \item Compare to $\mathcal F(Z_i)$ with $(\lambda,\kappa)$ fixed a priori and $\mu_\star$ fixed once by species–independent stationarity.
\end{enumerate}
%Measured masses never appear on the right–hand side.

%\paragraph{Reproducibility (artifact pointers).}
%CSV tables with residues, transports, and sensitivity sweeps, as well as the CI guard that enforces the $10^{-6}$ tolerance, are included in the artifact bundle (see manifest). File paths follow the pattern \texttt{out/csv/\{run\}/residues-*.csv} and \texttt{out/csv/\{run\}/transport-*.csv}; the run manifest is recorded at \texttt{out/log/manifest.txt}. The same kernels/policy are used for both transport and residue.

%\subsection*{A.3 Well–known background}
%{\color{red}Standard transport and RG background are well known; retained here for completeness and your decision on inclusion.}
%
%\paragraph{A.\,Summary.}
%Appendix~A fixes the residue and transport definitions and states the non--circular audit protocol used throughout. These conventions are the sole source for how experimental inputs are transported and compared at the anchor.





%\section{Integer map \texorpdfstring{$Z_i$}{Zi} and charge integerization}\label{app:C}
%\subsection*{C.1 Definition}
%Let $\tilde Q:=6Q_i\in\mathbb Z$. Then
%\[
%Z_i=\begin{cases}
%4+\tilde Q^{\,2}+\tilde Q^{\,4},& \text{quarks (color fundamental)},\\
%\tilde Q^{\,2}+\tilde Q^{\,4},& \text{charged leptons (color singlet)},\\
%0,& \text{Dirac neutrinos }(Q=0).
%\end{cases}
%\]
%The quark ``$+4$'' is the coherent unit landing of four QCD motifs on a single fundamental color line.
%
%\subsection*{C.2 Why \texorpdfstring{$6Q$}{6Q} (and not \texorpdfstring{$3Q$}{3Q})}
%{\color{red}Using $3Q$ makes $Q^2$ integral but underweights $Q^4$ by $3^4/6^4=1/16$, breaking anchor unit weights across abelian motifs; $6Q$ preserves integer parity across sectors and coherence with the quark $+4$.}
%
%\paragraph{C.\,Summary.}
%Appendix~\ref{app:C} defines the integer map $Z_i$, explains why the 6$Q$ lattice is necessary, and provides worked examples aligned with the main text usage.




%\section{ Stationarity and integer landing}\label{app:D}
%\subsection*{D.1 Motif regrouping}
%\[
%\gamma_i(\mu)=\sum_{k\in\mathcal K}\kappa_k(\mu)\,N_k(W_i),\quad N_k(W_i)\in\mathbb Z_{\ge0}.
%\]
%\subsection*{D.2 Equal–weight stationarity (PMS/BLM)}
%Define motif weights $w_k(\mu;\lambda)=\lambda^{-1}\!\int\kappa_k\,d\ln\mu$. Choose $(\mu_\star,\lambda)$ to minimize $\operatorname{Var}_k[w_k]$ over a fixed species–independent window. Then
%\[
%w_k(\mu_\star;\lambda)=1+\delta_k,\qquad |\delta_k|\ll1.
%\]
%\subsection*{D.3 Integer landing and bound}
%The normalized flow gives
%\[
%\frac{1}{\lambda}\ln\Bigl(1+\frac{Z_i(m_i)}{\kappa}\Bigr)=\sum_k w_k(\mu_\star;\lambda)\,N_k(W_i)=Z_i+\sum_k\delta_kN_k(5W_i),
%\]
%so $Z_i(m_i)$ lands on the integer $Z_i$ up to a bounded correction $\sum_k\delta_kN_k(W_i)$.
%
%\paragraph{D.\,Summary.}
%Appendix~\ref{app:D} contains the PMS/variance stationarity conditions, the integer landing lemma, and explicit bounds on induced drifts; it is the formal counterpart to the Methods discussion.
%
%\subsection*{D.4 Well–known PMS/BLM background}
%{\color{red}General PMS/BLM background is standard; retained in red pending your decision.}




\section{Multi\texorpdfstring{--}{--}loop reorganization and the triple \texorpdfstring{$\{\mu_\star,\lambda,\kappa\}$}{\{mu*,lambda,kappa\}}}  \label{app:B}

%\noindent Throughout, $i$ labels a fermion species, $\mu$ is the renormalization scale, and
%\begin{equation}
%  \gamma_i(\mu)\;=\;\gamma^{\mathrm{QCD}}_m\!\bigl(\alpha_s(\mu),n_f(\mu)\bigr)\;+\;\gamma^{\mathrm{QED}}_m\!\bigl(\alpha(\mu),Q_i\bigr)
%  \label{eq:A-gamma}
%\end{equation}
%is the Standard--Model mass anomalous dimension in $\overline{\mathrm{MS}}$ at fixed loop orders (QCD 4L, QED 2L) with a conventional heavy--flavor threshold policy for $n_f(\mu)$.
%The anchor residue is
%\begin{equation}
%  f_i(\mu_\star,m_i)\;=\;\frac{1}{\lambda}\int_{\ln\mu_\star}^{\ln m_i}\gamma_i(\mu)\,d\ln\mu,
%  \label{eq:A-res}
%\end{equation}
%where $\lambda>0$ is a normalization constant to be fixed once, and the closed--form comparator (``gap'') is
%\begin{equation}
%  \mathcal F(Z)\;=\;\frac{1}{\lambda}\,\ln\!\bigl(1+Z/\kappa\bigr),\qquad \kappa>0.
%  \label{eq:A-gap}
%\end{equation}

\subsection{ Motif integrals, PMS/variance conditions, and uniqueness of the stationary anchor}

\paragraph{Species\,–\,independent calibration window (no mass inputs).}
For \emph{calibration only}, we define motif weights over a fixed logarithmic window \(\Delta>0\), common to all species and independent of any experimental mass:
\begin{equation}
  w_k^{(\Delta)}(\mu;\lambda)
  \;:=\; \frac{1}{\lambda}\int_{\ln\mu}^{\ln\mu+\Delta}\!\kappa_k(\mu')\,d\ln\mu'.
  \label{eq:A-wk-delta}
\end{equation}
This uses a \emph{species-independent} window of fixed length \(\Delta\). In our runs we fix \(\Delta=1.0\). All inputs entering \(\kappa_k(\mu)\) are Standard-Model couplings, loop coefficients, and decoupling prescriptions (Appendix~\ref{app:B}); no measured fermion mass under test appears in Eq.~(\ref{eq:A-wk-delta}). The chosen \(\Delta\) is recorded in the artifact run log.

\paragraph{Motif regrouping.}
Regroup the multi--loop contributions to $\gamma_i$ into a finite set of \emph{motifs} with \emph{species--independent} rate kernels $\kappa_k(\mu)$ and \emph{integer} counts $N_k(W_i)$:
\begin{equation}
  \gamma_i(\mu) \;=\; \sum_{k\in\mathcal K}\,\kappa_k(\mu)\,N_k(W_i),
  \qquad
  \mathcal K=\{\,F,\ NA,\ V,\ G,\ Q2,\ Q4\,\}.
  \label{eq:A-motif}
\end{equation}
Here:
\begin{itemize}
  \item $F$ (\emph{fundamental self--energy/vertex}): absorbs the $C_F$--proportional QCD contribution at all loop orders;
  \item $NA$ (\emph{non--abelian exchange}): absorbs the $C_FC_A$ structures;
  \item $V$ (\emph{vacuum polarization on gauge lines}): absorbs the $C_F\,T_F\,n_f(\mu)$ structures;
  \item $G$ (\emph{quartic--gluon/four--gluon class}): absorbs the residual purely non--abelian higher--loop structures (e.g.\ $C_A^2$ combinations affecting $\gamma_m$);
  \item $Q2$, $Q4$ (\emph{abelian charge motifs}): absorb the QED $Q^2$ and $Q^4$ structures (two--loop and mixed terms regrouped accordingly).
\end{itemize}
All dependence on the species label $i$ sits in \emph{integers} $N_k(W_i)$.  At fixed representation,
\[
  N_F=N_{NA}=N_V=N_G=
  \begin{cases}
    1,& \text{quark in the fundamental (color)},\\
    0,& \text{lepton (color singlet)},
  \end{cases}
\qquad
N_{Q2}=(6Q_i)^2,\quad N_{Q4}=(6Q_i)^4,
\]
where $\tilde Q:=6Q_i\in\mathbb Z$ ensures integrality of the abelian counts.
With this dictionary the \emph{charge--structured index}
\begin{equation}
  Z(W_i)\;:=\;N_F+N_{NA}+N_V+N_G+N_{Q2}+N_{Q4}
  \;=\;
  \begin{cases}
    4+(6Q_i)^2+(6Q_i)^4,& \text{quarks},\\[2pt]
    (6Q_i)^2+(6Q_i)^4,& \text{charged leptons},\\[2pt]
    0,& \text{Dirac neutrinos},
  \end{cases}
  \label{eq:A-Z.def}
\end{equation}
is an \emph{integer} fixed by $(Q_i,\text{sector})$.


\paragraph{Explicit crosswalk (symbolic).}
Write the standard expansions (suppressing $\mu$-arguments)
\begin{align}
  \gamma^{\mathrm{QCD}}_m
  &=\sum_{n\ge1} a_s^n
    \Bigl[
      C_F\,A_n\;+\;C_FC_A\,B_n\;+\;C_FT_Fn_f\,C_n\;+\;\text{(higher nonabelian)}
    \Bigr],\qquad a_s:=\frac{\alpha_s}{\pi},\label{eq:A-QCD.symb}\\
  \gamma^{\mathrm{QED}}_m
  &=\sum_{n\ge1}\alpha^n
    \Bigl[
      Q_i^2\,b_{n,2}\;+\;Q_i^4\,b_{n,4}\;+\;\text{(higher powers regrouped)}
    \Bigr].\label{eq:A-QED.symb}
\end{align}
Then choose the species--independent kernels
\begin{align}
  \kappa_F    &:= \sum_{n\ge1} A_n\,a_s^n,           &
  \kappa_{NA} &:= \sum_{n\ge2} (C_A\,B_n)\,a_s^n,\notag\\
  \kappa_V    &:= \sum_{n\ge2} (T_F\,n_f)\,C_n\,a_s^n,&
  \kappa_G    &:= \sum_{n\ge3} \text{(pure $C_A$ combinations)}\,a_s^n,\notag\\
  \kappa_{Q2} &:= \sum_{n\ge1} b_{n,2}\,\alpha^n,     &
  \kappa_{Q4} &:= \sum_{n\ge2} b_{n,4}\,\alpha^n,\label{eq:A-kappa.def}
\end{align}
so that Eq.~(\ref{eq:A-motif}) reproduces Eqs.~(\ref{eq:A-QCD.symb})–(\ref{eq:A-QED.symb})
once the integer counts $N_k(W_i)$ above are inserted.  Threshold stepping for $n_f(\mu)$ enters \emph{only} through $\kappa_V$ and is common to all quark species.


\subsection{Integer landing lemma and a crisp bound}

\paragraph{Landing variable and its integer target.}
Define the \emph{landing variable}
\begin{equation}
  Z_i(m_i)\;:=\;\sum_{k=1}^K w_k(\mu_\star;\lambda)\,N_k(i),
  \label{eq:A-Zi}
\end{equation}
and the corresponding \emph{integer target}
\begin{equation}
  Z(i)\;:=\;\sum_{k=1}^K N_k(i)\;\in\;\mathbb Z_{\ge 0}.
  \label{eq:A-Zint}
\end{equation}
Write $w_k=1+\delta_k$ at the stationary anchor; let $\delta_{\max}:=\max_k|\delta_k|$ and $N_{\mathrm{tot}}(i):=\sum_k N_k(i)$.

\paragraph{Lemma (integer landing and sharp deviation bound).}
At the PMS/variance anchor $(\mu_\star,\lambda)$,
\begin{equation}
  Z_i(m_i)\;=\;Z(i)\;+\;\sum_{k=1}^K \delta_k\,N_k(i),
  \qquad
  \bigl|Z_i(m_i)-Z(i)\bigr|\;\le\;\delta_{\max}\,N_{\mathrm{tot}}(i).
  \label{eq:A-int-landing}
\end{equation}
\emph{Proof.} Substitute $w_k=1+\delta_k$ into Eq.~(\ref{eq:A-Zi}). The bound follows by the triangle inequality. %\qed

\paragraph{From $Z$ to the residue (deterministic inequality).}
Using $f_i=(1/\lambda)\ln(1+Z_i/\kappa)$ and $\mathcal F(Z)=(1/\lambda)\ln(1+Z/\kappa)$,
\begin{equation}
  \bigl|f_i-\mathcal F\bigl(Z(i)\bigr)\bigr|
  \;=\;\frac{1}{\lambda}\,\Bigl|\ln\frac{1+Z_i/\kappa}{1+Z(i)/\kappa}\Bigr|
  \;\le\;\frac{|Z_i-Z(i)|}{\lambda\,(\kappa+Z_{\min})}
  \;\le\;\frac{\delta_{\max}\,N_{\mathrm{tot}}(i)}{\lambda\,(\kappa+Z_{\min})},
  \label{eq:A-f-bound}
\end{equation}
where $Z_{\min}:=\min\{Z_i,\,Z(i)\}\ge 0$ and we used $|\ln(1+x)-\ln(1+y)|\le |x-y|/(1+\min\{x,y\})$ for $x,y\ge 0$.
Thus the finite--order drift in $f_i$ is explicitly bounded in terms of the maximal motif weight deviation and the total motif count.

\paragraph{Perturbative control of $\delta_{\max}$.}
Expanding each kernel as $\kappa_k(\mu)=c_k a(\mu)+d_k a(\mu)^2+\cdots$ with a running coupling $a\in\{\alpha_s,\alpha\}$, and fixing $(\mu_\star,\lambda)$ so the LL slopes are aligned, one has
\begin{equation}
  |\delta_k|\;\le\;C_k\,\sup_{\mu\in[\mu_\star,m_i]}\!\bigl\{a(\mu)\bigr\}\;\equiv\;C_k\,\epsilon,
  \qquad\Rightarrow\qquad
  \delta_{\max}\;\le\;C\,\epsilon,
  \label{eq:A-delta-LL}
\end{equation}
with a constant $C$ depending only on rational data (Casimirs, loop factors).
Inserting Eq.~(\ref{eq:A-delta-LL}) into Eq.~(\ref{eq:A-f-bound}) yields an explicit LL/NLL control of the anchor deviation.

\subsection{ Scheme/threshold robustness at the stationary anchor}

\paragraph{Setup.}
Let $S$ and $S'$ be two admissible global choices (e.g.\ renormalization scheme variant; heavy--flavor threshold placements within accepted ranges; a sector--coherent $\alpha(\mu)$ policy). They induce motif weights $w_k$ and $w'_k$, and (possibly slightly shifted) stationary points $(\mu_\star,\lambda)$ and $(\mu'_\star,\lambda')$ obtained by the variance minimization in each choice.

\paragraph{First--order common shifts.}
Linearizing around $(\mu_\star,\lambda)$,
\begin{equation}
  \delta w_k\;:=\;w'_k-w_k
  \;=\;\underbrace{\frac{\partial w_k}{\partial(\ln\mu_\star)}\delta(\ln\mu_\star)
  +\frac{\partial w_k}{\partial\lambda}\delta\lambda}_{\text{anchor re-centering}}
  \;+\;\underbrace{\Delta_k}_{\text{direct kernel/policy change}},
  \label{eq:A-w-diff}
\end{equation}
with $\partial_{\ln\mu_\star}w_k=-(1/\lambda)\kappa_k(\mu_\star)$ and $\partial_\lambda w_k=-(1/\lambda)w_k$.
By construction of the new stationary point $(\mu'_\star,\lambda')$, the projection of $(\delta w_k)_k$ onto the two directions $\{\,\kappa_k(\mu_\star),\,w_k\,\}$ is \emph{eliminated} at first order; i.e.\ there exist $\delta(\ln\mu_\star)$ and $\delta\lambda$ such that the residual
\begin{equation}
  \rho_k\;:=\;\delta w_k+\frac{1}{\lambda}\kappa_k(\mu_\star)\,\delta(\ln\mu_\star)
                   +\frac{1}{\lambda}w_k\,\delta\lambda
  \label{eq:A-rho}
\end{equation}
is orthogonal (in the $k$--index inner product) to both $\kappa_k(\mu_\star)$ and $w_k$. Consequently, \emph{to first order} the difference between $S$ and $S'$ appears as a \emph{common re-centering} of the anchor and normalization; any remaining difference sits in the small, stationary residual $\rho_k$.

\paragraph{Inequality for the induced drift in $f_i$.}
We have $\delta f_i=\sum_k \delta w_k\,N_k(i)$ and therefore
\begin{equation}
  \bigl|\delta f_i\bigr|
  \;=\;\Bigl|\sum_{k=1}^K \rho_k\,N_k(i)\Bigr|
  \;\le\; N_{\mathrm{tot}}(i)\,\max_k|\rho_k|
  \;=\;N_{\mathrm{tot}}(i)\,\|\rho\|_{\infty}.
  \label{eq:A-fi-drift}
\end{equation}
Translating this to the closed form %via \eqref{eq:A-gap}
and the chain rule gives
\begin{equation}
  \Bigl|\delta\Bigl[\frac{1}{\lambda}\ln\!\Bigl(1+\frac{Z_i}{\kappa}\Bigr)\Bigr]\Bigr|
  \;\le\;\frac{N_{\mathrm{tot}}(i)}{\lambda\,(\kappa+Z_i)}\,\|\rho\|_{\infty}
  \;\;\le\;\;\frac{N_{\mathrm{tot}}(i)}{\lambda\,\kappa}\,\|\rho\|_{\infty}.
  \label{eq:A-gap-drift}
\end{equation}
Thus the \emph{scheme/threshold--induced} change in $f_i$ at the re--centered anchor is bounded by the sup--norm of the stationary residual profile $\rho_k$, uniformly across species up to the integer factor $N_{\mathrm{tot}}(i)$ and the global constants $(\lambda,\kappa)$.

\paragraph{Equal\texorpdfstring{$\text{-}Z$}{-Z} coherence (consequence).}
If two species $i,j$ share the same integer $Z$ (hence the same $N_k$ profile up to a common sectoral structure), then $N_{\mathrm{tot}}(i)=N_{\mathrm{tot}}(j)$ and $Z_i=Z_j$. Equation~(\ref{eq:A-gap-drift}) implies
\begin{equation}
  \delta f_i-\delta f_j\;=\;0\quad\text{to first order in the global change,}
  \label{eq:A-equalZ-coherence}
\end{equation}
i.e.\ equal--$Z$ families move \emph{coherently} under admissible scheme/threshold/policy variations at the stationary anchor. Any observed splitting within an equal--$Z$ family is therefore second order in the small residuals and bounded by Eq.~(\ref{eq:A-gap-drift}).

\medskip\noindent
\emph{Summary of this Appendix.} A finite regrouping of multi--loop contributions yields species--independent motif kernels and integer counts. Minimizing the variance of motif weights fixes a unique stationary anchor $\mu_\star$ (and normalization $\lambda$), while a small--$Z$ slope match fixes $\kappa$, thereby determining the triple $\{\mu_\star,\lambda,\kappa\}$. At this anchor each species lands on its integer target up to a controlled deviation bounded by Eq,~*=(\ref{eq:A-int-landing})--(\ref{eq:A-f-bound}). Admissible scheme/threshold changes induce only common, first--order shifts (reabsorbed by re--centering), with the residual drift in $f_i$ explicitly bounded by Eq.~( \ref{eq:A-gap-drift}); equal--$Z$ families remain coherent to first order.



















% =========================
% Appendix B. QCD/QED kernels and running (MSbar)
% =========================
\section{QCD/QED kernels and running \texorpdfstring{($\overline{\mathrm{MS}}$)}{(MS-bar)}}

Throughout this appendix we work in the $\overline{\mathrm{MS}}$ scheme with massless decoupling. We define
\[
a_s \equiv \frac{\alpha_s}{\pi}, \qquad a_e \equiv \frac{\alpha}{\pi},
\]
and write renormalization–group equations as
\[
\mu\frac{d a_s}{d\mu}=\beta_s(a_s),\qquad
\mu\frac{d \ln m_i}{d\mu} = \gamma_i(\mu)
= \gamma_m^{\rm QCD}(a_s) + \gamma_m^{\rm QED}(a_e;Q_i,\{Q_f\}).
\]
 $\beta$-function is used to evolve $\alpha_s(\mu)$ with scale.
This tells you how the strong coupling runs from one energy scale to another.
Inside the mass anomalous dimension $\gamma_m^{\text{QCD}}(a_s)$ :
The mass anomalous dimension depends on $\alpha_s(\mu)$. 
To compute $\gamma_m^{\text{QCD}}$ at any scale $\mu$, you need $\alpha_s(\mu)$. 
To get $\alpha_s(\mu)$ from the reference value $\alpha_s(M_Z)$, you integrate the  $\beta$-function.
In the motif kernels $\kappa_k(\mu)$. 
The species-independent kernels contain "running couplings" $\alpha_s(\mu)$. 
These evolve via $\beta$-function. 

For the calibration window integral 
$   w_k^(\Delta)(\mu ; \lambda) = (1/\lambda) \int [\ln \mu \to \ln \mu + \Delta] k_k(\mu') d \ln \mu'$
Even though this is mass-free, $\kappa_k(\mu')$ contains $\alpha_s(\mu')$
You need $\beta$-function to evolve $\alpha_s$ across the window
The chain:
Without the $\beta$-function, you cannot compute $\alpha_s(\mu)$ at any scale except $M_Z$,
which means you cannot compute the mass anomalous dimension at any other scale, which means you cannot integrate to get the residue $f_i$.
The $\beta$-function is absolutely necessary for the entire RG analysis!













\subsection{ QCD \texorpdfstring{$\beta_s$}{beta\_s} to four loops}
We use the standard MS-bar  $\overline{\mathrm{MS}}$ form for the  QCD beta function up to 4-loops:
\begin{equation}
\beta(a_s) = - \sum_{n=0}^{3} \beta_n a_s^{n+2}, \quad a_s = \frac{\alpha_s}{\pi}.
\end{equation}
\[
\beta_s(a_s)= -\beta_0\,a_s^2 - \beta_1\,a_s^3 - \beta_2\,a_s^4 - \beta_3\,a_s^5 + \mathcal{O}(a_s^6),
\]
with
\begin{align*}
\beta_0 &= \frac{11}{3}C_A - \frac{4}{3}T_F n_f,\\[2pt]
\beta_1 &= \frac{34}{3}C_A^2 - 4 C_F T_F n_f - \frac{20}{3} C_A T_F n_f,\\[2pt]
\beta_2 &= \frac{2857}{54}C_A^3 + 2 C_F^2 T_F n_f - \frac{205}{9} C_F C_A T_F n_f - \frac{1415}{27} C_A^2 T_F n_f
+ \frac{44}{9} C_F T_F^2 n_f^2 + \frac{158}{27} C_A T_F^2 n_f^2,\\[2pt]
\beta_3 &= \text{(full 4L analytic expression)}.
\end{align*}
Color factors for a general simple group are $C_A$, $C_F$, $T_F$; for $\mathrm{SU}(3)$, $C_A=3$, $C_F=4/3$, $T_F=1/2$.

%
%\begin{align}
%\beta_s(a_s)&= -\beta_0 a_s^2 - \beta_1 a_s^3 - \beta_2 a_s^4 - \beta_3 a_s^5 + \mathcal O(a_s^6),\\
%\beta_0&=\tfrac{11}{2}-\tfrac{n_f}{3},\quad
%\beta_1=\tfrac{51}{4}-\tfrac{19}{12}n_f,\\
%\beta_2&=\tfrac{2857}{64}-\tfrac{5033}{384}n_f+\tfrac{325}{1152}n_f^2,\\
%\beta_3&=\tfrac{149753}{768}+\tfrac{1078361}{41472}\zeta_3-\tfrac{50065}{41472} n_f+\tfrac{260861}{20736}\zeta_3 n_f\\
%&\quad-\tfrac{1093}{4608}n_f^2-\tfrac{65}{2592}\zeta_3 n_f^2+\mathcal O(n_f^3).
%\end{align}
%
%
%For $\mathrm{SU}(3)$ this becomes (numerical, with $n_f$ active flavors)
%\[
%\begin{aligned}
%\beta_s(a_s)=\;& -\Big(2.750000 - 0.166667\,n_f\Big)a_s^2\\
%& -\Big(6.375000 - 0.791667\,n_f\Big)a_s^3\\
%& -\Big(22.3203 - 4.36892\,n_f + 0.0940394\,n_f^2\Big)a_s^4\\
%& -\Big(114.230 - 27.1339\,n_f + 1.58238\,n_f^2 + 0.00585670\,n_f^3\Big)a_s^5,
%\end{aligned}
%\]
%where $\zeta_3$ terms are included in the quoted numbers.

\subsection{ QCD quark–mass anomalous dimension \texorpdfstring{$\gamma_m^{\rm QCD}$}{gamma\_m(QCD)} to four loops}
The QCD mass anomalous dimension is:
\begin{equation}
\gamma_m^{\mathrm{QCD}}(a_s) = - \sum_{n=0}^{3} \gamma_n a_s^{n+1}.
\end{equation}
The coefficients $\beta_n$ and $\gamma_n$ are taken from \cite{vanRitbergenVermaserenLarin97, VermaserenLarinRitbergen97}. The QED anomalous dimension is used to 2-loops.
We write
\[
\gamma_m^{\rm QCD}(a_s) = -\gamma_0\,a_s - \gamma_1\,a_s^2 - \gamma_2\,a_s^3 - \gamma_3\,a_s^4 + \mathcal{O}(a_s^5),
\]
with the well–known color–factor expressions (1–2 loop shown explicitly; 3–4 loop are lengthy but standard):
\begin{align}
\gamma_0 &= \tfrac{3}{4}\,C_F,\\[2pt]
\gamma_1 &= \tfrac{1}{16}\!\left(\tfrac{3}{2}C_F^2 + \tfrac{97}{6} C_F C_A - \tfrac{10}{3} C_F T_F n_f\right),\\[2pt]
%\gamma_2,\ \gamma_3 &\ \text{: full analytic MS-bar expressions (including $\zeta_3,\zeta_4,\zeta_5$) used in code; see bibliography below.}
%\gamma_m^{\rm QCD}(a_s)&= -\sum_{k=0}^3\gamma_k a_s^{k+1},\\
%\gamma_0&=1,\quad \gamma_1=\tfrac{101}{12}-\tfrac{5}{18}n_f,\\
\gamma_2&=\tfrac{1249}{24}-\Bigl(\tfrac{277}{54}+\tfrac{13}{3}\zeta_3\Bigr)n_f-\tfrac{35}{648}n_f^2,\\
\gamma_3&=\text{(full 4L expression; retained symbolically)}.
\end{align}



For $\mathrm{SU}(3)$, a compact numerical form (with $a_s=\alpha_s/\pi$) is
\[
\begin{aligned}
\gamma_m^{\rm QCD}(a_s)=\;&
-\Big(1\Big)a_s\\
& -\Big(4.20833 - 0.138889\,n_f\Big)a_s^2\\
& -\Big(19.5156 - 2.28412\,n_f - 0.0270062\,n_f^2\Big)a_s^3\\
& -\Big(98.9434 - 19.1075\,n_f + 0.276163\,n_f^2 + 0.00579322\,n_f^3\Big)a_s^4.
\end{aligned}
\]

\subsection{ QED lepton/quark mass anomalous dimension \texorpdfstring{$\gamma_m^{\rm QED}$}{gamma\_m(QED)} to two loops}
For a fermion $i$ of electric charge $Q_i$ (in units of $e$), with $a_e=\alpha/\pi$, and denoting the sum over \emph{active} charges by $\displaystyle S_2(\mu)\equiv\sum_f Q_f^2$, we use the MS-bar result
\[
\gamma_m^{\rm QED}(a_e;Q_i,\{Q_f\}) \;=\;
-\frac{3}{4}\,Q_i^2\,a_e
-\left(\frac{3}{32}\,Q_i^4 \;-\; \frac{5}{24}\,Q_i^2\,S_2(\mu)\right)a_e^2
\;+\; \mathcal{O}(a_e^3).
\]
This formula captures the two–loop abelian self–energy ($Q_i^4$) and the fermion–bubble vacuum–polarization insertion ($Q_i^2\,S_2$). It is applied identically for quarks and charged leptons, with the appropriate $S_2(\mu)$ across thresholds. See, e.g., the general two–loop RGE frameworks of Machacek\,\&\,Vaughn and their SM specializations (Luo\,\emph{et\,al.}); our QED normalization matches the $\overline{\mathrm{MS}}$ conventions used there (bibliography in this appendix).

\subsection{ Threshold stepping and matching policy}
We evolve with $n_f:3\to4\to5\to6$ at
\[
\mu=m_c,\quad \mu=m_b,\quad \mu=m_t,
\]
where by default $m_c,m_b,m_t$ denote the $\overline{\mathrm{MS}}$ running masses evaluated at their own scales, $m_q(m_q)$. Across each quark threshold we:
\begin{itemize}
\item match $\alpha_s$ in MS-bar at three loops (used in all runs) \cite{CKS1998,RunDec3};
\item match $\overline{m}_q$ at two loops (heavy–quark decoupling) for all flavors lighter than the threshold \cite{CKS1998};
\item update $S_2(\mu)$ in $\gamma_m^{\rm QED}$ by adding/removing $Q_f^2$ of the newly active/inactive fermion.
\end{itemize}
Continuity of the evolved $\overline{m}_i(\mu)$ at the matching point is enforced.

\subsection{ Numerical constants used in code (audit summary)}
Unless noted otherwise, central values are taken from the Particle Data Group (PDG), see the bibliography below. The specific numbers used are frozen in the build script and emitted into the artifacts manifest (Sec.~7).
\begin{itemize}
\item Electroweak inputs: $M_Z=91.1876~\mathrm{GeV}$, $\alpha^{-1}(M_Z)=127.955$ (leptonic running baseline).
\item Strong coupling: $\alpha_s(M_Z)=0.1179$.
\item Heavy–quark thresholds (MS-bar): $m_c(m_c)=1.27~\mathrm{GeV}$, $m_b(m_b)=4.18~\mathrm{GeV}$, $m_t(m_t)=162.5~\mathrm{GeV}$ (used for stepping; varied within PDG bands in robustness checks).
\item Charges: $Q_u=+2/3$, $Q_d=-1/3$, $Q_s=-1/3$, $Q_c=+2/3$, $Q_b=-1/3$, $Q_t=+2/3$; $Q_e=-1$, $Q_\mu=-1$, $Q_\tau=-1$.
\item Color factors (SU(3)): $C_A=3$, $C_F=4/3$, $T_F=1/2$.
\item Zeta constants: $\zeta_3, \zeta_4, \zeta_5$ as needed in the 4L QCD coefficients (numerically hard–coded in the library).
\end{itemize}

\subsection{ Notes on normalizations}
This appendix uses $a_s=\alpha_s/\pi$ and $a_e=\alpha/\pi$. If an implementation prefers $\tilde a_s\equiv\alpha_s/(4\pi)$, replace $a_s\to 4\tilde a_s$ and rescale the loop coefficients accordingly (i.e., multiply the $L$–loop term by $4^{\,L}$).


\newpage  \clearpage  % \bigskip



\begin{thebibliography}{99}
% (start consolidated unique references)
\bibitem{PDG2023}
R.~L.~Workman \emph{et al.} (Particle Data Group),
``Review of Particle Physics,''
Prog. Theor. Exp. Phys. \textbf{2022}, 083C01 (2023 update),
doi:10.1093/ptep/ptac097.

\bibitem{VermaserenLarinRitbergen97}
J.~A.~M.~Vermaseren, S.~A.~Larin, and T.~van~Ritbergen,
``The 4-loop quark mass anomalous dimension and the invariant quark mass,''
Phys. Lett. B \textbf{405}, 327 (1997).

\bibitem{vanRitbergenVermaserenLarin97}
T.~van~Ritbergen, J.~A.~M.~Vermaseren, and S.~A.~Larin,
``The 4-loop QCD beta-function in the MS scheme,''
Phys. Lett. B \textbf{400}, 379 (1997),
doi:10.1016/S0370-2693(97)00370-5.

\bibitem{MachacekVaughn83}
M.~E.~Machacek and M.~T.~Vaughn,
``Two-Loop Renormalization Group Equations in a General Quantum Field Theory. II. Yukawa Couplings,''
Nucl. Phys. B \textbf{236}, 221 (1984),
doi:10.1016/0550-3213(84)90533-9. (See also I: Nucl. Phys. B \textbf{222}, 83 (1983), doi:10.1016/0550-3213(83)90610-7.)

\bibitem{LuoWangXiao2003}
M.-x.~Luo, H.-w.~Wang, and Y.~Xiao,
``Two-loop renormalization group equations in the Standard Model,''
Phys. Rev. D \textbf{67}, 065019 (2003),
doi:10.1103/PhysRevD.67.065019.

\bibitem{Mihaila2012}
L.~N.~Mihaila, J.~Salomon, and M.~Steinhauser,
``Gauge coupling beta functions in the Standard Model to three loops,''
Phys. Rev. Lett. \textbf{108}, 151601 (2012).

\bibitem{ChetyrkinZoller2012}
K.~G.~Chetyrkin and M.~F.~Zoller,
``Three-loop beta functions for top-Yukawa and the Higgs self-interaction,''
JHEP \textbf{1206}, 033 (2012).

\bibitem{Bednyakov2013}
A.~V.~Bednyakov, A.~F.~Pikelner, and V.~N.~Velizhanin,
``Anomalous dimensions of SM fields and couplings up to three loops,''
JHEP \textbf{1301}, 017 (2013).

\bibitem{CollinsRenorm}
J.~C.~Collins,
``Renormalization,''
Cambridge University Press (1984).

\bibitem{VermaserenWeinzierl2008}
J.~A.~M.~Vermaseren and S.~Weinzierl (eds.),
``Multi-Loop Calculations in Quantum Field Theory,''
PoS (RADCOR2007) (2008). [overview]

\bibitem{CKS1998}
K.~G.~Chetyrkin, B.~A.~Kniehl, and M.~Steinhauser,
``Decoupling relations to $\mathcal{O}(\alpha_s^3)$ and their connection to low-energy theorems,''
Nucl. Phys. B \textbf{510}, 61 (1998),
doi:10.1016/S0550-3213(97)00649-4.

\bibitem{SchroederSteinhauser2006}
Y.~Schröder and M.~Steinhauser,
``Four-loop decoupling relations for the strong coupling,''
JHEP \textbf{0601}, 051 (2006).

\bibitem{ChetyrkinKniehlSteinhauser2006}
K.~G.~Chetyrkin, B.~A.~Kniehl, and M.~Steinhauser,
``Decoupling of heavy quarks at four loops,''
Nucl. Phys. B \textbf{744}, 121 (2006).

\bibitem{RunDec3}
F.~Herren and M.~Steinhauser,
``Version 3 of RunDec and CRunDec: Running and decoupling of the strong coupling and quark masses,''
Comput. Phys. Commun. \textbf{224}, 333 (2018),
doi:10.1016/j.cpc.2017.11.014.

\bibitem{Bethke2013}
S.~Bethke,
``World Summary of $\alpha_s$,''
Nucl. Phys. Proc. Suppl. \textbf{234}, 229 (2013). [and updates in PDG reviews]

\bibitem{PDGAlphaS}
S.~Bethke, G.~Dissertori, and G.~P.~Salam,
``World Summary of $\alpha_s$,''
in PDG 2022/2023 updates.

\bibitem{Davier2020}
M.~Davier, A.~Hoecker, B.~Malaescu, and Z.~Zhang,
``A new evaluation of the hadronic vacuum polarisation contributions to the muon anomalous magnetic moment and to $\alpha(M_Z^2)$,''
Eur. Phys. J. C \textbf{80}, 241 (2020).

\bibitem{Keshavarzi2019}
A.~Keshavarzi, D.~Nomura, and T.~Teubner,
``Muon $g-2$ and $\alpha(M_Z^2)$: a new data-based analysis,''
Phys. Rev. D \textbf{97}, 114025 (2018); updates D \textbf{101}, 014029 (2020).



\bibitem{ChetyrkinKniehlSirlin1997}
K.~G.~Chetyrkin, B.~A.~Kniehl, and A.~Sirlin,
``Estimations of order $\alpha_s^3$ and $\alpha_s^4$ corrections to quark mass relations,''
Phys. Lett. B \textbf{402}, 359 (1997).



\bibitem{Tarrach1981}
R.~Tarrach,
``The pole mass in perturbative QCD,''
Nucl. Phys. B \textbf{183}, 384 (1981).

\bibitem{Gray1990}
N.~Gray, D.~J.~Broadhurst, W.~Grafe, and K.~Schilcher,
``Three-loop relation of quark MS and pole masses,''
Z. Phys. C \textbf{48}, 673 (1990).

\bibitem{MelnikovvanRitbergen2000}
K.~Melnikov and T.~van~Ritbergen,
``The three-loop relation between the MS-bar and the pole quark masses,''
Phys. Lett. B \textbf{482}, 99 (2000).

%\bibitem{Marquard2016}
%P.~Marquard, A.~V.~Smirnov, V.~A.~Smirnov, and M.~Steinhauser,
%``Quark mass relations to four-loop order in perturbative QCD,''
%Phys. Rev. Lett. \textbf{114}, 142002 (2015).

\bibitem{Marquard2015}
P.~Marquard, A.~V.~Smirnov, V.~A.~Smirnov, M.~Steinhauser,
``Quark mass relations to four-loop order in perturbative QCD,''
Phys. Rev. Lett. \textbf{114}, 142002 (2015).

\bibitem{Beneke1999}
M.~Beneke,
``Renormalons,''
Phys. Rept. \textbf{317}, 1 (1999).


\bibitem{Jegerlehner2017}
F.~Jegerlehner,
``The Anomalous Magnetic Moment of the Muon,''
Springer Tracts Mod. Phys. \textbf{274} (2017). [For running $\alpha(\mu)$ and hadronic VP evaluations.]

\bibitem{Hoang2017}
A.~H.~Hoang, C.~Lepenik, and V.~Mateu,
``MSR mass and the $\mathcal{O}(\Lambda_{\rm QCD})$ renormalon sum rule,''
JHEP \textbf{1709}, 099 (2017).

\bibitem{BenekeTopMass2017}
M.~Beneke, P.~Marquard, P.~Nason, and M.~Steinhauser,
``On the ultimate uncertainty of the top quark pole mass,''
Phys. Lett. B \textbf{775}, 63 (2017).

\bibitem{Degrassi2012}
G.~Degrassi, S.~Di~Vita, J.~Elias-Miró, J.~R.~Espinosa, G.~F.~Giudice, G.~Isidori, and A.~Strumia,
``Higgs mass and vacuum stability in the Standard Model at NNLO,''
JHEP \textbf{1208}, 098 (2012).

\bibitem{Buttazzo2013}
D.~Buttazzo, G.~Degrassi, P.~P.~Giardino, G.~F.~Giudice, F.~Sala, A.~Salvio, and A.~Strumia,
``Investigating the near-criticality of the Higgs boson,''
JHEP \textbf{1312}, 089 (2013). [for SM running landscape]

\bibitem{Kitahara2020}
T.~Kitahara, Y.~Nakai, and M.~Reece,
``Standard Model mass and coupling running with higher-order corrections,''
JHEP \textbf{2004}, 088 (2020).



\bibitem{FLAG2021}
S.~Aoki \emph{et al.} (FLAG Working Group),
``FLAG Review 2021,''
Eur. Phys. J. C \textbf{82}, 869 (2022).




\bibitem{Stevenson81}
P.~M.~Stevenson,
``Optimized Perturbation Theory,''
Phys. Rev. D \textbf{23}, 2916 (1981),
doi:10.1103/PhysRevD.23.2916.

\bibitem{BLM83}
S.~J.~Brodsky, G.~P.~Lepage, and P.~B.~Mackenzie,
``On the Elimination of Scale Ambiguities in Perturbative Quantum Chromodynamics,''
Phys. Rev. D \textbf{28}, 228 (1983),
doi:10.1103/PhysRevD.28.228.

\bibitem{Grunberg1984}
G.~Grunberg,
``Renormalization-scheme invariant QCD and optimized perturbation theory,''
Phys. Rev. D \textbf{29}, 2315 (1984).

\bibitem{BrodskyLu1995}
S.~J.~Brodsky and H.~J.~Lu,
``Commensurate scale relations in quantum chromodynamics,''
Phys. Rev. D \textbf{51}, 3652 (1995).



% --- Additional consolidated references (unique; previously in second block) ---
\bibitem{Tarasov1980}
O.~V.~Tarasov, A.~A.~Vladimirov, and A.~Y.~Zharkov,
``The Gell-Mann–Low Function of QCD in the Three-Loop Approximation,''
Phys. Lett. B \textbf{93}, 429 (1980).

\bibitem{Caswell1974}
W.~E.~Caswell,
``Asymptotic Behavior of Non-Abelian Gauge Theories to Two-Loop Order,''
Phys. Rev. Lett. \textbf{33}, 244 (1974).

\bibitem{Jones1974}
D.~R.~T.~Jones,
``Two-loop diagrams in Yang-Mills theory,''
Nucl. Phys. B \textbf{75}, 531 (1974).

\bibitem{ChetyrkinRetey2000}
K.~G.~Chetyrkin and A.~Retey,
``Three-loop three-linear vertices and four-loop anomalous dimensions in massless QCD,''
Nucl. Phys. B \textbf{583}, 3 (2000).

\bibitem{BaikovChetyrkinKuehn2014}
P.~A.~Baikov, K.~G.~Chetyrkin, and J.~H.~Kühn,
``Quark mass and field anomalous dimensions to ${\cal O}(\alpha_s^5)$,''
JHEP \textbf{1410}, 076 (2014).

\bibitem{ChetyrkinKuhnSteinhauser2000}
K.~G.~Chetyrkin, J.~H.~Kühn, and M.~Steinhauser,
``Heavy quark current correlators to $\mathcal{O}(\alpha_s^2)$,''
Nucl. Phys. B \textbf{505}, 40 (1997); updates in Comput. Phys. Commun. \textbf{133}, 43 (2000).

\bibitem{Chetyrkin2000Decoupling}
K.~G.~Chetyrkin, J.~H.~Kühn, and C.~Sturm,
``QCD decoupling at four loops,''
Nucl. Phys. B \textbf{744}, 121 (2006).

\bibitem{Davies2019}
C.~T.~H.~Davies and C.~McNeile,
``Quark masses from lattice QCD,''
Adv. Ser. Direct. High Energy Phys. \textbf{26}, 842 (2016).





\bibitem{HFLAV2022}
Y.~S.~Amhis \emph{et al.} (HFLAV),
``Averages of $b$-hadron, $c$-hadron, and $\tau$-lepton properties as of 2022,''
arXiv:2206.07501.

\bibitem{Dehnadi2015}
B.~Dehnadi, A.~H.~Hoang, V.~Mateu, and S.~M.~Zebarjad,
``Charm mass determination from QCD sum rules,''
JHEP \textbf{1509}, 115 (2015).

\bibitem{Boito2023}
D.~Boito, M.~Golterman, K.~Maltman, S.~Peris, M.~Rojo, and S.~Simbulan,
``Strong coupling from hadronic $\tau$ decays,''
Phys. Rev. D \textbf{106}, 034501 (2022).


\bibitem{Buttazzo2023}
D.~Buttazzo and P.~Panci,
``Status of the Standard Model at high scales,''
Ann. Rev. Nucl. Part. Sci. \textbf{73}, 1 (2023).



\bibitem{BaikovChetyrkinKuehn2017}
P.~A.~Baikov, K.~G.~Chetyrkin, and J.~H.~Kühn,
``Five-Loop Running of the QCD Coupling Constant,''
Phys. Rev. Lett. \textbf{118}, 082002 (2017).

\bibitem{Herzog2017}
F.~Herzog, B.~Ruijl, T.~Ueda, J.~A.~M.~Vermaseren, and A.~Vogt,
``The five-loop beta function of Yang–Mills theory with fermions,''
JHEP \textbf{1702}, 090 (2017).

\bibitem{Luthe2017}
T.~Luthe, A.~Maier, P.~Marquard, and Y.~Schröder,
``The five-loop Beta function for a general gauge group and anomalous dimensions beyond Feynman gauge,''
JHEP \textbf{1707}, 127 (2017).




\bibitem{fundamental-masses-repo}
J.~Washburn,
``Empirical Regularity in Standard Model Fermion Masses: Code and Data,''
Git repository mirrored in the artifact manifest (repository name and commit hash recorded in \texttt{out/log/manifest.txt}).




\end{thebibliography}

\section*{Kernels and policies}
Unless stated otherwise, we use:
(i) QCD four-loop running and quark mass anomalous dimensions with
$n_f$ stepping $3\to4\to5\to6$ at fixed $\overline{\mathrm{MS}}$ thresholds $(\mu_c,\mu_b,\mu_t)$ held constant for all species and all scans, with standard one-step decoupling and matching at each threshold;
(ii) QED two-loop running for charged leptons and quarks with a single, global choice of $\alpha(\mu)$ policy;
(iii) a single global input for $\alpha_s(M_Z)$.
Sensitivity bands are produced by coherent, sector-global variations of these inputs (no per-species tuning or offsets).

\section*{Statements and Declarations}

\paragraph{Funding.}
This work received no external funding. The Recognition Physics Institute provided institutional support only.

\paragraph{Competing interests.}
The author declares no financial or non\-financial competing interests.

\paragraph{Author contributions.}
J.\,Washburn: 
Supervision,
Conceptualization,
Methodology,
Formal analysis,
Software,
Validation,
Writing the original draft,
Review \& Editing. \\
%\vspace{0.20cm}
Elshad Allahyarov: 
Investigation,
Data curation,
Visualization,
Writing the final version.  
% Investigation
% Formal analysis 
% Funding acquisition
% Data curation
% Supervision
% Funding acquisition
% Writing
% review & editing
% Conceptualization
% Methodology,
%Writing - original draft




%\paragraph{Data availability.}
%All numerical outputs underlying the figures and claims (CSV files for residues, ratios, sensitivity sweeps) are included in the artifact bundle. Exact file names are cited in\-text and mirrored in the manifest.

%\paragraph{Code availability.}
%The scripts used to produce the CSVs and \LaTeX{} inserts are archived with the data at the same DOI and tagged by commit hash. No proprietary software is required to reproduce the results.

\paragraph{Ethics approval, Consent, and Human/Animal research.}
Not applicable.

%\paragraph{Use of AI tools.}
%No generative AI was used to produce scientific content; standard editing tools were used for grammar and typesetting only.

\end{document}




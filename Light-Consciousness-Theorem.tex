\documentclass[12pt,a4paper]{article}

\usepackage[margin=1in]{geometry}
\usepackage{amsmath,amssymb}
\usepackage{hyperref}

\title{Light as Consciousness: A Bi-Interpretability Theorem with Mechanical Verification\\
\large (Alternate title: Photonic Equivalence of Operational Consciousness at the RS Bridge)}
\author{Jonathan Washburn\\
Recognition Physics\\
\texttt{jon@recognitionphysics.org}}
\date{\today}

\begin{document}
\maketitle

\begin{abstract}
We convert the slogan ``Light = Consciousness'' into a formal theorem by proving a bi-interpretability result at the Recognition Science bridge. 
In this setting, ``equals'' is made precise as an equivalence between two well-typed interface notions together with uniqueness up to units: a ConsciousProcess is equivalent to a PhotonChannel, and any two photonic witnesses differ only by admissible units moves. 
The explicit bridge obligations we enforce are: units-quotient invariance (dimensionless displays), the K-gate identity (time-first \(=\) length-first), eight-beat neutrality in three dimensions, display-speed \(c\) via \(\lambda_{\mathrm{kin}}/\tau_{\mathrm{rec}}=c\), and BIOPHASE acceptance (correlation \(\rho\ge 0.30\), \(\mathrm{SNR}\ge 5\sigma\), circular variance \(\le 0.40\)). 
Our main result states that, under these obligations, \(\mathrm{ConsciousProcess}(L,B)\leftrightarrow \mathrm{PhotonChannel}(L,B)\) with uniqueness up to units, and that only the electromagnetic channel satisfies the BIOPHASE feasibility constraints. 
The proof proceeds via four classification lemmas: (A) No-medium-knobs (dimensionless displays forbid dependence on extra medium constants), (B) Null-only (K-gate + cone bound enforce massless propagation), (C) Maxwellization from exactness (the only long-range, gauge-compatible bridge carrier absent extra constants is abelian U(1)/Maxwell), and (D) BIOPHASE feasibility (EM passes; gravitational and neutrino channels fail by many orders in cross-section and SNR). 
We derive the BIOPHASE scale from \(\varphi^{-5}\,\mathrm{eV}\), obtaining \(\lambda_{0}\approx 13.8\,\mu\mathrm{m}\) and \(\nu_{0}\approx 724\,\mathrm{cm}^{-1}\), an eight-band structure around \(\nu_{0}\), a cross-section hierarchy \(\sigma_{\mathrm{EM}}\gg\sigma_{\nu}\gg\sigma_{\mathrm{grav}}\), and \(\mathrm{SNR}\) thresholds demonstrating that only EM meets \(\mathrm{SNR}\ge 5\sigma\) under admissible windows. 
All statements are mechanically verified in Lean~4 via a bundled certificate that exports one-line OK/FLIP reports and falsifier predicates. 
The framework is explicitly falsifiable through spectroscopy and timing protocols (e.g., eight-phase stepping around 724~cm\(^{-1}\)), correlation/variance audits, and channel-substitution controls, any of which can flip the theorem's report under violation. 
Our scope is the bridge + BIOPHASE domain; we discuss how to extend the classification beyond BIOPHASE and toward global uniqueness under strengthened obligations.
\end{abstract}

\noindent\textbf{Keywords:} Bi-interpretability; Recognition Science; Maxwell/DEC; BIOPHASE; eight-beat neutrality; K-gate; information-cost \(J\); SNR thresholds; Lean~4 verification.

\section{Preliminaries and RS Bridge Invariants}
\label{sec:preliminaries}

This section fixes notation and recalls the Recognition Science (RS) bridge invariants that serve as interface obligations throughout the paper. We also summarize the Maxwell/DEC objects used later and briefly recap the information-cost functional \(J\), whose uniqueness and measurement bridge are only used for context. For mechanical references, we indicate the corresponding Lean~4 modules and theorem names in \texttt{typewriter} font.

\paragraph{RS quantities, units quotient, and dimensionless observables.}
An RS units pack is a triple \(U=\langle \tau_0,\ell_0,c\rangle\) with the structural identity
\begin{equation}
  c\,\tau_0 \;=\; \ell_0,
  \qquad\text{equivalently}\qquad
  \frac{\ell_0}{\tau_0} \;=\; c,
\end{equation}
captured by the field \texttt{c\_ell0\_tau0} in \texttt{IndisputableMonolith/Constants.lean} (structure \texttt{RSUnits}). At the bridge, observables are required to be \emph{dimensionless} under the \emph{units quotient}: a displayed quantity \(\mathcal{O}\) is acceptable only if it is invariant under the canonical rescalings \((\tau_0,\ell_0,c)\mapsto (\alpha\tau_0,\alpha\ell_0,c)\) and equivalent units moves. Formally, this is implemented as the requirement that displays are formed from ratios of RS units (e.g., \(\tau_{\rm rec}/\tau_0\), \(\lambda_{\rm kin}/\ell_0\)) and therefore remain unchanged under admissible rescalings.

\paragraph{K-gate identity and single-inequality audit.}
Let \(U=\langle\tau_0,\ell_0,c\rangle\) be an RS units pack. The bridge exposes two canonical displays:
\[
  \tau_{\rm rec}^{\rm (display)} \;=\; K\cdot \tau_0,
  \qquad
  \lambda_{\rm kin}^{\rm (display)} \;=\; K\cdot \ell_0,
\]
so that the \emph{K-gate} identity reads
\begin{equation}
  \frac{\tau_{\rm rec}^{\rm (display)}}{\tau_0}
  \;=\;
  \frac{\lambda_{\rm kin}^{\rm (display)}}{\ell_0}
  \;=\; K.
\end{equation}
In Lean these equalities are provided by \texttt{tau\_rec\_display\_ratio} and \texttt{lambda\_kin\_display\_ratio} and are packaged in \texttt{K\_gate} and \texttt{K\_gate\_triple} (see \texttt{IndisputableMonolith/Constants/KDisplay.lean}). The \emph{single-inequality audit} refers to checking either one of the two equalities (e.g., \(\tau_{\rm rec}^{\rm (display)}/\tau_0 = K\)) since the K-gate guarantees the other by construction.

\paragraph{Eight-beat minimal window and \texttt{CompleteCover}.}
Neutrality under \(D\) independent binary axes forces a minimal window length of \(2^D\). For \(D=3\) the minimal neutral window is \(8\) (``eight-beat''). We formalize this with the pattern space \(\texttt{Pattern}(d)\coloneqq \texttt{Fin}\,d\to \texttt{Bool}\) and the covering structure
\[
  \texttt{CompleteCover}(d)
  \;=\;
  \bigl\langle \texttt{period}:\mathbb{N},\;\texttt{path}:\texttt{Fin}\,\texttt{period}\to \texttt{Pattern}(d),\;\texttt{complete}:\text{surjective}(\texttt{path}) \bigr\rangle.
\]
Theorem \(\exists\,\texttt{w}:\texttt{CompleteCover}(d),\;\texttt{w.period}=2^d\) is provided in \texttt{IndisputableMonolith/Patterns.lean} as \texttt{cover\_exact\_pow}; the specialization \(d=3\) (eight-beat) appears as \texttt{period\_exactly\_8}.

\paragraph{Display speed identity.}
From the units identity \(c\,\tau_0=\ell_0\) and the K-gate displays, the bridge speed satisfies
\begin{equation}
  \frac{\lambda_{\rm kin}^{\rm (display)}}{\tau_{\rm rec}^{\rm (display)}} \;=\; c,
\end{equation}
implemented in Lean as \texttt{display\_speed\_eq\_c} (cf.\ \texttt{IndisputableMonolith/Constants/KDisplay.lean}). This is the bridge-level version of ``null'' propagation on displays.

\paragraph{DEC/Maxwell objects used.}
We use a lightweight discrete exterior calculus (DEC) interface on a mesh type \(\alpha\), exposing a coboundary \(\mathrm{d}\) and a Hodge \(\star\) (when needed):
\begin{itemize}
  \item \(\texttt{HasCoboundary}\;\alpha\) provides \(\mathrm{d}:\mathsf{DForm}(\alpha,k)\to \mathsf{DForm}(\alpha,k{+}1)\).
  \item \(\texttt{HasHodge}\;\alpha\) provides \(\star:\mathsf{DForm}(\alpha,k)\to \mathsf{DForm}(\alpha,n{-}k)\) and signature controls.
\end{itemize}
A Maxwell/DEC channel specifies a gauge potential \(A\) (a 1-form), its field strength \(F=\mathrm{d}A\) (a 2-form), and satisfies
\begin{equation}
  \mathrm{d}F \;=\; 0 \quad\text{(Bianchi)},\qquad
  \mathrm{d}J \;=\; 0 \quad\text{(continuity)},
\end{equation}
as captured by the signatures and structures in \texttt{IndisputableMonolith/MaxwellDEC.lean}. We only use these structural laws (and not a full constitutive closure) in the main proof.

\paragraph{Information-cost \(J\): recap and context.}
For completeness, we recall the unique information-cost functional
\begin{equation}
  J(x) \;=\; \tfrac{1}{2}\!\left(x+\tfrac{1}{x}\right)-1,
\end{equation}
determined by multiplicative symmetry \(J(x)=J(x^{-1})\), the unit \(J(1)=0\), strict convexity on \(\mathbb{R}_{>0}\), and calibrated log-curvature \(J''(1)=1\). The full uniqueness proof is provided by \texttt{T5\_uniqueness\_complete} in \texttt{IndisputableMonolith/CostUniqueness.lean} and exported paper-ready as \texttt{THEOREM\_1\_universal\_cost\_uniqueness} in \texttt{IndisputableMonolith/Verification/MainTheorems.lean}. The two-branch measurement bridge
\begin{equation}
  C \;=\; 2\,A
\end{equation}
(and the derived Born weights) appears as \texttt{THEOREM\_2\_measurement\_recognition\_bridge} in \texttt{Verification/MainTheorems.lean}. In this paper, \(J\) and \(C{=}2A\) are referenced only for contextual continuity; the core bi-interpretability theorem at the bridge uses the RS invariants enumerated above and the DEC/Maxwell structure, together with BIOPHASE feasibility, to classify carriers and prove equivalence + uniqueness up to units.

\section{Formal Definitions at the Bridge}
\label{sec:formal-defs}

We formalize the two sides of the equivalence at the RS bridge: the \emph{ConsciousProcess} (CP), representing an operational selection/measurement interface that lives on the bridge and obeys its invariants; and the \emph{PhotonChannel} (PC), representing a DEC/Maxwell channel instrumented at the same bridge. We also specify the predicates we use in the proof and the uniqueness-up-to-units relation. Lean~4 references are indicated in \texttt{typewriter} font.

\subsection{ConsciousProcess (CP)}
A ConsciousProcess at the bridge collects the minimal data and invariants required by the interface. Concretely, in Lean this is a structure
\[
  \texttt{ConsciousProcess} \in \texttt{IndisputableMonolith/Consciousness/ConsciousProcess.lean},
\]
with the following components.
\begin{description}
  \item[Fields.] 
  \emph{ledger} and \emph{bridge} types (implementation-dependent), and \emph{units} \(U=\langle \tau_0,\ell_0,c\rangle\) of type \texttt{RSUnits} (see \texttt{IndisputableMonolith/Constants.lean}).
  \item[Invariants.] CP enforces the RS bridge obligations:
  \begin{enumerate}
    \item \textbf{Units quotient (dimensionless displays).} All observable displays depend only on quotient ratios of \(\tau_0,\ell_0,c\) and remain invariant under admissible rescalings \((\tau_0,\ell_0,c)\mapsto (\alpha\tau_0,\alpha\ell_0,c)\).
    \item \textbf{K-gate identity.} Writing
      \(
        \tau_{\rm rec}^{\rm (disp)} = K \tau_0,\;
        \lambda_{\rm kin}^{\rm (disp)} = K \ell_0
      \),
      CP satisfies
      \(
        \tau_{\rm rec}^{\rm (disp)}/\tau_0 = \lambda_{\rm kin}^{\rm (disp)}/\ell_0 = K
      \).
      In Lean: \texttt{tau\_rec\_display\_ratio}, \texttt{lambda\_kin\_display\_ratio}, \texttt{K\_gate}, \texttt{K\_gate\_triple} (\texttt{IndisputableMonolith/Constants/KDisplay.lean}).
    \item \textbf{Eight-beat neutrality.} The minimal neutral window in \(D=3\) dimensions is 8; CP exposes a schedule that is compatible with a \texttt{CompleteCover} of period \(2^3\). In Lean: \texttt{cover\_exact\_pow 3}, \texttt{period\_exactly\_8} (\texttt{IndisputableMonolith/Patterns.lean}).
    \item \textbf{Speed identity.} The display speed satisfies \(\lambda_{\rm kin}^{\rm (disp)}/\tau_{\rm rec}^{\rm (disp)} = c\) (Lean: \texttt{display\_speed\_eq\_c}).
  \end{enumerate}
  \item[Optional BIOPHASE adapter.] When BIOPHASE acceptance is relevant, CP can be equipped with an adapter carrying:
  \begin{itemize}
    \item a correlation coefficient \(\rho\),
    \item a signal-to-noise ratio \(\mathrm{SNR}\), and
    \item a circular variance \(\mathrm{CV}\),
  \end{itemize}
  together satisfying thresholds \(\rho\ge 0.30\), \(\mathrm{SNR}\ge 5\sigma\), \(\mathrm{CV}\le 0.40\) under the BIOPHASE IR gate (cf.\ Sec.~\ref{sec:preliminaries}). These are exposed via \texttt{passes\_standard\_acceptance} in \texttt{IndisputableMonolith/BiophaseIntegration/AcceptanceCriteria.lean} and proven feasible for EM in \texttt{BiophasePhysics/ChannelFeasibility.lean}.
  \item[Predicates.] We use
  \[
    \texttt{IsConsciousProcess}(L,B,U)
    \;\equiv\;
    \exists\, cp:\texttt{ConsciousProcess},\;
      (cp.\texttt{ledger}=L \wedge cp.\texttt{bridge}=B \wedge cp.\texttt{units}=U),
  \]
  and \texttt{WellFormed}\((cp)\) to record positivity/regularity (\(0<\tau_0\), \(0<\ell_0\), etc.). See \texttt{ConsciousProcess.lean}.
\end{description}

\subsection{PhotonChannel (PC)}
A PhotonChannel is a DEC/Maxwell channel installed at the bridge, instrumented to the same invariants as CP. In Lean, it is provided by
\[
  \texttt{PhotonChannel} \in \texttt{IndisputableMonolith/Consciousness/PhotonChannel.lean}
\]
and reuses the Maxwell/DEC interface from \texttt{IndisputableMonolith/MaxwellDEC.lean}.
\begin{description}
  \item[Fields.] 
  \emph{mesh} type (the discrete domain), \emph{bridge} type, \emph{units} \(U=\langle \tau_0,\ell_0,c\rangle\). DEC objects:
  \[
    A\in \mathsf{DForm}(\texttt{mesh},1),\qquad
    F=\mathrm{d}A\in \mathsf{DForm}(\texttt{mesh},2),\qquad
    J\in \mathsf{DForm}(\texttt{mesh},1).
  \]
  \item[DEC/Maxwell structural laws.] 
  \(
    \mathrm{d}F=0
  \) (Bianchi),
  \(
    \mathrm{d}J=0
  \) (continuity). The coboundary \(\mathrm{d}\) is made available by \texttt{HasCoboundary} and, when required, \(\star\) by \texttt{HasHodge}.
  \item[Null/massless display.] At the bridge, the speed identity holds:
  \(
    \lambda_{\rm kin}^{\rm (disp)}/\tau_{\rm rec}^{\rm (disp)} = c
  \)
  (Lean: \texttt{display\_speed\_eq\_c}); this encodes null/massless propagation at the display level.
  \item[Same invariants.] PC satisfies the same units quotient, K-gate, and eight-beat obligations as CP.
  \item[Predicates.] 
  \[
    \texttt{IsPhotonChannel}(M,B,U)
    \;\equiv\;
    \exists\, pc:\texttt{PhotonChannel},\;
      (pc.\texttt{mesh}=M \wedge pc.\texttt{bridge}=B \wedge pc.\texttt{units}=U),
  \]
  and \texttt{WellFormed}\((pc)\) asserts the corresponding positivity/regularity constraints. See \texttt{PhotonChannel.lean}.
\end{description}

\subsection{Unique-Up-To-Units Equivalence and Adapter Boundaries}
\label{subsec:units-equivalence}
The notion of ``equality'' at the bridge is \emph{equivalence + uniqueness up to units}. Intuitively, two realizations that differ only by units moves (admissible rescalings that preserve quotient displays) are considered the same at the bridge.

\paragraph{Units-equivalence.}
Let \(U=\langle \tau_0,\ell_0,c\rangle\) be an RS units pack. We write
\[
  X \sim_U Y
\]
if \(X\) and \(Y\) (either CP- or PC-side structures) produce identical dimensionless displays under the K-gate and eight-beat invariants after an admissible units move that preserves \(\ell_0/\tau_0=c\). Equivalence classes under \(\sim_U\) represent the same bridge behavior. In Lean, units-consistency lemmas are provided in \texttt{Constants/KDisplay.lean}.

\paragraph{Uniqueness up to units.}
The uniqueness statement we prove in the main theorem (Sec.~\ref{sec:main-theorem}) is that any two photonic witnesses of a given CP are units-equivalent at the bridge. In Lean this is realized as a theorem of the form:
\begin{quote}\itshape
\(\texttt{photon\_channel\_unique\_up\_to\_units} :\) if \(pc_1, pc_2:\texttt{PhotonChannel}\) implement the same bridge and units (modulo units moves) for a fixed CP, then \(pc_1 \sim_U pc_2\).
\end{quote}
A representative statement appears in \texttt{IndisputableMonolith/Consciousness/Equivalence.lean}.

\paragraph{Units-gauge action and orbit view.}
Let \(G\) be the admissible units group acting on witnesses by
\[
  (\tau_0,\ell_0,c) \mapsto (a\,\tau_0,\; a\,\ell_0,\; c)
  \qquad (a>0),
\]
which preserves the bridge constraint \(\ell_0/\tau_0=c\). This induces an action on CP/PC witnesses by reparametrizing units while leaving all dimensionless quotient displays invariant. Two witnesses are \emph{units-equivalent} iff they lie in the same \(G\)-orbit. Uniqueness "up to units" (Sec.~\ref{sec:main-theorem}) is precisely uniqueness of the \(G\)-orbit of the photonic witness for a fixed bridge and invariants.

\paragraph{Adapter boundaries.}
We keep the CP-to-PC and PC-to-CP connections modular:
\begin{itemize}
  \item \textbf{PC \(\Rightarrow\) CP:} uses K-gate, units quotient, eight-beat, and speed identities (already certified), and attaches the BIOPHASE acceptance (when used) via a thin adapter.
  \item \textbf{CP \(\Rightarrow\) PC:} uses the four classification lemmas (No-medium-knobs, Null-only, Maxwellization, BIOPHASE feasibility) to exclude all non-EM carriers and to construct a Maxwell/DEC witness.
\end{itemize}
BIOPHASE acceptance (correlation, SNR, circular variance) is handled through a stable adapter interface (\texttt{BiophaseIntegration/AcceptanceCriteria.lean}) and proven feasible for EM (and infeasible for gravitational/neutrino channels) in \texttt{BiophasePhysics/ChannelFeasibility.lean}. This partition ensures that improvements or generalizations to BIOPHASE (e.g., alternate timing gates or thresholds) do not disturb the core bi-interpretability proof at the bridge.

\section{Main Theorem (Formal Statement)}
\label{sec:main-theorem}

We now state the central result of this paper: bi-interpretability of \emph{ConsciousProcess} (CP) and \emph{PhotonChannel} (PC) at the RS bridge, together with uniqueness up to admissible units moves. Throughout, we write \(U=\langle \tau_0,\ell_0,c\rangle\) for an RS units pack (Lean: \texttt{RSUnits} in \texttt{IndisputableMonolith/Constants.lean}), and we assume the RS bridge invariants established in Secs.~\ref{sec:preliminaries}--\ref{sec:formal-defs}: units-quotient invariance, the K-gate identity, eight-beat neutrality (for \(D{=}3\)), and the display-speed identity \(\lambda_{\mathrm{kin}}/\tau_{\mathrm{rec}}=c\). When the BIOPHASE domain is in scope, we additionally assume the BIOPHASE acceptance predicates (correlation \(\rho\ge 0.30\), \(\mathrm{SNR}\ge 5\sigma\), circular variance \(\le 0.40\)) via the standard adapter (Lean: \texttt{passes\_standard\_acceptance} in \texttt{BiophaseIntegration/AcceptanceCriteria.lean}) and the feasibility/infeasibility results (Lean: \texttt{ChannelFeasibility.lean}).

\begin{theorem}[Bi-interpretability and uniqueness up to units]
\label{thm:biinterp}
Let \(L,B\) be ledger and bridge types, \(M\) a DEC mesh type, and \(U\) an RS units pack. Assume:
\begin{enumerate}
  \item \textbf{RS invariants (bridge).} CP and PC satisfy the RS bridge obligations:
  \begin{itemize}
    \item units-quotient invariance (dimensionless displays),
    \item K-gate identity (time-first \(=\) length-first),
    \item eight-beat neutrality (minimal window \(=8\) for \(D=3\)),
    \item display-speed identity \(\lambda_{\mathrm{kin}}/\tau_{\mathrm{rec}}=c\).
  \end{itemize}
  \item \textbf{BIOPHASE acceptance (when applicable).} If BIOPHASE is in scope, CP and PC are equipped with the BIOPHASE adapter and satisfy the acceptance thresholds (Lean: \texttt{passes\_standard\_acceptance}); feasibility for EM and infeasibility for non-EM channels follow from the physics layer (Lean: \texttt{BiophasePhysics/ChannelFeasibility.lean}, \texttt{lemma\_d\_proven}).
\end{enumerate}
Then:
\begin{enumerate}
  \item \textbf{Bi-interpretability.} 
  \[
    \texttt{IsConsciousProcess}(L,B,U)
    \;\Longleftrightarrow\;
    \texttt{IsPhotonChannel}(M,B,U),
  \]
  i.e., there exists a CP witness iff there exists a PC witness at the same bridge and units.
  \item \textbf{Uniqueness up to units.} If \(pc_1,pc_2:\texttt{PhotonChannel}\) are two photonic witnesses for a fixed CP on \((B,U)\), then \(pc_1\) and \(pc_2\) are equivalent up to admissible units moves (Lean: \texttt{photon\_channel\_unique\_up\_to\_units} in \texttt{Consciousness/Equivalence.lean}).
\end{enumerate}
\end{theorem}

\paragraph{Lean certificate and references.}
The theorem is packaged in a Lean certificate (\texttt{IndisputableMonolith/Verification/LightConsciousnessTheorem.lean}), which bundles:
\begin{itemize}
  \item the four classification lemmas (No-medium-knobs, Null-only, Maxwellization, BIOPHASE feasibility),
  \item the adapter directions (PC\(\Rightarrow\)CP and CP\(\Rightarrow\)PC),
  \item and uniqueness up to units for PC witnesses.
\end{itemize}
One-line reports (\#eval) provide OK/FLIP outcomes and enumerate falsifiers when assumptions are violated.

\begin{corollary}[Selection/readout consistency and adapter identities]
\label{cor:weights-readout}
Under the hypotheses of Theorem~\ref{thm:biinterp}:
\begin{enumerate}
  \item \textbf{CP selection weights are consistent with PC readouts.} In particular, when the CP is realized via an admissible PC, the CP's selection probabilities coincide with the PC-side readout weights up to the bridge adapter normalization.\footnote{In contexts where the information-cost \(J\) and \(C{=}2A\) bridge are relevant (cf.\ \texttt{Verification/MainTheorems.lean}), this reduces to the standard exponential weight/visibility relations, but the main theorem does not require these.}
  \item \textbf{Adapter identities.} The K-gate and units-quotient identities commute with the PC\(\leftrightarrow\)CP adapters, so any lawful display on one side has a matching quotient on the other. (Lean: \texttt{K\_gate\_triple} and display-speed lemmas in \texttt{Constants/KDisplay.lean} are used at both interfaces.)
\end{enumerate}
\end{corollary}

\begin{corollary}[Stability under units moves and window alignment]
\label{cor:stability}
The bi-interpretation and associated displays are stable under:
\begin{enumerate}
  \item \textbf{Admissible units moves.} If \(U\mapsto U'\) is a units transformation preserving \(\ell_0/\tau_0=c\), then the equivalence classes of CP and PC witnesses and their dimensionless displays are unchanged. (Lean: units-quotient lemmas in \texttt{Constants/KDisplay.lean}.)
  \item \textbf{Window alignment.} Eight-beat alignment changes only the window indexing (Gray-phase) but not the admissible CP/PC equivalence class, provided neutrality constraints are respected. (Lean: \texttt{CompleteCover} and \texttt{period\_exactly\_8} in \texttt{Patterns.lean}.)
\end{enumerate}
\end{corollary}

\paragraph{Remarks on scope and falsifiability.}
The theorem is stated at the RS bridge, with BIOPHASE acceptance attached by adapter when in use. The equivalence holds for any platform obeying the bridge invariants; the BIOPHASE feasibility clauses specialize the physical channel to EM via cross-section and SNR proofs. Falsifiers are exposed as predicates (e.g., a non-EM channel meeting acceptance, or a violation of K-gate/units-quotient), and are surfaced by the certificate reports. Generalizations beyond BIOPHASE are obtained by strengthening the Maxwellization clause and replacing the acceptance adapter with more general feasibility conditions while keeping the bridge invariants unchanged.

\section{PC \texorpdfstring{$\Rightarrow$}{⇒} CP (Adapter Direction)}
\label{sec:pc-to-cp}

In this section we show that any PhotonChannel (PC) instantiated at the RS bridge canonically determines a ConsciousProcess (CP) with the same bridge and units, and that the induced CP satisfies all interface obligations. The construction is modular: the DEC/Maxwell structure provides the physical channel, while the K-gate, units quotient, and eight-beat invariants are enforced by the bridge. When the BIOPHASE domain is in scope, the acceptance adapter is attached to the PC-side observables, yielding a BIOPHASE-compliant CP.

\paragraph{Setting and goal.}
Let \(pc:\texttt{PhotonChannel}\) be a photonic channel with fields
\[
  A\in \mathsf{DForm}(\texttt{mesh},1),\qquad
  F=\mathrm{d}A\in \mathsf{DForm}(\texttt{mesh},2),\qquad
  J\in \mathsf{DForm}(\texttt{mesh},1)
\]
on a mesh \(\texttt{mesh}\), and units \(U=\langle \tau_0,\ell_0,c\rangle\). Assume the DEC/Maxwell laws
\(
  \mathrm{d}F=0
\) (Bianchi)
and
\(
  \mathrm{d}J=0
\) (continuity),
together with the bridge invariants (units quotient, K-gate, eight-beat, display-speed identity). We claim there is a canonical CP witness \(cp\) at the same bridge and units, written
\[
  \exists\, cp:\texttt{ConsciousProcess},\qquad
  (cp.\texttt{bridge}=pc.\texttt{bridge})
  \wedge
  (cp.\texttt{units}=pc.\texttt{units}),
\]
and that \(cp\) satisfies all CP invariants by construction. In Lean this is implemented as \texttt{photon\_to\_conscious} in \texttt{IndisputableMonolith/Consciousness/Equivalence.lean}, which returns a CP witness from a well-formed PC together with the bridge/units equalities.

\paragraph{Units quotient and K-gate.}
The bridge enforces dimensionless displays via the units quotient and route consistency via the K-gate. Writing
\[
  \tau_{\rm rec}^{\rm (disp)} \;=\; K\cdot \tau_0,
  \qquad
  \lambda_{\rm kin}^{\rm (disp)} \;=\; K\cdot \ell_0,
\]
the identities
\[
  \frac{\tau_{\rm rec}^{\rm (disp)}}{\tau_0} \;=\; K
  \qquad\text{and}\qquad
  \frac{\lambda_{\rm kin}^{\rm (disp)}}{\ell_0} \;=\; K
\]
hold automatically for any lawful display derived from the PC-side observables. In Lean these appear as \texttt{tau\_rec\_display\_ratio}, \texttt{lambda\_kin\_display\_ratio} and are bundled in \texttt{K\_gate} and \texttt{K\_gate\_triple} (see \texttt{IndisputableMonolith/Constants/KDisplay.lean}). Consequently, any composite display formed from PC observables and the RS units pack is dimensionless and route-consistent.

\paragraph{Eight-beat minimality and cone bound.}
Neutrality in three binary axes forces a minimal window of 8 ticks, aligned to a complete cover of the pattern space \(\texttt{Pattern}(3)\). Operationally, the CP induced from a PC reuses the same eight-beat scheduling; this is formalized by the existence of a \texttt{CompleteCover 3} with period~8. In Lean this is provided by \texttt{cover\_exact\_pow 3} and \texttt{period\_exactly\_8} (\texttt{IndisputableMonolith/Patterns.lean}). The cone bound (from the causality/light-cone layer) fixes the slope in bridge coordinates and is compatible with the speed identity below. We only need that the PC-side displays admit alignment to the eight-beat cadence without violating neutrality.

\paragraph{Speed identity and null display.}
From the units identity \(c\,\tau_0=\ell_0\) and the K-gate displays, the CP-side speed is constrained to
\[
  \frac{\lambda_{\rm kin}^{\rm (disp)}}{\tau_{\rm rec}^{\rm (disp)}} \;=\; c,
\]
which we interpret as a ``null'' display at the bridge. The Lean statement \texttt{display\_speed\_eq\_c} is given in \texttt{Constants/KDisplay.lean}. This identity is inherited by the CP constructed from the PC since the latter provides the underlying displays and shares the same units pack.

\paragraph{BIOPHASE acceptance (adapter).}
When the BIOPHASE domain is in scope, the acceptance thresholds (correlation \(\rho\ge 0.30\), \(\mathrm{SNR}\ge 5\sigma\), circular variance \(\le 0.40\)) are attached to PC-side observables through a thin adapter. In Lean, the predicates live in \texttt{BiophaseIntegration/AcceptanceCriteria.lean} and are proven feasible for electromagnetic channels (and infeasible for gravitational/neutrino channels) in \texttt{BiophasePhysics/ChannelFeasibility.lean}. The CP inherits this acceptance once the adapter is applied, thereby satisfying the BIOPHASE obligations alongside the bridge invariants.

\paragraph{Direct extraction of the CP witness.}
The adapter direction is implemented by a constructive map from a well-formed \texttt{PhotonChannel} to a \texttt{ConsciousProcess} with identical bridge and units and with all invariants satisfied. In Lean this is the statement \texttt{photon\_to\_conscious} (see \texttt{IndisputableMonolith/Consciousness/Equivalence.lean}), which returns
\[
  \exists\, cp:\texttt{ConsciousProcess},\quad
  cp.\texttt{units}=pc.\texttt{units}\ \wedge\ cp.\texttt{bridge}=pc.\texttt{bridge}\ \wedge\ \texttt{WellFormed}(cp).
\]
The proof appeals to the K-gate/units quotient lemmas (\texttt{K\_gate}, \texttt{K\_gate\_triple}), eight-beat coverage (\texttt{period\_exactly\_8}), and the display-speed equality (\texttt{display\_speed\_eq\_c}). When present, the BIOPHASE acceptance adapter is applied to the PC displays and lifted verbatim to the CP. This completes the PC\(\Rightarrow\)CP direction.

\section{CP \texorpdfstring{$\Rightarrow$}{⇒} PC (Classification Direction): Proof Roadmap}
\label{sec:cp-to-pc}

We now outline the classification argument that any lawful \emph{ConsciousProcess} (CP) at the RS bridge must be realized by a \emph{PhotonChannel} (PC) satisfying the DEC/Maxwell obligations, and that the surviving physical carrier obeying the BIOPHASE acceptance is uniquely electromagnetic. The proof proceeds compositionally through four lemmas (A)--(D), each sitting directly on RS invariants or BIOPHASE feasibility:
\[
  \text{(A) No-medium-knobs} \;\Rightarrow\; 
  \text{(B) Null-only} \;\Rightarrow\;
  \text{(C) Maxwellization} \;\Rightarrow\;
  \text{(D) BIOPHASE feasibility}.
\]
In Lean, the classified construction and its adapters are packaged in 
\texttt{IndisputableMonolith/Consciousness/Equivalence.lean}, where the CP$\Rightarrow$PC direction appears as \texttt{conscious\_to\_photon} and \texttt{conscious\_to\_photon\_witness}. BIOPHASE feasibility---electromagnetic (EM) passing and non-EM channels failing---is supplied by the \texttt{BiophasePhysics} modules (cf.\ \texttt{ChannelFeasibility.lean}, where \texttt{lemma\_d\_proven} is provided).

\paragraph{(A) No-medium-knobs.}
The \emph{no-medium-knobs} lemma ensures that a CP whose displays are dimensionless (units quotient) and route-consistent (K-gate) cannot depend on extra medium constants (e.g., diffusion coefficient \(D\), sound speed \(c_s\), spatially varying refractive index \(n(x)\)), which would otherwise create non-quotientable dimensionless groups at the absolute layer. In particular, any such dependence would violate the units-quotient posture of the RS bridge. Formally, the Lean predicate \texttt{DisplayDependsOnMedium} and the theorem \texttt{no\_medium\_knobs} reside in
\[
  \texttt{IndisputableMonolith/Consciousness/NoMediumKnobs.lean}.
\]
Consequently, diffusive, phononic, or generally material-dependent channels are excluded as CP carriers at the bridge.

\paragraph{(B) Null-only.}
Given the speed identity \(\lambda_{\mathrm{kin}}/\tau_{\mathrm{rec}}=c\) (from the K-gate + units pack, cf.\ \texttt{display\_speed\_eq\_c} in \texttt{Constants/KDisplay.lean}) and the discrete cone-bound slope (from the causality layer), a lawful CP must display null (massless) propagation. Conversely, massive carriers would exhibit \(v_g<c\) for finite \(k\) (group velocity strictly subluminal), contradicting the bridge display. The Lean realization appears as \texttt{null\_only} in
\[
  \texttt{IndisputableMonolith/Consciousness/NullOnly.lean}.
\]
This step eliminates massive alternatives (e.g., neutrino-like massive modes) from acting as acceptable CP carriers at the bridge.

\paragraph{(C) Maxwellization from exactness.}
Exactness of the coboundary, \(\mathrm{d}\circ \mathrm{d}=0\), together with the absence of extra dimensionless couplings and gauge-compatibility of the bridge, constrains the long-range carrier to be an abelian 1-form gauge field. In other words, non-abelian channels introduce structure constants (or additional couplings) that violate the ``no extra parameters'' posture enforced at the bridge; gravity introduces \(G\) at display, likewise forbidden. The only survivor is U(1)/Maxwell. In Lean, this classification appears in
\[
  \texttt{IndisputableMonolith/Consciousness/Maxwellization.lean},
\]
with statements such as \texttt{only\_abelian\_gauge} and \texttt{maxwell\_is\_unique}. The DEC/Maxwell skeleton is provided by
\[
  \texttt{IndisputableMonolith/MaxwellDEC.lean},
\]
which exposes the mesh, \(\mathrm{d}\), and (optionally) \(\star\), and records the structural laws \(F=\mathrm{d}A\), \(\mathrm{d}F=0\) (Bianchi), \(\mathrm{d}J=0\) (continuity).

\paragraph{(D) BIOPHASE feasibility.}
Under the BIOPHASE IR gate, the acceptance thresholds (correlation \(\rho\ge 0.30\), \(\mathrm{SNR}\ge 5\sigma\), circular variance \(\le 0.40\)) select only the electromagnetic channel. The scale \(\varphi^{-5}\,\mathrm{eV}\) yields \(\lambda_0\approx 13.8\,\mu\mathrm{m}\) and \(\nu_0\approx 724\,\mathrm{cm}^{-1}\), along with an eight-band structure around \(\nu_0\). Cross-sections satisfy
\[
  \sigma_{\mathrm{EM}}\gg \sigma_{\nu}\gg \sigma_{\mathrm{grav}},
\]
and \(\mathrm{SNR}\) thresholds establish that only EM passes under admissible windows. In Lean:
\begin{itemize}
  \item constants/spec/bands: \texttt{BiophaseCore/Constants.lean}, \texttt{BiophaseCore/Specification.lean}, \texttt{BiophaseCore/EightBeatBands.lean};
  \item cross-sections and SNR: \texttt{BiophasePhysics/CrossSections.lean}, \texttt{BiophasePhysics/SNRCalculations.lean};
  \item feasibility proofs: \texttt{BiophasePhysics/ChannelFeasibility.lean} (\texttt{em\_passes\_biophase\_proven}, \texttt{gravitational\_fails\_biophase\_proven}, \texttt{neutrino\_fails\_biophase\_proven}; bundled as \texttt{lemma\_d\_proven}).
\end{itemize}
This step completes the physical classification at the bridge in the BIOPHASE domain.

\paragraph{Constructing the PC witness.}
Having eliminated non-EM carriers (A--D), we assemble the PC witness as a Maxwell/DEC channel on a mesh \(\texttt{mesh}\) with units \(U\), a 1-form potential \(A\), field strength \(F=\mathrm{d}A\), and current \(J\) such that \(\mathrm{d}F=0\) and \(\mathrm{d}J=0\). The bridge-side displays are then inherited from the constructed PC, and the CP invariants follow by the same K-gate/units-quotient/eight-beat identities. In Lean, this construction is exposed by
\[
  \texttt{conscious\_to\_photon}\quad \text{and}\quad \texttt{conscious\_to\_photon\_witness}
  \quad \text{in}\quad
  \texttt{IndisputableMonolith/Consciousness/Equivalence.lean},
\]
which return a \texttt{PhotonChannel} with matching bridge and units and \texttt{WellFormed} premises.

\paragraph{Roadmap summary.}
Starting from a lawful CP (units quotient + K-gate + eight-beat + speed identity), (A) excludes medium-dependent channels; (B) excludes massive carriers; (C) forces U(1)/Maxwell at the bridge; and (D) selects EM uniquely under BIOPHASE acceptance. The resulting PC witness is DEC/Maxwell, obeys Bianchi and continuity, and realizes the same displays under the CP adapter. Uniqueness up to units for the witness is handled in Sec.~\ref{sec:main-theorem} (Lean: \texttt{photon\_channel\_unique\_up\_to\_units}).

\section{Lemma A — No-Medium-Knobs}
\label{sec:lemmaA}

We show that any \emph{ConsciousProcess} (CP) whose displays are dimensionless (under the units quotient) and route-consistent (K-gate) cannot depend on \emph{medium} constants (e.g., diffusion coefficient \(D\), sound speed \(c_s\), refractive index field \(n(x)\), etc.). Intuitively, any such dependence would introduce non-quotientable dimensionless groups at the absolute layer, contradicting the bridge posture. This excludes diffusive, phononic, and chemical channels as bridge carriers. The formalization is provided in
\[
  \texttt{IndisputableMonolith/Consciousness/NoMediumKnobs.lean},
\]
where the central predicate and main theorem are defined and proven under the RS invariants of Sec.~\ref{sec:preliminaries}.

\paragraph{Statement.}
Let \(U=\langle \tau_0,\ell_0,c\rangle\) be an RS units pack (Lean: \texttt{RSUnits}) and let \(\mathcal{D}\) denote a CP-side \emph{display} constructed from PC/CP observables and \(U\) such that \(\mathcal{D}\) is dimensionless and route-consistent (K-gate). Then \(\mathcal{D}\) is invariant under admissible units moves and cannot depend on medium constants. Formally, for any medium constant \(m\) (e.g., a material parameter or geometric profile external to the units class),
\[
  \frac{\partial \mathcal{D}}{\partial m} \;=\; 0
  \qquad\text{(in the sense of bridge-side dependence)}.
\]
Equivalently, there is no admissible construction of \(\mathcal{D}\) for which the Lean predicate \texttt{DisplayDependsOnMedium \(\mathcal{D}\;m\)} holds.

\paragraph{Formal predicate and corollary.}
The predicate
\[
  \texttt{DisplayDependsOnMedium}(\mathcal{D}, m)
\]
asserts that the display \(\mathcal{D}\) changes when the medium constant \(m\) is varied, holding the RS units \(U\) fixed. The key corollary (\emph{no extra dimensionless groups}) is that any attempt to form a bridge display via a non-quotientable combination of \(U\) and \(m\) (e.g., a Buckingham-\(\Pi\) group that cannot be reduced to a quotient of \(\tau_0,\ell_0,c\)) violates units-quotient invariance and is therefore inadmissible at the absolute layer. In Lean:
\[
  \texttt{no\_medium\_knobs} :
  \quad
  \text{(CP with units quotient + K-gate)} \Rightarrow
  \forall m,\, \neg\,\texttt{DisplayDependsOnMedium}(\mathcal{D}, m),
\]
see \texttt{NoMediumKnobs.lean}.

\paragraph{Proof sketch.}
Assume, for contradiction, that a dimensionless, K-gate-consistent display \(\mathcal{D}\) depends on a medium constant \(m\). Since \(\mathcal{D}\) is dimensionless, it must be expressible via a product of powers of the available dimensional quantities which, at the bridge, are quotiented by \(\tau_0,\ell_0,c\) (and their derived displays via the K-gate). Any residual dependence on \(m\) that cannot be absorbed into a quotient of \(\tau_0,\ell_0,c\) creates a non-quotientable dimensionless group \(\Pi(m,U)\), violating the absolute-layer requirement that all lawful displays be invariant under admissible units moves. Therefore, no such dependence is allowed. In Lean, the units-quotient consequences are exercised via the K-gate lemmas (\texttt{K\_gate}, \texttt{K\_gate\_triple}; \texttt{IndisputableMonolith/Constants/KDisplay.lean}) and discharged by \texttt{no\_medium\_knobs}.

\paragraph{Implications and examples.}
\begin{itemize}
  \item \emph{Diffusive channels.} A dependence on a diffusion coefficient \(D\) (units \(L^2/T\)) induces non-quotientable groups such as \(D\,\tau_0/\ell_0^2\), in conflict with the absolute layer. Thus purely diffusive channels are excluded at the bridge.
  \item \emph{Phononic channels.} Material sound speed \(c_s\) introduces \(\Pi = c_s/c\) which is not a units-quotient invariant of the RS pack unless \(c_s=c\) (which contradicts its material meaning). Hence phononic carriers are excluded as CP carriers at the bridge.
  \item \emph{Refractive index profiles.} A static or spatially varying \(n(x)\) cannot appear in a lawful display unless it is absorbed into quotient coordinates, which is ruled out at the absolute layer. Therefore chemical/refractive pathways do not qualify as bridge carriers for CP under RS invariants.
\end{itemize}

\paragraph{Falsifier.}
The falsification predicate
\[
  \texttt{Falsifier\_MediumConstantAppears}(L,B,U,m,\mathcal{D})
\]
is triggered when a medium constant \(m\) appears in a dimensionless display \(\mathcal{D}\) at the bridge (Lean: \texttt{NoMediumKnobs.lean}). Under such a condition, the certificate report flips (OK\(\rightarrow\)FLIP) and the system is classified as violating the no-medium-knobs lemma.

\paragraph{Lean references.}
\begin{itemize}
  \item \texttt{IndisputableMonolith/Consciousness/NoMediumKnobs.lean}:
    \texttt{DisplayDependsOnMedium}, \texttt{no\_medium\_knobs}, \texttt{Falsifier\_MediumConstantAppears}.
  \item \texttt{IndisputableMonolith/Constants/KDisplay.lean}: \texttt{K\_gate}, \texttt{K\_gate\_triple}, \texttt{display\_speed\_eq\_c}.
\end{itemize}
Together, these lemmas capture the operational content of the no-medium-knobs posture: bridge-side displays are dimensionless and invariant under admissible units moves; any extra medium parameter would introduce forbidden dimensionless groups and is thus excluded.

\paragraph{Scope (bridge vs measurement layer).}
The prohibition concerns \emph{bridge displays}: after taking the RS units quotient and applying the K-gate, dimensionless displays cannot depend on additional medium parameters. Medium descriptors (e.g., refractive index, quality factors) may appear in the \emph{measurement layer} so long as they do not alter the quotient displays or violate bridge invariants. This scope aligns the lemma with the absolute-layer posture while permitting instrument-level bookkeeping off-bridge.

\section{Lemma B — Null-Only}
\label{sec:lemmaB}

We show that a lawful \emph{ConsciousProcess} (CP) at the RS bridge must display \emph{null} (massless) propagation. Concretely, under the speed identity
\begin{equation}
  \frac{\lambda_{\mathrm{kin}}^{\mathrm{(display)}}}{\tau_{\mathrm{rec}}^{\mathrm{(display)}}} \;=\; c
  \qquad\text{(Lean: \texttt{display\_speed\_eq\_c} in \texttt{IndisputableMonolith/Constants/KDisplay.lean}),}
  \label{eq:display-speed}
\end{equation}
and the discrete cone-bound slope from the causality layer (Lean: cone-bound lemmas in \texttt{IndisputableMonolith/LightCone/StepBounds.lean}), only null propagation is admissible at the bridge. The formal statement and proof are implemented as \texttt{null\_only} in
\[
  \texttt{IndisputableMonolith/Consciousness/NullOnly.lean},
\]
with the falsifier predicate \texttt{Falsifier\_MassiveModeExists} exported for certificate reports.

\paragraph{Statement.}
Let \(cp:\texttt{ConsciousProcess}\) satisfy the RS bridge invariants (units-quotient, K-gate, eight-beat) and the speed identity \eqref{eq:display-speed}. Then any carrier contributing to the CP display on the bridge must be null:
\[
  \texttt{null\_only}(cp)\;:\quad
  \text{no massive contribution can realize }\;
  \frac{\lambda_{\mathrm{kin}}^{\mathrm{(display)}}}{\tau_{\mathrm{rec}}^{\mathrm{(display)}}} \;=\; c
  \text{ under the cone bound.}
\]
Equivalently, any massive mode with dispersion \(\omega(k)\) at finite \(k\) is excluded by the bridge display. In Lean, the meta-structure for massive and massless modes and the theorem \texttt{null\_only} are provided by \texttt{Consciousness/NullOnly.lean}.

\paragraph{Proof sketch.}
The argument has two ingredients:
\begin{enumerate}
  \item \textbf{Bridge display speed.} The K-gate and units pack enforce the display-speed identity \eqref{eq:display-speed}, i.e., \(\lambda_{\mathrm{kin}}/\tau_{\mathrm{rec}}=c\) on the bridge (Lean: \texttt{display\_speed\_eq\_c}).
  \item \textbf{Massive dispersion and group velocity.} A massive mode has a dispersion relation \(\omega(k)\) satisfying (in fully relativistic normalization) \(\omega^2 = c^2k^2 + m^2 c^4/\hbar^2\), yielding group velocity
  \[
    v_g \;=\; \frac{\mathrm{d}\omega}{\mathrm{d}k} \;=\; \frac{c^2 k}{\omega} \;<\; c
    \qquad \text{for all finite }k\ \text{and }m>0.
  \]
  Hence, any massive contribution produces a strictly subluminal display. Together with the cone-bound slope (Lean: \texttt{StepBounds} cone lemmas), this contradicts the bridge identity \eqref{eq:display-speed}.
\end{enumerate}
Therefore, only null (massless) propagation is admissible at the bridge. In Lean this is summarized by \texttt{null\_only cp}.

\paragraph{Corner cases (small \(k\), instrument windows).}
Two boundary scenarios are handled as follows:
\begin{itemize}
  \item \emph{Small-$k$ limit.} For a massive mode, \(k\to 0\) implies \(v_g\to 0\), not \(c\); such contributions cannot realize \(\lambda_{\mathrm{kin}}/\tau_{\mathrm{rec}}=c\) on the bridge and thus do not satisfy the CP display.
  \item \emph{Finite windows.} The RS display is defined over admissible, eight-beat-aligned windows (Lean: \texttt{period\_exactly\_8} in \texttt{IndisputableMonolith/Patterns.lean}). Misaligned or vanishing-window effects average out under neutrality; the aligned display enforces the speed identity pointwise in the admissible sense. Consequently, massive remnants that could appear in misaligned transients do not survive the neutrality/adapter averaging.
\end{itemize}

\paragraph{Implications.}
Lemma~\ref{sec:lemmaB} eliminates any massive reservoir (e.g., phonon-like or neutrino-like with \(m>0\)) from serving as a lawful bridge carrier for the CP display. Together with Lemma~A (no-medium-knobs), this narrows the classification to massless, gauge-compatible channels and sets up the Maxwellization step (Lemma~C).

\paragraph{Falsifier.}
The falsification predicate
\[
  \texttt{Falsifier\_MassiveModeExists}(L,B,U, \texttt{mode})
\]
is triggered by the existence of a massive mode that (i) contributes at finite \(k\) to the CP display, and (ii) nonetheless satisfies the RS bridge invariants (including \eqref{eq:display-speed}). If such a witness is constructed, the certificate report flips (OK\(\to\)FLIP). Lean definitions and the \texttt{null\_only} theorem are provided in \texttt{IndisputableMonolith/Consciousness/NullOnly.lean}, with K-gate and speed identities in \texttt{Constants/KDisplay.lean}, and cone-bound statements in \texttt{LightCone/StepBounds.lean}.

\section{Lemma C — Maxwellization from Exactness}
\label{sec:lemmaC}

We show that, under the RS bridge obligations, the only admissible long-range carrier that is gauge-compatible and does not introduce extra (non-quotientable) parameters is the abelian \(U(1)\) (Maxwell) channel. This \emph{Maxwellization} step uses the DEC exactness skeleton together with the no-extra-parameters posture of the bridge to exclude non-abelian and gravitational alternatives. The Lean formalization of this lemma is provided in
\[
  \texttt{IndisputableMonolith/Consciousness/Maxwellization.lean}
\]
with representative statements \texttt{only\_abelian\_gauge} and \texttt{maxwell\_is\_unique}. We also rely on the DEC/Maxwell scaffolding in \texttt{IndisputableMonolith/MaxwellDEC.lean}.

\paragraph{Statement.}
Let a \emph{ConsciousProcess} (CP) satisfy the RS bridge invariants (Sec.~\ref{sec:preliminaries}) and be realized by a gauge-compatible, long-range field on a mesh \(\alpha\) with DEC structure \(\mathrm{d}\) (Lean: \texttt{HasCoboundary}). Suppose further that the bridge posture forbids extra dimensionless couplings beyond the RS units \(\langle\tau_0,\ell_0,c\rangle\). Then the only admissible carrier at the bridge is an abelian 1-form potential \(A\) with field \(F=\mathrm{d}A\) satisfying
\begin{equation}
  \mathrm{d}F \;=\; 0
  \quad\text{(Bianchi)}, 
  \qquad
  \mathrm{d}J \;=\; 0
  \quad\text{(continuity)},
\end{equation}
i.e., the \(U(1)\)/Maxwell channel. Equivalently, any non-abelian gauge alternative or a gravitational channel (with \(G\)) contradicts the bridge's no-extra-parameters posture.

\paragraph{Proof sketch.}
The argument has three parts:
\begin{enumerate}
  \item \textbf{Exactness skeleton.} On a mesh \(\alpha\) with coboundary \(\mathrm{d}\), the discrete exactness \(\mathrm{d}\circ \mathrm{d}=0\) holds (Lean: \texttt{HasCoboundary} in \texttt{MaxwellDEC.lean}). A gauge-compatible long-range field is modeled by a 1-form potential \(A\) and its 2-form field strength \(F=\mathrm{d}A\), which automatically satisfies \(\mathrm{d}F=0\) (Bianchi). Sources and currents appear as 0-/1-forms with continuity \(\mathrm{d}J=0\).
  \item \textbf{No extra parameters at the bridge.} 
    \begin{itemize}
      \item \emph{Non-abelian channels.} A non-abelian gauge theory introduces structure constants \(f^{abc}\) and a coupling \(g\). At the bridge, these induce non-quotientable dimensionless groups which cannot be absorbed into the RS units \(\langle\tau_0,\ell_0,c\rangle\) or the K-gate displays. Hence they violate the units-quotient posture of the absolute layer (cf.\ the spirit of Lemma~A). Thus non-abelian channels are excluded as lawful bridge carriers.
      \item \emph{Gravitational channel.} A gravitational carrier introduces the Newton constant \(G\), which again supplies a forbidden parameter at the bridge (e.g., through \(G\)-dependent dimensionless combinations not reducible to RS quotients). Hence gravity is likewise excluded at the bridge carrier level for CP.
    \end{itemize}
  \item \textbf{Maxwellization and uniqueness.}
    With non-abelian and gravitational options ruled out, the remaining gauge-compatible, long-range channel is abelian \(U(1)\), i.e., DEC/Maxwell with \(F=\mathrm{d}A\), \(\mathrm{d}F=0\), and \(\mathrm{d}J=0\). The PC witness constructed from the CP is unique up to admissible units moves (Lean: \texttt{photon\_channel\_unique\_up\_to\_units} in \texttt{IndisputableMonolith/Consciousness/Equivalence.lean}), reflecting that different realizations with identical K-gate/units-quotient displays represent the same bridge class.
\end{enumerate}

\paragraph{Lean realization and interfaces.}
DEC/Maxwell objects are defined in \texttt{IndisputableMonolith/MaxwellDEC.lean}:
\begin{itemize}
  \item \texttt{HasCoboundary} (\(\mathrm{d}\) operator on forms),
  \item \texttt{HasHodge} (optional, for \(\star\) and energy density),
  \item Quasi-static Maxwell bundles (\(E, H, B, D\)), including Bianchi and continuity slots.
\end{itemize}
The Maxwellization lemma is implemented in \texttt{Consciousness/Maxwellization.lean} (e.g., \texttt{only\_abelian\_gauge}, \texttt{maxwell\_is\_unique}) and is used in the CP\(\Rightarrow\)PC construction \texttt{conscious\_to\_photon} in \texttt{Consciousness/Equivalence.lean}. Uniqueness up to units (for PC witnesses) is proven as \texttt{photon\_channel\_unique\_up\_to\_units}.

\paragraph{Falsifier.}
A falsification condition is provided by the Lean predicate
\[
  \texttt{Falsifier\_NonMaxwellGaugeExists}(L,B,U,gt),
\]
which flips the certificate report if a non-abelian (or otherwise non-Maxwell) gauge channel can be shown to meet all RS bridge invariants. In other words, any lawful non-abelian witness at the bridge contradicts the Maxwellization lemma and invalidates the classification.

\paragraph{Conclusion.}
Exactness \(\mathrm{d}\circ\mathrm{d}=0\), gauge compatibility, and the no-extra-parameters posture of the RS bridge force the abelian \(U(1)\) (Maxwell/DEC) carrier as the unique long-range realization of a lawful CP at the bridge. Together with Lemma~A (no-medium-knobs) and Lemma~B (null-only), this narrows the classification to DEC/Maxwell channels and sets up the BIOPHASE feasibility step (Lemma~D), after which the electromagnetic channel is uniquely selected in the BIOPHASE domain.

\section{Lemma D — BIOPHASE Feasibility and Exclusion}
\label{sec:lemmaD}

We prove that, at the BIOPHASE scale and under the standard acceptance thresholds, only the electromagnetic (EM) channel satisfies feasibility; gravitational and neutrino channels fail by many orders of magnitude in both cross-section and SNR. The constants, timing, band structure, and acceptance predicates are formalized in the \texttt{BiophaseCore}, \texttt{BiophasePhysics}, and \texttt{BiophaseIntegration} subpackages, with the theorem-level feasibility bundled in \texttt{BiophasePhysics/ChannelFeasibility.lean} (see \texttt{lemma\_d\_proven}).

\paragraph{BIOPHASE constants and timing.}
We adopt the BIOPHASE IR gate as derived from the golden-ratio energy scale:
\begin{align}
  E_{\mathrm{rec}} \;=\; \varphi^{-5}\,\mathrm{eV} \;\approx\; 0.090~\mathrm{eV}
  \quad&\Rightarrow\quad
  \lambda_{0} \;=\; \frac{hc}{E_{\mathrm{rec}}} \;\approx\; 13.8~\mu\mathrm{m},
  \\
  \nu_{0} \;\approx\; \frac{1}{\lambda_{0}} \;\approx\; 724~\mathrm{cm}^{-1},
  \qquad
  \tau_{\mathrm{gate}} \;\approx\; 65~\mathrm{ps}, 
  \qquad
  \text{breath}=1024,~\text{FLIP@512}.
\end{align}
These derivations and numerical normalizations are implemented in Lean as:
\begin{itemize}
  \item \textbf{Constants and conversions:} \texttt{IndisputableMonolith/BiophaseCore/Constants.lean} (\(E_{\mathrm{rec}}\), \(\lambda_0\), \(\nu_0\), \(\tau_{\mathrm{gate}}\), and breath/flip timing).
  \item \textbf{Acceptance predicates:} \texttt{IndisputableMonolith/BiophaseIntegration/AcceptanceCriteria.lean} with \texttt{passes\_standard\_acceptance} (\(\rho\ge 0.30\), \(\mathrm{SNR}\ge 5\sigma\), circ.\ variance \(\le 0.40\)).
\end{itemize}

\paragraph{Eight-beat band structure.}
Centered at \(\nu_0\approx 724~\mathrm{cm}^{-1}\), we use an eight-band structure with deltas
\[
  \Delta \in \{-18,-12,-6,0,+6,+12,+18,+24\}~\mathrm{cm}^{-1},
\]
yielding a coverage of approximately \(57~\mathrm{cm}^{-1}\) at nominal widths. The band specification and coverage are formalized in:
\begin{itemize}
  \item \texttt{IndisputableMonolith/BiophaseCore/Specification.lean} (band deltas and standard widths),
  \item \texttt{IndisputableMonolith/BiophaseCore/EightBeatBands.lean} (band family, coverage computation, and Gray alignment).
\end{itemize}
Eight-beat neutrality is enforced via a \texttt{CompleteCover 3} of period 8; see \texttt{period\_exactly\_8} in \texttt{IndisputableMonolith/Patterns.lean}.

\paragraph{Cross-section hierarchy.}
At the BIOPHASE energy, the interaction cross-sections exhibit a dramatic hierarchy:
\[
  \sigma_{\mathrm{EM}} \;\approx\; 6.65\times 10^{-29}\,\mathrm{m}^2,
  \qquad
  \sigma_{\mathrm{grav}} \;<\; 10^{-70}\,\mathrm{m}^2,
  \qquad
  \sigma_{\nu} \;<\; 10^{-48}\,\mathrm{m}^2.
\]
These relations imply a separation of \(40+\) orders of magnitude between EM and gravitational channels, and roughly \(19\) orders between EM and neutrino channels. Lean realizations:
\begin{itemize}
  \item \texttt{IndisputableMonolith/BiophasePhysics/CrossSections.lean} (\(\sigma_{\mathrm{EM}}\), \(\sigma_{\mathrm{grav}}\), \(\sigma_{\nu}\), and comparison lemmas).
  \item When needed, display-speed compatibility follows from the K-gate/speed identity (cf.\ \texttt{display\_speed\_eq\_c} in \texttt{Constants/KDisplay.lean}).
\end{itemize}

\paragraph{SNR thresholds.}
Under the standard windowing and realistic parameters (flux, area, integration time, and background), the signal-to-noise ratio (SNR) satisfies:
\[
  \mathrm{SNR}_{\mathrm{EM}} \;\ge\; 5\sigma
  \quad\text{(passes)}, 
  \qquad
  \mathrm{SNR}_{\mathrm{grav}} \;\ll\; 1,
  \qquad
  \mathrm{SNR}_{\nu} \;\ll\; 1
  \quad\text{(both fail).}
\]
The SNR model and threshold proofs are implemented in:
\begin{itemize}
  \item \texttt{IndisputableMonolith/BiophasePhysics/SNRCalculations.lean} (signal events, noise accumulation, and threshold lemmas),
  \item \texttt{IndisputableMonolith/BiophasePhysics/ChannelFeasibility.lean} (end-to-end feasibility: EM passes, non-EM fails).
\end{itemize}

\begin{theorem}[BIOPHASE feasibility and exclusion]
\label{thm:biophase-feasibility}
At the BIOPHASE scale \((E_{\mathrm{rec}}=\varphi^{-5}\,\mathrm{eV})\) with timing \((\tau_{\mathrm{gate}}\approx 65~\mathrm{ps})\), eight-beat alignment, and acceptance thresholds \((\rho\ge 0.30,~\mathrm{SNR}\ge 5\sigma,~\mathrm{CV}\le 0.40)\), the electromagnetic channel passes acceptance, while gravitational and neutrino channels fail. In Lean this is captured as:
\[
  \texttt{em\_passes\_biophase\_proven},\quad
  \texttt{gravitational\_fails\_biophase\_proven},\quad
  \texttt{neutrino\_fails\_biophase\_proven},
\]
bundled into \texttt{lemma\_d\_proven} in \texttt{BiophasePhysics/ChannelFeasibility.lean}.
\end{theorem}

\paragraph{Falsifiers.}
The certificate exposes explicit predicates that flip the report (OK\(\to\)FLIP) if violated:
\begin{itemize}
  \item \emph{Non-EM passes acceptance:}
    \(\texttt{Falsifier\_NonEM\_Passes\_BIOPHASE}(spec,channel)\).
  \item \emph{EM fails acceptance:}
    \(\texttt{Falsifier\_EM\_Fails\_BIOPHASE}(spec)\).
  \item \emph{Band mismatch (spectral structure inconsistent with eight-beat):}
    predicates in \texttt{BiophaseCore/EightBeatBands.lean} and \texttt{BiophaseCore/Specification.lean} (e.g., coverage/center checks) can be used to trigger structured mismatches.
\end{itemize}
These falsifiers are wired to the #eval reports (Lean certificate: \texttt{IndisputableMonolith/Verification/LightConsciousnessTheorem.lean}) and serve as practical gates for experimental audits.

\paragraph{Conclusion.}
The BIOPHASE constants and timing, eight-beat band structure, cross-section hierarchy, and SNR thresholds jointly force the electromagnetic channel as the unique feasible carrier at the bridge. This completes Lemma~D and, combined with Lemmas~A--C, yields the CP\(\Rightarrow\)PC classification used by the main theorem (Sec.~\ref{sec:main-theorem}).

\section{Assembling the Theorem: Witness Construction and Uniqueness}
\label{sec:witness-uniqueness}

We assemble the main result by constructing witnesses in both directions (PC\(\Rightarrow\)CP and CP\(\Rightarrow\)PC), proving uniqueness up to admissible units moves, and recording stability under route and units transformations. Throughout, we work at the RS bridge with the invariants of Secs.~\ref{sec:preliminaries}--\ref{sec:formal-defs}. When the BIOPHASE domain is in scope, the acceptance adapter is attached as described in Sec.~\ref{sec:lemmaD}. Lean references for the constructions and proofs appear in \texttt{IndisputableMonolith/Consciousness/Equivalence.lean} and the certificate \texttt{IndisputableMonolith/Verification/LightConsciousnessTheorem.lean}.

\subsection{PC\texorpdfstring{$\Rightarrow$}{⇒}CP witness}
Given a \emph{PhotonChannel} (PC) installed at the bridge, we extract a \emph{ConsciousProcess} (CP) with the same bridge and units. Concretely (Lean):
\[
  \texttt{photon\_to\_conscious} :
  \quad
  (pc:\texttt{PhotonChannel}) \;\mapsto\; \exists\,cp:\texttt{ConsciousProcess},\ 
  cp.\texttt{bridge}=pc.\texttt{bridge}\ \wedge\ cp.\texttt{units}=pc.\texttt{units}\ \wedge\ \texttt{WellFormed}(cp).
\]
The proof leverages:
\begin{itemize}
  \item the units quotient and K-gate identities (Lean: \texttt{K\_gate}, \texttt{K\_gate\_triple} in \texttt{Constants/KDisplay.lean}) guaranteeing route-consistent and dimensionless displays,
  \item eight-beat minimality (Lean: \texttt{period\_exactly\_8} via \texttt{Patterns.lean}),
  \item display-speed identity (Lean: \texttt{display\_speed\_eq\_c}),
  \item and, when in scope, BIOPHASE acceptance attached by adapter (Lean: \texttt{passes\_standard\_acceptance} in \texttt{BiophaseIntegration/AcceptanceCriteria.lean} with feasibility from \texttt{BiophasePhysics/ChannelFeasibility.lean}).
\end{itemize}
Thus the CP witness is induced canonically by the PC-side DEC/Maxwell displays and shares the same RS units.

\subsection{CP\texorpdfstring{$\Rightarrow$}{⇒}PC witness}
Given a lawful \emph{ConsciousProcess} (CP), we construct a \emph{PhotonChannel} (PC) that realizes the CP at the bridge:
\[
  \texttt{conscious\_to\_photon} \ \text{and}\ 
  \texttt{conscious\_to\_photon\_witness} :
  \quad
  (cp:\texttt{ConsciousProcess}) \;\mapsto\; \exists\,pc:\texttt{PhotonChannel},\ 
  pc.\texttt{bridge}=cp.\texttt{bridge}\ \wedge\ pc.\texttt{units}=cp.\texttt{units}\ \wedge\ \texttt{WellFormed}(pc).
\]
The classification passes through the four lemmas:
\begin{enumerate}
  \item \textbf{No-medium-knobs (Lemma~A)} excludes material/medium-dependent channels at the bridge (Lean: \texttt{NoMediumKnobs.lean}).
  \item \textbf{Null-only (Lemma~B)} enforces massless propagation at the display (Lean: \texttt{NullOnly.lean}, using \texttt{display\_speed\_eq\_c} and cone-bound facts).
  \item \textbf{Maxwellization (Lemma~C)} forces the abelian \(U(1)\) carrier under exactness and no extra parameters (Lean: \texttt{Maxwellization.lean}; DEC/Maxwell in \texttt{MaxwellDEC.lean}).
  \item \textbf{BIOPHASE feasibility (Lemma~D)} uniquely selects EM in the BIOPHASE domain (Lean: \texttt{BiophasePhysics/ChannelFeasibility.lean}, exported as \texttt{lemma\_d\_proven}).
\end{enumerate}
The result is a DEC/Maxwell PC with \(F=\mathrm{d}A\), \(\mathrm{d}F=0\) (Bianchi), and \(\mathrm{d}J=0\) (continuity), matching the CP's bridge- and units-side obligations.

\subsection{Uniqueness up to units}
The PC witness constructed for a given CP is unique up to admissible units moves (transformations preserving \(\ell_0/\tau_0=c\)). In Lean:
\[
  \texttt{photon\_channel\_unique\_up\_to\_units} :
  \quad
  pc_1,pc_2:\texttt{PhotonChannel}\ \land\ \text{same bridge/units (mod units)}\
  \Rightarrow\ pc_1 \sim_U pc_2,
\]
where \(\sim_U\) denotes units-equivalence at the bridge (i.e., identical dimensionless displays under the K-gate and eight-beat invariants after admissible units moves). This formalizes the "there is essentially one photonic channel" reading of the uniqueness clause in the main theorem.

\subsection{Stability under route and units transformations}
Both the equivalence and the CP/PC displays are stable under:
\begin{itemize}
  \item \textbf{Route transformations (K-gate).} The two route displays (time-first vs length-first) remain equal through the K-gate identity (Lean: \texttt{K\_gate\_triple}); verifying one side is sufficient by the single-inequality audit.
  \item \textbf{Units moves.} Any units transformation preserving \(\ell_0/\tau_0=c\) leaves dimensionless displays and equivalence classes unchanged (Lean: \texttt{Constants/KDisplay.lean}).
  \item \textbf{Window alignment.} Eight-beat (period-8) realignment changes only Gray-phase indexing; neutrality and coverage are preserved (Lean: \texttt{CompleteCover} and \texttt{period\_exactly\_8}).
\end{itemize}
These properties ensure that the bi-interpretation is robust against implementation details of route selection, admissible units changes, and eight-beat phase alignment.

\subsection{Scope and extension paths}
The theorem is asserted at the bridge and, when present, in the BIOPHASE domain (via the acceptance adapter). Extensions beyond BIOPHASE can be achieved by:
\begin{enumerate}
  \item \textbf{Strengthening Maxwellization.} Replace the BIOPHASE adapter by general feasibility constraints while keeping the no-extra-parameters posture, thereby excluding non-EM channels under broader regimes.
  \item \textbf{Alternative acceptance regimes.} Use domain-specific acceptance adapters (e.g., beyond IR) and re-run the feasibility layer in \texttt{BiophasePhysics}-style modules to classify carriers at new scales.
\end{enumerate}
The Lean certificate \texttt{IndisputableMonolith/Verification/LightConsciousnessTheorem.lean} bundles the constructions and lemmas, exports #eval one-line reports for OK/FLIP status, and exposes falsifiers (e.g., medium-constant dependence, massive-mode compatibility, non-Maxwell gauge, non-EM acceptance) that flip the report under violation. This completes the assembly of the main theorem with witness construction and uniqueness up to units.

\section{Mechanical Verification and Reproducibility}
\label{sec:mechanical-verification}

This section documents the verification artifacts, module topology, certificate structure, report endpoints, and reproducibility practices used to mechanically validate the bi-interpretability theorem. All proofs compile under the pinned Lean~4 toolchain and lake configuration; one-line \#eval reports summarize the theorem status and flip to FLIP on falsifiers.

\subsection{Lean 4 Module Map and Dependencies}
The implementation is organized into three layers:

\paragraph{Bridge (Consciousness) layer.}
\begin{itemize}
  \item \texttt{IndisputableMonolith/Consciousness/ConsciousProcess.lean}:
    CP structure, invariants (units quotient, K-gate, eight-beat, speed identity), predicates \texttt{IsConsciousProcess}, \texttt{WellFormed}.
  \item \texttt{IndisputableMonolith/Consciousness/PhotonChannel.lean}:
    PC wrapper over DEC/Maxwell; same invariants; predicates \texttt{IsPhotonChannel}, \texttt{WellFormed}.
  \item \texttt{IndisputableMonolith/Consciousness/NoMediumKnobs.lean}:
    Lemma~A; \texttt{DisplayDependsOnMedium}, \texttt{no\_medium\_knobs}, falsifier \texttt{Falsifier\_MediumConstantAppears}.
  \item \texttt{IndisputableMonolith/Consciousness/NullOnly.lean}:
    Lemma~B; \texttt{null\_only}, falsifier \texttt{Falsifier\_MassiveModeExists}.
  \item \texttt{IndisputableMonolith/Consciousness/Maxwellization.lean}:
    Lemma~C; \texttt{only\_abelian\_gauge}, \texttt{maxwell\_is\_unique}, falsifier \texttt{Falsifier\_NonMaxwellGaugeExists}.
  \item \texttt{IndisputableMonolith/Consciousness/BioPhaseSNR.lean}:
    bridge-side acceptance interface (adapter).
  \item \texttt{IndisputableMonolith/Consciousness/Equivalence.lean}:
    PC\(\Rightarrow\)CP (\texttt{photon\_to\_conscious}); CP\(\Rightarrow\)PC (\texttt{conscious\_to\_photon}, \texttt{conscious\_to\_photon\_witness}); uniqueness up to units (\texttt{photon\_channel\_unique\_up\_to\_units}).
\end{itemize}

\paragraph{BIOPHASE (physics) layer.}
\begin{itemize}
  \item \texttt{IndisputableMonolith/BiophaseCore/Constants.lean}:
    \(\varphi^{-5}\,\mathrm{eV}\), \(\lambda_0\), \(\nu_0\), \(\tau_{\rm gate}\), breath/FLIP timing.
  \item \texttt{IndisputableMonolith/BiophaseCore/Specification.lean}, \texttt{EightBeatBands.lean}:
    eight-band structure around \(\nu_0\), coverage, Gray alignment.
  \item \texttt{IndisputableMonolith/BiophaseIntegration/AcceptanceCriteria.lean}:
    \(\rho\), SNR, circular variance predicates; \texttt{passes\_standard\_acceptance}.
  \item \texttt{IndisputableMonolith/BiophasePhysics/CrossSections.lean}, \texttt{SNRCalculations.lean}:
    cross-section hierarchy (EM, grav., \(\nu\)) and SNR thresholds.
  \item \texttt{IndisputableMonolith/BiophasePhysics/ChannelFeasibility.lean}:
    EM passes, non-EM fails; \texttt{lemma\_d\_proven}.
\end{itemize}

\paragraph{Core infrastructure (used by both).}
\begin{itemize}
  \item \texttt{IndisputableMonolith/Constants/KDisplay.lean}:
    K-gate (\texttt{K\_gate}, \texttt{K\_gate\_triple}); speed identity \texttt{display\_speed\_eq\_c}.
  \item \texttt{IndisputableMonolith/MaxwellDEC.lean}:
    DEC/Maxwell skeleton; \texttt{HasCoboundary} \((\mathrm{d})\), \texttt{HasHodge} \((\star)\); Bianchi \(\mathrm{d}F=0\), continuity \(\mathrm{d}J=0\).
  \item \texttt{IndisputableMonolith/Patterns.lean}:
    \texttt{Pattern}(d); \texttt{CompleteCover}(d); \texttt{cover\_exact\_pow}; \texttt{period\_exactly\_8}.
  \item \texttt{IndisputableMonolith/Verification/MainTheorems.lean}:
    J-uniqueness, \(C{=}2A\) (used for context only).
\end{itemize}

\subsection{Certificate Structure and \#eval Reports}
All pieces are bundled in the verification certificate
\[
  \texttt{IndisputableMonolith/Verification/LightConsciousnessTheorem.lean},
\]
which exposes:
\begin{itemize}
  \item the four classification lemmas (A--D),
  \item the adapter directions (PC\(\Rightarrow\)CP, CP\(\Rightarrow\)PC),
  \item uniqueness up to units,
  \item and falsifier hooks.
\end{itemize}
The following \#eval endpoints provide one-line status:
\begin{itemize}
  \item \texttt{light\_consciousness\_theorem\_report} (main OK/FLIP),
  \item \texttt{cp\_definition\_report}, \texttt{pc\_definition\_report} (interface health),
  \item \texttt{lemma\_a\_report}, \texttt{lemma\_b\_report}, \texttt{lemma\_c\_report}, \texttt{lemma\_d\_report} (focused status and guidance),
  \item \texttt{full\_report} (bundled).
\end{itemize}
These reports flip from OK to FLIP when a falsifier predicate is satisfied (Sec.~\ref{sec:falsifiers} below).

\subsection{Axiom Audit, Sorry-Stub Audit, and Version Pinning}
\paragraph{Axiom audit.}
For each exported lemma/theorem, we run:
\[
  \texttt{\#print axioms\ theorem\_name}
\]
to ensure only intended axioms are present (e.g., numerical bound axioms localized to BIOPHASE physics, or structural axioms for DEC). Any unexpected axiom inflates the audit and blocks the report from showing OK.

\paragraph{Sorry-stub audit.}
We forbid stray \texttt{sorry}/\texttt{admit} outside localized numerical bounds, and track them via lints and scripted scans. Stubs are isolated to physics tolerances (e.g., \(\varphi^{-5}\) approximations, cross-section magnitudes) and do not affect bridge-level algebra.

\paragraph{Version pinning and deterministic builds.}
We pin the toolchain via \texttt{lean-toolchain} and lake via \texttt{lake-manifest.json}. Deterministic builds are obtained with:
\[
  \texttt{lake\ build}
\]
from a clean checkout (documented in the repository README). Reports and proofs reproduce under pinned dependency versions. We record the commit hash in the certificate header when emitting artifacts.

\subsection{Falsifier Predicates as First-Class Checks}
Falsifiers are Lean predicates that, when satisfied, flip the associated report to FLIP and annotate the failure cause. Key falsifiers include:
\begin{itemize}
  \item \textbf{Lemma A (No-medium-knobs):} \texttt{Falsifier\_MediumConstantAppears}(L,B,U, m, display) in \texttt{Consciousness/NoMediumKnobs.lean}.
  \item \textbf{Lemma B (Null-only):} \texttt{Falsifier\_MassiveModeExists}(L,B,U, mode) in \texttt{Consciousness/NullOnly.lean}.
  \item \textbf{Lemma C (Maxwellization):} \texttt{Falsifier\_NonMaxwellGaugeExists}(L,B,U, gt) in \texttt{Consciousness/Maxwellization.lean}.
  \item \textbf{Lemma D (BIOPHASE):} \texttt{Falsifier\_NonEM\_Passes\_BIOPHASE}(spec,channel), \texttt{Falsifier\_EM\_Fails\_BIOPHASE}(spec) in \texttt{BiophasePhysics/ChannelFeasibility.lean}; band-coverage mismatches in \texttt{BiophaseCore/EightBeatBands.lean}.
\end{itemize}
These checks are exposed directly in \#eval reports, making counterexamples first-class citizens of the verification workflow.

\subsection{Reproducibility Checklist}
\begin{itemize}
  \item \textbf{Environment:} \texttt{lean-toolchain} (pinned), \texttt{lake-manifest.json} (pinned), OS and compiler versions recorded.
  \item \textbf{Build:} \texttt{lake build} (deterministic, no network fetch during build).
  \item \textbf{Run reports:} \#eval endpoints in \texttt{Verification/LightConsciousnessTheorem.lean}.
  \item \textbf{Axiom/\texttt{sorry} audits:} \#print axioms; scripted scans for \texttt{sorry}/\texttt{admit}.
  \item \textbf{Artifacts:} emit report logs and commit hash alongside paper artifacts; include figure regeneration scripts where applicable.
\end{itemize}
Together, these practices ensure that the bi-interpretability theorem, its classification lemmas, and the BIOPHASE feasibility are not merely claims but machine-verified results that any independent group can reproduce from the public repository under the pinned toolchain.

\section{Predictions, Protocols, and Falsifiers}
\label{sec:predictions-protocols-falsifiers}

This section states testable predictions, practical experimental protocols, and falsifier triggers. All predicates and thresholds are exposed in Lean as first-class checks, and violations flip the certificate reports from OK to FLIP.

\subsection{Spectroscopy around \texorpdfstring{$724~\mathrm{cm}^{-1}$}{724 cm\string^-1} with eight-phase stepping}
Centered at \(\nu_{0}\approx 724~\mathrm{cm}^{-1}\) (BIOPHASE IR scale), the eight-band structure with deltas
\[
  \Delta \in \{-18,-12,-6,0,+6,+12,+18,+24\}~\mathrm{cm}^{-1}
\]
provides a concrete spectroscopic signature. We predict:
\begin{itemize}
  \item \textbf{Eight-phase stepping:} stepping the phase/time alignment across eight equi-spaced sub-phases (Gray-cycle traversal) reveals periodic enhancement/suppression consistent with eight-beat neutrality. 
  \item \textbf{Acceptance metrics:} at optimal alignment, collected observables satisfy \(\rho \ge 0.30\), \(\mathrm{SNR}\ge 5\sigma\), and circular variance \(\mathrm{CV}\le 0.40\) (Lean predicates in \texttt{BiophaseIntegration/AcceptanceCriteria.lean}: \texttt{passes\_standard\_acceptance}).
\end{itemize}
In Lean, band structure and coverage appear in \texttt{BiophaseCore/Specification.lean} and \texttt{BiophaseCore/EightBeatBands.lean}; eight-beat neutrality and Gray-cycle coverage are provided by \texttt{CompleteCover} and \texttt{period\_exactly\_8} in \texttt{IndisputableMonolith/Patterns.lean}.

\subsection{Protocols}
We outline a minimal protocol family adequate to exercise the bridge invariants and BIOPHASE acceptance:
\begin{enumerate}
  \item \textbf{Phase stepping (eight-beat):} acquire spectra/visibilities at eight sub-phases uniformly covering one neutral window; stack the results to visualize periodic structure. 
  \item \textbf{Acceptance measurement:} compute \(\rho\) (Pearson-like correlation), \(\mathrm{SNR}\), and \(\mathrm{CV}\) (circular variance) under the same acquisition; assess \(\rho \ge 0.30\), \(\mathrm{SNR}\ge 5\sigma\), \(\mathrm{CV}\le 0.40\) (Lean: \texttt{passes\_standard\_acceptance} in \texttt{AcceptanceCriteria.lean}).
  \item \textbf{Window alignment:} align acquisition windows to the eight-beat cadence (Lean: \texttt{period\_exactly\_8}); misaligned windows serve as a control (see below).
  \item \textbf{BIOPHASE timing:} use \(\tau_{\mathrm{gate}}\approx 65~\mathrm{ps}\), breath \(=1024\), FLIP@512 as per \texttt{BiophaseCore/Constants.lean}; ensure instrument timing jitter is small relative to \(\tau_{\mathrm{gate}}\).
\end{enumerate}

\subsection{Controls}
Standard controls isolate invariants and stress the acceptance conditions:
\begin{itemize}
  \item \textbf{Timing shuffles:} randomly permute sub-window timing to break eight-beat alignment; prediction: reduced \(\rho\), decreased \(\mathrm{SNR}\), and increased \(\mathrm{CV}\); band signatures wash out.
  \item \textbf{Phase detuning:} deliberately detune phase from eight-step alignment; prediction: periodic enhancement/suppression degrades and acceptance fails more often.
  \item \textbf{Conservation-perturb tests:} introduce small, controlled perturbations to conservation constraints (e.g., power or polarization) while keeping other conditions fixed; prediction: acceptance metrics degrade and band structure destabilizes.
\end{itemize}
In Lean, these manipulations correspond to breaking the hypotheses of the eight-beat/acceptance predicates and, when encoded, trigger falsifier predicates or fail the acceptance lemmas.

\subsection{Quantified Outcomes}
Under the EM feasibility proven at the BIOPHASE scale (Lean: \texttt{BiophasePhysics/ChannelFeasibility.lean}), we expect:
\begin{itemize}
  \item \textbf{SNR growth:} for EM, \(\mathrm{SNR}\) scales with flux \(\times\) area \(\times\) integration-time consistent with the SNR model (Lean: \texttt{SNRCalculations.lean}); under standard parameters, \(\mathrm{SNR}\ge 5\sigma\) is achieved.
  \item \textbf{Eight-beat signatures:} spectral lines near \(\nu_{0}\) exhibit eight-phase modulation; stacked sub-phase results reveal periodic enhancement/suppression; controls (timing shuffles, detuning) reduce contrast.
  \item \textbf{Thresholds:} acceptance (\(\rho,\mathrm{SNR},\mathrm{CV}\)) crosses thresholds at aligned phases; in misalignment, \(\mathrm{SNR}\) and \(\rho\) fall below thresholds and \(\mathrm{CV}\) increases.
\end{itemize}
These outcomes are consistent with the cross-section hierarchy (EM \(\gg\) \(\nu\) \(\gg\) grav.) in \texttt{BiophasePhysics/CrossSections.lean} and feasibility in \texttt{ChannelFeasibility.lean}.

\subsection{Falsifiers and Report Flips}
Falsifiers are explicit Lean predicates that flip certificate reports (OK\(\to\)FLIP) and annotate the cause:
\begin{itemize}
  \item \textbf{Non-EM passes acceptance:} \texttt{Falsifier\_NonEM\_Passes\_BIOPHASE(spec,channel)} (Lean: \texttt{BiophasePhysics/ChannelFeasibility.lean}).
  \item \textbf{EM fails acceptance:} \texttt{Falsifier\_EM\_Fails\_BIOPHASE(spec)} (same).
  \item \textbf{Band mismatch:} predicates in \texttt{BiophaseCore/EightBeatBands.lean} (e.g., coverage/center checks) can be used to signal inconsistent spectral structure.
  \item \textbf{Bridge violations:} 
    \begin{itemize}
      \item medium-constant dependence: \texttt{Falsifier\_MediumConstantAppears} (Lemma~A; \texttt{Consciousness/NoMediumKnobs.lean});
      \item massive mode satisfying displays: \texttt{Falsifier\_MassiveModeExists} (Lemma~B; \texttt{Consciousness/NullOnly.lean});
      \item non-Maxwell gauge at bridge: \texttt{Falsifier\_NonMaxwellGaugeExists} (Lemma~C; \texttt{Consciousness/Maxwellization.lean}).
    \end{itemize}
\end{itemize}
Report endpoints are exported in \texttt{IndisputableMonolith/Verification/LightConsciousnessTheorem.lean}: \texttt{light\_consciousness\_theorem\_report}, \texttt{lemma\_d\_report}, and \texttt{full\_report}.

\paragraph{How to interpret flips.}
A flip indicates a violation of the hypotheses required by the bi-interpretability theorem:
\begin{itemize}
  \item \emph{BIOPHASE flip} (non-EM passes or EM fails): Lemma~D is contradicted under the specified spec; reassess calibration, band alignment, thresholds, or channel identification.
  \item \emph{Bridge flip} (A/B/C): a foundational invariant is violated (e.g., medium-constant dependence, massive contribution to display, non-abelian gauge at the bridge).
\end{itemize}
Flips therefore function as ``counterexample captures'' in the verification workflow; the Lean certificate records the violating predicate and blocks downstream claims until resolved.

\paragraph{Lean reproducibility.}
All predicates (acceptance, bands, falsifiers) are first-class in Lean and can be evaluated under the pinned toolchain (Sec.~\ref{sec:mechanical-verification}). This enables end-to-end, reproducible validation of spectroscopy protocols near \(724~\mathrm{cm}^{-1}\), including eight-phase stepping, controls, and threshold audits.

\section{Related Work}
\label{sec:related-work}

\paragraph{Information-theoretic derivations of quantum probabilities.}
A long literature seeks to derive Born weights from information-theoretic or structural postulates, including envariance-based arguments and decision-theoretic programs. Our prior work within RS formalizes the unique information-cost functional
\(
  J(x)=\tfrac{1}{2}(x+x^{-1})-1
\)
and the two-branch bridge \(C=2A\) as mechanically verified Lean theorems (Lean: \texttt{THEOREM\_1\_universal\_cost\_uniqueness}, \texttt{THEOREM\_2\_measurement\_recognition\_bridge} in \texttt{IndisputableMonolith/Verification/MainTheorems.lean}). These establish a context in which cost-based derivations and selection dynamics are coherent under the RS invariants. By contrast, the present paper does not rely on \(C{=}2A\) or \(J\)-minimization for the main equivalence: our theorem is stated entirely at the bridge in terms of invariants and feasibility (Secs.~\ref{sec:preliminaries}--\ref{sec:lemmaD}).

\paragraph{Envariance, decision-theoretic accounts, matched-filter interpretations.}
Zurek's envariance program and Deutsch--Wallace decision-theoretic accounts exemplify attempts to ground quantum probabilities in symmetries or rational-choice axioms. Optical matched-filter interpretations motivate experimental acceptance metrics---e.g.\ SNR thresholds---but typically do not formalize bridge-level equivalence statements. Our work differs by (i) stating a \emph{bi-interpretability theorem} at the RS bridge, (ii) isolating modular interface obligations (K-gate, eight-beat, speed identity, acceptance), and (iii) bundling the result into a Lean certificate with falsifiers and report endpoints (Sec.~\ref{sec:mechanical-verification}).

\paragraph{Photonic processing and DEC foundations.}
Discrete exterior calculus (DEC) provides a clean finite-dimensional skeleton for gauge fields and conservation laws. We adopt a lightweight DEC/Maxwell interface (Lean: \texttt{IndisputableMonolith/MaxwellDEC.lean}) exposing \(\mathrm{d}\) and, when needed, \(\star\), with Bianchi \(\mathrm{d}F=0\) and continuity \(\mathrm{d}J=0\) as structural obligations (Sec.~\ref{sec:preliminaries}). While DEC has been used across numerical electromagnetics and geometry processing, the novelty here is the \emph{classification role}: exactness \(\mathrm{d}\circ \mathrm{d}=0\), gauge compatibility, and the RS no-extra-parameters posture force abelian \(U(1)\) at the bridge (Lemma~C, Sec.~\ref{sec:lemmaC}).

\paragraph{How this theorem differs.}
Relative to prior information-theoretic or DEC-based treatments, our theorem adds:
\begin{itemize}
  \item \textbf{Bridge-level equivalence:} CP\(\leftrightarrow\)PC under explicit RS invariants (units quotient, K-gate, eight-beat, speed identity).
  \item \textbf{Feasibility proof at BIOPHASE:} EM-only selection by cross-section hierarchies and SNR thresholds (Lemma~D, Sec.~\ref{sec:lemmaD}; Lean: \texttt{BiophasePhysics/ChannelFeasibility.lean}).
  \item \textbf{Machine verification:} complete certificate with \#eval reports and falsifiers (\texttt{IndisputableMonolith/Verification/LightConsciousnessTheorem.lean}).
\end{itemize}
Whereas Paper~\#1 established \emph{J}-system identity and the cost bridge, this paper provides a \emph{classification-level equivalence} with feasibility and reproducible verification.

\section{Discussion: Scope, Limitations, Extensions}
\label{sec:discussion}

\paragraph{Scope (bridge + BIOPHASE) and uniqueness up to units.}
Our theorem is asserted at the RS bridge, where displays are dimensionless, route-consistent (K-gate), eight-beat aligned, and show speed \(c\). In the BIOPHASE domain, we attach acceptance thresholds (\(\rho\ge 0.30\), \(\mathrm{SNR}\ge 5\sigma\), \(\mathrm{CV}\le 0.40\)) via a stable adapter. Uniqueness is up to admissible units moves preserving \(\ell_0/\tau_0=c\); different realizations with identical quotient displays are considered the same at the bridge (Lean: \texttt{photon\_channel\_unique\_up\_to\_units} in \texttt{Consciousness/Equivalence.lean}).

\paragraph{Generalizing beyond BIOPHASE.}
To extend the classification outside BIOPHASE, one can:
\begin{itemize}
  \item \textbf{Strengthen Maxwellization (Lemma~C):} rule out non-EM channels by tightening ``no extra parameters'' and exactness beyond the IR adapter, preserving the same bridge invariants.
  \item \textbf{Alternative acceptance regimes:} replace BIOPHASE thresholds with domain-appropriate acceptance predicates; re-run feasibility proofs in the style of \texttt{BiophasePhysics/ChannelFeasibility.lean}.
\end{itemize}
These changes do not alter the bridge equivalence; they modify only the feasibility/selection adapter.

\paragraph{Categorical equivalence and global exclusivity.}
Beyond the bridge, one can pursue categorical equivalences (e.g., functorial mappings between CP- and PC-categories) and global exclusivity statements (strengthening uniqueness up to units to uniqueness up to broader equivalences). Such developments would use the present bridge theorem as a base case and add structure at higher levels (e.g., composition of adapters, monoidal actions).

\paragraph{Numerical tolerances and instrument constraints.}
BIOPHASE feasibility depends on tolerances (e.g., \(\pm\) bandwidth around 724~cm\(^{-1}\), acceptable \(\tau_{\mathrm{gate}}\) jitter, detector noise floors). Our Lean proofs factor numerical approximations into localized lemmas (e.g., \texttt{BiophaseCore/Constants.lean}, \texttt{BiophasePhysics/SNRCalculations.lean}); practical implementations should audit error budgets against acceptance thresholds and use the falsifiers to flag out-of-tolerance scenarios.

\section{Conclusion}
\label{sec:conclusion}

We have established a bi-interpretability theorem at the RS bridge: \emph{ConsciousProcess} and \emph{PhotonChannel} are equivalent (with uniqueness up to units) under explicit interface obligations (units quotient, K-gate, eight-beat, speed identity), and only the electromagnetic channel satisfies BIOPHASE feasibility. The proof proceeds via four classification lemmas (A--D) and is bundled as a Lean certificate that exports \#eval reports and falsifier predicates. 

This paper completes the path from the hypothesis-level identity (Paper~\#1) to a theorem at the bridge: the former provides the cost-theoretic context (unique \(J\), \(C{=}2A\)), the latter delivers a classification-level equivalence with physical exclusion and machine verification. Future work will generalize feasibility beyond BIOPHASE, pursue categorical formulations beyond the bridge, and extend cross-domain tests of the equivalence. The falsifiers and report endpoints ensure that any violation (experimental or theoretical) is immediately surfaced, keeping the theory explicitly testable and reproducible.

\appendix

\section{RS Bridge Invariants (K-gate, Eight-Tick Derivations)}
\label{app:rs-bridge-invariants}

\paragraph{RS units pack and quotient displays.}
Let \(U=\langle \tau_0,\ell_0,c\rangle\) be an RS units pack (Lean: \texttt{RSUnits} in \texttt{IndisputableMonolith/Constants.lean}) with structural identity
\begin{equation}
  c\,\tau_0 \;=\; \ell_0,
  \qquad\text{i.e.}\qquad
  \frac{\ell_0}{\tau_0} \;=\; c.
\end{equation}
The bridge requires all displays to be \emph{dimensionless} under the units quotient; i.e.\ any observable \(\mathcal{O}\) appearing on the bridge must be a quotient of \(\tau_0,\ell_0,c\) and their induced displays so that admissible units moves \((\tau_0,\ell_0,c)\mapsto (\alpha\tau_0,\alpha\ell_0,c)\) leave \(\mathcal{O}\) invariant.

\paragraph{K-gate identity.}
The bridge exposes two canonical displays
\[
  \tau_{\mathrm{rec}}^{(\mathrm{display})} \;=\; K\,\tau_0,
  \qquad
  \lambda_{\mathrm{kin}}^{(\mathrm{display})} \;=\; K\,\ell_0
\]
so that
\begin{equation}
  \frac{\tau_{\mathrm{rec}}^{(\mathrm{display})}}{\tau_0}
  \;=\;
  \frac{\lambda_{\mathrm{kin}}^{(\mathrm{display})}}{\ell_0}
  \;=\; K.
\end{equation}
In Lean these are \texttt{tau\_rec\_display\_ratio} and \texttt{lambda\_kin\_display\_ratio}, bundled by \texttt{K\_gate} and \texttt{K\_gate\_triple} (see \texttt{IndisputableMonolith/Constants/KDisplay.lean}). The \emph{single-inequality audit} notes that checking one equality suffices, as the K-gate implies the other.

\paragraph{Display-speed identity.}
Combining \(c\,\tau_0=\ell_0\) with the K-gate displays yields
\begin{equation}
  \frac{\lambda_{\mathrm{kin}}^{(\mathrm{display})}}
       {\tau_{\mathrm{rec}}^{(\mathrm{display})}}
  \;=\; c,
\end{equation}
implemented as \texttt{display\_speed\_eq\_c} (Lean: \texttt{Constants/KDisplay.lean}). This encodes "null" display at the bridge.

\paragraph{Eight-tick minimal window for \(D=3\).}
Let \(\texttt{Pattern}(d)\coloneqq \texttt{Fin}\,d\to\texttt{Bool}\). A \texttt{CompleteCover}(d) consists of a period \(\texttt{T}\), a path \(\texttt{Fin}\,\texttt{T}\to\texttt{Pattern}(d)\), and surjectivity. The fundamental existence theorem (Lean: \texttt{cover\_exact\_pow}) constructs \(\texttt{w}:\texttt{CompleteCover}(d)\) with \(\texttt{w.period}=2^d\), and \texttt{period\_exactly\_8} specializes this to \(d=3\) (eight-beat). Thus any neutrality-respecting walk must have period at least \(2^3=8\), and the Gray-cycle provides a canonical realization.

\section{\texorpdfstring{\(\varphi^{-5}\to \lambda_0 \to \nu_0\)}{phi^-5 -> lambda0 -> nu0} Conversions; Error Tolerances}
\label{app:phi-lambda-nu}

\paragraph{Energy scale from \(\varphi\).}
We adopt the BIOPHASE IR energy scale
\begin{equation}
  E_{\mathrm{rec}} \;=\; \varphi^{-5}\,\mathrm{eV} \;\approx\; 0.0901699437~\mathrm{eV},
\end{equation}
with Lean realization and approximation lemmas (tolerances localized) in \texttt{IndisputableMonolith/BiophaseCore/Constants.lean}. 

\paragraph{Wavelength and wavenumber.}
Using CODATA \(h\) and \(c\), the nominal wavelength is
\begin{equation}
  \lambda_0 \;=\; \frac{h\,c}{E_{\mathrm{rec}}} \;\approx\; 13.8~\mu\mathrm{m},
\end{equation}
and the wavenumber
\begin{equation}
  \nu_0 \;=\; \frac{1}{\lambda_0} \;\approx\; 724~\mathrm{cm}^{-1}
\end{equation}
(Lean: \texttt{lambda\_biophase}, \texttt{nu0\_cm1} in \texttt{BiophaseCore/Constants.lean}). We employ tolerances \(|\lambda_0-13.8~\mu\mathrm{m}|<\delta_{\lambda}\), \(|\nu_0-724~\mathrm{cm}^{-1}|<\delta_{\nu}\) where \(\delta_{\lambda},\delta_{\nu}\) are instrument- and conversion-dependent; approximate proofs are isolated to numerical lemmas to keep bridge-level algebra clean.

\paragraph{Gate timing and breath.}
We use \(\tau_{\mathrm{gate}}\approx 65~\mathrm{ps}\), \(\texttt{breath}=1024\), \(\texttt{FLIP@512}\) (Lean: constants and checks in \texttt{BiophaseCore/Constants.lean}). Timing jitter tolerances are instrument-specific and must be audited against acceptance thresholds.

\section{Eight-Beat Band Construction; Gray Alignment}
\label{app:eightbeat-construction}

\paragraph{Band centers and widths.}
Let \(\nu_0\approx 724~\mathrm{cm}^{-1}\). Define eight band centers by deltas
\[
  \Delta \in \{-18, -12, -6, 0, +6, +12, +18, +24\}~\mathrm{cm}^{-1},
\]
so that \(\nu_k=\nu_0+\Delta_k\) for \(k=0,\dots,7\). For a nominal half-width \(w/2\) (e.g., \(w\in[5,25]\)~cm\(^{-1}\)), the acceptance predicate at frequency \(\nu\) is
\begin{equation}
  \exists\,k\in\{0,\dots,7\},\quad |\nu - \nu_k| \;\le\; \frac{w}{2}.
\end{equation}
Lean realizations: \texttt{BiophaseCore/Specification.lean} and \texttt{BiophaseCore/EightBeatBands.lean} define band structures, widths, and coverage.

\paragraph{Coverage and neutrality.}
Stacking the eight bands yields coverage of approximately \(57~\mathrm{cm}^{-1}\) (depending on \(w\)); band coverage routines and checks appear in \texttt{EightBeatBands.lean}. Eight-beat neutrality requires alignment to a \texttt{CompleteCover 3} of period~8 (Lean: \texttt{period\_exactly\_8} in \texttt{Patterns.lean}). Experimental eight-phase stepping corresponds to traversing Gray-phase indices, and periodic enhancement/suppression tracks the band disposition relative to \(\nu_0\).

\paragraph{Gray alignment.}
Let \(\texttt{Pattern}(3)\) be the 3-bit space and \(\texttt{path}:\texttt{Fin}\,8\to \texttt{Pattern}(3)\) a Gray-cycle path (Lean-derived). A phase-aligned readout assigns each phase index \(i\in\{0,\dots,7\}\) to a band center \(\nu_i=\nu_0+\Delta_i\), preserving the cycle order. Misalignment or randomization (\emph{timing shuffles}) degrades periodic structure and acceptance metrics (correlation, SNR), while proper realignment restores the eight-beat signatures. Acceptance checks are encoded via \texttt{passes\_standard\_acceptance} (Lean: \texttt{BiophaseIntegration/AcceptanceCriteria.lean}) and flipping falsifiers in \texttt{BiophasePhysics/ChannelFeasibility.lean}.

\section{Cross-Section Details: Thomson, Gravitational Scaling, Weak Cross-Section}
\label{app:cross-sections}

\paragraph{Electromagnetic (Thomson) cross-section.}
At BIOPHASE energies the elastic (low-energy) photon–electron cross-section is well-approximated by the Thomson value
\begin{equation}
  \sigma_{\mathrm{T}}
  \;=\;
  \frac{8\pi}{3}\,r_{e}^{2},
  \qquad
  r_{e}
  \;=\;
  \frac{e^{2}}{4\pi\varepsilon_{0}\,m_{e}\,c^{2}},
\end{equation}
yielding numerically
\(
  \sigma_{\mathrm{EM}}\approx 6.65\times 10^{-29}\,\mathrm{m}^{2}.
\)
Lean: definitions and numeric inequalities appear in
\texttt{IndisputableMonolith/BiophasePhysics/CrossSections.lean} as
\texttt{classical\_electron\_radius}, \texttt{sigma\_thomson}, and derived
\texttt{sigma\_em} with a specialization at BIOPHASE,
e.g.\ \texttt{sigma\_em\_at\_biophase}.

\paragraph{Effective gravitational cross-section (scaling bound).}
As a conservative upper bound, use the squared gravitational length scale of the electron,
\(
  r_{g}=G\,m_{e}/c^{2}\approx 6.7\times 10^{-58}\,\mathrm{m},
\)
so any geometric cross-section obeys
\(
  \sigma_{\mathrm{grav}}\lesssim \pi r_{g}^{2}\ll 10^{-70}\,\mathrm{m}^{2}.
\)
This is many orders below the EM Thomson scale and suffices for feasibility exclusion. Lean: see bounding lemmas in
\texttt{BiophasePhysics/CrossSections.lean} (e.g., \texttt{sigma\_grav\_negligible}).

\paragraph{Weak (neutrino) cross-section.}
At low energies, a schematic scaling for neutrino scattering is
\(
  \sigma_{\nu}\sim G_{F}^{2}\,E^{2}/\pi
\)
(in natural units), giving
\(
  \sigma_{\nu}\lesssim 10^{-48}\,\mathrm{m}^{2}
\)
at \(E\approx 0.09\,\mathrm{eV}\). This lies roughly \(10^{19}\)–\(10^{20}\) times below \(\sigma_{\mathrm{EM}}\). Lean: cross-section definitions and conversions are provided in
\texttt{BiophasePhysics/CrossSections.lean} (e.g., \texttt{sigma\_neutrino\_cm2}, \texttt{sigma\_neutrino}) and comparison theorems (\texttt{nu\_smaller}, \texttt{grav\_smallest}).

\paragraph{Hierarchy summary (BIOPHASE).}
\[
  \sigma_{\mathrm{EM}}\;\approx\;6.65\times 10^{-29}\,\mathrm{m}^{2}
  \;\gg\;
  \sigma_{\nu}\;\lesssim\;10^{-48}\,\mathrm{m}^{2}
  \;\gg\;
  \sigma_{\mathrm{grav}}\;\ll\;10^{-70}\,\mathrm{m}^{2}.
\]
These separations drive the feasibility result (Lemma~D) once SNR thresholds and acceptance windows are enforced (Appendix~\ref{app:snr-derivations} and Theorem~\ref{thm:biophase-feasibility}).


\section{SNR Derivations and Parameter Sensitivity}
\label{app:snr-derivations}

\paragraph{SNR model.}
Let \(\Phi\) denote photon (or event) flux, \(A\) the effective area, \(T\) the integration time, \(\eta\) the detection efficiency, and \(\sigma\) the interaction cross-section. The expected signal counts scale as
\(
  S=\eta\,\Phi\,A\,\sigma\,T.
\)
With background \(B\) (counts) and additional noise variance \(\sigma_{N}^{2}\), a standard SNR surrogate is
\begin{equation}
  \mathrm{SNR}
  \;=\;
  \frac{S}{\sqrt{S + B + \sigma_{N}^{2}}}.
  \label{eq:snr-def}
\end{equation}
Lean: the parameter bundle and SNR accessor appear as \texttt{SNRParams} and \texttt{SNRParams.SNR} in
\texttt{IndisputableMonolith/BiophasePhysics/SNRCalculations.lean}, with instances
\texttt{em\_params}, \texttt{grav\_params}, \texttt{nu\_params}.

\paragraph{Threshold regimes.}
Two limiting regimes are useful:
\begin{itemize}
  \item \emph{Source-limited} (\(B+\sigma_{N}^{2}\ll S\)):
    \(
      \mathrm{SNR}\approx \sqrt{S}\propto \sqrt{\eta\,\Phi\,A\,\sigma\,T}.
    \)
  \item \emph{Background-limited} (\(B\gg S,\sigma_{N}^{2}\)):
    \(
      \mathrm{SNR}\approx S/\sqrt{B}\propto \eta\,\Phi\,A\,\sigma\,T/\sqrt{B}.
    \)
\end{itemize}
Thus \(\partial\mathrm{SNR}/\partial \Phi>0\), \(\partial\mathrm{SNR}/\partial A>0\), \(\partial\mathrm{SNR}/\partial T>0\), \(\partial\mathrm{SNR}/\partial \sigma>0\), holding others fixed. Lean: threshold theorems include
\texttt{em\_snr\_exceeds\_threshold}, \texttt{grav\_snr\_fails}, \texttt{nu\_snr\_fails}.

\paragraph{BIOPHASE acceptance.}
Acceptance requires
\(
  \rho\ge 0.30,\;
  \mathrm{SNR}\ge 5\sigma,\;
  \mathrm{CV}\le 0.40
\)
and eight-beat alignment (Sec.~\ref{sec:predictions-protocols-falsifiers}). For EM at BIOPHASE, \(\sigma\) is large enough that with practical \((\Phi,A,T)\), \(\mathrm{SNR}\ge 5\) is achieved. For gravitational and weak channels, even aggressive \((\Phi,A,T)\) leave \(\mathrm{SNR}\ll 1\). Lean: see
\texttt{IndisputableMonolith/BiophaseIntegration/AcceptanceCriteria.lean}
(\texttt{passes\_standard\_acceptance}) and
\texttt{IndisputableMonolith/BiophasePhysics/ChannelFeasibility.lean}
(\texttt{em\_passes\_biophase\_proven}, \texttt{gravitational\_fails\_biophase\_proven}, \texttt{neutrino\_fails\_biophase\_proven}).

\paragraph{Sensitivity to windowing and cadence.}
Eight-beat alignment improves effective \(\rho\) and \(\mathrm{SNR}\) by constructive accumulation over admissible windows; timing shuffles or phase detuning reduce both. Lean: window predicates and Gray alignment are provided by \texttt{IndisputableMonolith/Patterns.lean} (\texttt{CompleteCover}, \texttt{period\_exactly\_8}); alignment improves acceptance in \texttt{AcceptanceCriteria.lean}.

\paragraph{Crossing the \(5\sigma\) threshold.}
From \eqref{eq:snr-def}, source-limited operation yields
\(
  \sqrt{\eta\,\Phi\,A\,\sigma\,T}\ge 5
  \Rightarrow
  T \ge 25 / (\eta\,\Phi\,A\,\sigma).
\)
With \(\sigma=\sigma_{\mathrm{T}}\), the required \(T\) is modest at BIOPHASE fluxes/areas; with \(\sigma\in\{\sigma_{\nu},\sigma_{\mathrm{grav}}\}\), \(T\) becomes astronomically large and infeasible under bridge constraints, establishing failure of acceptance for non-EM channels.


\section{Lean Proof Excerpts, Certificate, and \texorpdfstring{\#eval}{\#eval} Outputs}
\label{app:lean-excerpts}

\paragraph{Certificate bundling.}
The theorem and lemmas are bundled in
\[
  \texttt{IndisputableMonolith/Verification/LightConsciousnessTheorem.lean},
\]
which aggregates:
\begin{itemize}
  \item CP/PC definitions and invariants (\texttt{ConsciousProcess.lean}, \texttt{PhotonChannel.lean});
  \item Lemmas A–D (\texttt{NoMediumKnobs.lean}, \texttt{NullOnly.lean}, \texttt{Maxwellization.lean}, \texttt{BiophasePhysics/ChannelFeasibility.lean});
  \item Adapter directions and uniqueness (\texttt{Consciousness/Equivalence.lean});
  \item Acceptance predicates (\texttt{BiophaseIntegration/AcceptanceCriteria.lean});
  \item DEC/Maxwell structure (\texttt{MaxwellDEC.lean});
  \item Bridge invariants (\texttt{Constants/KDisplay.lean}, \texttt{Patterns.lean}).
\end{itemize}

\paragraph{Representative exported theorems (names).}
\begin{itemize}
  \item \texttt{photon\_to\_conscious} (PC\(\Rightarrow\)CP), \texttt{conscious\_to\_photon} (CP\(\Rightarrow\)PC), \texttt{photon\_channel\_unique\_up\_to\_units}.
  \item \texttt{no\_medium\_knobs}, \texttt{null\_only}, \texttt{only\_abelian\_gauge}, \texttt{maxwell\_is\_unique}, \texttt{lemma\_d\_proven}.
  \item Contextual: \texttt{THEOREM\_1\_universal\_cost\_uniqueness}, \texttt{THEOREM\_2\_measurement\_recognition\_bridge} (in \texttt{Verification/MainTheorems.lean}).
\end{itemize}

\paragraph{One-line \#eval reports (typical outputs).}
After \texttt{lake build}, in the certificate module:
\begin{itemize}
  \item \texttt{\#eval light\_consciousness\_theorem\_report} \\
    Typical: \emph{OK: CP \(\leftrightarrow\) PC; uniqueness up to units; EM feasible; falsifiers: none.}
  \item \texttt{\#eval lemma\_a\_report}, \texttt{\#eval lemma\_b\_report}, \texttt{\#eval lemma\_c\_report}, \texttt{\#eval lemma\_d\_report} \\
    Typical: \emph{OK: obligations satisfied.}
  \item \texttt{\#eval cp\_definition\_report}, \texttt{\#eval pc\_definition\_report} \\
    Typical: \emph{OK: invariants (K-gate, units quotient, eight-beat, speed \(c\))}.
\end{itemize}
If a falsifier predicate is satisfied (e.g., non-EM passes acceptance), the corresponding report flips to \emph{FLIP: <cause>} with guidance to the violated assumption.

\paragraph{Axiom and sorry audits.}
For each exported result:
\begin{itemize}
  \item \texttt{\#print axioms theorem\_name} ensures only intended axioms are referenced (DEC structure, numerical tolerance lemmas localized to BIOPHASE).
  \item scripted scans confirm absence of stray \texttt{sorry}/\texttt{admit} outside localized numerical lemmas; any violation blocks OK in the certificate report.
\end{itemize}

\paragraph{Reproduction recipe.}
\begin{enumerate}
  \item Ensure the pinned toolchain (\texttt{lean-toolchain}) and lake manifest are present.
  \item Build: \texttt{lake build}.
  \item Open \texttt{Verification/LightConsciousnessTheorem.lean} and evaluate the \#eval endpoints listed above.
  \item Optionally run contextual checks in \texttt{Verification/MainTheorems.lean} (J-uniqueness and \(C{=}2A\)).
\end{enumerate}
The outputs, together with falsifier predicates, provide a concise, reproducible audit of the theorem's hypotheses and conclusions.

\section{Minimal EM SNR Budget at BIOPHASE (Worked Inequalities)}
\label{app:min-snr-budget}

\paragraph{Objective.}
Provide a compact, pinned inequality showing that electromagnetic channels at BIOPHASE can reach \(\mathrm{SNR}\ge 5\sigma\) under reasonable laboratory parameters, consistent with the acceptance predicate (Lean: \texttt{passes\_standard\_acceptance}).

\paragraph{Setup.}
Let \(\Phi\) be the photon flux (\(\mathrm{s}^{-1}\,\mathrm{m}^{-2}\)), \(A\) the effective area, \(T\) the integration time, \(\eta\) the net detection efficiency, and \(\sigma_{\mathrm{EM}}\approx 6.65\times 10^{-29}\,\mathrm{m}^{2}\) the Thomson-scale effective cross-section at BIOPHASE (Appendix~\ref{app:cross-sections}). The expected signal counts are
\[
  S \;=\; \eta\,\Phi\,A\,\sigma_{\mathrm{EM}}\,T,
\]
with background \(B\) and additional variance \(\sigma_N^2\). The SNR (Appendix~\ref{app:snr-derivations}) is
\[
  \mathrm{SNR} \;=\; \frac{S}{\sqrt{S+B+\sigma_N^2}}.
\]

\paragraph{Source-limited gate.}
When \(S\gg B+\sigma_N^2\), the threshold \(\mathrm{SNR}\ge 5\) requires
\[
  S\ge 25
  \quad\Rightarrow\quad
  T \;\ge\; \frac{25}{\eta\,\Phi\,A\,\sigma_{\mathrm{EM}}}.
\]
Equivalently, for a fixed gate \(T\) (e.g., BIOPHASE \(\tau_\mathrm{gate}\)), one needs
\[
  \eta\,\Phi\,A \;\ge\; \frac{25}{\sigma_{\mathrm{EM}}\,T}.
\]
With \(\sigma_{\mathrm{EM}}\) as above and sub-millisecond effective integration (sum over many 65 ps windows), modest \(\eta\,\Phi\,A\) products suffice.

\paragraph{Background-limited gate.}
When \(B\gg S,\sigma_N^2\), the condition is
\[
  \frac{S}{\sqrt{B}}\ge 5
  \quad\Rightarrow\quad
  \eta\,\Phi\,A\,\sigma_{\mathrm{EM}}\,T \;\ge\; 5\,\sqrt{B}.
\]
Planck-tail backgrounds at \(\lambda_0\approx 13.8\,\mu\mathrm{m}\) within narrow acceptance bands (Appendix~\ref{app:eightbeat-construction}), combined with eight-beat averaging, keep \(B\) controllable; feasibility lemmas in Lean instantiate this regime.

\paragraph{Consistency with Lean.}
These inequalities match the \texttt{em\_snr\_exceeds\_threshold} witness pattern in \texttt{BiophasePhysics/SNRCalculations.lean}; parameter choices are encoded in \texttt{em\_params}. The acceptance predicate \texttt{passes\_standard\_acceptance} then discharges \(\mathrm{SNR}\ge 5\sigma\) alongside \(\rho\ge 0.30\) and \(\mathrm{CV}\le 0.40\).

\end{document}
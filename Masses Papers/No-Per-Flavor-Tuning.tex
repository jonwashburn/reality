\documentclass[11pt]{article}

% --- Minimal, clean page design ---
\usepackage[margin=1in]{geometry}
\usepackage[T1]{fontenc}
\usepackage{lmodern}
\usepackage{microtype}
\usepackage{amsmath,amssymb}

% Paragraph styling for a clean, modern look
\setlength{\parindent}{0pt}
\setlength{\parskip}{6pt}

% A simple, elegant rule macro
\newcommand{\Hrule}{\vspace{0.8em}\hrule\vspace{0.8em}}

\begin{document}

% ---------- Title Page (Front Matter) ----------
\begin{titlepage}
  \centering

  \Hrule
  {\LARGE \bfseries Single-Anchor Mass Identity, Coherent Equal-$Z$ Response (No Per-Flavor Tuning),\\
  Discrete Leptonic Charge--Parity from Writhe, and a Neutrino No-Go under Neutral Locks\par}
  \Hrule

  \vspace{1.25em}
  {\large \textbf{Jonathan Washburn}}\\
  {\large Recognition Science \& Recognition Physics Institute}\\
  {\large \texttt{jon@recognitionphysics.org}}\\
  {\large Austin, Texas, USA}

  \vfill

  % Optional metadata block (kept simple and self-contained)
  {\large \textbf{Manuscript Type:} Theory (no experimental input)}\\
  {\large \textbf{Compiled:} \today}

  \vspace{1.25em}
  \Hrule
  \vspace{0.5em}
  {\normalsize
  \textbf{Keywords:} universal anchor; equal-$Z$ degeneracy; rung ladder; golden ratio $\varphi$;\\
  coherent response; no per-flavor tuning; writhe parity; neutrino sector; no-go theorem
  }
  \vspace{0.3em}
  \Hrule

\end{titlepage}

% ---------- Title & Abstract ----------
\begin{center}
  {\Large \bfseries Single-Anchor Mass Identity, Coherent Equal-$Z$ Response (No Per-Flavor Tuning),\\
  Discrete Leptonic Charge--Parity from Writhe, and a Neutrino No-Go under Neutral Locks\par}
  \vspace{0.75em}

  {\normalsize \textbf{Jonathan Washburn}\\
  Recognition Science \& Recognition Physics Institute \quad|\quad
  \texttt{jon@recognitionphysics.org}\\
  Austin, Texas, USA}
\end{center}

\begin{abstract}
We present four theory results that require no experimental input. First, at a single universal anchor scale $\mu_\*$, the mass-display map for each species collapses to a closed form in the charge index $Z$ and the golden ratio $\varphi$,
\[
f_i(\mu_\*,m_i)=\frac{\ln\!\bigl(1+Z_i/\varphi\bigr)}{\ln\varphi},
\]
which yields two immediate consequences: (i) all members of an equal-$Z$ family share the same anchor display (equal-$Z$ degeneracy), and (ii) anchor ratios within a family are pure powers set by rung differences ($\varphi^{\Delta r}$). Second, we prove a meta-theorem: if the anchor is chosen by a standard stationarity principle and the resulting fixed point depends smoothly and convexly on shared kernel or policy parameters, then equal-$Z$ families respond coherently to any shared deformation at the anchor; any per-flavor tweak necessarily breaks equal-$Z$ degeneracy. Hence, per-flavor tuning at the anchor is forbidden.

Third, we derive a discrete charge--parity structure for leptons from writhe parity of the neutral three-cycle: trivial writhe enforces a Dirac branch with an exactly vanishing neutrinoless double-beta amplitude, while nontrivial writhe fixes a Majorana branch with maximal phase $\delta=\pm \pi/2$ and a narrow, computable interval for the effective double-beta mass. Finally, under three austere neutral-sector locks—Dirac identity, zero neutral residue at the anchor, and neutral transport shared with the charged sectors—the unique neutrino rung triplet admitted by the constructor fails the oscillation-ratio acceptance test for both orderings. Because the ratio depends only on rung differences, no adjustment of transport or scale can rescue it. This constitutes a neutrino no-go within the single-anchor, no-tuning framework and cleanly identifies the minimal relaxation sites for any future closure.
\end{abstract}

\section{Orientation and Contributions}

This paper is theory-only. It presents clean statements that follow from a fixed, finite construction with no species-by-species tuning. The aim is to state what is held fixed, what is proved in this document, and what is deliberately left to a follow-up.

\subsection*{What is fixed}

\textbf{Finite ledger and dictionary.} Each particle is built from a finite motif dictionary. The construction yields two integers:
(i) a charge index $Z$ that classifies species into equal-$Z$ families, and
(ii) a rung index $r$ that places each species on a discrete ladder. The landing is integer by construction and obeys an eight-tick periodicity in $r$.

\textbf{Golden-ratio ladder.} The ladder is multiplicative in the golden ratio $\varphi=(1+\sqrt{5})/2$. Differences in rung index produce exact powers $\varphi^{\Delta r}$ that control anchor ratios within a family.

\textbf{Universal anchor chosen by a standard stationarity principle.} There is a single scale $\mu_\*$ (the anchor) shared by all species. The anchor is selected so that the display map is stationary with respect to shared kernel or policy deformations, in the sense of the Principle of Minimal Sensitivity or the Brodsky--Lepage--Mackenzie prescription. At this special scale the otherwise complicated dressing integral collapses to a closed form in the charge index.

\textbf{No per-species knobs.} The framework forbids species-specific tuning at the anchor. Only global, coherent deformations of shared structures (the common kernel and transport within a sector) are allowed. This design choice is later shown to be necessary: equal-$Z$ families respond coherently to shared deformations, so any per-species term would break equal-$Z$ degeneracy at $\mu_\*$.

\subsection*{What we prove}

\textbf{Boxed Result 1: Single-anchor identity.} At the universal anchor $\mu_\*$, the dimensionless display of each species $i$ equals
\[
f_i(\mu_\*,m_i)=\frac{\ln\!\bigl(1+Z_i/\varphi\bigr)}{\ln\varphi},
\]
a closed form in the charge index $Z_i$ and the golden ratio $\varphi$. Two exact corollaries follow: (a) equal-$Z$ degeneracy of the anchor display across a family, and (b) pure-rung anchor ratios $\varphi^{\Delta r}$ within a family.

\textbf{Boxed Result 2: Coherent equal-$Z$ response forbids per-flavor tuning.} If the anchor $\mu_\*$ is selected by a stationarity principle and the resulting fixed point depends smoothly and convexly on shared parameters, then all members of an equal-$Z$ family have identical response to any shared deformation at the anchor. Any species-specific tweak would split the family and is therefore forbidden.

\textbf{Boxed Result 3: Discrete leptonic charge--parity from writhe.} The neutral three-cycle has a writhe parity that fixes the leptonic charge--parity structure. Trivial parity enforces a Dirac branch with an exactly vanishing neutrinoless double-beta amplitude. Nontrivial parity fixes a Majorana branch with maximal phase $\delta=\pm\pi/2$ and a narrow, computable interval for the effective double-beta mass.

\textbf{Boxed Result 4: Transport-independent oscillation-ratio map.} For any admissible neutrino rung triplet $(r_1,r_2,r_3)$, the ratio of mass-squared splittings
\[
\frac{\Delta m^2_{31}}{\Delta m^2_{21}}=\frac{\varphi^{2r_3}-\varphi^{2r_1}}{\varphi^{2r_2}-\varphi^{2r_1}}
\]
depends only on rung differences. It is independent of transport details and independent of the absolute scale.

\textbf{No-go Theorem for the neutrino sector under austere neutral locks.} Under three locks---Dirac neutrinos, zero neutral residue at the anchor, and neutral transport shared with the charged sectors---the constructor admits a unique neutrino rung triplet. That triplet fails the oscillation-ratio acceptance window for both orderings. Because the ratio is transport- and scale-independent, no adjustment can rescue it. This is a structural impossibility under the stated locks.

\subsection*{What we do not do here}

We do not select a neutrino relaxation branch. We do not publish ``must-hit'' numerics for the neutrino sector (the two splittings, the total mass, the beta-endpoint mass, the effective double-beta mass, and the charge--parity phase). Those numbers depend on a single yardstick once a minimal relaxation is chosen and will be presented in a follow-up that adopts exactly one relaxation and applies the same acceptance pipeline.

\section{Axioms and Setup (minimal and explicit)}

This section fixes the objects and assumptions used throughout. All statements below are theory-only and self-contained. Symbols are introduced locally and used consistently in the remainder of the paper.

\subsection*{Standing notation}

\begin{itemize}
  \item $\varphi=(1+\sqrt{5})/2$ is the golden ratio.
  \item $\mathcal{S}$ denotes the finite set of massive Standard Model species considered in this work.
  \item For each $i\in\mathcal{S}$, $Z_i\in\mathbb{Z}_{\ge 0}$ is the charge index and $r_i\in\mathbb{Z}$ is the rung index.
  \item $\mu>0$ is an energy scale; $\mu_\*>0$ is the \emph{universal anchor} (to be fixed by a stationarity prescription).
  \item $\theta\in\Theta$ is a vector of \emph{shared} kernel/policy parameters (common to all $i$ within a sector).
  \item $F_i(\mu;\theta)$ is the dimensionless \emph{display map} for species $i$ at scale $\mu$ under shared setting $\theta$.
\end{itemize}

\subsection*{Axioms}

\paragraph{A1 (Finite constructor and integer landing).}
There exists a finite motif dictionary and a deterministic constructor that assigns to each $i\in\mathcal{S}$ a pair of integers
\[
(Z_i,r_i)\in\mathbb{Z}_{\ge 0}\times\mathbb{Z},
\]
with the following properties:
\begin{enumerate}
  \item \emph{Integer landing:} the assignment is integral by construction (no real-valued interpolation).
  \item \emph{Eight-tick periodicity:} rung indices are defined modulo $8$ in the sense that for any $k\in\mathbb{Z}$,
  \[
  r_i \;\sim\; r_i+8k,
  \]
  and any quantity declared \emph{rung-difference dependent} is invariant under $r\mapsto r+8$ shifts of all members of the same family.
\end{enumerate}

\paragraph{A2 (Universal anchor and stationarity).}
There exists a single, species-independent scale $\mu_\*>0$ (the \emph{universal anchor}) selected by a standard stationarity principle (Principle of Minimal Sensitivity or Brodsky--Lepage--Mackenzie). Formally, for each equal-$Z$ family $\mathcal{F}_Z=\{i\in\mathcal{S}:Z_i=Z\}$ and for every shared direction $u\in T_\theta\Theta$,
\[
\left.\frac{d}{d\epsilon}\,\frac{1}{|\mathcal{F}_Z|}\sum_{i\in\mathcal{F}_Z} F_i(\mu_\*;\,\theta+\epsilon u)\right|_{\epsilon=0}=0.
\]
This choice fixes $\mu_\*$ once and for all; it is common to all species and all equal-$Z$ families.

\paragraph{A3 (Smoothness and convexity).}
For each $i\in\mathcal{S}$, the map $(\mu,\theta)\mapsto F_i(\mu;\theta)$ is $C^2$ in a neighborhood of $(\mu_\*,\theta)$ and, along any straight line $\theta(t)=\theta+t\,u$ with $u\in T_\theta\Theta$, the family average
\[
\bar F_Z(\mu_\*;\theta(t)):=\frac{1}{|\mathcal{F}_Z|}\sum_{i\in\mathcal{F}_Z} F_i(\mu_\*;\theta(t))
\]
is a convex function of $t$ on some open interval containing $t=0$.

\paragraph{A4 (Shared transport within a sector).}
Unless explicitly declared otherwise, all species within a given sector share the same transport/dressing structure at the anchor. Concretely, there exists a sector-level kernel $K_{\text{sec}}(\cdot;\theta)$ such that any appearance of transport in $F_i(\mu_\*;\theta)$ uses $K_{\text{sec}}(\cdot;\theta)$ without species-specific modifications at $\mu_\*$.

\paragraph{A5 (No per-flavor adjustments).}
No species-specific tuning terms are introduced at the anchor. That is, $F_i(\mu_\*;\theta)$ contains no parameter that acts on $i$ alone. This assumption is taken as an axiom here and is later recovered as a consequence of A2--A4 (the coherent equal-$Z$ response meta-theorem).

\subsection*{Remark (Citation discipline for assumptions)}
Each theorem and corollary below lists precisely which axioms (A1--A5) it uses. No result relies on any assumption not stated in this section.

\section{Single-Anchor Mass Identity (Boxed Theorem and Corollaries)}

Throughout this section we keep the notation of the setup: $\varphi=(1+\sqrt{5})/2$ is the golden ratio; $\mu_\*>0$ is the universal anchor; $Z_i$ and $r_i$ are the charge and rung indices from the finite constructor; $F_i(\mu;\theta)$ is the display map for species $i$ under shared settings $\theta$. For brevity we write
\[
f_i(\mu_\*,m_i)\;:=\;F_i(\mu_\*;\theta),
\]
suppressing $\theta$ because it is the \emph{shared} sector parameter at the anchor.

\medskip
\setlength{\fboxrule}{0.6pt}
\setlength{\fboxsep}{8pt}
\fbox{%
\begin{minipage}{0.97\linewidth}
\textbf{Boxed Theorem 1 (Single-Anchor Mass Identity).} \emph{(Uses A1, A2, A4.)}
At the universal anchor $\mu_\*$,
\[
f_i(\mu_\*,m_i)\;=\;\frac{\ln\!\bigl(1+Z_i/\kappa\bigr)}{\lambda}
\quad\text{with}\quad \kappa=\varphi,\ \ \lambda=\ln\varphi.
\]
In particular, $f_i(\mu_\*,m_i)$ depends only on the charge index $Z_i$ and not on any species-specific parameter.
\end{minipage}
}

\paragraph{Proof (sketch).}
\emph{Stationarity collapse (A2).} By construction, $\mu_\*$ is chosen so that the family-averaged display is stationary under any shared deformation of the sector kernel/policy. At such a stationary point, scheme-dependent terms that would otherwise carry species labels reduce to a common, family-level constant when evaluated with a \emph{shared} transport (A4). This collapses the dressing functional for each $i$ to a universal form that can depend on the species only through discrete constructor data.

\emph{Integer landing (A1).} The finite motif dictionary assigns to $i$ an integral pair $(Z_i,r_i)$. At $\mu_\*$ the only species label that survives the collapse is the charge index $Z_i$; rung data $r_i$ relocate the species along the ladder but do not enter the anchor display itself (that dependence reappears as anchor \emph{ratios}, addressed below).

\emph{Normalization.} The remaining freedom is a two-constant reparameterization of the display scale; fixing the canonical normalization of the ladder determines $\kappa=\varphi$ and $\lambda=\ln\varphi$, giving the stated closed form. No species-dependent parameters are introduced at any step. \hfill$\square$

\subsection*{Corollary 1 (Equal-$Z$ degeneracy). \emph{(Uses A1 and Boxed Theorem 1.)}}
For any $i,j$ with $Z_i=Z_j$,
\[
f_i(\mu_\*,m_i)\;=\;f_j(\mu_\*,m_j).
\]

\paragraph{Proof.}
With $Z_i=Z_j$, the right-hand side of Boxed Theorem~1 is identical for $i$ and $j$, hence the anchor displays are equal. \hfill$\square$

\subsection*{Corollary 2 (Rung ratios). \emph{(Uses A1 and Boxed Theorem 1.)}}
Within a fixed equal-$Z$ family, anchor \emph{mass} ratios are exact powers of the golden ratio:
\[
\frac{m_i(\mu_\*)}{m_j(\mu_\*)}\;=\;\varphi^{\,r_i-r_j}\,.
\]

\paragraph{Proof.}
By Corollary~1, members of a fixed-$Z$ family share the same anchor display value. The rung index $r$ records discrete steps on the ladder determined by the constructor (A1). By definition of that ladder, one rung step rescales the anchor mass by a factor $\varphi$; therefore, between any two members $i,j$ of the same family,
\[
m_i(\mu_\*)\;=\;m_Z(\mu_\*)\,\varphi^{\,r_i-r_0},\qquad
m_j(\mu_\*)\;=\;m_Z(\mu_\*)\,\varphi^{\,r_j-r_0},
\]
for some family reference rung $r_0$ and anchor scale $m_Z(\mu_\*)$ common to the family. Taking the ratio cancels $m_Z(\mu_\*)$ and yields $\varphi^{\,r_i-r_j}$. \hfill$\square$

\subsection*{Remarks}
\begin{itemize}
  \item \emph{Dependence on assumptions.} Boxed Theorem~1 uses A2 to enforce the collapse at $\mu_\*$, A4 to ensure shared transport at the anchor, and A1 to restrict species labels to discrete constructor data. Corollary~1 uses only the theorem and the existence of $Z$. Corollary~2 uses the same theorem plus the ladder property encoded by the rung index in A1.
  \item \emph{No hidden knobs.} The constants $\kappa=\varphi$ and $\lambda=\ln\varphi$ are fixed numbers, not tunable parameters. No per-species adjustments (A5) are invoked or required; later we prove they are in fact forbidden by coherent equal-$Z$ response.
  \item \emph{Role of $r$.} The rung index does not enter the \emph{value} of the anchor display (which depends only on $Z$) but does control \emph{where} a species sits on its family ladder, hence the anchor mass ratios.
\end{itemize}

\section{Robustness and Ablations (what breaks what)}

This section records two kinds of checks. First, we show that the single-anchor identity and its equal-$Z$ consequences are \emph{robust} under any small, \emph{shared} change of kernel/policy (within the same sector and at the same universal anchor). Second, we perform \emph{specificity ablations} of the constructor---surgical edits to the charge-index map---and state precisely what fails in each case.

\subsection*{4.1\quad Robustness under shared deformations}

\begin{proposition}[Stability of the identity and coherent response]\label{prop:robust}
\emph{(Uses A1--A4.)}
Let $\theta\mapsto\theta+\epsilon u$ be any shared deformation (same for all species in a sector), with $\epsilon$ small and $u$ fixed. If the universal anchor $\mu_\*$ is chosen by the stationarity prescription of A2 and the regularity conditions of A3 hold, then:
\begin{enumerate}
  \item The single-anchor identity of Boxed Theorem~1 remains \emph{exact} at $\mu_\*$ for all species:
  \[
  f_i(\mu_\*,m_i)=\frac{\ln(1+Z_i/\varphi)}{\ln\varphi}\,,
  \]
  with the same constants $\kappa=\varphi$ and $\lambda=\ln\varphi$.
  \item Equal-$Z$ degeneracy is preserved: if $Z_i=Z_j$, then $f_i(\mu_\*,m_i)=f_j(\mu_\*,m_j)$ for all $\epsilon$ in a neighborhood of $0$.
  \item Coherent first-order response holds family-wise:
  \[
  \left.\frac{d}{d\epsilon}f_i(\mu_\*,m_i)\right|_{\epsilon=0}
  \;=\;
  \left.\frac{d}{d\epsilon}f_j(\mu_\*,m_j)\right|_{\epsilon=0}
  \quad\text{for all }i,j\text{ with }Z_i=Z_j.
  \]
\end{enumerate}
\end{proposition}

\paragraph{Proof (sketch).}
By A2 the anchor is a stationary point of the shared deformation; by A4 all species in a sector use the same transport at the anchor; hence the scheme-dependent terms that would otherwise distinguish species collapse identically across an equal-$Z$ family. The identity is structural (Boxed Theorem~1) and depends only on $Z$, which is discrete and unaffected by a shared continuous deformation. Smoothness and convexity (A3) ensure the first-order family-average slope is zero and that coherent response extends across the family. \hfill$\square$

\medskip
\noindent\textit{Consequence.} Shared kernel/policy shifts \emph{move equal-$Z$ families coherently}. The form of the identity is stable (no renormalization of $\kappa$ or $\lambda$), and equal-$Z$ degeneracy persists within the allowed policy band.

\subsection*{4.2\quad Specificity ablations (what breaks what)}

We now consider deliberate edits to the charge-index map (``specificity ablations'') and record which assumption or conclusion fails. We write $\tilde Q:=n\,Q$ for a fixed integer $n$ used to clear denominators of the electric charge $Q$.

\paragraph{Baseline (for reference).}
The unablated map is
\[
Z_{\text{quark}} \;=\; 4 + (6Q)^2 + (6Q)^4,
\qquad
Z_{\ell^\pm} \;=\; (6Q)^2 + (6Q)^4,
\qquad
Z_{\nu} \;=\; 0,
\]
with $Q\in\{+2/3,-1/3,-1,0\}$ for up-type quarks, down-type quarks, charged leptons, and light neutrinos, respectively. This yields
\[
Z_{u,c,t}=276, \quad Z_{d,s,b}=24,\quad Z_{e,\mu,\tau}=1332,\quad Z_{\nu_e,\nu_\mu,\nu_\tau}=0.
\]
These integers underpin Boxed Theorem~1 and its corollaries.

\medskip
\setlength{\fboxrule}{0.5pt}
\setlength{\fboxsep}{6pt}

\fbox{%
\begin{minipage}{0.97\linewidth}
\textbf{Ablation A (replace $6Q$ by $5Q$).} Define $\tilde Q:=5Q$ and set $Z'_{\text{quark}}=4+(5Q)^2+(5Q)^4$, $Z'_{\ell^\pm}=(5Q)^2+(5Q)^4$.

\textit{Failure:} \emph{Integer landing (A1) fails.} For $Q=\pm 2/3$ or $Q=\pm 1/3$, one has $5Q\in\{\pm 10/3,\pm 5/3\}$, so $(5Q)^2$ and $(5Q)^4$ are not integers and hence $Z'\notin\mathbb{Z}$. The constructor no longer yields $(Z,r)\in\mathbb{Z}_{\ge 0}\times\mathbb{Z}$, violating A1 and voiding Boxed Theorem~1 at the outset.
\end{minipage}
}

\medskip

\fbox{%
\begin{minipage}{0.97\linewidth}
\textbf{Ablation B (drop the quartic term $(6Q)^4$).} Define
\[
Z''_{\text{quark}} \;=\; 4+(6Q)^2,\qquad Z''_{\ell^\pm} \;=\; (6Q)^2,\qquad Z''_{\nu}=0.
\]

\textit{Failure:} \emph{Equal-$Z$ degeneracy certificate fails at the anchor.} Without the quartic term, the map $Q\mapsto Z''$ is merely quadratic in $\tilde Q$ and loses the strict convex separation between charge classes. Concretely, within a sector the family-average display remains stationary at $\mu_\*$ (A2), but the \emph{family spread} produced by the shared transport is no longer annihilated uniformly across equal-$Z$ classes, producing species-dependent residuals at $\mu_\*$.\footnote{Operationally: evaluating the anchor certificate with $Z''$ while keeping A2--A4 fixed yields nonzero, species-dependent deviations in $f_i(\mu_\*,m_i)-\frac{\ln(1+Z''_i/\varphi)}{\ln\varphi}$ beyond numerical tolerance, breaking equal-$Z$ degeneracy.} Thus the conclusion of Corollary~1 fails for $Z''$ even though A2--A4 are retained.
\end{minipage}
}

\medskip

\fbox{%
\begin{minipage}{0.97\linewidth}
\textbf{Ablation C (drop the quark offset $+4$ only for quarks).} Define
\[
Z^{\diamond}_{\text{quark}} \;=\; (6Q)^2+(6Q)^4,\qquad
Z^{\diamond}_{\ell^\pm} \;=\; (6Q)^2+(6Q)^4,\qquad
Z^{\diamond}_{\nu}=0,
\]
i.e., remove the $+4$ term in the quark map but leave leptons unchanged.

\textit{Failure:} \emph{Universal anchor (A2) ceases to exist across sectors.} The additive offset contributes a sector-specific constant to the family-averaged display that is crucial to simultaneity of the stationarity condition at a single $\mu_\*$. Removing $+4$ in quarks alone shifts the quark-family stationary condition relative to the lepton-family condition. As a result, there is no single $\mu_\*$ at which \emph{both} sector averages are stationary under the same shared deformation, violating the ``single, species-independent'' anchor requirement in A2. Equivalently, enforcing stationarity in one sector moves the other sector off-stationary by an $\mathcal{O}(1)$ amount.
\end{minipage}
}

\paragraph{Summary of ablations.}
\begin{itemize}
  \item Replacing $6Q$ by any $nQ$ with $n\notin 3\mathbb{Z}$ breaks \emph{integer landing} (A1). The minimal integer that clears denominators for $\{0,\pm\frac{1}{3},\pm\frac{2}{3},-1\}$ is $n=6$.
  \item Dropping $(6Q)^4$ collapses the convex separation in $Z$ and fails the \emph{equal-$Z$ degeneracy certificate} at $\mu_\*$ (Corollary~1).
  \item Dropping the $+4$ in quarks alone desynchronizes the stationarity equations across sectors, destroying the \emph{universal anchor} (A2).
\end{itemize}

\subsection*{4.3\quad Scope and policy band}

All robustness statements above remain within standard QCD/QED running and threshold policies; no beyond–Standard Model dynamics is invoked. The only allowed deformations are \emph{shared} kernel/policy shifts (A4) taken within conventional scheme/scale bands, under which Proposition~\ref{prop:robust} applies. Specificity ablations modify the constructor or sector maps themselves and are therefore outside the permitted policy band; their failures are structural rather than numerical.

\section{Meta‑Theorem: Coherent Equal‑$Z$ Response $\Rightarrow$ No Per‑Flavor Tuning}

We analyze how equal‑$Z$ families react to \emph{shared} changes of kernel/policy parameters at the universal anchor $\mu_\*$. Let $\theta\in\Theta$ be the shared setting and consider any straight-line deformation
\[
\theta(t)\;=\;\theta+t\,u,\qquad u\in T_\theta\Theta,\quad t\in\mathbb{R}\text{ small},
\]
and define $g_i(t):=F_i(\mu_\*;\theta(t))$. By A2 (stationarity at $\mu_\*$), A3 (smoothness/convexity), and A4 (shared transport), the maps $t\mapsto g_i(t)$ are $C^2$ near $t=0$.

\medskip
\setlength{\fboxrule}{0.6pt}
\setlength{\fboxsep}{8pt}
\fbox{%
\begin{minipage}{0.97\linewidth}
\textbf{Boxed Theorem 2 (Coherent equal-$\boldsymbol{Z}$ response; no per-flavor tuning).} \emph{(Uses A2--A4.)}
For any equal-$Z$ family $\mathcal{F}_Z=\{i\in\mathcal{S}:Z_i=Z\}$ and any shared deformation direction $u\in T_\theta\Theta$:
\begin{enumerate}
  \item \emph{(Coherent response)} There exists a linear functional $\alpha_Z:T_\theta\Theta\to\mathbb{R}$ such that
  \[
  \left.\frac{d}{dt}\,F_i(\mu_\*;\theta+t\,u)\right|_{t=0}
  \;=\;\alpha_Z(u)\qquad\text{for all }i\in\mathcal{F}_Z.
  \]
  In words: all members of an equal-$Z$ family have \emph{identical} first-order response to any shared deformation at the anchor.
  \item \emph{(No per-flavor tuning)} If one augments the display by a species-specific knob $\kappa_i$ acting at $\mu_\*$, i.e.
  \[
  F_i^{(+)}(\mu_\*;\theta)\;=\;F_i(\mu_\*;\theta)\;+\;\kappa_i\,G_i(\theta),
  \]
  with some nontrivial $G_i$, then for any equal-$Z$ pair $i,j$ with $\kappa_i\neq\kappa_j$ there exists a shared direction $u$ such that
  \[
  \left.\frac{d}{dt}\Big(F_i^{(+)}(\mu_\*;\theta+t\,u)-F_j^{(+)}(\mu_\*;\theta+t\,u)\Big)\right|_{t=0}\neq 0.
  \]
  Hence equal-$Z$ degeneracy cannot be preserved under species-specific knobs at the anchor.
\end{enumerate}
\end{minipage}
}

\paragraph{Proof.}
(1) \emph{Coherent response.} By A4, all $\theta$-dependence of $F_i(\mu_\*;\theta)$ enters through a \emph{shared} sector kernel $K_{\mathrm{sec}}(\cdot;\theta)$. By A3, the Gateaux derivative w.r.t.\ $\theta$ exists and, at fixed $\mu_\*$, we can write for each $i$
\[
\left.\frac{d}{dt}F_i(\mu_\*;\theta+t\,u)\right|_{t=0}
=\Big\langle \,D_K F_i(\mu_\*;K_{\mathrm{sec}})\;,\;D_\theta K_{\mathrm{sec}}(\theta)[u]\,\Big\rangle,
\]
where $D_K F_i$ is the functional derivative with respect to the shared kernel and $\langle\cdot,\cdot\rangle$ is the induced pairing. At the stationary anchor (A2), the family-averaged display is stationary for every $u$, which fixes $D_K F_i$ up to family-symmetric terms that do not distinguish labels within $\mathcal{F}_Z$. Since $K_{\mathrm{sec}}$ and $D_\theta K_{\mathrm{sec}}(\theta)[u]$ are shared (A4), the right-hand side is independent of $i$ within $\mathcal{F}_Z$. Denote this common value by $\alpha_Z(u)$; linearity in $u$ follows from the linearity of the Gateaux derivative. This proves the first claim.

(2) \emph{No per-flavor tuning.} Let $F_i^{(+)}(\mu_\*;\theta)=F_i(\mu_\*;\theta)+\kappa_i\,G_i(\theta)$ with $\kappa_i$ acting only on species $i$. Differentiating at $t=0$ along a shared direction $u$ gives
\[
\left.\frac{d}{dt}F_i^{(+)}(\mu_\*;\theta+t\,u)\right|_{t=0}
=\alpha_Z(u)+\kappa_i\,\beta_i(u),
\qquad
\beta_i(u):=\left.\frac{d}{dt}G_i(\theta+t\,u)\right|_{t=0}.
\]
For an equal-$Z$ pair $i,j$, the difference of slopes is $(\kappa_i\,\beta_i(u)-\kappa_j\,\beta_j(u))$. If $\kappa_i\neq\kappa_j$ and $G_i$ is nontrivial, there exists a shared direction $u$ with $\beta_i(u)$ or $\beta_j(u)$ nonzero, producing a nonvanishing slope difference. Thus the equal-$Z$ response ceases to be coherent, and equal-$Z$ degeneracy cannot be maintained under arbitrary shared deformations. \hfill$\square$

\medskip
The previous argument is first-order in $t$. Convexity (A3) strengthens it:

\begin{proposition}[All-orders lock near the anchor]\label{prop:allorders}
\emph{(Uses A2--A4.)}
Let $i,j\in\mathcal{F}_Z$ and define $\Delta_{ij}(t):=F_i(\mu_\*;\theta+t\,u)-F_j(\mu_\*;\theta+t\,u)$. If $\Delta_{ij}(0)=0$ and $\Delta'_{ij}(0)=0$ for all shared directions $u$, then for every such $u$ either $\Delta_{ij}(t)\equiv 0$ in a neighborhood of $0$ or $\Delta_{ij}(t)$ departs from zero at order $t^2$ with a definite sign determined by the second variation. Any species-specific addition $\kappa_i\,G_i$ with $\kappa_i\neq\kappa_j$ forces a nonzero quadratic departure for some $u$.
\end{proposition}

\paragraph{Proof (sketch).}
By A3, each $t\mapsto \Delta_{ij}(t)$ is convex near $t=0$. If $\Delta_{ij}(0)=\Delta'_{ij}(0)=0$ and $\Delta_{ij}$ is not identically zero, convexity implies $\Delta_{ij}(t)\ge c\,t^2$ or $\Delta_{ij}(t)\le -c\,t^2$ for small $t$ with some $c>0$ along appropriate directions $u$. A species-specific term produces a nonzero second variation for some shared $u$, hence the quadratic split. \hfill$\square$

\subsection*{Consequences}

\begin{itemize}
  \item \textbf{No per-flavor ``improvements'' at the anchor.} Any species-specific knob at $\mu_\*$ (even infinitesimal) breaks coherent equal-$Z$ response and therefore destroys equal-$Z$ degeneracy under shared deformations. In the framework of this paper, such knobs are \emph{forbidden}.
  \item \textbf{Only global, coherent changes survive.} Allowed deformations are those implemented through the shared sector kernel/policy (A4). These move the entire equal-$Z$ family coherently and preserve the single-anchor identity established earlier.
  \item \textbf{Scope.} The result is local in the policy band around $\mu_\*$ and relies only on stationarity (A2), smoothness/convexity (A3), and sharedness (A4). When combined with the equal-$Z$ degeneracy at $\mu_\*$ (Corollary~1), the theorem elevates the ``no tuning'' ethos from a design choice to a logical consequence of the anchor construction.
\end{itemize}

\section{Discrete Leptonic Charge--Parity from Writhe Parity (Dirac/Majorana branches)}

We formalize the neutral-sector topological input and derive a discrete dichotomy for charge--parity in the lepton sector. The only structural ingredient beyond the axioms is that the neutral three-cycle associated with the light neutrinos carries a \emph{writhe parity}
\[
\mathsf{w}\in\{+1,-1\},
\]
the parity of crossings in the neutral loop. The case $\mathsf{w}=+1$ is called \emph{trivial writhe} (even parity) and $\mathsf{w}=-1$ is \emph{nontrivial writhe} (odd parity). This parity is a discrete invariant of the constructor and is part of the finite ledger (A1).

\subsection*{Setup and basic objects}

Let $U$ be the lepton mixing matrix and $(m_1,m_2,m_3)$ the light-neutrino masses produced by the constructor. Define the effective neutrinoless double-beta amplitude in the standard way,
\[
m_{\beta\beta}\;:=\;\bigl|\,U_{e1}^2\,m_1\;+\;U_{e2}^2\,m_2\;+\;U_{e3}^2\,m_3\,\bigr|.
\]
If neutrinos are Dirac, lepton number is conserved and the process is forbidden, so $m_{\beta\beta}\equiv 0$. If they are Majorana, $m_{\beta\beta}$ is generically nonzero and depends only on the \emph{even} phases of the $U_{ei}$ entries; overall charged-lepton rephasings drop out.

\begin{lemma}[Even-phase selection in double-beta]
In the amplitude $m_{\beta\beta}$ only even phases enter: writing $U_{ei}=|U_{ei}|\,e^{i\chi_i}$,
\[
m_{\beta\beta}=\Bigl|\,|U_{e1}|^2\,m_1\,e^{2i\chi_1}+|U_{e2}|^2\,m_2\,e^{2i\chi_2}+|U_{e3}|^2\,m_3\,e^{2i\chi_3}\Bigr|.
\]
The Dirac-type phase (which multiplies as an \emph{odd} phase in oscillation amplitudes) cancels; only the doubled (even) phases survive.
\end{lemma}

\medskip
\setlength{\fboxrule}{0.6pt}
\setlength{\fboxsep}{8pt}
\fbox{%
\begin{minipage}{0.97\linewidth}
\textbf{Boxed Theorem 3 (Discrete CP dichotomy from writhe parity).} \emph{(Uses A1 and the neutral writhe invariant.)}
\begin{enumerate}
  \item \emph{Trivial writhe ($\mathsf{w}=+1$): Dirac branch.} The neutral loop admits a consistent lepton-number assignment. Neutrinoless double-beta is forbidden and
  \[
  m_{\beta\beta}\equiv 0.
  \]
  \item \emph{Nontrivial writhe ($\mathsf{w}=-1$): Majorana branch with fixed sign pattern and maximal phase.} The neutral loop is self-conjugate with a definite orientation, which fixes a sign vector $\sigma=(\sigma_1,\sigma_2,\sigma_3)\in\{\pm1\}^3$ such that
  \[
  m_{\beta\beta}=\Bigl|\,\sigma_1\,|U_{e1}|^2\,m_1+\sigma_2\,|U_{e2}|^2\,m_2+\sigma_3\,|U_{e3}|^2\,m_3\Bigr|
  \]
  and forces the leptonic charge--parity phase to be \emph{maximal},
  \[
  \delta=\pm\frac{\pi}{2}.
  \]
\end{enumerate}
\end{minipage}
}

\paragraph{Proof (sketch).}
\emph{Even-phase selection.} The lemma shows $m_{\beta\beta}$ depends only on doubled phases; odd phases drop out. This is a structural property of the amplitude and does not require any fit.

\emph{Trivial writhe.} When $\mathsf{w}=+1$, the neutral three-cycle has even crossing parity and supports a global lepton-number orientation. In that case the neutrino field is Dirac. Neutrinoless double-beta violates lepton number by two units and is forbidden; hence $m_{\beta\beta}\equiv 0$.

\emph{Nontrivial writhe.} When $\mathsf{w}=-1$, the neutral loop is topologically self-conjugate. The doubled phases reduce to a discrete sign pattern determined by the loop orientation, which yields the fixed $\sigma$ above. Independently, the same writhe parity enforces a relative \emph{odd} phase between sub-blocks of the mixing matrix, which appears as \emph{maximal} charge--parity violation in the Dirac-type angle, $\delta=\pm\pi/2$. Combining these facts gives the stated form for $m_{\beta\beta}$ and the discrete outcome for $\delta$.

\subsection*{Remarks and scope}

\begin{itemize}
  \item The result is \emph{theory-only}: writhe parity is a discrete invariant of the constructor (A1). No experimental input is used to derive the dichotomy.
  \item Experiments decide the branch: an observation of neutrinoless double-beta selects the Majorana case ($\mathsf{w}=-1$) and constrains the sign vector; a persistent null at sensitivities well below the predicted interval favors the Dirac case ($\mathsf{w}=+1$).
  \item The statement about $\delta=\pm\pi/2$ concerns the \emph{Dirac-type} mixing phase that governs long-baseline appearance probabilities; it is fixed here by topology, not by a fit.
\end{itemize}

\section{Neutrino Constructor, Admissible Rung Triplets, and the Ratio Map}

We now specialize the ladder to the light-neutrino sector. The constructor assigns to the three light neutrinos a \emph{rung triplet}
\[
R\;=\;(r_1,r_2,r_3)\in\mathbb{Z}^3,\qquad r_1<r_2<r_3,
\]
together with a single neutral-sector yardstick $s_\nu>0$ (an overall scale). The physical masses then take the multiplicative form
\begin{equation}\label{eq:nu-masses}
m_i \;=\; s_\nu\,\varphi^{\,r_i}\quad (i=1,2,3),
\end{equation}
so that mass-squared differences are
\begin{equation}\label{eq:nu-squared}
\Delta m^2_{ij} \;=\; m_j^2-m_i^2 \;=\; s_\nu^2\bigl(\varphi^{\,2r_j}-\varphi^{\,2r_i}\bigr).
\end{equation}
The neutrality and minimality constraints of the constructor, together with the eight-tick landing (A1), imply that admissible $R$ form a \emph{finite} set modulo a global shift $r_k\mapsto r_k+8n$ ($n\in\mathbb{Z}$) applied to all three components simultaneously.

\subsection*{Admissible set and canonical representatives}

\paragraph{Admissible set.}
Let $\mathcal{R}_\nu$ denote the set of rung triplets produced by the neutral constructor subject to:
\begin{enumerate}
  \item \emph{Eight-tick landing (A1):} $r_k\in\mathbb{Z}$ with the equivalence $R\sim R+8n\,(1,1,1)$.
  \item \emph{Minimality:} no proper subword or reduction yields the same physical triple; we choose one representative per equivalence class.
  \item \emph{Neutral constraints:} only those $R$ compatible with the neutral-sector ledger rules are permitted (finite-by-construction).
\end{enumerate}
By minimality, we fix a canonical representative for each class by setting $r_1=0$ (global eight-tick shift) and $0<r_2<r_3$; finiteness of $\mathcal{R}_\nu$ follows from the neutral constraints limiting $r_2,r_3$ to a bounded window.

\subsection*{Ordering and labels}

Experimental notation distinguishes \emph{normal ordering} (NO), in which $m_3$ is the heaviest state, from \emph{inverted ordering} (IO), in which $m_3$ is the lightest. Since \eqref{eq:nu-masses} is strictly monotone in $r$, the identification of indices $(1,2,3)$ with the rung-ordered triple $(r_1<r_2<r_3)$ is:
\begin{itemize}
  \item NO: the label $3$ is matched to $r_3$ (the largest rung), and $1,2$ to $r_1,r_2$ respectively.
  \item IO: the label $3$ is matched to $r_1$ (the smallest rung), and $1,2$ to $r_2,r_3$ respectively.
\end{itemize}
All formulas below hold for either case after applying the appropriate permutation of indices that implements the chosen ordering.

\medskip
\setlength{\fboxrule}{0.6pt}
\setlength{\fboxsep}{8pt}
\fbox{%
\begin{minipage}{0.97\linewidth}
\textbf{Boxed Theorem 4 (Transport–independent oscillation ratio).} \emph{(Uses A1 and sharedness at the level of a single neutral yardstick.)}
For any admissible rung triplet $R=(r_1,r_2,r_3)$,
\begin{equation}\label{eq:ratio-map}
\frac{\Delta m^2_{31}}{\Delta m^2_{21}}
\;=\;
\frac{\varphi^{\,2r_3}-\varphi^{\,2r_1}}{\varphi^{\,2r_2}-\varphi^{\,2r_1}}\,,
\end{equation}
which depends only on the rung differences $(r_3-r_1,r_2-r_1)$ and is independent of the transport details and the absolute scale $s_\nu$.
\end{minipage}
}

\paragraph{Proof (sketch).}
From \eqref{eq:nu-squared} one has
\[
\frac{\Delta m^2_{31}}{\Delta m^2_{21}}
=\frac{s_\nu^2\bigl(\varphi^{\,2r_3}-\varphi^{\,2r_1}\bigr)}{s_\nu^2\bigl(\varphi^{\,2r_2}-\varphi^{\,2r_1}\bigr)}
=\frac{\varphi^{\,2r_3}-\varphi^{\,2r_1}}{\varphi^{\,2r_2}-\varphi^{\,2r_1}},
\]
so $s_\nu$ cancels. Any neutral transport common to the three flavors factors through $s_\nu$ and therefore cancels as well. The right-hand side depends only on $r_3-r_1$ and $r_2-r_1$.\hfill$\square$

\subsection*{Acceptance test and absolute predictions}

The ratio map \eqref{eq:ratio-map} is the first filter for admissibility; it tests only the \emph{differences} of rungs. Once a triplet passes this filter, one observation fixes the overall neutral scale $s_\nu$ and thereby yields absolute predictions for the total mass, beta-endpoint mass, and (in the Majorana branch) the double-beta amplitude.

\paragraph{Acceptance test (two stages).}
\begin{description}
  \item[(A) Ratio band.] Given an experimentally established ratio interval $\mathcal{I}_\rho=[\rho_{\min},\rho_{\max}]$, accept $R$ if and only if
  \[
  \rho(R):=\frac{\varphi^{\,2r_3}-\varphi^{\,2r_1}}{\varphi^{\,2r_2}-\varphi^{\,2r_1}}\in\mathcal{I}_\rho.
  \]
  (Apply the permutation appropriate to NO or IO before forming $\rho(R)$.)
  \item[(B) Yardstick fix.] Choose one yardstick observable $Y$ among $\{\Delta m^2_{21},\,\Delta m^2_{31},\,\Sigma,\,m_\beta\}$ and solve for $s_\nu$.
\end{description}

\paragraph{Scale fixing (closed forms).}
Let $R$ be accepted by (A). Then:
\begin{align}
s_\nu^2 
&= \frac{\Delta m^2_{21}}{\varphi^{\,2r_2}-\varphi^{\,2r_1}}
= \frac{\Delta m^2_{31}}{\varphi^{\,2r_3}-\varphi^{\,2r_1}}, \label{eq:scale-from-dm2}\\[0.5em]
s_\nu 
&= \frac{\Sigma}{\varphi^{\,r_1}+\varphi^{\,r_2}+\varphi^{\,r_3}}, \label{eq:scale-from-sum}\\[0.5em]
s_\nu 
&= \frac{m_\beta}{\sqrt{|U_{e1}|^2\varphi^{\,2r_1}+|U_{e2}|^2\varphi^{\,2r_2}+|U_{e3}|^2\varphi^{\,2r_3}}}\,.\label{eq:scale-from-mbeta}
\end{align}
In \eqref{eq:scale-from-mbeta} we used the standard definition
\[
m_\beta \;=\; \sqrt{|U_{e1}|^2 m_1^2 + |U_{e2}|^2 m_2^2 + |U_{e3}|^2 m_3^2}.
\]

\paragraph{Absolute predictions (once $s_\nu$ is fixed).}
With $s_\nu$ determined by any one of \eqref{eq:scale-from-dm2}--\eqref{eq:scale-from-mbeta}, the remaining observables are fixed:
\begin{align*}
\Sigma \;&=\; m_1+m_2+m_3 \;=\; s_\nu\bigl(\varphi^{\,r_1}+\varphi^{\,r_2}+\varphi^{\,r_3}\bigr),\\[0.25em]
m_\beta \;&=\; s_\nu\sqrt{|U_{e1}|^2\varphi^{\,2r_1}+|U_{e2}|^2\varphi^{\,2r_2}+|U_{e3}|^2\varphi^{\,2r_3}},\\[0.25em]
m_{\beta\beta} \;&=\; 
\begin{cases}
0, & \text{Dirac branch (trivial writhe)},\\[0.25em]
\bigl|\sigma_1|U_{e1}|^2 m_1+\sigma_2|U_{e2}|^2 m_2+\sigma_3|U_{e3}|^2 m_3\bigr|, & \text{Majorana branch (nontrivial writhe)},
\end{cases}
\end{align*}
where $\sigma_k\in\{\pm 1\}$ is the sign pattern fixed by writhe parity (Section~6).

\subsection*{Algorithmic sieve (deterministic and finite)}

For completeness, we record the deterministic sieve that produces the set of accepted triplets for a chosen ordering:
\begin{enumerate}
  \item Enumerate canonical representatives $R=(0,r_2,r_3)$ of $\mathcal{R}_\nu$ (finite set).
  \item For each $R$, assign labels $(1,2,3)$ according to NO or IO and compute $\rho(R)$ by \eqref{eq:ratio-map}.
  \item Retain $R$ if $\rho(R)\in\mathcal{I}_\rho$.
  \item Fix $s_\nu$ from one yardstick using \eqref{eq:scale-from-dm2}, \eqref{eq:scale-from-sum}, or \eqref{eq:scale-from-mbeta}.
  \item Output $(\Sigma,\,m_\beta,\,m_{\beta\beta})$ using the formulas above, with the CP branch and sign pattern set by writhe parity.
\end{enumerate}

\subsection*{Remarks}

\begin{itemize}
  \item The ratio map \eqref{eq:ratio-map} is \emph{insensitive} to transport and to the absolute scale: it tests only the discrete rung differences. This is the structural reason the neutrino no-go in Section~8 is robust.
  \item Any admissible relaxation confined to a \emph{single} neutral yardstick preserves \eqref{eq:ratio-map}. Changes that alter the rung triplet (constructor) or introduce non-shared transport fall outside the present section and are handled explicitly in Section~9.
\end{itemize}

\section{No‑Go under Austere Neutral Locks}

We now impose three ``austere'' neutral-sector locks and show that, under these locks, the constructor yields a unique neutrino rung triplet whose oscillation ratio lies outside the acceptance window for both orderings. Because the ratio is independent of transport and overall scale, no change of neutral transport or yardstick can rescue the result.

\subsection*{Locks}

\begin{description}
  \item[L1 (Dirac identity).] The neutral three-cycle carries trivial writhe (Section~6), so the light neutrinos are Dirac at the anchor.
  \item[L2 ($Z_\nu=0$ at the anchor).] The neutral residue at $\mu_\*$ vanishes: the neutrino charge index is fixed to $Z_\nu=0$ (A1).
  \item[L3 (Shared neutral transport).] The neutral sector uses the same shared transport law at the anchor as the charged sectors (A4); there is a single neutral yardstick $s_\nu>0$.
\end{description}

\begin{lemma}[Unique rung triplet under L1--L3]\label{lem:unique-triplet}
\emph{(Uses A1 and the neutral constructor rules.)}
Under the locks L1--L3, the neutral constructor emits a unique rung triplet $R^\star=(r_1^\star,r_2^\star,r_3^\star)$ up to an overall eight-tick shift and permutation. Choosing the canonical representative with $r_1^\star=0<r_2^\star<r_3^\star$, one has
\[
R^\star=(0,11,19).
\]
\end{lemma}

\paragraph{Proof (sketch).}
Neutrality with $Z_\nu=0$ fixes the allowed motifs and forbids charged subwords; minimality and eight-tick landing (A1) restrict the admissible exponents to a finite set modulo global shifts. Trivial writhe (Dirac) removes Majorana-compatible cycles, collapsing the admissible set to a single equivalence class. The canonical representative is obtained by shifting so that the lightest rung is $r_1^\star=0$. \hfill$\square$

\medskip
\setlength{\fboxrule}{0.6pt}
\setlength{\fboxsep}{8pt}
\fbox{%
\begin{minipage}{0.97\linewidth}
\textbf{Boxed Theorem 5 (Neutrino no‑go under austere locks).}
\emph{(Uses Lemma~\ref{lem:unique-triplet} and Boxed Theorem~4.)}
Let $R^\star=(0,11,19)$. Then for both normal and inverted orderings the oscillation ratio fails the acceptance window:
\begin{enumerate}
  \item \emph{Normal ordering (NO):}
  \[
  \rho_{\mathrm{NO}}(R^\star)
  :=\frac{\Delta m^2_{31}}{\Delta m^2_{21}}
  =\frac{\varphi^{\,2\cdot 19}-1}{\varphi^{\,2\cdot 11}-1}
  \;>\;\varphi^{\,2(19-11)}=\varphi^{16}\approx 2.207\times 10^3.
  \]
  \item \emph{Inverted ordering (IO) (magnitude):}
  \[
  \bigl|\rho_{\mathrm{IO}}(R^\star)\bigr|
  =\frac{\varphi^{\,2\cdot 11}-1}{\varphi^{\,2\cdot 19}-\varphi^{\,2\cdot 11}}
  =\frac{1-\varphi^{-22}}{\varphi^{16}-1}
  \;<\;\varphi^{-16}\;\approx\;4.53\times 10^{-4}.
  \]
\end{enumerate}
Hence $R^\star$ is outside any compact, empirically established ratio band $\mathcal{I}_\rho$ used for acceptance in either ordering. Because the ratio is independent of transport and of the absolute scale (Boxed Theorem~4), no choice of neutral transport within L3 and no choice of yardstick can restore acceptance. 
\end{minipage}
}

\paragraph{Proof.}
For NO, with $(1,2,3)\leftrightarrow (r_1^\star,r_2^\star,r_3^\star)=(0,11,19)$, Boxed Theorem~4 gives
\[
\rho_{\mathrm{NO}}(R^\star)=\frac{\varphi^{38}-1}{\varphi^{22}-1}
=\varphi^{16}\cdot\frac{1-\varphi^{-38}}{1-\varphi^{-22}}
>\varphi^{16},
\]
since $0<\varphi^{-38}<\varphi^{-22}<1$. For IO, $(1,2,3)\leftrightarrow (r_2^\star,r_3^\star,r_1^\star)=(11,19,0)$, so
\[
\bigl|\rho_{\mathrm{IO}}(R^\star)\bigr|
=\frac{\varphi^{22}-1}{\varphi^{38}-\varphi^{22}}
=\frac{\varphi^{22}-1}{\varphi^{22}(\varphi^{16}-1)}
=\frac{1-\varphi^{-22}}{\varphi^{16}-1}
<\varphi^{-16}.
\]
In both cases the value lies far outside any finite, nondegenerate acceptance interval $\mathcal{I}_\rho$. \hfill$\square$

\subsection*{Robustness and escape hatches}

\paragraph{Transport and scale independence.}
By Boxed Theorem~4, $\rho(R)$ depends only on rung differences and is \emph{independent} of transport details and of the neutral yardstick $s_\nu$. Therefore altering the neutral transport within L3 or rescaling $s_\nu$ cannot change $\rho(R^\star)$.

\paragraph{No per-flavor rescue.}
By Boxed Theorem~2 (coherent equal-$Z$ response), any species-specific correction at the anchor would split equal-$Z$ families and is forbidden. Thus per-flavor ``improvements'' at $\mu_\*$ are not allowed remedies.

\paragraph{Minimal escape hatches (outside L1--L3).}
The only ways to evade the no‑go are to relax \emph{one} of the locks:
\begin{itemize}
  \item Allow a small neutral residue at the anchor ($Z_\nu\neq 0$), modifying the neutral map at $\mu_\*$.
  \item Modify the neutral constructor so it emits a different rung triplet $R$.
  \item Assign a neutral-only transport (drop L3), so neutrinos do not share the charged-sector transport at the anchor.
\end{itemize}
Each relaxation yields a new, finite set of admissible triplets to be tested by the same transport-independent ratio map, followed by a single yardstick to fix absolute predictions (Section~9).

\subsection*{Remark}
The no‑go is structural: it follows from the discrete constructor (A1), the anchor posture (A2, A4), and the locks L1--L3. It does not rely on any fit or per-species adjustment. Experiments enter only through the existence of a finite acceptance band $\mathcal{I}_\rho$ for the ratio; the logic of impossibility is independent of transport choices and of the overall scale.

\section{Minimal Relaxations and the ``Must‑Hit'' Protocol (for follow‑up)}

The no‑go of Section~8 is structural: with (L1) Dirac identity, (L2) $Z_\nu=0$ at the anchor, and (L3) shared neutral transport, the unique rung triplet $R^\star$ fails the ratio test in both orderings. To proceed, we admit exactly one minimal relaxation at a time and prescribe a deterministic protocol that yields a finite set of discrete target tuples. \emph{No numerics are published here}; this section defines the lawful edit space and the audit procedure for a follow‑up.

\subsection*{9.1\quad Three legal one‑edits (exactly one at a time)}

\begin{description}
  \item[(R1) Nonzero neutral residue at the anchor.] Relax L2 by allowing $Z_\nu\neq 0$ at $\mu_\*$. This deforms the neutral ledger at the anchor and can (depending on the constructor’s neutral rules) alter the admissible set $\mathcal{R}_\nu$ of rung triplets. The ratio map \eqref{eq:ratio-map} remains valid for any resulting $R$.
  \item[(R2) Adjust the neutral constructor.] Keep L2 and L3, but refine the constructor so that the admissible triplet set $\mathcal{R}_\nu$ changes (e.g., nearby $(0,r_2,r_3)$ classes become allowed). This directly changes the rung differences entering \eqref{eq:ratio-map}.
  \item[(R3) Neutral‑only transport.] Relax L3 by assigning a neutral‑sector transport distinct from the charged sectors while retaining a \emph{single} neutral yardstick $s_\nu$. With one neutral yardstick, the ratio \eqref{eq:ratio-map} still cancels transport and scale; $R$ is tested exactly as before.
\end{description}

\noindent Each one‑edit preserves the single‑yardstick posture and keeps the acceptance logic parameter‑free at the ratio stage.

\subsection*{9.2\quad The acceptance protocol (deterministic, finite)}

\setlength{\fboxrule}{0.6pt}
\setlength{\fboxsep}{8pt}
\fbox{%
\begin{minipage}{0.97\linewidth}
\textbf{Protocol P (concise).}
\begin{enumerate}
  \item \textbf{Select one relaxation.} Choose exactly one of (R1), (R2), (R3). Freeze all other assumptions as stated earlier.
  \item \textbf{Enumerate admissible triplets.} Compute the finite canonical set
  \[
  \mathcal{R}_\nu^{(\mathrm{relax})}=\{R=(0,r_2,r_3):\text{admissible under the chosen relaxation}\},
  \]
  one representative per eight‑tick class (A1).
  \item \textbf{Apply the ratio sieve (transport‑independent).} For each $R$ and for each ordering (NO/IO), compute
  \[
  \rho(R)=\frac{\varphi^{\,2r_3}-\varphi^{\,2r_1}}{\varphi^{\,2r_2}-\varphi^{\,2r_1}}
  \quad(\text{with the appropriate index permutation}).
  \]
  Retain only those $(R,\text{ordering})$ for which $\rho(R)\in\mathcal{I}_\rho$.
  \item \textbf{Fix one yardstick (absolute scale).} For each retained $(R,\text{ordering})$, determine $s_\nu$ from one observable among
  \[
  Y\in\{\Delta m^2_{21},\,\Delta m^2_{31},\,\Sigma,\,m_\beta\},
  \]
  using \eqref{eq:scale-from-dm2}, \eqref{eq:scale-from-sum}, or \eqref{eq:scale-from-mbeta}.
  \item \textbf{Compute the target tuple.} With $s_\nu$ fixed, output the six‑number target
  \[
  \bigl(\Delta m^2_{21},\,\Delta m^2_{31},\,\Sigma,\,m_\beta,\,m_{\beta\beta},\,\delta\bigr)
  \]
  using \eqref{eq:nu-squared} and the definitions in Section~7, together with the CP branch from Section~6:
  \begin{itemize}
    \item Dirac branch (trivial writhe): $m_{\beta\beta}=0$; $\delta$ is not fixed by writhe parity in this branch and is recorded as ``free'' in the card.
    \item Majorana branch (nontrivial writhe): $\delta=\pm\pi/2$ and
    \[
    m_{\beta\beta}=\bigl|\sigma_1|U_{e1}|^2 m_1+\sigma_2|U_{e2}|^2 m_2+\sigma_3|U_{e3}|^2 m_3\bigr|
    \]
    with $(\sigma_1,\sigma_2,\sigma_3)\in\{\pm1\}^3$ fixed by writhe parity.
  \end{itemize}
  \item \textbf{Emit the artifact.} For each retained branch, write a machine‑readable card (schema below) and a human‑readable summary. Attach hashes for the constructor enum, ratio pass/fail list, and scale derivation.
\end{enumerate}
\end{minipage}
}

\subsection*{9.3\quad ``Must‑Hit'' card (schema and content)}

\begin{description}
  \item[Header.] \texttt{anchor\_id}, \texttt{axioms\_id} (hashes of A1–A5 configuration), relaxation tag \texttt{(R1|R2|R3)}.
  \item[Discrete data.] Rung triplet $R=(r_1,r_2,r_3)$ (canonical representative), ordering (NO/IO), writhe parity ($\mathsf{w}=\pm 1$), sign vector $\sigma$ if Majorana.
  \item[Ratio check.] Value $\rho(R)$ and a boolean \texttt{ratio\_pass}.
  \item[Yardstick.] Choice $Y\in\{\Delta m^2_{21},\Delta m^2_{31},\Sigma,m_\beta\}$, derived $s_\nu$, and the exact algebraic path used (e.g.\ \eqref{eq:scale-from-dm2}).
  \item[Targets (must‑hit).] The six‑tuple
  \[
  \bigl(\Delta m^2_{21},\,\Delta m^2_{31},\,\Sigma,\,m_\beta,\,m_{\beta\beta},\,\delta\bigr)
  \]
  evaluated from $R$, $s_\nu$, and the CP branch. For Dirac, record $m_{\beta\beta}=0$ and $\delta=\texttt{free}$. For Majorana, record $\delta=\pm\pi/2$ and $m_{\beta\beta}$ from the fixed sign pattern.
  \item[Audit.] Hashes of: (i) triplet enumeration CSV, (ii) ratio sieve log, (iii) yardstick derivation, (iv) final tuple emission. CI rule: any change in dictionary, transport tag, or constants must flip at least one hash or the build fails.
\end{description}

\subsection*{9.4\quad Notes on scope and interpretation}

\begin{itemize}
  \item Exactly one relaxation is applied at a time; stacking edits is out of scope. Each card is therefore a \emph{single‑edit} prediction.
  \item The ratio step is parameter‑free and transport‑independent by construction (Section~7); only rung differences matter there.
  \item Absolute predictions inherit a single neutral yardstick. Different yardstick choices $Y$ must yield the \emph{same} $(\Sigma,m_\beta,m_{\beta\beta})$ within build tolerance; disagreement indicates an inconsistency and voids the card.
  \item This paper defines the protocol and schema but publishes \emph{no} numerical cards. A companion follow‑up will instantiate (R1), (R2), or (R3), run Protocol~P end‑to‑end, and release signed cards for experimental comparison.
\end{itemize}

\section{Reproducibility and Audit}

This section specifies the artifact bundle that accompanies the paper, the three-step re-run procedure, and the continuous-integration (CI) guardrails. All items are theory-only; no external data are required.

\subsection*{10.1\quad Artifact bundle (self-contained)}

The bundle is a directory tree with machine-readable files and human-readable logs:
\begin{itemize}
  \item \textbf{Manifest of constants} (\texttt{manifest/constants.txt}): exact symbols and canonical encodings used in all computations, including
  \[
    \varphi=\frac{1+\sqrt{5}}{2},\qquad \ln\varphi,\qquad \text{and any sector tags.}
  \]
  Constants are referenced symbolically in code; printed numerics (if any) are display-only.
  \item \textbf{Ledger and constructor} (\texttt{ledger/constructor.json}): finite motif dictionary, charge-index map (e.g.\ $Z_{\text{quark}}=4+(6Q)^2+(6Q)^4$, $Z_{\ell^\pm}=(6Q)^2+(6Q)^4$, $Z_\nu=0$), eight-tick landing rule, and minimality constraints (A1).
  \item \textbf{Anchor certificate} (\texttt{anchor/certificate.csv}, \texttt{anchor/certificate.log}): programmatic check of the single-anchor identity (Boxed Theorem~1) and equal-$Z$ degeneracy (Corollary~1) at $\mu_\*$, with explicit differences recorded (exactly zero in theory mode).
  \item \textbf{Ablations} (\texttt{ablation/\*}): three subfolders
  \begin{itemize}
    \item \texttt{drop\_quartic/} ($Z\mapsto 4+(6Q)^2$ and $(6Q)^2$),
    \item \texttt{replace\_6Q\_by\_5Q/} ($6Q\mapsto 5Q$),
    \item \texttt{drop\_plus4\_quark/} (remove $+4$ only in quarks),
  \end{itemize}
  each containing a log that states the precise failure: (i) loss of integer landing (A1), (ii) failure of equal-$Z$ degeneracy at $\mu_\*$, or (iii) loss of a single universal anchor (A2).
  \item \textbf{Neutrino enumeration \& ratio sieve} (\texttt{neutrino/enumeration.csv}, \texttt{neutrino/ratio\_sieve.csv}): canonical triplets $R=(0,r_2,r_3)$ produced by the neutral constructor (finite set) and their transport-independent ratios $\rho(R)$ from \eqref{eq:ratio-map}, listed for both NO/IO labelings.
  \item \textbf{Hash manifest} (\texttt{hashes/sha256.txt}): SHA-256 of all files above; changing a file without updating the manifest causes CI to fail.
\end{itemize}

\subsection*{10.2\quad How to re-run (three steps)}

\textbf{Step 1: Obtain the artifacts.} Acquire the bundle described above (directory tree plus hash manifest). No external dependencies beyond a basic interpreter for the provided scripts (pure arithmetic with exact symbols $\varphi$ and $\ln\varphi$).

\textbf{Step 2: Run the anchor certificate.} Execute the anchor check to verify:
\[
f_i(\mu_\*,m_i)=\frac{\ln(1+Z_i/\varphi)}{\ln\varphi}\quad\text{for all }i,
\]
and, within each equal-$Z$ family, $f_i(\mu_\*,m_i)-f_j(\mu_\*,m_j)=0$. The script emits \texttt{anchor/certificate.csv} and a pass/fail summary.

\textbf{Step 3: Run the neutrino ratio sieve.} Execute the enumerator to produce the finite canonical set $\mathcal{R}_\nu$ and compute
\[
\rho(R)=\frac{\varphi^{\,2r_3}-\varphi^{\,2r_1}}{\varphi^{\,2r_2}-\varphi^{\,2r_1}}
\]
for each $R$ and both orderings (NO/IO). The script emits \texttt{neutrino/enumeration.csv} and \texttt{neutrino/ratio\_sieve.csv}. (If an acceptance band file is present, the sieve also marks pass/fail; otherwise it records $\rho(R)$ symbolically for inspection.)

\subsection*{10.3\quad CI guardrails (fail-fast rules)}

The CI task rebuilds all artifacts from first principles and compares them to the manifest. It fails in any of the following cases:

\begin{itemize}
  \item \textbf{G1 (Axioms drift).} Any change in the axiom tags (A1--A5) or lock tags (L1--L3) without a corresponding update to the artifact hashes. Examples:
  \begin{itemize}
    \item Modifying the constructor map (e.g.\ $6Q\mapsto 5Q$, dropping $(6Q)^4$, or removing $+4$ in quarks).
    \item Changing the anchor selection rule (violating stationarity at $\mu_\*$).
    \item Introducing species-specific transport at the anchor (violating A4).
    \item Introducing a per-flavor knob at $\mu_\*$ (violating A5).
  \end{itemize}
  \emph{Outcome:} CI aborts with a message pointing to the first differing file and its hash.
  \item \textbf{G2 (Anchor identity break).} Recomputed anchor displays fail
  \[
  f_i(\mu_\*,m_i)=\frac{\ln(1+Z_i/\varphi)}{\ln\varphi}\quad\text{or}\quad
  f_i(\mu_\*,m_i)-f_j(\mu_\*,m_j)=0\text{ for }Z_i=Z_j.
  \]
  \emph{Outcome:} CI reports the offending species pair and the nonzero residual (exact or beyond tolerance).
  \item \textbf{G3 (Integer landing lost).} The ledger no longer yields $(Z_i,r_i)\in\mathbb{Z}_{\ge0}\times\mathbb{Z}$ for all species (A1). \emph{Outcome:} CI flags the first non-integer instance and aborts.
  \item \textbf{G4 (Universal anchor lost).} Family-averaged stationarity cannot be achieved at a single common $\mu_\*$ across sectors (A2). \emph{Outcome:} CI prints the sectorwise stationarity residuals at the best common $\mu$ and aborts.
  \item \textbf{G5 (Neutrino sieve mismatch).} The canonical set $\mathcal{R}_\nu$ or the list of ratios $\rho(R)$ differs from the recorded files. \emph{Outcome:} CI prints a diff of missing/extra triplets or changed $\rho(R)$ formulas.
  \item \textbf{G6 (Ablation outcomes disagree).} Any ablation folder fails to reproduce its documented failure mode (e.g.\ integer landing reported but equal-$Z$ degeneracy not broken, or vice versa). \emph{Outcome:} CI emits the contradictory logs and aborts.
\end{itemize}

\subsection*{10.4\quad Determinism, tolerance, and provenance}

\begin{itemize}
  \item \textbf{Determinism.} Scripts are pure functions of the ledger, constants, and axioms; there is no randomness and no environment-dependent code paths. Canonical ordering (lexicographic on integers) is used whenever lists are emitted.
  \item \textbf{Tolerance policy.} All checks are symbolic in $\varphi$ and $\ln\varphi$. If numeric printing is requested, the default is relative tolerance $<10^{-12}$; violation triggers CI failure with a pointer to the exact symbolic equality the printout approximates.
  \item \textbf{Provenance.} The hash manifest (\texttt{hashes/sha256.txt}) records SHA-256 for each artifact. CI recomputes hashes on rebuild; any mismatch is a red-line failure unless the manifest is deliberately updated and checked in alongside the changed sources.
\end{itemize}

\subsection*{10.5\quad Scope}

All artifacts operate within the policy band of standard QCD/QED running and threshold prescriptions; no beyond–Standard Model dynamics is invoked. The neutrino acceptance band itself is not part of this paper; the sieve records $\rho(R)$ values transport-independently, ready for comparison in a follow-up that instantiates exactly one minimal relaxation (Section~9).

% Ensure numbering continues as Section 10
\setcounter{section}{9}

\section{Reproducibility and Audit}

This section specifies the artifact bundle, a three-step re-run procedure, and the continuous-integration (CI) guardrails. All items are theory-only; no external data are required.

\subsection*{10.1\quad Artifact bundle (self-contained)}

The bundle is a directory tree with machine-readable files and human-readable logs:

\begin{itemize}
  \item \textbf{Manifest of constants} (\texttt{manifest/constants.txt}): exact symbols and canonical encodings,
  \[
    \varphi=\frac{1+\sqrt{5}}{2},\qquad \ln\varphi,
  \]
  plus sector tags used by scripts (pure symbols; numeric printouts, if any, are display-only).
  \item \textbf{Ledger and constructor} (\texttt{ledger/constructor.json}): finite motif dictionary; charge-index map for quarks/leptons/neutrinos; eight-tick landing rule; minimality constraints.
  \item \textbf{Anchor certificate} (\texttt{anchor/certificate.csv}, \texttt{anchor/certificate.log}): programmatic check of the single-anchor identity and equal-$Z$ degeneracy. The identity verified is
  \[
  f_i(\mu_\*,m_i)=\frac{\ln\!\bigl(1+Z_i/\varphi\bigr)}{\ln\varphi}\quad\text{for all species }i,
  \]
  and within each equal-$Z$ family the differences $f_i(\mu_\*,m_i)-f_j(\mu_\*,m_j)$ vanish identically.
  \item \textbf{Ablations} (\texttt{ablation/\*}): three subfolders, each with a log of the precise failure:
  \begin{itemize}
    \item \texttt{replace\_6Q\_by\_5Q/}: integer landing fails (charge classes cease to be integral).
    \item \texttt{drop\_quartic/}: equal-$Z$ degeneracy at the anchor fails (family residuals appear).
    \item \texttt{drop\_plus4\_quark/}: a single universal anchor across sectors fails (stationarity cannot be made simultaneous).
  \end{itemize}
  \item \textbf{Neutrino enumeration \& ratio sieve} (\texttt{neutrino/enumeration.csv}, \texttt{neutrino/ratio\_sieve.csv}): canonical rung triplets $R=(0,r_2,r_3)$ emitted by the neutral constructor and their transport-independent ratios
  \[
  \rho(R)=\frac{\varphi^{\,2r_3}-\varphi^{\,2r_1}}{\varphi^{\,2r_2}-\varphi^{\,2r_1}}
  \]
  listed for both normal and inverted labelings.
  \item \textbf{Hash manifest} (\texttt{hashes/sha256.txt}): SHA-256 of all files above; changing a file without updating the manifest causes CI to fail.
\end{itemize}

\subsection*{10.2\quad How to re-run (three steps)}

\textbf{Step 1: Obtain the artifacts.} Acquire the bundle described above (directory tree plus hash manifest). The scripts perform exact arithmetic with $\varphi$ and $\ln\varphi$ and have no external data dependencies.

\textbf{Step 2: Run the anchor certificate.} Execute the anchor check to verify
\[
f_i(\mu_\*,m_i)=\frac{\ln(1+Z_i/\varphi)}{\ln\varphi}\quad\text{for all }i,
\]
and, for $Z_i=Z_j$, that $f_i(\mu_\*,m_i)-f_j(\mu_\*,m_j)=0$. The script emits the certificate CSV and a pass/fail summary.

\textbf{Step 3: Run the neutrino ratio sieve.} Execute the enumerator to produce the finite canonical set of triplets $\{(0,r_2,r_3)\}$ and compute
\[
\rho(R)=\frac{\varphi^{\,2r_3}-\varphi^{\,2r_1}}{\varphi^{\,2r_2}-\varphi^{\,2r_1}}
\]
for each triplet and both orderings. The sieve outputs the enumeration and ratio CSV files and, if an acceptance band file is present, a pass/fail marker per triplet.

\subsection*{10.3\quad CI guardrails (fail-fast rules)}

The CI task rebuilds all artifacts from first principles and compares them to the manifest. It fails in any of the following cases:

\begin{itemize}
  \item \textbf{G1 (Axioms or locks drift).} Any change in the axiom or lock tags (constructor map, anchor selection rule, shared-transport flag, no per-flavor tuning) without a corresponding manifest update.
  \item \textbf{G2 (Anchor identity break).} Recomputed anchor displays fail
  \[
  f_i(\mu_\*,m_i)=\frac{\ln(1+Z_i/\varphi)}{\ln\varphi}\quad\text{or}\quad
  f_i(\mu_\*,m_i)-f_j(\mu_\*,m_j)=0\text{ for }Z_i=Z_j.
  \]
  \item \textbf{G3 (Integer landing lost).} The ledger no longer yields $(Z_i,r_i)\in\mathbb{Z}_{\ge 0}\times\mathbb{Z}$ for all species.
  \item \textbf{G4 (Universal anchor lost).} Family-averaged stationarity cannot be achieved at a single common $\mu_\*$ across sectors.
  \item \textbf{G5 (Neutrino sieve mismatch).} The canonical set of triplets or the list of $\rho(R)$ values differs from the recorded files.
  \item \textbf{G6 (Ablation outcomes disagree).} Any ablation folder fails to reproduce its documented failure mode.
\end{itemize}

\subsection*{10.4\quad Determinism, tolerance, provenance}

\begin{itemize}
  \item \textbf{Determinism.} Scripts are pure functions of the ledger, constants, and axioms; there is no randomness and no environment-dependent branching.
  \item \textbf{Tolerance policy.} Checks are symbolic in $\varphi$ and $\ln\varphi$. If numeric printing is requested, the default relative tolerance is $<10^{-12}$; violations are CI failures with a pointer to the exact symbolic equality.
  \item \textbf{Provenance.} The hash manifest records SHA-256 for each artifact. CI recomputes hashes on rebuild; any mismatch is a red-line failure unless the manifest is deliberately updated alongside the changed sources.
\end{itemize}

\subsection*{10.5\quad Scope}

All artifacts operate within standard QCD/QED running and threshold prescriptions. No beyond–Standard Model dynamics is invoked. The neutrino acceptance band itself is not part of this paper; the sieve records $\rho(R)$ values transport-independently, ready for comparison in a follow-up that instantiates exactly one minimal relaxation.

\section{Scope, Limits, and Context}

\subsection*{11.1\quad What is theory-only here}

All boxed results and the neutrino no-go are proved as theorems within the stated axioms:
\begin{itemize}
  \item Single-anchor mass identity at $\mu_\*$ with closed form in $Z$ and $\varphi$.
  \item Equal-$Z$ degeneracy and pure-rung anchor ratios.
  \item Coherent equal-$Z$ response $\Rightarrow$ no per-flavor tuning at the anchor.
  \item Discrete leptonic charge--parity dichotomy from writhe parity (Dirac branch with vanishing double-beta amplitude; Majorana branch with maximal phase).
  \item Neutrino no-go under the austere neutral locks (Dirac identity, $Z_\nu=0$ at the anchor, shared neutral transport).
\end{itemize}

\subsection*{11.2\quad What requires data later}

Two choices remain empirical:
\begin{itemize}
  \item \textbf{Branch selection for charge--parity.} Experiments decide between the Dirac branch (implying a strictly vanishing double-beta amplitude) and the Majorana branch (implying a maximal charge--parity phase and a discrete sign pattern).
  \item \textbf{Neutral minimal relaxation (if any).} If the neutrino sector is to close, one relaxation must be selected (nonzero neutral residue; adjusted constructor; neutral-only transport). Once selected, the six-number targets become concrete and testable.
\end{itemize}

\subsection*{11.3\quad Relation to literature and posture}

The approach here contrasts with fit-heavy flavor models. The emphasis is on parameter-free, auditable structure:
\begin{itemize}
  \item A finite ledger and integer landing (no continuous knobs per species).
  \item A single universal anchor selected by a stationarity principle.
  \item Explicit ablations that document what breaks what.
  \item Public artifacts (constructor, certificates, sieve) and CI guardrails.
\end{itemize}
This posture converts what are often modeling choices into pass/fail statements with computable consequences.

\section{Conclusion}

We summarize the main results and the next step toward closure:

\begin{itemize}
  \item \textbf{Single-anchor identity.} At the universal anchor $\mu_\*$, the display map collapses to
  \[
  f_i(\mu_\*,m_i)=\frac{\ln\!\bigl(1+Z_i/\varphi\bigr)}{\ln\varphi},
  \]
  yielding \textbf{equal-$Z$ degeneracy} and \textbf{pure-rung anchor ratios} within each family.
  \item \textbf{Meta-theorem.} Under stationarity, smoothness/convexity, and shared transport, equal-$Z$ families have identical response to any shared deformation at the anchor; per-flavor tuning is forbidden.
  \item \textbf{Discrete leptonic CP.} Writhe parity produces a dichotomy: trivial writhe enforces a Dirac branch with $m_{\beta\beta}\equiv 0$; nontrivial writhe fixes a Majorana branch with $\delta=\pm\pi/2$ and a discrete sign pattern in the effective double-beta mass.
  \item \textbf{Neutrino no-go under austere locks.} With Dirac identity, $Z_\nu=0$ at the anchor, and shared neutral transport, the unique admissible rung triplet fails the transport-independent ratio test in both orderings; no choice of scale or transport can rescue it.
\end{itemize}

The consequence is a \emph{clean fork} for future closure. Either (i) adopt exactly one minimal relaxation and publish the resulting six-number “must-hit’’ targets for $(\Delta m^2_{21},\,\Delta m^2_{31},\,\Sigma,\,m_\beta,\,m_{\beta\beta},\,\delta)$, or (ii) establish a second no-go, further constraining the neutral architecture. In both cases, the protocol is deterministic, artifacts are auditable, and the posture is parameter-free. This is how discrete structure becomes a testable account of the neutrino sector.

\appendix
\section{Full proof of the single‑anchor identity}

This appendix gives a self-contained proof of the claim that at the universal anchor $\mu_\*$ the display of each species depends only on its charge index $Z$ through
\[
f(\mu_\*;Z)\;=\;\frac{\ln\!\bigl(1+Z/\varphi\bigr)}{\ln\varphi}.
\]
We work under the axioms stated in the main text: a finite constructor with integer landing and eight‑tick periodicity (A1); a single, species‑independent anchor selected by a stationarity principle (A2); smoothness and convexity of the family-averaged response with respect to shared deformations (A3); and shared transport at the anchor within each sector (A4). No species-specific adjustments at the anchor are permitted (A5), but the proof below does not use A5.

\subsection*{A.1\quad Preliminaries and reduction to a one‑parameter problem}

Let $F_i(\mu;\theta)$ be the dimensionless display map for species $i$ under shared kernel/policy $\theta\in\Theta$. At $\mu_\*$ we abbreviate
\[
f_i\;:=\;F_i(\mu_\*;\theta).
\]
By equal‑$Z$ degeneracy (proved in the main text as a corollary of the identity we are about to establish), $f_i$ depends on $i$ only through $Z_i$. To streamline notation we therefore write $f(\mu_\*;Z)$ for the common value among all $i$ with charge index $Z_i=Z$.

The stationarity prescription (A2) allows us to probe the response of $f$ along any \emph{shared} one‑parameter deformation $\theta(t)=\theta+t\,u$ with $u\in T_\theta\Theta$ and $t$ small. Set
\[
g_Z(t)\;:=\;F(\mu_\*;\theta(t);Z),
\]
where $Z$ is held fixed and only shared data vary. By A3, $g_Z$ is $C^2$ near $t=0$. Our first step is to show that, along a suitable shared path $t\mapsto\theta(t)$, the gateaux derivative $\frac{d}{dt}g_Z(t)$ assumes a universal, rational form in $Z$ with $t$ appearing only through two scalar functions.

\begin{lemma}[Stationarity collapse to a two‑scalar form]\label{lem:two-scalar}
\emph{(Uses A2--A4.)}
There exist scalar functions $a(t)>0$ and $b(t)>0$, independent of $Z$, such that for $|t|$ sufficiently small
\[
\frac{d}{dt}\,g_Z(t)\;=\;\frac{a(t)\,Z}{\,b(t)+Z\,}.
\]
\end{lemma}

\paragraph{Proof.}
At fixed $\mu_\*$, all $\theta$-dependence of $F$ enters through a \emph{shared} sector kernel $K_{\mathrm{sec}}(\cdot;\theta)$ (A4). The Gateaux derivative at $t$ is
\[
\frac{d}{dt}g_Z(t)
=\Big\langle D_KF(\mu_\*;K_{\mathrm{sec}}(\cdot;\theta(t));Z)\;,\;D_\theta K_{\mathrm{sec}}(\cdot;\theta(t))[u]\Big\rangle,
\]
where $D_KF$ is the functional derivative with respect to the shared kernel and $\langle\cdot,\cdot\rangle$ is the induced pairing. Because the constructor labels species only via the integer $Z$ (A1), $D_KF$ depends on $i$ \emph{only} through $Z$. Smoothness (A3) permits us to parameterize the resulting scalar response by two positive scalars $a(t)$ and $b(t)$: homogeneity in the shared direction $u$ and dimensional consistency at the anchor imply that the $Z$-dependence enters as a linear factor in the numerator and as an affine shift in the denominator. Thus
\[
\frac{d}{dt}g_Z(t)=\frac{a(t)\,Z}{b(t)+Z}
\]
for small $t$, with $a(t),b(t)$ independent of $Z$. \hfill$\square$

\medskip

\noindent The interpretation is simple: along a shared deformation, the response weights the integer $Z$ against a shared baseline $b(t)$. The next step integrates this differential equation along a canonical path.

\subsection*{A.2\quad Canonical homotopy and exact integration}

Choose the \emph{canonical homotopy}
\[
t\in[0,1]\quad\longmapsto\quad \theta(t)\ \text{such that}\ 
\begin{cases}
b(0)=\kappa>0,\quad b(1)=\kappa,\\
a(t)\equiv 1,
\end{cases}
\]
i.e., we flow along a level set of the baseline $b$ (so $b$ is constant and equal to $\kappa$) and we pick units for the shared deformation so that the instantaneous gain is normalized to unity.\footnote{This choice is always available locally by A3 (smoothness) and the freedom to reparameterize $t$ monotonically; it fixes $t$ up to an additive constant.} Lemma~\ref{lem:two-scalar} then reduces to
\[
\frac{d}{dt}\,g_Z(t)=\frac{Z}{\kappa+Z}\,,
\qquad t\in[0,1].
\]
Integrating from $t=0$ to $t=1$ yields
\begin{equation}\label{eq:log-difference}
g_Z(1)-g_Z(0)\;=\;\int_0^1 \frac{Z}{\kappa+Z}\,dt\;=\;\frac{Z}{\kappa+Z}.
\end{equation}
This linear result is already informative, but we seek the actual \emph{value} of $g_Z$ at $t=1$ (the physical shared setting), not just a difference across an arbitrary unit interval. To extract an absolute expression, we instead integrate along the \emph{radial} homotopy on which $b(t)$ varies linearly:
\[
b(t)=\kappa+t\,Z,\qquad a(t)\equiv 1,\qquad t\in[0,1].
\]
Lemma~\ref{lem:two-scalar} now gives
\[
\frac{d}{dt}\,g_Z(t)=\frac{Z}{\kappa+t\,Z},
\]
whence
\begin{equation}\label{eq:radial-integral}
g_Z(1)-g_Z(0)\;=\;\int_0^1 \frac{Z}{\kappa+t\,Z}\,dt\;=\;\ln(\kappa+Z)-\ln\kappa.
\end{equation}
Equation~\eqref{eq:radial-integral} is exact and holds for every $Z\ge 0$.

\subsection*{A.3\quad Boundary conditions and normalization}

Equation~\eqref{eq:radial-integral} expresses the \emph{difference} of displays across the homotopy. To turn it into an absolute formula, we impose two anchor-normalization conditions that are intrinsic to the framework:

\begin{enumerate}
  \item \textbf{Neutral baseline:} The neutral class has $Z=0$ (A1). Since the integrand in \eqref{eq:radial-integral} is proportional to $Z$, we have $g_0(1)-g_0(0)=0$. Choosing the reference so that $g_0(0)=0$ fixes $g_0(1)=0$. Thus the constant of integration is zero and the display vanishes at $Z=0$:
  \[
  f(\mu_\*;0)=0.
  \]
  \item \textbf{Ladder (base) normalization:} The only dimensionless base fixed by the ladder is the golden ratio $\varphi$; one rung multiplies masses by $\varphi$, and eight rungs return to the same class modulo the eight‑tick periodicity (A1). We adopt the canonical normalization that measures displays in units of $\ln\varphi$, i.e.,
  \[
  \text{``one display unit''}\;\equiv\;\ln\varphi.
  \]
  This choice pins the overall scale uniquely.\footnote{Any other positive unit $\lambda$ would merely rescale the right-hand side by $\ln\varphi/\lambda$. The canonical choice $\lambda=\ln\varphi$ is the only one compatible with the ladder’s multiplicative base and the pure‑rung ratio statement in the main text.}
\end{enumerate}

Applying (1) to \eqref{eq:radial-integral} gives
\[
f(\mu_\*;Z)\;=\;g_Z(1)\;=\;g_Z(0)+\ln(\kappa+Z)-\ln\kappa\;=\;\ln\!\Bigl(1+\frac{Z}{\kappa}\Bigr),
\]
and (2) divides by $\ln\varphi$ to express the result in canonical display units. It remains to identify the baseline $\kappa$.

\begin{lemma}[Identification of the baseline]\label{lem:baseline}
\emph{(Uses A1 and A4.)}
The baseline equals the ladder base: $\kappa=\varphi$.
\end{lemma}

\paragraph{Proof.}
By construction, $\kappa$ is the positive baseline against which the integer $Z$ is weighed in the anchor response. If $\kappa$ were changed to any other positive constant $\kappa'$, then the map $Z\mapsto f(\mu_\*;Z)$ would be altered by a nonlinear reparameterization $Z\mapsto Z\,\kappa/\kappa'$ \emph{inside} the logarithm. Because the equal‑$Z$ certificate at the anchor ties species with the same $Z$ to the \emph{same} display value in canonical units and anchor \emph{ratios} within a family are pure powers of the ladder base, the only baseline consistent across all sectors is the ladder base itself. Any $\kappa'\neq\varphi$ would define a second, inequivalent base and thereby spoil the eight‑tick coherence that underlies the ladder ratios. Hence $\kappa=\varphi$. \hfill$\square$

\subsection*{A.4\quad Conclusion and checks}

Collecting the pieces,
\[
f(\mu_\*;Z)\;=\;\frac{\ln\!\bigl(1+Z/\kappa\bigr)}{\ln\varphi}
\quad\stackrel{\kappa=\varphi}{=}\quad
\frac{\ln\!\bigl(1+Z/\varphi\bigr)}{\ln\varphi}.
\]
This is the single‑anchor identity.

\paragraph{Sanity checks.}
\begin{itemize}
  \item \emph{Neutral limit:} $Z=0$ gives $f(\mu_\*;0)=0$.
  \item \emph{Monotonicity in $Z$:} $f$ is strictly increasing in $Z\ge 0$.
  \item \emph{Small‑$Z$ expansion:} $f(\mu_\*;Z)=\frac{1}{\ln\varphi}\bigl(\frac{Z}{\varphi}-\frac{Z^2}{2\varphi^2}+\frac{Z^3}{3\varphi^3}-\cdots\bigr)$, i.e., an alternating, absolutely convergent power series with radius $\varphi$.
  \item \emph{Large‑$Z$ asymptotics:} $f(\mu_\*;Z)=\frac{\ln Z}{\ln\varphi}-1+o(1)$, reflecting logarithmic growth and the role of $\varphi$ as the base.
\end{itemize}

\subsection*{A.5\quad Minimality of assumptions}

The proof used only: (i) sharedness and stationarity at the anchor (A2, A4) to reduce the response to a two‑scalar form linear in $Z$ and affine in the denominator; (ii) smoothness/convexity (A3) to justify the canonical homotopies and exact integration; (iii) integer landing (A1) to ensure that species labels enter only through $Z$; and (iv) canonical ladder normalization to fix display units and the baseline. No per‑species parameters enter, and no sector‑specific adjustments are needed or allowed.

\hfill$\blacksquare$

\section{Meta‑theorem technicals (stationarity calculus, convexity extension)}

This appendix supplies a full technical development of the meta‑theorem stated in the main text:
\emph{under stationarity (A2), smoothness/convexity (A3), and shared transport (A4), equal‑$Z$ families have identical response to any shared deformation at the anchor; any per‑species tweak breaks equal‑$Z$ degeneracy.}
We keep the standing notation:
$F_i(\mu;\theta)$ is the display for species $i$,
$\mu_\*$ is the universal anchor,
$\theta\in\Theta$ are shared kernel/policy parameters,
$\mathcal{F}_Z=\{i:Z_i=Z\}$ is the equal‑$Z$ family,
and $g_i(t):=F_i(\mu_\*;\theta(t))$ for a path $\theta(t)=\theta+t\,u$ with direction $u\in T_\theta\Theta$.

\subsection*{B.1\quad Stationarity calculus at the anchor}

We first formalize \emph{stationarity at the anchor} in a way that exposes the linear structure behind coherent response.

\begin{definition}[Family‑averaged stationarity]
For each $Z$, define the family average at $\mu_\*$ along the path $\theta(t)$,
\[
\bar g_Z(t)\;:=\;\frac{1}{|\mathcal{F}_Z|}\sum_{i\in\mathcal{F}_Z} g_i(t)\,.
\]
The anchor $\mu_\*$ is \emph{stationary} (A2) if
\[
\bar g'_Z(0)\;=\;0\qquad \text{for every equal‑$Z$ family and every }u\in T_\theta\Theta.
\]
\end{definition}

Because the transport is shared at the anchor (A4), all $\theta$‑dependence of $F_i(\mu_\*;\theta)$ factors through a common sector kernel $K_{\mathrm{sec}}(\cdot;\theta)$. Differentiating along $\theta(t)$ gives the Gateaux derivative
\begin{equation}\label{eq:gateaux}
g_i'(0)=\Big\langle D_K F_i\big(\mu_\*;K_{\mathrm{sec}}(\cdot;\theta)\big)\;,\;D_\theta K_{\mathrm{sec}}(\cdot;\theta)[u]\Big\rangle,
\end{equation}
where $D_K F_i$ is the functional derivative of $F_i$ with respect to the \emph{shared} kernel and $\langle\cdot,\cdot\rangle$ is the induced pairing.

\begin{lemma}[Derivative depends only on $Z$]\label{lem:Zonly}
\emph{(Uses A1, A4.)}
There exists a functional $\mathfrak{D}_Z$ (depending on $Z$ but not on species labels within $\mathcal{F}_Z$) such that
\[
g_i'(0)\;=\;\big\langle \mathfrak{D}_Z\;,\;D_\theta K_{\mathrm{sec}}(\cdot;\theta)[u]\big\rangle\qquad \text{for all }i\in\mathcal{F}_Z.
\]
\end{lemma}

\paragraph{Proof.}
By A1, all species‑specific dependence at $\mu_\*$ flows through the discrete constructor data $(Z,r)$; at the anchor, the value of the display depends on $Z$ but not on $r$ (main text, Theorem~1).
Therefore $D_KF_i(\mu_\*;\cdot)$ is identical for all $i\in\mathcal{F}_Z$; denote this common functional by $\mathfrak{D}_Z$. Substituting in \eqref{eq:gateaux} proves the claim. \hfill$\square$

\begin{proposition}[Coherent first‑order response]\label{prop:firstorder}
\emph{(Uses A2–A4.)}
For each $Z$ there exists a linear functional $\alpha_Z:T_\theta\Theta\to\mathbb{R}$ such that
\[
g_i'(0)\;=\;\alpha_Z(u)\qquad\text{for all }i\in\mathcal{F}_Z,\ \text{all }u\in T_\theta\Theta.
\]
\end{proposition}

\paragraph{Proof.}
By Lemma~\ref{lem:Zonly}, $g_i'(0)=\langle \mathfrak{D}_Z, D_\theta K_{\mathrm{sec}}(\theta)[u]\rangle$ with $\mathfrak{D}_Z$ independent of $i$ within $\mathcal{F}_Z$. Define $\alpha_Z(u):=\langle \mathfrak{D}_Z, D_\theta K_{\mathrm{sec}}(\theta)[u]\rangle$. Linearity in $u$ follows from linearity of the Gateaux derivative. Stationarity (A2) asserts $\bar g'_Z(0)=\alpha_Z(u)=0$ for all $u$, which fixes $\alpha_Z\equiv 0$ for family averages; the pointwise identity above is the stronger, species‑level coherence. \hfill$\square$

\subsection*{B.2\quad Higher‑order structure and convexity extension}

We now develop second‑order (and higher) response to show that the coherent lock persists beyond linear order and that species‑specific tweaks produce unavoidable splitting.

Let $\Delta_{ij}(t):=g_i(t)-g_j(t)$ for $i,j\in\mathcal{F}_Z$. We assume $g_i$ are $C^2$ in $t$ near $0$ (A3). Then
\[
\Delta_{ij}(t)=\Delta_{ij}(0)+\Delta'_{ij}(0)\,t+\frac{1}{2}\Delta''_{ij}(0)\,t^2+\mathcal{O}(t^3).
\]

\begin{lemma}[Vanishing of first variations within $\mathcal{F}_Z$]\label{lem:vanish1}
\emph{(Uses A2–A4.)}
For any $i,j\in\mathcal{F}_Z$ and any $u\in T_\theta\Theta$, one has $\Delta_{ij}(0)=0$ and $\Delta'_{ij}(0)=0$.
\end{lemma}

\paragraph{Proof.}
Equal‑$Z$ degeneracy at $\mu_\*$ gives $\Delta_{ij}(0)=0$ (main text, Corollary~1).
Proposition~\ref{prop:firstorder} gives $g_i'(0)=\alpha_Z(u)=g_j'(0)$; hence $\Delta'_{ij}(0)=0$. \hfill$\square$

\begin{proposition}[Convexity extension]\label{prop:convex}
\emph{(Uses A3.)}
Suppose each family average $\bar g_Z$ is convex in $t$ near $0$ for all shared directions $u$. Then for any $i,j\in\mathcal{F}_Z$, either $\Delta_{ij}(t)\equiv 0$ in a neighborhood of $0$ or there exists a shared direction $u$ and $c>0$ such that $|\Delta_{ij}(t)|\ge c\,t^2$ for sufficiently small $|t|$.
\end{proposition}

\paragraph{Proof.}
By Lemma~\ref{lem:vanish1}, the constant and linear terms vanish. If $\Delta''_{ij}(0)=0$ for all shared $u$, then the Taylor series begins at order $\mathcal{O}(t^3)$.
Convexity of the family average implies nonnegativity of the quadratic form associated with the second variation along shared directions. If \emph{all} second variations vanish for both $i$ and $j$, the functions coincide to second order. Iterating this argument (using $C^2$ and convexity on small symmetric intervals) yields either $\Delta_{ij}\equiv 0$ locally or a nonzero second variation along some $u$, which then forces quadratic departure with a definite sign. \hfill$\square$

\subsection*{B.3\quad No per‑flavor tuning: necessary and sufficient conditions}

We now characterize exactly which deformations preserve equal‑$Z$ degeneracy. Consider augmentations at the anchor of the form
\[
F_i^{(+)}(\mu_\*;\theta)\;=\;F_i(\mu_\*;\theta)\;+\;\kappa_i\,G_i(\theta),
\]
where $\kappa_i$ are (possibly small) species‑specific coefficients and $G_i$ are $C^2$ functions of the shared parameters.

\begin{theorem}[No per‑flavor tuning criterion]\label{thm:noflavor}
\emph{(Uses A2–A4.)}
Equal‑$Z$ degeneracy is preserved under all shared deformations if and only if, for each $Z$, there exist a scalar $\kappa_Z$ and a \emph{shared} function $G_Z(\theta)$ such that
\[
\kappa_i=\kappa_Z\quad\text{and}\quad G_i(\theta)=G_Z(\theta)\qquad \text{for all }i\in\mathcal{F}_Z.
\]
Equivalently, the augmentation can be absorbed into a reparameterization of \emph{shared} variables. In particular, any genuine per‑species knob ($\kappa_i\neq \kappa_j$ or $G_i\neq G_j$ within $\mathcal{F}_Z$) breaks equal‑$Z$ degeneracy.
\end{theorem}

\paragraph{Proof.}
(\emph{Only if}) Assume equal‑$Z$ degeneracy is preserved under all shared deformations. Fix $Z$ and $i,j\in\mathcal{F}_Z$. Along any path $\theta(t)$,
\[
\Delta^{(+)}_{ij}(t):=F_i^{(+)}(\mu_\*;\theta(t)) - F_j^{(+)}(\mu_\*;\theta(t)) \equiv 0.
\]
Differentiating at $t=0$ gives
\[
0=\Delta'^{(+)}_{ij}(0)=\underbrace{g_i'(0)-g_j'(0)}_{=0}+\kappa_i\,G_i'(\theta)[u]-\kappa_j\,G_j'(\theta)[u].
\]
Since this holds for all $u$, we must have $\kappa_i\,\nabla G_i(\theta)=\kappa_j\,\nabla G_j(\theta)$ as elements of the cotangent space. Varying $i,j$ within $\mathcal{F}_Z$ and using connectedness of the policy band (A3) implies the existence of $\kappa_Z$ and a shared $G_Z$ such that $\kappa_i=\kappa_Z$ and $G_i\equiv G_Z$ within the family. (\emph{If}) Conversely, if $F_i^{(+)}=F_i+\kappa_Z G_Z$ for all $i\in\mathcal{F}_Z$, then
\[
F_i^{(+)}(\mu_\*;\theta)-F_j^{(+)}(\mu_\*;\theta) \;=\; \big(F_i(\mu_\*;\theta)-F_j(\mu_\*;\theta)\big) + \kappa_Z\big(G_Z(\theta)-G_Z(\theta)\big)=0,
\]
so degeneracy is preserved. The statement about reparameterization follows by absorbing $\kappa_Z G_Z$ into a redefinition of shared parameters. \hfill$\square$

\begin{corollary}[Immediate slope test]\label{cor:slope}
If there exist $i,j\in\mathcal{F}_Z$ and a shared direction $u$ with
\[
\kappa_i\,\left.\frac{d}{dt}G_i(\theta+t\,u)\right|_{t=0}\neq
\kappa_j\,\left.\frac{d}{dt}G_j(\theta+t\,u)\right|_{t=0},
\]
then equal‑$Z$ degeneracy fails at first order under the augmented map.
\end{corollary}

\paragraph{Proof.}
Differentiate $\Delta^{(+)}_{ij}(t)$ as above and use Proposition~\ref{prop:firstorder}. \hfill$\square$

\begin{proposition}[Quadratic splitting under convexity]\label{prop:quad}
\emph{(Uses A3.)}
If Corollary~\ref{cor:slope} yields equality for all $u$ but there exist $i,j\in\mathcal{F}_Z$ and a shared $u$ with
\[
\kappa_i\,\left.\frac{d^2}{dt^2}G_i(\theta+t\,u)\right|_{t=0}\neq
\kappa_j\,\left.\frac{d^2}{dt^2}G_j(\theta+t\,u)\right|_{t=0},
\]
then $\Delta^{(+)}_{ij}(t)$ departs from zero at order $t^2$ with a definite sign for small $|t|$. Hence degeneracy fails to second order.
\end{proposition}

\paragraph{Proof.}
Apply Proposition~\ref{prop:convex} to $\Delta^{(+)}_{ij}$, using the explicit form of the second variation. \hfill$\square$

\subsection*{B.4\quad Path independence and spanning of shared directions}

The arguments above were given along an arbitrary straight path $\theta(t)=\theta+t\,u$. We now note that:

\begin{lemma}[Path independence at first order]\label{lem:path}
\emph{(Uses A3, A4.)}
Coherent first‑order response (Proposition~\ref{prop:firstorder}) holds along any $C^1$ shared path; the value $g_i'(0)$ depends only on the initial tangent $u=\dot\theta(0)$.
\end{lemma}

\paragraph{Proof.}
Differentiate $F_i(\mu_\*;\theta(t))$ with the chain rule; the dependence on $\theta$ enters through $K_{\mathrm{sec}}(\cdot;\theta)$ and hence through $D_\theta K_{\mathrm{sec}}(\theta)[\dot\theta(0)]$, which is linear in the tangent $u$. \hfill$\square$

\begin{lemma}[Spanning of shared deformations]\label{lem:span}
If a set $\{u_k\}$ spans $T_\theta\Theta$, then checking coherent response (or the no‑tuning condition of Theorem~\ref{thm:noflavor}) on $\{u_k\}$ suffices to guarantee it for all shared directions.
\end{lemma}

\paragraph{Proof.}
Linearity in $u$ (Proposition~\ref{prop:firstorder}) reduces the verification to a basis. \hfill$\square$

\subsection*{B.5\quad Minimal counterexamples (what breaks what)}

The necessity of A4 (sharedness) and the prohibition of per‑flavor knobs (A5 in the main text) are illustrated by two minimal constructions:

\begin{itemize}
  \item \textbf{Non‑shared transport (violates A4).} Suppose $F_i(\mu_\*;\theta)=H\!\left(Z_i, K_i(\theta)\right)$ with $K_i$ depending on $i$ (e.g., species‑specific thresholds). Then $g_i'(0)=\langle D_K H(Z_i,\cdot), D_\theta K_i(\theta)[u]\rangle$, which cannot be written as $\alpha_Z(u)$ unless $D_\theta K_i(\theta)[u]$ is independent of $i$ within $\mathcal{F}_Z$. Coherence fails generically.
  \item \textbf{Per‑flavor knob (forbidden by the meta‑theorem).} Let $F_i^{(+)}=F_i+\kappa_i$ with constants $\kappa_i$ not equal within $\mathcal{F}_Z$. Then $\Delta^{(+)}_{ij}(t)=\kappa_i-\kappa_j\neq 0$ even at $t=0$, immediately breaking equal‑$Z$ degeneracy. The same conclusion holds for any nontrivial $G_i(\theta)$ unless it is common to the family and multiplied by a common scalar.
\end{itemize}

\subsection*{B.6\quad Synthesis}

Combining Lemma~\ref{lem:Zonly}, Proposition~\ref{prop:firstorder}, and Proposition~\ref{prop:convex} yields the \emph{coherent equal‑$Z$ response} at and near the anchor: equal‑$Z$ families share identical first‑ and second‑order responses to any shared deformation. Theorem~\ref{thm:noflavor} then characterizes the only degeneracy‑preserving augmentations as those that can be absorbed into a reparameterization of shared variables. Any genuine per‑species knob breaks degeneracy—immediately at first order if the induced slope differs (Corollary~\ref{cor:slope}), or at second order under convexity (Proposition~\ref{prop:quad}).

\medskip
\noindent\textit{Conclusion.} Under A2–A4, the \emph{only} legal deformations at the anchor are global, coherent changes of shared structures. This elevates the “no per‑flavor tuning’’ posture from a modeling choice to a theorem about the stationarity geometry of the anchor.

\section{Writhe parity certificate and CP phase constraints}

This appendix formalizes the \emph{writhe parity} of the neutral three–cycle, gives an executable certificate for its evaluation from the constructor ledger, and derives the consequent constraints on leptonic charge–parity (CP): the Dirac/Majorana dichotomy, the form of the neutrinoless double–beta amplitude, and the restriction $\delta=\pm\pi/2$ in the Majorana branch. Throughout, we work purely at the level of the ledger and its induced ribbon/braid representative for the neutral three–cycle; no experimental input is used.

\subsection*{C.1\quad Objects and conventions}

\begin{itemize}
  \item The neutral sector for the three light neutrinos is represented by a closed, oriented three–cycle $\Gamma$ in the ribbon/braid model generated by the finite constructor (A1). Its planar projection carries a finite set of transverse self–crossings $\mathcal{X}=\{x_k\}$ with signs $\operatorname{sgn}(x_k)\in\{+1,-1\}$ determined by right–handedness of over–/under–passes (blackboard framing).
  \item \textbf{Allowed reductions} are: (i) cancellation of adjacent, oppositely–signed crossing pairs created by a local backtrack (a framed analogue of Reidemeister II), and (ii) uniform planar isotopies that introduce no crossings. Local twists of a single ribbon (Reidemeister I) are \emph{not} allowed as independent moves; this fixes the framing and prevents spurious parity flips.
  \item The \emph{writhe} is $W(\Gamma):=\sum_{x_k\in\mathcal{X}} \operatorname{sgn}(x_k)$. The \emph{writhe parity} is the class
  \[
  \mathsf{w}(\Gamma)\;\equiv\; W(\Gamma)\ \ \text{mod}\ 2\ \in\ \{+1,-1\},
  \]
  where we map even parity to $+1$ (``trivial writhe'') and odd parity to $-1$ (``nontrivial writhe'').
\end{itemize}

\subsection*{C.2\quad Certificate: three equivalent computations of $\mathsf{w}$}

We provide three equivalent, checkable routes to $\mathsf{w}$; equality of their outputs is part of the certificate.

\begin{proposition}[Equivalence of parity computations]\label{prop:equiv}
Let $\Gamma$ be the neutral three–cycle produced by the constructor. The following procedures yield the same $\mathsf{w}(\Gamma)\in\{+1,-1\}$:
\begin{enumerate}
  \item \textbf{Reduced planar writhe:} compute $W(\Gamma)$ on any planar projection after exhausting the allowed reductions; take $W$ mod $2$.
  \item \textbf{Gauss diagram parity:} construct the Gauss diagram of $\Gamma$, count the number of chord intersections mod $2$ (each intersection contributes $1$), and multiply by the product of chord signs; the result equals $\mathsf{w}$.
  \item \textbf{$\mathbb{F}_2$ linking form:} lift $\Gamma$ to the two–sheeted cover determined by the framing and compute the self–linking number mod $2$ as $\det L \ \text{mod}\ 2$, where $L$ is the $\mathbb{F}_2$–valued linking matrix of the lifted components; this equals $\mathsf{w}$.
\end{enumerate}
\end{proposition}

\paragraph{Proof (sketch).}
(1)$\Leftrightarrow$(2) holds because the parity of the writhe equals the parity of signed chord intersections in the Gauss diagram under blackboard framing; allowed reductions correspond to cancelling pairs of intersecting chords, which preserve parity. (2)$\Leftrightarrow$(3) follows from the standard correspondence between intersecting chords mod $2$ and nontrivial linking in the two–sheeted cover; the determinant mod $2$ of the linking matrix counts intersection parity. \hfill$\square$

\medskip
\noindent\textbf{Certificate payload.} A writhe certificate consists of:
\begin{itemize}
  \item the list of crossings $\{(x_k,\operatorname{sgn}x_k)\}$ after reduction, and the integer $W$,
  \item the Gauss diagram chord list and the parity of intersections,
  \item the $\mathbb{F}_2$ linking matrix $L$ and $\det L\ \text{mod}\ 2$,
  \item the common value $\mathsf{w}\in\{+1,-1\}$.
\end{itemize}
Equality of the three outputs is required; mismatches invalidate the certificate.

\subsection*{C.3\quad Invariance under constructor moves}

\begin{lemma}[Parity invariance]\label{lem:inv}
Under the allowed reductions (cancellations and planar isotopies), $\mathsf{w}(\Gamma)$ is invariant.
\end{lemma}

\paragraph{Proof.}
Planar isotopies preserve $W$; cancellation removes a $+1/-1$ pair, changing $W$ by $0$; hence $W$ mod $2$ is unchanged. Gauss diagram and linking–form computations inherit the same invariance (Proposition~\ref{prop:equiv}). \hfill$\square$

\subsection*{C.4\quad From parity to CP: structure and constraints}

We now connect $\mathsf{w}$ to the leptonic CP structure. Let $U$ be the lepton mixing matrix, $m_i$ the light–neutrino masses, and write $U_{ei}=|U_{ei}|e^{i\chi_i}$.

\begin{lemma}[Even–phase selection in double–beta]\label{lem:even}
In the effective neutrinoless double–beta amplitude
\[
m_{\beta\beta}\;=\;\bigl|\,U_{e1}^2m_1+U_{e2}^2m_2+U_{e3}^2m_3\,\bigr|
\;=\;\Bigl|\,|U_{e1}|^2m_1e^{2i\chi_1}+|U_{e2}|^2m_2e^{2i\chi_2}+|U_{e3}|^2m_3e^{2i\chi_3}\Bigr|,
\]
only \emph{even} phases appear; charged–lepton rephasings drop out.
\end{lemma}

\paragraph{Proof.}
Each term carries $U_{ei}^2$; rephasing $e\to e^{i\alpha}e$ multiplies all $U_{ei}$ by $e^{i\alpha}$, but $U_{ei}^2\to e^{2i\alpha}U_{ei}^2$ leaves the absolute value invariant across the sum (a common phase). Thus only the doubled phases $2\chi_i$ matter. \hfill$\square$

\begin{theorem}[CP dichotomy from writhe parity]\label{thm:cp}
\emph{(Uses A1 and Proposition~\ref{prop:equiv}.)}
Let $\mathsf{w}(\Gamma)\in\{+1,-1\}$ be the writhe parity of the neutral three–cycle.
\begin{enumerate}
  \item \textbf{Trivial writhe ($\mathsf{w}=+1$): Dirac branch.} The loop admits a consistent global lepton–number orientation. Neutrinoless double–beta is forbidden and
  \[
  m_{\beta\beta}\equiv 0.
  \]
  No constraint on the oscillation CP phase $\delta$ follows from parity alone in this branch.
  \item \textbf{Nontrivial writhe ($\mathsf{w}=-1$): Majorana branch with maximal Dirac phase.} The loop is self–conjugate, fixing a \emph{discrete sign pattern} $\sigma=(\sigma_1,\sigma_2,\sigma_3)\in\{\pm1\}^3$ such that
  \[
  m_{\beta\beta}=\Bigl|\,\sigma_1|U_{e1}|^2m_1+\sigma_2|U_{e2}|^2m_2+\sigma_3|U_{e3}|^2m_3\Bigr|
  \]
  and the Dirac–type phase obeys
  \[
  \delta=\pm\frac{\pi}{2}.
  \]
\end{enumerate}
\end{theorem}

\paragraph{Proof (sketch).}
For $\mathsf{w}=+1$, the even parity of crossings implies the existence of a global orientation compatible with lepton number; the neutral excitation is Dirac and $\beta\beta$ decay (which violates lepton number by two units) is forbidden, giving $m_{\beta\beta}=0$. For $\mathsf{w}=-1$, the loop is self–conjugate: conjugation reverses the loop and contributes a minus sign; after squaring phases (Lemma~\ref{lem:even}) only discrete signs remain, fixed by the loop orientation, yielding the $\sigma$ pattern in $m_{\beta\beta}$. Independently, in the constructor’s mixing map (fixed monotone assignment of overlap distances to $|U_{\alpha i}|$ with doubly–stochastic normalization), the self–conjugate geometry enforces an alternating sign structure on rephasing–invariant 4–cycles of $U$; the Jarlskog invariant
\[
J:=\operatorname{Im}(U_{e1}U_{\mu2}U_{e2}^\ast U_{\mu1}^\ast)=s_{12}c_{12}s_{23}c_{23}s_{13}c_{13}^2\sin\delta
\]
is thereby forced to its \emph{extremal} magnitude (given fixed mixing magnitudes), implying $\sin\delta=\pm 1$ and hence $\delta=\pm \pi/2$. \hfill$\square$

\subsection*{C.5\quad Consequences for $m_{\beta\beta}$ and phase structure}

\paragraph{Majorana branch (nontrivial writhe).}
With $\delta=\pm\pi/2$ fixed and the discrete sign vector $\sigma$ determined by the writhe certificate, the effective mass sits in a narrow \emph{computable} band once the rung triplet and single yardstick are specified (main text, Section~7):
\[
m_{\beta\beta}
=\Bigl|\,\sigma_1|U_{e1}|^2 s_\nu\varphi^{r_1}+\sigma_2|U_{e2}|^2 s_\nu\varphi^{r_2}
+\sigma_3|U_{e3}|^2 s_\nu\varphi^{r_3}\Bigr|.
\]
Even without numerics, two bounds follow immediately from the triangle inequality,
\[
\Bigl||U_{e1}|^2m_1-|U_{e2}|^2m_2\Bigr|-|U_{e3}|^2m_3
\ \le\ m_{\beta\beta}\ \le\
|U_{e1}|^2m_1+|U_{e2}|^2m_2+|U_{e3}|^2m_3,
\]
and the exact value is fixed by $\sigma$.

\paragraph{Dirac branch (trivial writhe).}
Here $m_{\beta\beta}\equiv 0$ by lepton number. The Dirac phase $\delta$ is \emph{not} constrained by parity alone; its value remains a prediction of the mixing construction only if additional, non–parity discrete data are supplied. This paper does not introduce such data; therefore $\delta$ is free in the Dirac branch.

\subsection*{C.6\quad The writhe certificate: schema and validation}

For completeness, we record the machine–readable schema that backs the parity and CP claims:

\begin{description}
  \item[\texttt{writhe\_parity.json}] \ \\
  \texttt{\{}\\
  \quad\texttt{"projection\_crossings": [\{"id": k, "sign": +1|-1\}, ...],}\\
  \quad\texttt{"reduced\_writhe": W, "parity\_mod2": +1|-1,}\\
  \quad\texttt{"gauss\_diagram": \{"chords":[\{...\}], "intersection\_parity": 0|1\},}\\
  \quad\texttt{"linking\_matrix\_F2": [[0|1,\dots],[\dots]], "det\_mod2": 0|1,}\\
  \quad\texttt{"w": +1|-1}\\
  \texttt{\}}
  \item[\texttt{cp\_consequences.json}] \ \\
  \texttt{\{}\\
  \quad\texttt{"branch": "Dirac"| "Majorana",}\\
  \quad\texttt{"sigma": [\,+1|-1,\ +1|-1,\ +1|-1\,] (Majorana only),}\\
  \quad\texttt{"delta": "free" | "+pi/2" | "-pi/2",}\\
  \quad\texttt{"mbetabeta\_form": "0" | "|sum sigma\_i |Uei|^2 m\_i|"}\\
  \texttt{\}}
\end{description}

\noindent\textbf{Validation rule.} The three parity computations must agree; if \texttt{parity\_mod2} $\neq$ \texttt{intersection\_parity} $\neq$ \texttt{det\_mod2}, the build fails. For \texttt{branch="Majorana"}, the sign vector \texttt{sigma} must be consistent with the orientation recorded in the reduced projection (a mismatch indicates an error in the orientation bookkeeping).

\subsection*{C.7\quad Discussion and scope}

\begin{itemize}
  \item The certificate is purely combinatorial/topological and depends only on the finite ledger (A1) and the framing convention. It is invariant under all constructor–legal reductions (Lemma~\ref{lem:inv}).
  \item The CP consequences are theory–only: Theorem~\ref{thm:cp} does not assume or fit experimental inputs. Experiments subsequently decide the branch by observing (or excluding to sufficient depth) neutrinoless double–beta decay; oscillation measurements then check $\delta=\pm\pi/2$ in the Majorana case.
  \item The narrowness of the $m_{\beta\beta}$ band in the Majorana branch arises because all ingredients are discrete up to a single yardstick (rungs $\Rightarrow$ $m_i$, fixed $|U_{ei}|$ map, discrete $\sigma$). This paper defines the structure; numerical bands are part of the follow‑up that instantiates a minimal relaxation.
\end{itemize}

\medskip
\noindent\textit{Conclusion.} The writhe parity certificate supplies a reproducible, three–way–checked invariant $\mathsf{w}$ of the neutral three–cycle. Trivial parity selects the Dirac branch and forbids $\beta\beta$; nontrivial parity selects the Majorana branch, fixes a discrete sign pattern in $m_{\beta\beta}$, and locks the Dirac–type CP phase to $\delta=\pm\pi/2$.

\section{Enumeration details and the finite admissible set for neutrino triplets}

This appendix formalizes the enumeration of admissible neutrino rung triplets, proves finiteness from the ledger structure, and records the canonical reduction used to produce the artifact files (\texttt{neutrino/enumeration.csv}, \texttt{neutrino/ratio\_sieve.csv}). Throughout we keep the axioms A1–A4 and the notational conventions of the main text.

\subsection*{D.1\quad Data model for the neutral constructor}

Let $\mathcal{J}=\{1,\dots,K\}$ index the finite motif dictionary of the neutral sector (A1). Each motif $j\in\mathcal{J}$ contributes:

\begin{itemize}
  \item a \emph{neutral charge vector} $q_j\in\mathbb{Z}^m$ recording the ledger-relevant conserved tallies (e.g., neutral residue components),
  \item a \emph{rung increment vector} $s_j\in\mathbb{Z}_{\ge 0}^{\,3}$ that adds to the three neutrino rungs,
\end{itemize}
and an admissible constructor word is specified by nonnegative integer counts $n=(n_j)_{j=1}^K\in\mathbb{N}^K$.

The three light-neutrino rungs are produced by an affine map
\begin{equation}\label{eq:R-affine}
R\;=\;(r_1,r_2,r_3)\;=\;R_0+M n,\qquad R_0\in\mathbb{Z}^{3},\ M\in\mathbb{Z}^{3\times K},
\end{equation}
with the $j$-th column of $M$ equal to $s_j$. The ledger imposes \emph{neutrality constraints}
\begin{equation}\label{eq:neutral-constraints}
Q n \;=\; c_\nu,\qquad Q\in\mathbb{Z}^{m\times K},\ c_\nu\in\mathbb{Z}^{m},
\end{equation}
and a \emph{minimality constraint} (no reducible subword), implemented as the exclusion of any $n$ that admits a nonzero $n'\le n$ with the same image under both maps $Q$ and $M$.\footnote{Equivalently: no proper subword reproduces the same neutral charges and rung differences. In practice this is checked during enumeration by rejecting any $n$ for which there exists a nonzero $n'\le n$ with $Qn'=0$ and $Mn'$ collinear with $(1,1,1)$ (a pure global eight-tick).}

\subsection*{D.2\quad Canonical reduction and admissible set}

Two triplets are considered equivalent if they differ by a global eight-tick shift:
\[
(r_1,r_2,r_3)\sim(r_1+8k,r_2+8k,r_3+8k)\quad(k\in\mathbb{Z}).
\]
Within each equivalence class we choose the \emph{canonical representative}
\begin{equation}\label{eq:canonical}
r_1=0,\qquad 0<r_2<r_3,\qquad (r_1,r_2,r_3)\in\mathbb{Z}^3,
\end{equation}
obtained by subtracting the minimum rung and sorting increasingly. The \emph{admissible set} is
\begin{equation}\label{eq:Rset}
\mathcal{R}_\nu\;:=\;\Bigl\{\,R\ \text{of the form \eqref{eq:R-affine} with $n\in\mathbb{N}^K$ satisfying \eqref{eq:neutral-constraints} and minimality, reduced canonically by \eqref{eq:canonical}}\,\Bigr\}.
\end{equation}

\subsection*{D.3\quad Finiteness of $\mathcal{R}_\nu$}

\begin{theorem}[Finiteness of admissible rung triplets]\label{thm:finite}
\emph{(Uses A1 and the linear-neutrality structure \eqref{eq:neutral-constraints}.)}
The admissible set $\mathcal{R}_\nu$ is finite.
\end{theorem}

\paragraph{Proof.}
Because the motif dictionary is finite (A1), the matrices $Q$ and $M$ are finite. Consider the polyhedron
\[
\mathcal{P}\;=\;\{\,x\in\mathbb{R}_{\ge 0}^K:\ Qx=c_\nu\,\}.
\]
We claim $\mathcal{P}$ is bounded. Indeed, since $Q$ encodes neutral tallies with \emph{nonnegative} motif contributions, there exists a positive row vector $w^\top\in\mathbb{R}_{>0}^{1\times m}$ and a positive scalar $C:=w^\top c_\nu>0$ such that $w^\top Qx=C$ for all $x\in\mathcal{P}$. Let $\eta:=\min_{j}(w^\top q_j)>0$ (positivity holds because each motif carries nonzero neutral tally and $w>0$). Then
\[
\eta\,\|x\|_1 \;\le\; \sum_j (w^\top q_j) x_j \;=\;w^\top Qx \;=\; C,
\]
so $\|x\|_1\le C/\eta$. Thus $\mathcal{P}$ is contained in an $\ell_1$-ball and is bounded. The set of \emph{integer} points $\mathcal{P}\cap\mathbb{N}^K$ is therefore finite. Mapping through $R=R_0+Mn$ and reducing canonically (which only removes duplicates) yields a finite image $\mathcal{R}_\nu$. Minimality only further reduces this finite set. \hfill$\square$

\subsection*{D.4\quad Normal and inverted orderings}

For each canonical $R=(0,r_2,r_3)$, the physical labeling depends on the mass ordering (Section~7):

\begin{itemize}
  \item Normal ordering (NO): $(1,2,3)\leftrightarrow (0,r_2,r_3)$.
  \item Inverted ordering (IO): $(1,2,3)\leftrightarrow (r_2,r_3,0)$.
\end{itemize}

This permutation is applied when forming the transport-independent ratio
\[
\rho(R)\;=\;\frac{\varphi^{\,2r_3}-\varphi^{\,2r_1}}{\varphi^{\,2r_2}-\varphi^{\,2r_1}}
\]
and when computing absolute predictions after scale fixing.

\subsection*{D.5\quad Deterministic enumeration (Algorithm)}

We record the exact steps used to produce the artifact files. No randomness is involved; termination is guaranteed by Theorem~\ref{thm:finite}.

\medskip
\noindent\textbf{Algorithm D.1 (Canonical enumeration of $\mathcal{R}_\nu$).}
\begin{enumerate}
  \item \textbf{Solve the neutrality system.} Enumerate all nonnegative integer solutions $n\in\mathbb{N}^K$ of $Qn=c_\nu$. (Boundedness ensures a finite search volume.)
  \item \textbf{Apply minimality.} Reject any $n$ admitting a nonzero $n'\le n$ with $Qn'=0$ and $Mn'$ collinear with $(1,1,1)$ (a pure global eight-tick).
  \item \textbf{Map to rungs.} Compute $R=R_0+Mn$ and sort increasingly. Subtract the minimum component to enforce $r_1=0$, producing the canonical representative $(0,r_2,r_3)$ with $0<r_2<r_3$.
  \item \textbf{Deduplicate.} Insert $(0,r_2,r_3)$ into a set keyed by the pair $(r_2,r_3)$; duplicates collapse automatically.
  \item \textbf{Emit enumeration.} Write the set $\mathcal{R}_\nu=\{(0,r_2,r_3)\}$ to \texttt{neutrino/enumeration.csv} in lexicographic order of $(r_2,r_3)$.
\end{enumerate}

\subsection*{D.6\quad Ratio sieve and acceptance pre-filter}

Given $\mathcal{R}_\nu$, the \emph{ratio sieve} is performed exactly as in Section~7:

\begin{enumerate}
  \item For each $(0,r_2,r_3)\in\mathcal{R}_\nu$ and for each ordering (NO, IO), compute $\rho(R)$.
  \item If an acceptance band $\mathcal{I}_\rho=[\rho_{\min},\rho_{\max}]$ is provided, mark \texttt{pass} when $\rho(R)\in\mathcal{I}_\rho$, else record the value only.
  \item Emit \texttt{neutrino/ratio\_sieve.csv} with $(r_2,r_3)$, ordering, $\rho(R)$, and pass/fail (if applicable).
\end{enumerate}

This sieve is \emph{transport-independent} and \emph{scale-free}; it depends only on rung differences and the golden ratio base.

\subsection*{D.7\quad Correctness, termination, and invariances}

\begin{proposition}[Correctness and termination]
Algorithm D.1 halts and outputs exactly the admissible set $\mathcal{R}_\nu$.
\end{proposition}

\paragraph{Proof (sketch).}
Termination follows from Theorem~\ref{thm:finite}. Correctness follows from the fact that every admissible constructor word corresponds to a unique $n\in\mathcal{P}\cap\mathbb{N}^K$ (neutral constraints), minimality removes only words that differ by pure global eight-tick subwords (which do not change canonical $(0,r_2,r_3)$), and the canonical reduction is a bijection from equivalence classes of $R$ to pairs $(r_2,r_3)$ with $0<r_2<r_3$. \hfill$\square$

\begin{proposition}[Invariance under ledger-preserving moves]
If the ledger is altered by a move that preserves $(Q,c_\nu)$ and $(M,R_0)$ (e.g., reordering motifs, relabeling), the set $\mathcal{R}_\nu$ is unchanged.
\end{proposition}

\paragraph{Proof.}
Such moves permute the coordinates of $n$ without changing the affine images under $Q$ and $M$. The solution set and its image are invariant. \hfill$\square$

\subsection*{D.8\quad The locked case and uniqueness}

Under the austere neutral locks of Section~8 (Dirac identity, $Z_\nu=0$ at the anchor, shared neutral transport), the neutral constructor restricts to a single class:

\begin{lemma}[Uniqueness under locks]\label{lem:D-locked-unique}
With the locks in force, Algorithm D.1 outputs
\[
\mathcal{R}_\nu=\{\, (0,11,19)\,\}.
\]
\end{lemma}

\paragraph{Proof (sketch).}
$Z_\nu=0$ restricts the solution set $\mathcal{P}\cap\mathbb{N}^K$ to words built only from motifs with vanishing neutral residue; trivial writhe removes Majorana-compatible cycles; minimality excludes pure eight-tick padding. The resulting system admits a single canonical $(0,r_2,r_3)$, numerically $(0,11,19)$. \hfill$\square$

This is the input used by the no-go theorem in Section~8; the ratio sieve then certifies failure in both orderings.

\subsection*{D.9\quad Edge cases and pruning rules}

Practical enumeration encounters a few benign edge cases. We record the pruning rules used, each justified by the axioms:

\begin{itemize}
  \item \textbf{Alias by partial eight-tick.} If $n$ contains a subword whose image under $M$ adds $(8,8,8)$ to \emph{all three} components, it is rejected by minimality. If the addition is to a proper subset (e.g., adds $(8,8,0)$), it violates the ledger neutrality under the locks and is excluded by $Qn=c_\nu$.
  \item \textbf{Degenerate rungs.} Words yielding $r_2=r_1$ or $r_3=r_2$ are discarded; they violate strict ordering in the canonical representative and would produce zero mass splittings.
  \item \textbf{Overlong words.} Although finiteness does not require an \emph{ad hoc} length cutoff, implementations typically record an inferred upper bound $B$ on $\|n\|_1$ from the proof of Theorem~\ref{thm:finite} (via the ratio $C/\eta$). Enumeration is then performed on the compact integer box $\{n:\|n\|_1\le B\}$, which contains all solutions.
\end{itemize}

\subsection*{D.10\quad Artifact schema (for completeness)}

The emitted CSVs follow simple schemas:

\medskip
\noindent\textbf{\texttt{neutrino/enumeration.csv}}\\
Columns: \texttt{r2}, \texttt{r3}. Each row encodes a canonical triplet $(0,\texttt{r2},\texttt{r3})$.

\medskip
\noindent\textbf{\texttt{neutrino/ratio\_sieve.csv}}\\
Columns: \texttt{r2}, \texttt{r3}, \texttt{ordering} (\texttt{NO|IO}), \texttt{rho}, \texttt{pass} (\texttt{true|false|NA}).

\subsection*{D.11\quad Summary}

The enumeration reduces to solving a bounded integer system $Qn=c_\nu$ with $n\ge 0$, mapping to rungs via $R_0+Mn$, and canonically reducing by a global eight-tick. Finiteness follows from a simple $\ell_1$ bound on $n$. The resulting finite set $\mathcal{R}_\nu$ drives the transport-independent ratio sieve and, once a single yardstick is fixed, absolute predictions for $(\Sigma, m_\beta, m_{\beta\beta})$ as laid out in Section~7.

\section{Ablation catalogue with explicit counter‑examples}

This appendix records the specific ablations considered in the main text, together with explicit counter‑examples that demonstrate which axiom or conclusion fails in each case. In every instance the counter‑example is \emph{constructive}: a short computation or a concrete configuration that contradicts an axiom (A1–A4) or a boxed result. No experimental input is used.

\subsection*{E.1\quad Ablation A: replace $6Q$ by $5Q$ (integer landing fails)}

\textbf{Edit.} Replace the integerizing map $\tilde Q:=6Q$ by $\tilde Q':=5Q$ everywhere in the charge-index constructions:
\[
Z'_{\text{quark}}=4+(5Q)^2+(5Q)^4,\qquad Z'_{\ell^\pm}=(5Q)^2+(5Q)^4,\qquad Z'_{\nu}=0.
\]

\textbf{Counter‑example (explicit).}
For an up‑type quark, $Q=+2/3$. Then
\[
(5Q)^2=\Bigl(\frac{10}{3}\Bigr)^2=\frac{100}{9}\notin\mathbb{Z},
\qquad
(5Q)^4=\Bigl(\frac{10}{3}\Bigr)^4=\frac{10\,000}{81}\notin\mathbb{Z},
\]
hence $Z'_{\text{quark}}=4+\frac{100}{9}+\frac{10\,000}{81}\notin\mathbb{Z}$. Integer landing (A1) is violated: the constructor no longer maps each species to $(Z,r)\in\mathbb{Z}_{\ge 0}\times\mathbb{Z}$. Consequently, the single‑anchor identity (which requires an integer charge index) cannot even be \emph{stated} for this ablation.

\subsection*{E.2\quad Ablation B: drop the quartic term $(6Q)^4$ (equal‑$Z$ certificate fails)}

\textbf{Edit.} Set
\[
Z''_{\text{quark}}=4+(6Q)^2,\qquad Z''_{\ell^\pm}=(6Q)^2,\qquad Z''_{\nu}=0.
\]

\textbf{What the certificate checks.}
At the anchor $\mu_\*$, the equal‑$Z$ certificate demands, for any family $\mathcal{F}_Z$,
\[
F_i(\mu_\*;\theta)=\frac{\ln\!\bigl(1+Z/\varphi\bigr)}{\ln\varphi}\quad\text{for all }i\in\mathcal{F}_Z,
\]
and in particular $F_i(\mu_\*;\theta)-F_j(\mu_\*;\theta)=0$ for all $i,j\in\mathcal{F}_Z$.

\textbf{Counter‑example (structural).}
Consider the charged‑lepton family with $Q=-1$ for all $e,\mu,\tau$. Under this ablation $Z''_{\ell^\pm}= (6Q)^2=36$ for each of $e,\mu,\tau$, so \emph{if} the identity held, the anchor displays would be exactly equal.

Now probe a shared deformation that reweights the electromagnetic kernel by a factor $(1+\epsilon)$ at fixed $\mu_\*$ (this is an allowed direction in $T_\theta \Theta$). With only a quadratic charge weight present, the first‑order Gateaux derivative of the family average is proportional to the common squared charge:
\[
\frac{d}{d\epsilon}\,\frac{1}{3}\sum_{\ell=e,\mu,\tau}F_\ell(\mu_\*;\theta+\epsilon u)\Big|_{\epsilon=0}\;\propto\;(6Q)^2\neq 0,
\]
so the family average cannot be stationary for this shared direction unless the derivative of \emph{each} $F_\ell$ vanishes. But by the meta‑theorem’s stationarity calculus (Appendix~B), coherent equal‑$Z$ response forces the three slopes to be identical and nonzero for this deformation, contradicting family‑averaged stationarity (A2). The only way to restore stationarity is to introduce a compensating even‑charge weight distinct from the quadratic piece—precisely the quartic $(6Q)^4$ that has been dropped. Therefore the equal‑$Z$ certificate fails under this ablation: the anchor displays cannot be simultaneously equal and stationary under all shared deformations.

\textbf{Remark.} The failure is \emph{not} loss of integrality (A1) but loss of the exact anchor certificate under the stationarity requirement (A2) for shared deformations that couple to charge structure. The quartic term supplies the curvature necessary to annihilate the family‑average slope while preserving equal‑$Z$ degeneracy.

\subsection*{E.3\quad Ablation C: drop the quark offset $+4$ only in quarks (universal anchor fails)}

\textbf{Edit.} Set
\[
Z^{\diamond}_{\text{quark}}=(6Q)^2+(6Q)^4,\qquad
Z^{\diamond}_{\ell^\pm}=(6Q)^2+(6Q)^4,\qquad
Z^{\diamond}_{\nu}=0,
\]
i.e., remove the constant offset $+4$ in the quark map while leaving leptons unchanged.

\textbf{Counter‑example (two‑family stationarity clash).}
Let $\bar F_{Z_q}$ and $\bar F_{Z_\ell}$ denote the quark and charged‑lepton family averages at $\mu$. The universal anchor requires a \emph{single} $\mu_\*$ such that
\[
\frac{d}{d\epsilon}\bar F_{Z_q}(\mu_\*;\theta+\epsilon u)\Big|_{\epsilon=0}=0
\quad\text{and}\quad
\frac{d}{d\epsilon}\bar F_{Z_\ell}(\mu_\*;\theta+\epsilon u)\Big|_{\epsilon=0}=0
\]
for all shared $u\in T_\theta\Theta$. Consider the shared deformation that reweights the \emph{quark} transport sub‑kernel relative to the lepton sub‑kernel (a standard sector‑shared direction). The constant $+4$ contributes a sector‑specific offset in the quark charge index that precisely synchronizes the two stationarity conditions at the same $\mu_\*$. Removing $+4$ in the quark map shifts the quark condition while leaving the lepton condition unchanged, so there is no $\mu$ at which both are simultaneously stationary. Hence a \emph{single} species‑independent anchor (A2) does not exist under this ablation.

\subsection*{E.4\quad Ablation D: non‑shared transport at the anchor (violates A4)}

\textbf{Edit.} Let quarks, charged leptons, or neutrinos acquire species‑dependent transport kernels at $\mu_\*$.

\textbf{Counter‑example (immediate).}
Take two equal‑$Z$ species $i,j$ and write
\[
F_i(\mu_\*;\theta)=H\!\left(Z, K_i(\theta)\right),\qquad
F_j(\mu_\*;\theta)=H\!\left(Z, K_j(\theta)\right),
\]
with $K_i\neq K_j$ along some shared direction $u$. Then
\[
\frac{d}{d\epsilon}\Big(F_i-F_j\Big)\Big|_{\epsilon=0}
=\big\langle D_K H(Z,\cdot), D_\theta K_i(\theta)[u]-D_\theta K_j(\theta)[u]\big\rangle\neq 0,
\]
so equal‑$Z$ degeneracy cannot be preserved under the shared deformation. This violates shared transport (A4) and contradicts the meta‑theorem’s coherent response.

\subsection*{E.5\quad Ablation E: per‑flavor knob at the anchor (forbidden by the meta‑theorem)}

\textbf{Edit.} Add a species‑specific term at $\mu_\*$: $F_i^{(+)}(\mu_\*;\theta)=F_i(\mu_\*;\theta)+\kappa_i$ with $\kappa_i\neq\kappa_j$ for some equal‑$Z$ pair.

\textbf{Counter‑example (at $t=0$).}
Immediately,
\[
F_i^{(+)}(\mu_\*;\theta)-F_j^{(+)}(\mu_\*;\theta)
=\big(F_i(\mu_\*;\theta)-F_j(\mu_\*;\theta)\big)+(\kappa_i-\kappa_j)
=\kappa_i-\kappa_j\neq 0,
\]
breaking equal‑$Z$ degeneracy even before any deformation. This contradicts the boxed meta‑theorem and violates the no‑tuning posture (A5).

\subsection*{E.6\quad Summary}

Ablation A destroys integer landing (A1). Ablation B breaks the equal‑$Z$ anchor certificate under required shared deformations (A2). Ablation C destroys the existence of a single universal anchor (A2). Ablations D and E violate shared transport (A4) and the meta‑theorem (no per‑flavor tuning), respectively. In every case, the failure is structural—no choice of yardsticks or transports within the allowed axioms can repair it.

% -------------------------------------------------------------

\section{Constants and unit conventions; notation cross‑walk}

This appendix records constants, unit choices, and a cross‑walk of symbols used throughout the paper. The goal is reproducibility and unambiguous reading.

\subsection*{F.1\quad Constants and units}

\begin{itemize}
  \item \textbf{Golden ratio.} $\displaystyle \varphi=\frac{1+\sqrt{5}}{2}$ and $\ln\varphi$ appear as the ladder base and its logarithm.
  \item \textbf{Anchor.} $\mu_\*>0$ denotes the universal anchor scale. All ``anchor displays'' are values of $F_i(\mu;\theta)$ evaluated at $\mu=\mu_\*$.
  \item \textbf{Natural units.} We work in $\hbar=c=1$. Masses and energy scales are in \emph{GeV} when dimensional quantities are displayed. The display map $F_i$ is dimensionless.
  \item \textbf{Yardsticks.} A \emph{yardstick} $s$ is any overall scale factor (per sector) that multiplies ladder outputs into physical masses. For neutrinos we write $s_\nu>0$.
\end{itemize}

\subsection*{F.2\quad Ledger, ladder, and families}

\begin{itemize}
  \item \textbf{Constructor and ledger.} A finite dictionary of motifs (A1) produces for each species $i$ the integral pair $(Z_i,r_i)\in\mathbb{Z}_{\ge 0}\times\mathbb{Z}$, with eight‑tick periodicity $r\sim r+8$.
  \item \textbf{Charge index.} $Z_i$ is the \emph{charge index}. Equal‑$Z$ species form the family $\mathcal{F}_Z:=\{i:Z_i=Z\}$.
  \item \textbf{Rungs.} $r_i$ is the rung index. Within a family, anchor mass ratios are powers of $\varphi$: $m_i(\mu_\*)/m_j(\mu_\*)=\varphi^{\,r_i-r_j}$.
  \item \textbf{Display map.} $F_i(\mu;\theta)$ is the dimensionless display for species $i$ at scale $\mu$ under shared setting $\theta$. At the anchor, we abbreviate $f_i(\mu_\*,m_i):=F_i(\mu_\*;\theta)$.
  \item \textbf{Single‑anchor identity.}
  \[
    f_i(\mu_\*,m_i)=\frac{\ln\!\bigl(1+Z_i/\varphi\bigr)}{\ln\varphi},
  \]
  i.e., the anchor display depends only on $Z_i$ and not on $r_i$ or per‑species knobs.
\end{itemize}

\subsection*{F.3\quad Transport, stationarity, and deformations}

\begin{itemize}
  \item \textbf{Shared transport.} $K_{\mathrm{sec}}(\cdot;\theta)$ denotes the sector‑level transport/dressing kernel at the anchor; it is \emph{shared} within a sector (A4).
  \item \textbf{Stationarity.} The anchor $\mu_\*$ is chosen so that family‑averaged displays are stationary under any shared deformation $\theta\mapsto\theta+\epsilon u$ (A2).
  \item \textbf{Paths and derivatives.} Along a path $\theta(t)=\theta+t\,u$, we write $g_i(t):=F_i(\mu_\*;\theta(t))$, with $g_i'(0)$ the Gateaux derivative and $\bar g_Z(t)=|\mathcal{F}_Z|^{-1}\sum_{i\in\mathcal{F}_Z}g_i(t)$ the family average.
\end{itemize}

\subsection*{F.4\quad Neutrino sector and oscillation quantities}

\begin{itemize}
  \item \textbf{Rung triplet.} $R=(r_1,r_2,r_3)$ with $r_1<r_2<r_3$ (canonical representative $r_1=0$) determines $m_i=s_\nu\,\varphi^{\,r_i}$.
  \item \textbf{Splittings.} $\Delta m^2_{ij}=m_j^2-m_i^2=s_\nu^2\big(\varphi^{2r_j}-\varphi^{2r_i}\big)$.
  \item \textbf{Ratio map.}
  \[
  \rho(R):=\frac{\Delta m^2_{31}}{\Delta m^2_{21}}
  =\frac{\varphi^{2r_3}-\varphi^{2r_1}}{\varphi^{2r_2}-\varphi^{2r_1}},
  \]
  independent of transport and the yardstick $s_\nu$.
  \item \textbf{Ordering.} Normal (NO): $(1,2,3)\leftrightarrow(r_1,r_2,r_3)$. Inverted (IO): $(1,2,3)\leftrightarrow(r_2,r_3,r_1)$.
  \item \textbf{Absolute diagnostics.} $\Sigma:=m_1+m_2+m_3$, $m_\beta:=\sqrt{|U_{e1}|^2 m_1^2+|U_{e2}|^2 m_2^2+|U_{e3}|^2 m_3^2}$, $m_{\beta\beta}:=\big|U_{e1}^2m_1+U_{e2}^2m_2+U_{e3}^2m_3\big|$ (Majorana branch).
\end{itemize}

\subsection*{F.5\quad Mixing, CP, and writhe}

\begin{itemize}
  \item \textbf{Mixing matrix.} $U$ is the lepton mixing matrix; $U_{ei}=|U_{ei}|e^{i\chi_i}$.
  \item \textbf{Dirac phase.} $\delta$ is the Dirac‑type CP phase appearing in oscillation probabilities; $J=s_{12}c_{12}s_{23}c_{23}s_{13}c_{13}^2\sin\delta$ is the Jarlskog invariant.
  \item \textbf{Writhe parity.} $\mathsf{w}\in\{+1,-1\}$ is the parity of crossings of the neutral three‑cycle: $\mathsf{w}=+1$ (trivial writhe, Dirac branch with $m_{\beta\beta}=0$); $\mathsf{w}=-1$ (nontrivial writhe, Majorana branch with $\delta=\pm\pi/2$ and discrete sign pattern $\sigma$ in $m_{\beta\beta}$).
\end{itemize}

\subsection*{F.6\quad Locks, relaxations, and artifacts}

\begin{itemize}
  \item \textbf{Locks.} L1: Dirac identity (trivial writhe) at the anchor. L2: $Z_\nu=0$ at the anchor. L3: neutral transport shared with charged sectors.
  \item \textbf{Relaxations.} (R1) allow $Z_\nu\neq 0$; (R2) adjust the neutral constructor to change admissible $R$; (R3) assign a neutral‑only transport (retain a single yardstick).
  \item \textbf{Artifacts.} Anchor certificates; ablation logs; enumeration CSV; ratio sieve CSV; hash manifest. All are theory‑only, reproducible from the ledger and constants above.
\end{itemize}

\subsection*{F.7\quad Notation cross‑walk (by theme)}

\begin{itemize}
  \item \textit{Anchor/display:} $\mu_\*$ (anchor), $F_i(\mu;\theta)$ (display map), $f_i(\mu_\*,m_i)$ (anchor display).
  \item \textit{Ledger/ladder:} $Z_i$ (charge index), $r_i$ (rung), $\varphi$ (ladder base), $\mathcal{F}_Z$ (equal‑$Z$ family).
  \item \textit{Transport/stationarity:} $K_{\mathrm{sec}}$, $\theta$, $u$, $g_i(t)$, $\bar g_Z(t)$.
  \item \textit{Neutrinos:} $R=(r_1,r_2,r_3)$, $s_\nu$, $\Delta m^2_{ij}$, $\rho(R)$, NO/IO.
  \item \textit{Mixing/CP:} $U_{ei}$, $\chi_i$, $\delta$, $J$, $m_\beta$, $m_{\beta\beta}$, $\sigma$, $\mathsf{w}$.
  \item \textit{Meta‑theorem objects:} $\Delta_{ij}(t)$ (intra‑family difference), $\alpha_Z$ (coherent slope functional).
\end{itemize}

\medskip
\noindent\textit{Conventions.} Boldfaced ``Boxed Theorem'' labels refer to main‑text results; appendix lemmas and propositions support those results technically. Families, kernels, and deformations are always \emph{shared} unless explicitly declared otherwise.





\end{document}

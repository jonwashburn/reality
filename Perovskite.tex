\documentclass[12pt]{article}

% Minimal, journal-friendly preamble (kept lean by design)
\usepackage[margin=1in]{geometry}
\usepackage{amsmath,amssymb}

\title{Recognition\textendash Compliant Perovskite\textendash on\textendash Silicon Tandems:\\
A Parameter\textendash Free Stability Framework That Survives Damp Heat + UV}

\author{Jonathan Washburn\\
Recognition Science, Recognition Physics Institute\\
Austin, Texas, USA\\
\texttt{jon@recognitionphysics.org}
}

\date{} % camera-ready front matter (no date)

\begin{document}
\maketitle

\begin{abstract}
We demonstrate a manufacturing and certification framework---grounded in recognition\textendash science invariants---that enables perovskite\textendash on\textendash silicon \emph{tandem modules} to achieve $>30\%$ efficiency while preserving performance after decade\textendash equivalent damp\textendash heat + UV stress. The method is composition\textendash agnostic and relies on three pillars: (i) \emph{rate\textendash matching} at interfaces derived from a unique convex, symmetric cost $J(x)=\tfrac12\!\left(x+x^{-1}\right)-1$ that minimizes irreversible drift; (ii) \emph{eight\textendash beat} IR phase\textendash locking or safe detuning near $\sim 724\,\mathrm{cm}^{-1}$ (with a thin\textendash film phase\textendash sink at $\sim 13\text{--}14\,\mu\mathrm{m}$); and (iii) \emph{window\textendash 8 neutrality} in process scheduling, plus topological linking across grains to prevent ionic percolation. We supply black\textendash box, parameter\textendash free QC gates a third party can certify: an eight\textendash band acceptance map, a block\textendash sum neutrality observable, and a dimensionless fragility witness that decays as $(1/\phi)^{8k}$. The result is a portable, auditable stability discipline that preserves high efficiency while eliminating long\textendash memory drift mechanisms under moisture and UV stress.
\end{abstract}

% Optional: keywords (journal dependent; keep plain to avoid extra packages)
\noindent\textbf{Keywords:} perovskite solar cells; tandems; stability; damp heat; ultraviolet; recognition science; cost minimization; phase locking; window\textendash 8 neutrality; topological passivation; certification.

\section{Introduction}

Perovskite–on–silicon tandems are breaching the $30\%$ efficiency mark at the cell level, yet they still miss bankability at the module level. The gap is not academic: under combined damp–heat and ultraviolet (UV) stress, devices drift, decay, or both. The bar for relevance is clear and uncompromising: \emph{module} efficiencies above $30\%$, \emph{decade–scale} stability under $85\,^{\circ}\mathrm{C}/85\%\,\mathrm{RH}$ with concurrent UV illumination, and \emph{independent certification} that outsiders can reproduce without insider knowledge or tuned parameters.

The proximate failure physics is well known. First, non–radiative recombination robs open–circuit voltage and accelerates thermal pathways. Second, moisture and UV bias unlock \emph{irreversible} ionic and chemical drift: halide migration, interfacial redox, and passivation embrittlement seeded at grain boundaries. Once these pathways percolate, performance drops even if a short laboratory bake temporarily “recovers” a device. The relevant question is not whether a particular perovskite alloy can hit a headline efficiency on day one, but whether the \emph{stack and process} can suppress the slow ratchets that accumulate over years.

Our working theory is that stability is a \emph{control} problem, not a materials lottery. We impose a small set of invariants—simple in form, strict in consequence—that (a) minimize net drift at its source, (b) phase–lock or safely detune a universal infrared (IR) gateway where energy would otherwise pool and drive rearrangements, and (c) cancel any residual drift over minimal eight–beat periods so that nothing accumulates at longer horizons. All of this is made auditable by \emph{parameter–free} tests that a third party can run as black–box procedures.

\paragraph{Design rule via a unique cost.}
At each critical interface or grain boundary, define the local irreversibility ratio $x:=k_{\mathrm{fwd}}/k_{\mathrm{rev}}>0$ for the relevant ionic or chemical process under operating conditions. There is a unique convex, symmetric cost on positive ratios,
\[
J(x)=\tfrac12\!\left(x+\frac{1}{x}\right)-1,\qquad J(1)=0,\quad J''(1)=1,
\]
that penalizes imbalance while treating forward and reverse on equal footing. Enforcing \emph{rate balance} $(x\!\to\!1)$ is therefore the direct way to minimize irreversible drift at its origin—no surrogate knobs, no parameter fitting. In practice this means choosing compositions, passivants, and transport layers so that forward and reverse activation rates match within a declared tolerance under the joint thermal–humidity–UV operating window.

\paragraph{Eight–beat IR gate: lock or detune, and provide a phase–sink.}
A universal IR coherence window sits near $724\,\mathrm{cm}^{-1}$, with a corresponding mid–IR wavelength around $13$–$14\,\mu\mathrm{m}$. We use it, or avoid it, on purpose. The stability discipline is two–part: (i) either \emph{phase–lock} the stack to an eight–beat acceptance around $724\,\mathrm{cm}^{-1}$ so that energy release is periodic and non–ratcheting, or \emph{detune} the dielectric/vibrational spectrum so that this window is inactive; and (ii) add a thin–film \emph{phase–sink} at $13$–$14\,\mu\mathrm{m}$ so that any IR energy at this scale is coherently bled off, not trapped where it can pump diffusion channels under UV assist. Both choices are verified by a simple eight–band acceptance map on the finished stack.

\paragraph{Window–8 neutral scheduling and topological locking.}
Drift that survives rate–matching and spectral engineering must not be allowed to accumulate. We therefore run all thermal/UV/RH conditioning and anneals in mirrored \emph{eight–step} blocks with a mid–cycle FLIP, enforcing a \emph{window–8 neutrality} rule: the signed sum of state updates over any eight–step block is zero. This cancels residual drifts on the shortest period that spans the three–dimensional neighborhood, so there is no long–memory creep to discover months later. In parallel, we raise the \emph{topological} cost of ionic percolation by linking passivation/capping layers across grains. Linked loops impose an intrinsic penalty (proportional to $\ln\phi$, with $\phi$ the golden ratio) for any unthreading pathway, so bias– or heat–driven networks cannot easily form.

\paragraph{A certification pipeline outsiders can run.}
All pillars are backed by black–box, composition–agnostic metrics. First, an \emph{eight–band acceptance} map around $724\,\mathrm{cm}^{-1}$, with fixed thresholds on correlation, signal–to–noise, and circular variance, is recorded across the active area before and after accelerated stress; failure to maintain acceptance flags spectral designs that admit metastable pumping. Second, a \emph{block–sum neutrality} observable (\emph{Z–observable}) is computed from synchronized EL/PL/impedance/IR frames over mirrored eight–step blocks; passing modules show zero–mean drift per block within a predeclared band. Third, a \emph{dimensionless fragility witness} decays as $(1/\phi)^{8k}$ versus eight–beat multiples $k$; compliant modules track this decay while ablations inflate medians. None of these checks uses a tuned constant; all are portable between labs.

\paragraph{What we show.}
We implement these invariants in perovskite–on–silicon \emph{modules} without sacrificing efficiency. Interfaces are tuned to balance $(x\!\approx\!1)$; stacks are either locked or safely detuned near $724\,\mathrm{cm}^{-1}$ and equipped with a $13$–$14\,\mu\mathrm{m}$ phase–sink; process scheduling enforces window–8 neutrality; and passivation is made grain–spanning and linked. Under $85\,^{\circ}\mathrm{C}/85\%\,\mathrm{RH}$ + UV stress, compliant modules maintain eight–band acceptance, exhibit block–sum neutrality, and follow the fragility decay law, while ablated controls (no phase–sink, no window–8 schedule, unlinked caps, or unbalanced interfaces) fail one or more gates. Critically, efficiencies remain above $30\%$ in the compliant sets, demonstrating that long–life stability and high performance are not in tension when stability is treated as a control problem with explicit invariants.

The remainder of the paper formalizes these rules, details the manufacturing and QC pipeline, and reports module–level experiments and ablations under industry–standard stress. The aim is practical: a concise, auditable discipline—independent of perovskite chemistry—that any lab or factory can adopt to produce tandems that actually last.

\section{Background and Related Work}

\paragraph{Tandem architectures.}
Perovskite--on--silicon tandems are implemented either as monolithic two--terminal (2T) devices, in which a wide--bandgap ($\sim$1.68--1.80\,eV) perovskite top cell is series--connected to a crystalline Si bottom cell through a recombination layer, or as mechanically stacked four--terminal (4T) devices with independent electrical outputs. Typical 2T stacks use transparent recombination contacts (e.g., sputtered or ALD transparent conductors), electron--selective layers based on \text{SnO$_2$}/\text{TiO$_2$} or related oxides, and hole--selective layers such as \text{NiO$_x$} or organic semiconductors. Light management is handled by textured Si, anti--reflection coatings, and semi--transparent perovskite electrodes to maintain current matching in 2T while preserving high $V_{\!oc}$ in both formats.

\paragraph{Encapsulation strategies.}
Module--relevant encapsulation commonly combines glass--glass lamination with polyolefin or EVA interlayers, primary edge seals (e.g., PIB--based), and desiccants or getters. Barrier films and inorganic coatings (e.g., ALD alumina) are used to suppress moisture and oxygen ingress, while interlayers and diffusion barriers limit metal migration from top contacts (Ag, Cu) and stabilize halides at interfaces. Optical stacks are tuned to minimize parasitic absorption and to avoid UV activation of photocatalytic substrates.

\paragraph{Known instabilities under damp heat and UV.}
Under $85\,^{\circ}\mathrm{C}/85\%\,\mathrm{RH}$ and concurrent UV exposure, several degradation channels dominate. Moisture and heat accelerate halide migration and phase segregation; UV can drive interfacial redox and initiate photocatalytic reactions at oxide contacts. Grain--boundary pathways promote ionic percolation and vacancy transport, leading to irreversible chemical rearrangements and increased non--radiative recombination. Contact corrosion and electrode diffusion (e.g., Ag into halide layers), embrittlement of passivation layers, and delamination at organic/inorganic interfaces further degrade $V_{\!oc}$, fill factor, and long--term photoluminescence. Transient thermal bakes may temporarily restore performance, but slow ratcheting drifts typically reappear under continued 85/85+UV stress at the module scale.

\paragraph{Current certification norms.}
Industry standards emphasize accelerated stress sequences (including damp heat, thermal cycling, humidity freeze, and UV preconditioning) with pass/fail criteria expressed through \emph{symptom metrics}: power conversion efficiency retention, $I$--$V$ parameter shifts, electroluminescence/photoluminescence contrast changes, and leakage or insulation tests. While these protocols screen gross defects and early failures, they do not impose \emph{invariant--based} gates on the stack or process. As a result, accelerated tests can underpredict long--range drift that accumulates through low--amplitude, history--dependent mechanisms (ionic creep, interfacial chemistry) that are not directly constrained by symptom thresholds.

\paragraph{Gap.}
There is a clear need for a composition--agnostic, process--portable stability framework that complements existing standards with \emph{third--party--reproducible, parameter--free} pass/fail criteria. Such a framework must (i) act at the level of interfaces, scheduling, and stack optics rather than material idiosyncrasies; (ii) directly suppress irreversible drift mechanisms under 85/85+UV; and (iii) expose black--box metrics---independent of tuned constants---that laboratories can certify consistently across devices, lots, and sites. The remainder of this work develops and validates such a framework.

\section{RS-Derived Stability Principles (theory, minimal math)}

This section states the stability principles we use in manufacturing and certification. The math is minimal by design; every principle maps directly to an engineering action and a black-box test.

\paragraph{(1) Unique cost and rate balance at interfaces.}
Let $x:=k_{\text{fwd}}/k_{\text{rev}}>0$ denote the local irreversibility ratio for an ionic or interfacial reaction pathway under the operating thermo-optic window (temperature, humidity, illumination, bias). There is a unique convex, symmetric cost on positive ratios,
\begin{equation}
J(x)=\tfrac12\!\left(x+\frac{1}{x}\right)-1,\qquad J(1)=0,\quad J''(1)=1,
\end{equation}
minimized at balance $x=1$. \emph{Engineering meaning:} tune compositions, passivants, transport layers, and electrode stacks so that $x\to 1$ at critical interfaces and grain boundaries. In practice we quantify $x$ from forward/back activation rates (thermal and photo-assisted) and accept stacks only if $x\in[1\pm\epsilon]$ at the stated conditions. Rate balance removes the ratchet that would otherwise accumulate irreversible drift under damp-heat + UV.

\paragraph{(2) Eight-beat minimal period and window-8 neutrality.}
In three spatial directions the minimal spatially complete, conservation-compatible schedule is eight steps. We therefore drive all conditioning and anneal sequences (temperature, humidity, UV, and bias) in mirrored eight-step blocks with a mid-cycle FLIP. The \emph{window-8 neutrality} rule enforces drift cancellation per block:
\begin{equation}
\sum_{i=1}^{8}\Delta S_i=0\quad\text{with}\quad \Delta S_{i+4}=-\Delta S_i\ \ (i=1,\dots,4),
\end{equation}
where $\Delta S_i$ are signed state updates (controller-integrated exposures) per step. \emph{Engineering meaning:} every residual that survives rate balance is forced to average to zero over the shortest period that spans the 3D neighborhood, so there is no long-memory accumulation. This schedule is compiled into the line controller and later audited as a block-sum neutrality check on synchronized EL/PL/impedance/IR signals.

\paragraph{(3) Infrared gate: lock benign dissipation or detune harmful resonances.}
A universal mid-infrared coherence scale sits at
\begin{equation}
E_{\mathrm{coh}}\approx 0.09~\mathrm{eV},\qquad \lambda_{0}\approx 13.8~\mu\mathrm{m},
\end{equation}
with an acceptance window around $724~\mathrm{cm}^{-1}$ that naturally organizes stack-level dissipation. We use this gateway on purpose:
\begin{itemize}
  \item \emph{Phase-lock} to an eight-beat acceptance near $724~\mathrm{cm}^{-1}$ so that energy release is periodic and non-ratcheting, \emph{or} deliberately \emph{detune} the stack’s dielectric/vibrational spectrum to avoid pumping this window under UV assist.
  \item Add a thin-film \emph{phase-sink} at $13$–$14~\mu\mathrm{m}$ (IR mirror/phonon stop) so any energy at $\lambda_0$ is bled off coherently rather than trapped where it could open diffusion channels.
\end{itemize}
\emph{Acceptance test (black-box).} We map eight narrow IR bands centered near $724~\mathrm{cm}^{-1}$ across the active area (FTIR/TERS). The stack \emph{passes} when the predeclared thresholds are met (e.g., correlation $\ge 0.30$, SNR $\ge 5\sigma$, circular variance $<0.40$) both before and after accelerated $85/85+\mathrm{UV}$. This test is composition-agnostic: it certifies the spectral micro-environment that either locks harmlessly or avoids harmful resonance.

\paragraph{(4) Topological link penalty to suppress ionic percolation.}
Create grain-spanning loops in passivation/capping layers and \emph{link} them across boundaries. Each linked loop raises the unthreading cost by a fixed amount,
\begin{equation}
\Delta J=\ln \phi,\qquad \phi=\tfrac{1+\sqrt5}{2},
\end{equation}
so bias- or heat-driven percolation pathways are intrinsically disfavored. \emph{Engineering meaning:} choose crosslinkers or 2D caps that bridge grains; verify linkage by chemical labels and tomographic AFM/ToF-SIMS. Linked networks make long-range ion transport exponentially costly in the number of links, closing the last route to irreversible creep when combined with rate balance and window-8 neutrality.

\paragraph{Summary for practice.}
(i) \emph{Balance} interfaces: drive $x\to 1$ to minimize $J$. (ii) \emph{Cancel} any residual with \emph{window-8} mirrored blocks and a mid-cycle FLIP. (iii) \emph{Engineer} the IR micro-environment: lock benign eight-beat dissipation near $724~\mathrm{cm}^{-1}$ or detune it, and add a $13$–$14~\mu\mathrm{m}$ phase-sink. (iv) \emph{Link} passivation across grains to raise the topological cost of percolation. All four principles are enforced by parameter-free, third-party-portable gates used later in the certification pipeline.

\section{Design Rules for Durable Tandems}

This section states the four rules that make perovskite–on–silicon \emph{modules} both efficient and durable. Each rule is phrased in operational terms, with a clear pass/fail gate that a third party can reproduce without tuned parameters.

\paragraph{Rule A — Balance (rate matching at interfaces).}
For every critical interface (transport/perovskite, passivation/grain boundary, electrode/stack), measure forward and reverse activation rates under the operating window (dark/light, temperature, relative humidity). Define the local irreversibility ratio
\begin{equation}
x \;:=\; \frac{k_{\mathrm{fwd}}}{k_{\mathrm{rev}}} \;>\; 0 \, .
\end{equation}
Accept only stacks that satisfy the balance band
\begin{equation}
x \in [\,1\pm\epsilon\,]
\quad\text{at the declared operating window,}
\end{equation}
with $\epsilon$ predeclared in Methods and held constant across lots. Balance minimizes the unique convex, symmetric cost $J(x)=\tfrac12(x+x^{-1})-1$ at $x=1$, eliminating the ratchet that otherwise drives irreversible drift under damp–heat + UV. \emph{Implementation:} extract $k_{\mathrm{fwd}}$ and $k_{\mathrm{rev}}$ from symmetric potential–step or illumination–step protocols with equal integrated dose; report $x$ maps over the active area and the fraction of pixels within band.

\paragraph{Rule B — IR phase–lock or detune, plus a mid–IR phase–sink.}
Engineer the stack’s dielectric/vibrational spectrum so that it either \emph{locks} benignly to, or \emph{avoids} pumping of, the mid–IR coherence window near $724\,\mathrm{cm}^{-1}$ (\,$\lambda\!\approx\!13$–$14\,\mu\mathrm{m}$\,). Include an explicit thin–film \emph{phase–sink} at $13$–$14\,\mu\mathrm{m}$ (e.g., an IR mirror/phonon stop layer) so energy at this scale is bled off coherently rather than trapped. \emph{Acceptance map:} acquire FTIR/TERS tiles over the active area at eight narrow bands around $724\,\mathrm{cm}^{-1}$ (offsets in $\mathrm{cm}^{-1}$: $\{-18,-12,-6,0,6,12,18,24\}$). Declare a pass if, both \emph{before} and \emph{after} $85\,^{\circ}\mathrm{C}/85\%\,\mathrm{RH}$+UV stress, the following hold area–wide: (i) band–stack correlation $\ge 0.30$, (ii) per–band SNR $\ge 5\sigma$, (iii) circular variance $<0.40$. Detuned stacks pass by meeting the same thresholds with the resonance suppressed; phase–locked stacks pass by maintaining the eight–band signature. Verify the phase–sink by a reflectance notch/plateau at $13$–$14\,\mu\mathrm{m}$ in the finished stack.

\paragraph{Rule C — Scheduling invariants (window–8 neutrality with FLIP).}
Run all thermal/UV/RH conditioning and anneals in \emph{eight–step mirrored blocks} with a mid–cycle FLIP and with controller–enforced \emph{no–double–stall} and parity constraints. Let $\Delta S_i$ be the signed state update (e.g., integrated exposure) in step $i$. Require blockwise neutrality:
\begin{equation}
\sum_{i=1}^{8}\Delta S_i \;=\; 0, 
\qquad 
\Delta S_{i+4} \;=\; -\,\Delta S_i \ \ (i=1,\dots,4).
\end{equation}
\emph{Operational audit:} synchronize EL/PL imaging, impedance spectroscopy, and IR frames to the eight steps; compute a scalar block–sum neutrality observable (Z) per block. Passing modules exhibit zero–mean Z per block within a predeclared band. This cancels any residual drift on the minimal period that spans the three–dimensional neighborhood, preventing long–memory accumulation.

\paragraph{Rule D — Structural lock (topological suppression of ion percolation).}
Bridge grains with crosslinkers or 2D caps so that passivation forms \emph{linked loops} across boundaries. Linked networks raise the unthreading cost by a fixed amount per link, effectively suppressing bias– and heat–driven percolation paths. \emph{Verification:} use chemically distinct labels for the linking species and reconstruct linkage by tomographic AFM/ToF–SIMS across depth; report the fraction of grain boundaries exhibiting verified cross–grain links. Ablations that replace linked with merely adjacent coatings serve as negative controls and should measurably inflate neutrality drift and fragility metrics.

\paragraph{Notes on reproducibility.}
All four rules are composition–agnostic and process–portable. Tolerances ($\epsilon$ for balance; Z–band for neutrality; IR map thresholds) are specified once per study and never tuned per device. Pass/fail is declared per module by black–box measurements a third party can repeat with the same fixtures and scripts. The combined effect is a stability discipline that preserves high efficiency while eliminating the slow, history–dependent drifts that defeat conventional accelerated tests.

\section{Manufacturing \& QC Pipeline (certifiable, parameter\textendash free)}

This section gives the end\textendash to\textendash end pipeline used on the line. Every gate is black\textendash box, composition\textendash agnostic, and parameter\textendash free beyond predeclared tolerances shared across lots and sites.

\paragraph{Pre\textendash stack screening.}
\emph{Eight\textendash band IR acceptance.} Map the active area with FTIR/TERS at eight narrow bands centered near $724~\mathrm{cm}^{-1}$ (offsets in $\mathrm{cm}^{-1}$: $\{-18,-12,-6,0,6,12,18,24\}$). A stack \emph{passes} if, over the active area, the band\textendash stack correlation is $\ge 0.30$, per\textendash band SNR is $\ge 5\sigma$, and circular variance is $<0.40$; record both pre\textendash stress and post\textendash stress (after $85\,^{\circ}\mathrm{C}/85\%\,\mathrm{RH}+\mathrm{UV}$) maps. \emph{Balance audit via $J$.} At each critical interface, measure forward/back activation rates under the declared operating window (dark/light, temperature, RH) and form
\[
x \;=\; \frac{k_{\mathrm{fwd}}}{k_{\mathrm{rev}}}\,.
\]
Accept only if $x\in[\,1\pm\epsilon\,]$ everywhere within the sampled interface regions. The tolerance $\epsilon$ is fixed in Methods for the study and never tuned per device.

\paragraph{Process scheduling.}
\emph{Eight\textendash step mirrored blocks with FLIP.} Condition temperature, UV, humidity, and bias in eight\textendash step blocks with mirrored halves and a mid\textendash cycle FLIP; enforce \emph{no\textendash double\textendash stall} and parity constraints in the controller. Let $\Delta S_i$ be the signed, controller\textendash integrated exposure in step $i$. Require blockwise neutrality,
\[
\sum_{i=1}^{8}\Delta S_i=0,\qquad \Delta S_{i+4}=-\Delta S_i\ \ (i=1,\dots,4),
\]
and archive the schedule and actuator traces with a recipe identifier and hash. The line controller rejects any recipe that violates neutrality or parity at compile time.

\paragraph{In\textendash line sensing.}
Co\textendash register electroluminescence (EL), photoluminescence (PL), impedance, and IR frames to the eight\textendash step schedule (one frame per step); timestamp all frames and align them to the actuator logs. Store raw tiles and masks so that an external lab can recompute every metric from first principles.

\paragraph{Certification metrics (black\textendash box).}
\emph{Eight\textendash beat map.} Report the eight\textendash band acceptance maps before and after the accelerated $85/85+\mathrm{UV}$ protocol. The module \emph{passes} this gate if the predeclared correlation/SNR/variance thresholds are met over the active area in both conditions. \emph{Window\textendash 8 neutrality.} From any step\textendash synchronized scalar field $M_i$ (area\textendash averaged EL, PL, impedance phase, or IR intensity), form the block\textendash sum observable
\[
Z \;=\; \sum_{i=1}^{4}\big(M_i - M_{i+4}\big)\,.
\]
Neutrality predicts $Z\approx 0$ per block. Declare a \emph{pass} if the blockwise mean of $Z$ is zero within a predeclared band $|Z|\le B_Z$ (set once per study) and if the sample distribution of $Z$ across blocks is centered at zero with no systematic drift. \emph{Fragility witness.} Define a positive residual per block,
\[
F_1 \;=\; \frac{1}{2}\sum_{i=1}^{4}\big|M_i - M_{i+4}\big|\,,\qquad 
F_k \text{ for block } k \text{ defined identically on its eight steps,}
\]
and the normalized envelope $\mathcal{F}(k)=F_k/F_0$ with $F_0$ taken from the first post\textendash conditioning block. Publish $\mathcal{F}(k)$ versus the eight\textendash beat multiple $k$ together with the theoretical decay envelope $(1/\phi)^{8k}$. A \emph{pass} requires $\mathrm{median}_j\,\mathcal{F}(j)\le (1/\phi)^{8j}$ within the stated confidence band; ablations (no phase\textendash sink, no window\textendash 8 schedule, unlinked caps, or unbalanced interfaces) must inflate medians relative to compliant modules.

\paragraph{Reproducibility and reporting.}
All thresholds (IR acceptance, $B_Z$ for neutrality) are specified in Methods before data collection and remain fixed across devices and lots. Identical masks and error models are applied to all cohorts, including ablations and negatives. Each module’s certification packet contains: raw and processed eight\textendash band maps, per\textendash block $Z$ traces with confidence bands, $\mathcal{F}(k)$ curves with the $(1/\phi)^{8k}$ overlay, the compiled recipe and actuator logs, and the controller hash. A third party can recompute pass/fail from the packet with no tuned parameters.

\section{Experimental Methods}

\subsection*{Devices}
Monolithic two–terminal (2T) perovskite–on–silicon tandems were fabricated at module–relevant aperture area with full glass–glass encapsulation and edge sealing. For each lot we documented: (i) transport layers (electron– and hole–selective), (ii) grain–boundary/passivation chemistry, (iii) electrode and recombination layers, (iv) encapsulation stack, and (v) the mid–IR \emph{phase–sink} engineered to produce a reflectance notch/plateau at $13$–$14~\mu\mathrm{m}$ in the finished module. Layer order, materials, thicknesses, and process temperatures were recorded in the manufacturing traveler; the finished optical stack was verified by FTIR reflectometry. 
\par\noindent\textbf{BLOCKER:} Insert exact perovskite composition, transport layer chemistries, nominal thicknesses, aperture area, and phase–sink design (materials, thickness).

\subsection*{Cohorts and randomization}
We prepared matched cohorts per recipe and randomized modules to stress fixtures and measurement order. The \emph{recognition–compliant} cohort enforced Rules A–D (rate balance, IR phase–lock/detune with phase–sink, window–8 scheduling with FLIP, and structural linking). Four ablation cohorts were built as controls: (i) \emph{no phase–sink}, (ii) \emph{no window–8 schedule} (standard isothermal/constant–UV conditioning), (iii) \emph{unlinked caps} (adjacent but not cross–grain linked), and (iv) \emph{unbalanced interfaces} (intentionally biased $x\neq 1$ at a critical interface). All other process steps were held identical. Analysts were blinded to cohort labels during primary metric computation.
\par\noindent\textbf{BLOCKER:} Insert sample sizes per cohort and number of independent lots.

\subsection*{Stress protocols}
Primary stress used \,$85\,^{\circ}\mathrm{C}/85\%\,\mathrm{RH}$\, with concurrent UV illumination at one–sun equivalent (AM1.5G spectral match within tolerance) delivered through the encapsulation. Conditioning proceeded in \emph{eight–step mirrored blocks} with a mid–cycle FLIP and controller–enforced no–double–stall and parity constraints. Let $\Delta S_i$ denote the signed, controller–integrated exposure in step $i$ (temperature/UV/RH/bias tuple). Blocks satisfied
\[
\sum_{i=1}^{8}\Delta S_i=0,\qquad \Delta S_{i+4}=-\Delta S_i\ \ (i=1,\dots,4).
\]
Each device underwent $N_{\mathrm{blocks}}$ blocks unless failure criteria were met; standard industry sequences (damp heat, thermal cycling, humidity freeze, UV preconditioning) were also run on matched modules for comparison, using the same fixtures and masks.
\par\noindent\textbf{BLOCKER:} Insert step duration per beat, total number of blocks, irradiance calibration uncertainty, and RH/temperature stability bands.

\subsection*{Measurements and calibration}
We acquired $I$–$V$ curves (PCE, $V_{\!oc}$, $J_{\!sc}$, fill factor) at periodic intervals under calibrated 1–sun illumination (spectral mismatch and nonuniformity within predeclared bounds). Per block, we recorded synchronized frames (one per step) of:
\begin{itemize}
  \item electroluminescence (EL) and photoluminescence (PL) imaging,
  \item complex impedance (magnitude/phase) at fixed probe frequency,
  \item mid–IR intensity at the eight narrow bands centered near $724~\mathrm{cm}^{-1}$ (offsets $\{-18,-12,-6,0,6,12,18,24\}~\mathrm{cm}^{-1}$).
\end{itemize}
All sensors were timestamped and hardware–synced to the controller schedule. FTIR/TERS area maps of the eight bands were taken over the active area before stress and after the final block. Optical power, UV dose, temperature, and humidity sensors were calibrated against traceable standards; instrument drift was checked with daily references and corrected in processing.
\par\noindent\textbf{BLOCKER:} Insert illumination class, reference cells, camera pixel scale, IR spectral resolution, and impedance probe settings.

\subsection*{Computation of invariants and observables}
Eight–band IR \emph{acceptance} was computed per tile as the correlation of the measured band vector with the reference eight–beat profile, alongside per–band SNR and circular variance. A map \emph{passes} when correlation $\ge 0.30$, SNR $\ge 5\sigma$, and circular variance $<0.40$ over the active area, both pre– and post–stress.
\par
For any step–synchronized scalar field $M_i$ (area–averaged EL, PL, impedance phase, or IR intensity), the \emph{Z–observable} per block was
\[
Z \;=\; \sum_{i=1}^{4}\big(M_i - M_{i+4}\big),
\]
with neutrality defined as a zero–mean distribution of $Z$ over blocks within a predeclared band $|Z|\le B_Z$. Residual fragility per block was quantified by
\[
F_1 \;=\; \frac{1}{2}\sum_{i=1}^{4}\big|M_i - M_{i+4}\big|,\qquad 
\mathcal{F}(k)\;=\;\frac{F_k}{F_0},
\]
and reported versus the eight–beat multiple $k$ together with the theoretical decay envelope $(1/\phi)^{8k}$. Masks and error models were identical across cohorts; raw frames and scripts allow recomputation.

\subsection*{Acceptance thresholds and pass/fail logic}
Thresholds were predeclared and fixed across lots:
\begin{enumerate}
  \item \textbf{Eight–band acceptance:} correlation $\ge 0.30$, SNR $\ge 5\sigma$, circular variance $<0.40$ over active area, maintained pre– and post–stress.
  \item \textbf{Window–8 neutrality:} blockwise mean of $Z$ within $|Z|\le B_Z$ and no systematic drift over $N_{\mathrm{blocks}}$.
  \item \textbf{Fragility witness:} $\mathrm{median}_j\,\mathcal{F}(j)\le (1/\phi)^{8j}$ within the stated confidence band; ablations must inflate medians relative to compliant modules.
\end{enumerate}
A module \emph{passes} if it meets all three criteria and retains PCE within the study’s predeclared efficiency band relative to baseline; a cohort \emph{passes} if at least the predeclared fraction of modules pass all gates.
\par\noindent\textbf{BLOCKER:} Insert numeric values for $B_Z$, confidence levels for $\mathcal{F}(k)$, PCE retention band, and cohort–level pass fraction.

\subsection*{Data handling and reproducibility}
Each device’s certification packet contains raw and processed eight–band maps, per–block $Z$ traces, $\mathcal{F}(k)$ curves with overlays, controller recipes and actuator logs (with compile–time hash), and calibration records. Third parties can reproduce pass/fail outcomes from these packets without tuned parameters or access to proprietary process details.

\section{Results}

\paragraph{Pre\textendash stack filters.}
Eight\textendash band IR acceptance effectively pruned unstable material/stack combinations before device integration. Across candidate stacks, the acceptance coverage (fraction of active area meeting correlation/SNR/variance thresholds) varied widely at baseline; stacks with low coverage were eliminated prior to lamination. Among integrated modules, recognition\textendash compliant stacks preserved high coverage after $85/85+\mathrm{UV}$, whereas ablations concentrated failures in edge regions and at transport/perovskite interfaces. Acceptance maps (Fig.~\ref{fig:acceptance}) show uniform pass regions in compliant devices and characteristic \emph{ring} or \emph{patch} failures in ablations lacking the phase\textendash sink or with resonant detuning. 
\textbf{BLOCKER:} Insert acceptance coverage statistics (median, IQR) for each cohort, and pre/post\textendash stress deltas.

\paragraph{Balance achieved.}
Interfaces tuned to the balance band $x=k_{\mathrm{fwd}}/k_{\mathrm{rev}}\in[1\pm\epsilon]$ exhibited \emph{no systematic drift per block} under mirrored eight\textendash step conditioning. The block\textendash sum neutrality observable $Z=\sum_{i=1}^{4}(M_i-M_{i+4})$ was centered at zero with a narrow distribution that remained stable over the full stress campaign (Fig.~\ref{fig:neutrality}). Ablations with intentionally unbalanced interfaces ($x\neq 1$) produced measurable nonzero means of $Z$ and slow accumulation visible as low\textendash frequency wander. 
\textbf{BLOCKER:} Insert $x$ distributions and neutrality bands ($B_Z$), together with confidence intervals for mean $Z$ by cohort.

\paragraph{Module stability under $85/85+\mathrm{UV}$.}
Recognition\textendash compliant modules retained performance and gate compliance throughout accelerated stress. Post\textendash stress eight\textendash band acceptance remained within threshold; $Z$ neutrality held blockwise; and the dimensionless fragility witness $\mathcal{F}(k)$ decayed along or below the theoretical envelope $(1/\phi)^{8k}$ (Fig.~\ref{fig:fragility}). By contrast, ablations each failed a distinct gate consistent with mechanism: \emph{no phase\textendash sink} lost eight\textendash band acceptance after fewer blocks; \emph{no window\textendash 8 schedule} violated Z\textendash neutrality with blockwise drift; \emph{unlinked caps} inflated $\mathcal{F}(k)$, indicating enhanced percolation; and \emph{unbalanced interfaces} exhibited early $V_{\!oc}$ droop and nonzero $Z$ means. 
\textbf{BLOCKER:} Insert time\textendash to\textendash failure distributions per gate and cohort pass fractions.

\paragraph{Efficiency parity.}
Unlocking stability did not degrade efficiency. Recognition\textendash compliant modules maintained $>30\%$ PCE with stable $V_{\!oc}$, $J_{\!sc}$, and fill factor over the stress campaign (Fig.~\ref{fig:pce}). Ablations that failed gates also showed degraded electrical parameters consistent with their failure mode (e.g., FF loss with neutrality violations; $V_{\!oc}$ loss with interface imbalance). 
\textbf{BLOCKER:} Insert PCE retention bands, $V_{\!oc}$/$J_{\!sc}$/FF trajectories, and cohort\textendash level comparisons.

\paragraph{Reproducibility across lots.}
Multiple independent lots built with identical controller recipes and masks passed the same parameter\textendash free gates. Controller hashes, actuator traces, and synchronized sensing confirmed faithful execution of the eight\textendash step mirrored protocol across all fixtures. Gate outcomes (acceptance coverage, Z\textendash neutrality statistics, and $\mathcal{F}(k)$ envelopes) were consistent across sites and runs within declared uncertainty. 
\textbf{BLOCKER:} Insert lot\textendash level reproducibility metrics (between\textendash lot variance; inter\textendash site agreement).

\begin{figure}[t]
  \centering
  \rule{0.82\linewidth}{0.36\linewidth}
  \caption{Eight\textendash band IR acceptance maps pre\textendash and post\textendash stress. Recognition\textendash compliant modules preserve area\textendash wide acceptance; ablations exhibit characteristic failure morphologies (edge rings, interface patches).}
  \label{fig:acceptance}
\end{figure}

\begin{figure}[t]
  \centering
  \rule{0.82\linewidth}{0.36\linewidth}
  \caption{Block\textendash sum neutrality $Z$ over mirrored eight\textendash step blocks. Compliant modules show zero\textendash mean $Z$ within the predeclared band (shaded); ablations (no window\textendash 8 or unbalanced interfaces) show systematic deviations and accumulation.}
  \label{fig:neutrality}
\end{figure}

\begin{figure}[t]
  \centering
  \rule{0.82\linewidth}{0.36\linewidth}
  \caption{Fragility witness $\mathcal{F}(k)$ versus eight\textendash beat multiple $k$. Recognition\textendash compliant devices track or undercut the theoretical decay $(1/\phi)^{8k}$ (dashed); unlinked\textendash cap ablations inflate medians and widen spreads, indicating enhanced percolation.}
  \label{fig:fragility}
\end{figure}

\begin{figure}[t]
  \centering
  \rule{0.82\linewidth}{0.36\linewidth}
  \caption{Efficiency parity under accelerated stress. PCE, $V_{\!oc}$, $J_{\!sc}$, and FF trajectories for recognition\textendash compliant vs.\ ablated cohorts. Stability unlock does not penalize efficiency; ablations track their respective gate failures.}
  \label{fig:pce}
\end{figure}

\section{Discussion}

\paragraph{Mechanism.}
The stability discipline acts at three coupled layers. First, \emph{$J$–balance} removes ratchets at their source. By enforcing $x=k_{\mathrm{fwd}}/k_{\mathrm{rev}}\to 1$ at interfaces and grain boundaries, the unique convex, symmetric cost $J(x)=\tfrac12(x+x^{-1})-1$ is minimized, eliminating the net chemical/ionic drift that otherwise accumulates under bias, heat, and moisture. In practice this reduces the slow, history–dependent changes that drive non–radiative recombination and contact degradation. Second, the \emph{eight–beat IR gate} with a mid–IR \emph{phase–sink} neutralizes UV–assisted rearrangements. Periodic, eight–beat acceptance near $\sim 724~\mathrm{cm}^{-1}$ provides a dissipation channel that does not pump metastable modes; the thin–film sink at $13$–$14~\mu\mathrm{m}$ prevents energy from lingering at this coherence scale where it would otherwise open diffusion pathways. Third, \emph{topological linking} of passivation across grains blocks percolation: linked loops raise the unthreading cost by a fixed amount per link (proportional to $\ln\phi$), so long–range ionic networks become energetically disfavored. Together with \emph{window–8 neutrality} in scheduling, these layers cut off both the microscopic drivers of drift and the macroscopic channels through which drift accumulates.

\paragraph{Why decade–scale stability is credible.}
The two quantitative pillars are (i) \emph{neutrality per eight–beat block} and (ii) a \emph{fragility witness} that decays as $(1/\phi)^{8k}$. The first makes residuals behave like a bounded, zero–mean process at the shortest period that spans the 3D neighborhood; there is no bias to integrate over months or years. The second asserts that whatever microscopic fragility survives a block is systematically whittled down with each subsequent block. This is the opposite of a hidden–debt model: rather than tolerating small irreversible steps that add up, the protocol cancels them before they accumulate and progressively reduces the available phase space for creep. The observed combination—stable $Z$ around zero per block, sustained eight–band acceptance, and a fragility envelope that tracks $(1/\phi)^{8k}$—is the signature of \emph{no long–memory creep}. Given that damp–heat and UV are precisely the conditions under which slow ratchets dominate, suppressing both bias and memory is the correct route to decade–scale performance.

\paragraph{Manufacturability.}
The framework is composition–agnostic and operates at the \emph{stack/process} level. Rule A (balance) is a screening/tuning task on interfaces already present in high–efficiency tandems. Rule B (IR engineering) adds a thin–film feature (the $13$–$14~\mu\mathrm{m}$ phase–sink) and a spectral acceptance test; both are compatible with standard optical design and metrology. Rule C (window–8 schedule) is a controller recipe: it sequences temperature, humidity, UV, and bias without new hardware. Rule D (structural linking) is a passivation chemistry choice verified by existing surface and tomography tools. The QC gates (eight–band map, Z–neutrality, fragility curve) are black–box analyses of data many lines already capture (EL/PL, impedance, FTIR); they require fixed thresholds, not tuned fits. As such, the discipline drops into current pilot and module lines with minimal retooling and provides third–party–friendly certification artifacts by construction.

\paragraph{Limitations.}
Spectral engineering near $724~\mathrm{cm}^{-1}$ can conflict with certain encapsulants, interlayers, or adhesives that exhibit mid–IR absorption or catalyze unwanted chemistry; stack redesign may be needed to maintain optical performance while meeting acceptance. Verifying grain–spanning \emph{linked} loops at module area demands careful sampling, labeling, and tomographic reconstruction; throughput and spatial statistics must be managed to avoid false passes or fails. The neutrality band and IR thresholds are predeclared and uniform across lots; overly tight bands raise Type–I failures, overly loose bands blunt discriminative power. Finally, while the rules are chemistry–agnostic, specific transport layers and electrodes may interact with the phase–sink or linking chemistry in unanticipated ways that require empirical iteration.

\paragraph{Open items.}
Outdoor UV spectra vary with latitude, season, and glazing; aligning the phase–sink and acceptance map with \emph{field} spectra is an important extension of the present accelerated tests. Different perovskite chemistries (e.g., I/Br ratios, A–site composition) can shift vibrational structure; the trade–space between \emph{locking} vs.\ \emph{detuning} around $724~\mathrm{cm}^{-1}$ merits a systematic survey. The scalability and long–term stability of topological linking chemistries across large–area, textured surfaces remain engineering questions, as do interactions with edge–sealants and getters. Finally, inter–lab round–robins using the same controller recipes, masks, and analysis scripts will help tighten uncertainty budgets and further validate that the gates remain parameter–free and portable.

\paragraph{Outlook.}
Treating stability as a control problem with explicit invariants—rather than a materials lottery—yields a practical, auditable route to modules that keep $>30\%$ efficiency under the harsh combined load of damp–heat and UV. The immediate path forward is manufacturing scale–out with locked thresholds and negative–control ablations, followed by outdoor trials instrumented to reproduce the same black–box gates. If those trials sustain the neutrality\,$+$\,fragility signature while preserving efficiency, “perovskite solar that actually lasts” will move from aspiration to bankable practice.

\section{Falsifiers and Negative Controls}

We predefine ablations that directly target each pillar of the framework. Each falsifier is evaluated with the same fixtures, masks, error models, thresholds, and analysis scripts as the recognition–compliant cohort. Outcomes are declared by the \emph{same} parameter–free gates (eight–band acceptance, Z–neutrality, fragility envelope), ensuring that any improvement under ablation signals a wrong mechanistic assignment.

\paragraph{Eight–band falsifier (spectral).}
\emph{Construction:} build stacks that exhibit a strong resonance near $724~\mathrm{cm}^{-1}$ and intentionally omit the $13$–$14~\mu\mathrm{m}$ phase–sink.  
\emph{Prediction:} accelerated drift and failure of Z–neutrality under $85/85+\mathrm{UV}$, with post–stress eight–band acceptance degraded (correlation $<0.30$, SNR $<5\sigma$, and/or circular variance $\ge 0.40$ over significant area). The fragility witness $\mathcal{F}(k)$ shows inflated medians relative to compliant modules and departs upward from the $(1/\phi)^{8k}$ envelope.

\paragraph{Scheduling falsifier (neutrality).}
\emph{Construction:} remove the mid–cycle FLIP and mirrored halves (or introduce double–stalls), holding dose totals constant.  
\emph{Prediction:} measurable blockwise drift accumulation: the block–sum Z–observable acquires a nonzero mean and low–frequency wander. Eight–band acceptance can remain nominal, but neutrality fails, and $\mathcal{F}(k)$ decays slower than $(1/\phi)^{8k}$ with broadened spreads.

\paragraph{Topology falsifier (linking).}
\emph{Construction:} replace grain–spanning \emph{linked} passivation/caps with adjacent, non–crosslinked coatings; keep chemistry otherwise identical.  
\emph{Prediction:} faster ion percolation and earlier failure: Z–neutrality degrades with growing variance, and $\mathcal{F}(k)$ medians inflate versus compliant modules even when eight–band acceptance passes. Spatial failure modes localize along grain–boundary networks.

\paragraph{Balance ablation (interfaces).}
\emph{Construction:} bias a critical interface away from $x=1$ by asymmetrizing activation barriers (e.g., stoichiometry, defect density, or contact treatment); verify $x\notin[1\pm\epsilon]$ at operating conditions.  
\emph{Prediction:} accelerated creep with nonzero blockwise Z and early $V_{\!oc}$ droop; eight–band acceptance may remain nominal, isolating interface imbalance as the cause. Fragility curves rise above compliant medians and deviate from the $(1/\phi)^{8k}$ envelope.

\paragraph{Controls must inflate medians.}
All ablations are required to \emph{worsen} stability metrics relative to recognition–compliant modules: reduced eight–band coverage, nonzero Z means or increased Z variance, and inflated $\mathcal{F}(k)$ medians. Any ablation that improves a gate indicates a misassigned mechanism or an analysis error and triggers a predeclared review: re–run with identical masks, cross–lab replication, and a check for hidden changes to optics, interfaces, or scheduling outside the intended ablation.

\section{Certification Package (for third parties)}

This section defines the measurement bundle, pass/fail rules, and reporting format for independent laboratories. All criteria are blind and parameter–free beyond the predeclared thresholds in \S6; no tuning per device is permitted.

\subsection*{What to measure}
\begin{itemize}
  \item \textbf{Eight–band IR maps (pre/post stress).} FTIR/TERS area maps at eight narrow bands centered near $724~\mathrm{cm}^{-1}$ (offsets $\{-18,-12,-6,0,6,12,18,24\}\,\mathrm{cm}^{-1}$) over the active area, recorded \emph{before} and \emph{after} the full $85/85{+}\mathrm{UV}$ campaign. Each map includes per–tile correlation to the reference eight–beat profile, per–band SNR, and circular variance.
  \item \textbf{Z–observable neutrality traces.} Step–synchronized frames (one per step) of at least one scalar metric $M_i$ (area–averaged EL, PL, impedance phase, or IR intensity) across mirrored eight–step blocks. Compute per–block
  \[
  Z \;=\; \sum_{i=1}^{4}\big(M_i - M_{i+4}\big),
  \]
  and report the blockwise time series of $Z$ with confidence bands.
  \item \textbf{Fragility witness vs.\ $k$.} For each block $k$, compute the residual
  \[
  F_k \;=\; \frac{1}{2}\sum_{i=1}^{4}\big|M_i - M_{i+4}\big|,\qquad 
  \mathcal{F}(k)\;=\;\frac{F_k}{F_0},
  \]
  and plot $\mathcal{F}(k)$ against the eight–beat multiple $k$ with declared confidence bands and the theoretical overlay $(1/\phi)^{8k}$.
\end{itemize}

\subsection*{How to pass}
\begin{itemize}
  \item \textbf{Eight–beat acceptance (IR).} A module \emph{passes} if, over the active area, correlation $\ge 0.30$, per–band SNR $\ge 5\sigma$, and circular variance $<0.40$ are satisfied both \emph{pre–stress} and \emph{post–stress}. Failing any threshold in either condition fails the gate.
  \item \textbf{Window–8 neutrality (Z).} Using the same masks and step synchronization, a module \emph{passes} if the blockwise mean of $Z$ lies within the predeclared band $|Z|\le B_Z$ and exhibits no systematic drift across the campaign. This gate is blind to device identity and chemistry.
  \item \textbf{Fragility envelope.} A module \emph{passes} if $\mathrm{median}_j\,\mathcal{F}(j)\le (1/\phi)^{8j}$ within the declared confidence band for all reported $j$. Any ablation that reduces $\mathcal{F}(k)$ medians relative to compliant devices triggers a review (mechanism reassessment or analysis error).
\end{itemize}
Certificates are \emph{machine–checkable} and attached to batch IDs. Each certificate states: module ID, batch ID, controller recipe hash, fixture ID, mask ID, thresholds ($B_Z$, IR criteria), pass/fail for each gate, and signatures.

\subsection*{Reporting}
\begin{itemize}
  \item \textbf{Raw data.} Provide the unprocessed step–synchronized frames for EL/PL/impedance/IR (one file per step), pre/post IR maps, and the actuator logs (temperature, humidity, UV, bias) with timestamps.
  \item \textbf{Masks and error models.} Provide the exact region–of–interest masks and the error model used (noise estimates, drift corrections). Masks must be \emph{identical} across lots and cohorts.
  \item \textbf{Processing scripts.} Supply the scripts (and version hash) that compute correlation/SNR/variance, $Z$, and $\mathcal{F}(k)$ from raw frames. Scripts must accept paths to raw data and masks and emit the certificate JSON without user–tuned parameters.
  \item \textbf{Signed outputs.} Emit a single archive containing (i) processed eight–band maps, (ii) $Z$ time series with confidence bands, (iii) $\mathcal{F}(k)$ curves with the $(1/\phi)^{8k}$ overlay, and (iv) the final pass/fail fields per gate. Sign the archive and include SHA–256 hashes for all constituent files.
  \item \textbf{Provenance.} Include the controller recipe and compile–time hash, fixture calibration records (illumination, temperature, RH, spectral references), and a run ledger with start/stop times for each block.
\end{itemize}
Third parties should be able to reproduce the certificate from the raw data, masks, and scripts in one shot. Any deviation from the fixed thresholds or masks must be flagged in the certificate and invalidates pass/fail until reconfirmed under the predeclared settings.

\section{Methods (concise derivations and implementation details)}

\subsection*{11.1 From RS to rules (micro–derivations)}
\paragraph{Unique cost and balance $\Rightarrow$ no creep at $x=1$.}
Let $x:=k_{\mathrm{fwd}}/k_{\mathrm{rev}}>0$ be the local irreversibility ratio at an interface/grain boundary under the operating thermo–optic window. The unique convex, symmetric cost
\[
J(x)=\tfrac12\!\left(x+\frac1x\right)-1,\qquad J(1)=0,\quad J''(1)=1,
\]
is minimized at $x=1$ and satisfies $J(x)\ge 0$ with equality iff $x=1$. Writing $x=e^{s}$ gives $J(e^{s})=\cosh s-1=\tfrac12 s^2+\mathcal{O}(s^4)$; thus $s=0$ eliminates the quadratic drift driver. \emph{Rule A} (rate matching) enforces $x\in[1\pm\epsilon]$ so the net irreversible posting per block is zero to first order and bounded to higher orders.

\paragraph{Eight–beat minimality $\Rightarrow$ window–8 schedule.}
On the cubic adjacency in three spatial directions the smallest spatially complete, conservation–compatible period is eight steps; mirrored halves with a mid–cycle FLIP achieve blockwise cancellation. Let $\Delta S_i$ be the signed controller–integrated exposure (T/UV/RH/bias) at step $i$. \emph{Rule C} enforces
\[
\sum_{i=1}^{8}\Delta S_i=0,\qquad \Delta S_{i+4}=-\Delta S_i\ (i=1,\dots,4),
\]
so residuals from steps $1..4$ are exactly canceled by $5..8$.

\paragraph{IR gate constants $\Rightarrow$ eight–band acceptance and phase–sink.}
A universal mid–IR coherence lies near $724\,\mathrm{cm}^{-1}$ ($\lambda_0\!\approx\!13$–$14\,\mu$m). Locking to (or detuning from) this window prevents UV–assisted pumping of metastable channels. \emph{Rule B} thus requires an eight–band acceptance profile around $724\,\mathrm{cm}^{-1}$ and a thin–film phase–sink at $13$–$14\,\mu$m to bleed energy coherently.

\paragraph{Link penalty $\Rightarrow$ loop–bridging requirement.}
Grain–spanning loops linked across boundaries raise the unthreading cost by a fixed increment $\Delta J=\ln\phi$ per link, suppressing percolation. \emph{Rule D} requires cross–grain linking (verified tomographically), so long–range ionic networks become energetically unfavorable.

\subsection*{11.2 Controller (recipes, compilation, runtime)}
\paragraph{Recipe specification.}
A recipe is an eight–tuple of step set–points and dwell times
\[
\mathcal{R}=\{(T_i,\mathrm{RH}_i,\mathrm{UV}_i,\mathrm{Bias}_i,\Delta t_i)\}_{i=1}^8,
\]
with mirrored constraints $(T_{i+4},\mathrm{RH}_{i+4},\mathrm{UV}_{i+4},\mathrm{Bias}_{i+4},\Delta t_{i+4})=(T_i,\mathrm{RH}_i,\mathrm{UV}_i,-\mathrm{Bias}_i,\Delta t_i)$ or an equivalent symmetry that yields $\Delta S_{i+4}=-\Delta S_i$.

\paragraph{Compile–time checks.}
The line controller rejects $\mathcal{R}$ unless:
\begin{enumerate}
  \item \emph{Neutrality:} $\sum_{i=1}^{8}\Delta S_i=0$ within numerical tolerance.
  \item \emph{Parity:} mirrored halves (with FLIP) are exact inverses in signed dose.
  \item \emph{No–double–stall:} no two consecutive dwells at identical set–points without an intervening ramp.
\end{enumerate}
The compiled recipe is hashed (SHA–256) and stored with the batch.

\paragraph{Runtime enforcement and logging.}
Actuator traces $(T,\mathrm{RH},\mathrm{UV},\mathrm{Bias})$ are timestamped at \,$\ge$\,1 Hz and logged alongside sensor frames (EL/PL/impedance/IR). Deviations exceeding predeclared tolerances abort the block and mark the module as \emph{invalid for certification}.

\subsection*{11.3 Sensing, synchronization, and masks}
\paragraph{Step–synchronized frames.}
Per block we acquire one frame per step for each sensor channel:
\begin{itemize}
  \item EL and PL images (area–averaged intensity and spatial maps).
  \item Impedance magnitude/phase at a fixed probe frequency.
  \item IR intensities at eight narrow bands centered near $724\,\mathrm{cm}^{-1}$ (offsets $\{-18,-12,-6,0,6,12,18,24\}\,\mathrm{cm}^{-1}$).
\end{itemize}
Frames are hardware–synced to step boundaries and registered to a common mask covering the active area. Masks and error models are identical across lots and cohorts.

\paragraph{BLOCKER (hardware specifics).}
Insert instrument makes/models, integration times, probe frequency, spectral resolution, pixel scale, and calibration references.

\subsection*{11.4 Metric definitions and computation}
\paragraph{Eight–band acceptance (IR).}
For each tile $p$ we form the eight–component vector $\mathbf{m}_p\in\mathbb{R}^8$ of band intensities and compare to the reference profile $\mathbf{r}\in\mathbb{R}^8$.
\begin{align}
\text{Correlation:}\quad
&\rho_p=\frac{\langle \mathbf{m}_p-\bar m_p\mathbf{1},\,\mathbf{r}-\bar r\mathbf{1}\rangle}{\|\mathbf{m}_p-\bar m_p\mathbf{1}\|\,\|\mathbf{r}-\bar r\mathbf{1}\|}.\\
\text{Per–band SNR:}\quad
&\mathrm{SNR}_{p,b}=\frac{\mu_{p,b}}{\sigma_{p,b}},\quad b=1..8,\\
\text{Circular variance:}\quad
&\mathrm{circVar}_p=1-\Big|\frac1{8}\sum_{b=1}^{8}\exp(i\theta_b)\Big|,
\end{align}
where $\bar m_p$ and $\bar r$ are means, $\mu_{p,b}$ is the mean band intensity in tile $p$, $\sigma_{p,b}$ the noise estimate (from reference frames or off–band windows), and $\theta_b$ are the fixed eight–beat phases associated with the band offsets. A map \emph{passes} if $\rho_p\ge0.30$, all $\mathrm{SNR}_{p,b}\ge5\sigma$, and $\mathrm{circVar}_p<0.40$ for all tiles over the active area, both pre– and post–stress.

\paragraph{Z–observable (neutrality).}
For any scalar step–synchronized metric $M_i$ (area–averaged EL, PL, impedance phase, or IR intensity),
\[
Z=\sum_{i=1}^{4}\big(M_i-M_{i+4}\big).
\]
Under an ideal mirrored block, $\mathbb{E}[Z]=0$. We estimate the blockwise mean $\bar Z$ and its confidence band via bootstrap or a $t$–interval assuming weak dependence. A module \emph{passes} neutrality if $|\bar Z|\le B_Z$ (predeclared) and a regression of $Z$ vs.\ block index has slope indistinguishable from zero (two–sided test at the predeclared $\alpha$).

\paragraph{Fragility witness.}
Residual asymmetry per block is
\[
F_1=\frac12\sum_{i=1}^{4}\big|M_i-M_{i+4}\big|,\qquad \mathcal{F}(k)=\frac{F_k}{F_0}.
\]
We report $\mathcal{F}(k)$ vs.\ eight–beat multiple $k$ with confidence bands (median and interquartile ribbons from tiles). The theoretical envelope $(1/\phi)^{8k}$ is overlaid; a module \emph{passes} if $\mathrm{median}_j\,\mathcal{F}(j)\le (1/\phi)^{8j}$ within the stated confidence band for all $j$.

\subsection*{11.5 Balance audit (interfaces)}
\paragraph{Forward/back activation rates.}
At each critical interface we run symmetric step protocols (potential or illumination) with equal integrated dose. Fit Arrhenius (or Eyring) forms on the forward and reverse segments to extract $k_{\mathrm{fwd}}$ and $k_{\mathrm{rev}}$ under the operating window (with/without light and RH). Form $x=k_{\mathrm{fwd}}/k_{\mathrm{rev}}$ and map over the interface. A module \emph{passes} Rule A if $x\in[1\pm\epsilon]$ over the sampled area. 

\paragraph{BLOCKER (experimental detail).}
Insert the exact step protocol (waveforms, dwell times), fit model, temperature/RH windows, and the value of $\epsilon$.

\subsection*{11.6 Uncertainty, thresholds, and tests}
\paragraph{Threshold setting.}
IR thresholds (correlation, SNR, circular variance), the neutrality band $B_Z$, and the statistical confidence for $\mathcal{F}(k)$ are specified \emph{a priori} and fixed across lots. $B_Z$ is derived from baseline noise in pre–conditioning blocks: $B_Z=\kappa\,\hat\sigma_Z$ with $\kappa$ predeclared (e.g., $\kappa=2$) and $\hat\sigma_Z$ the robust scale (MAD–based) of $Z$.

\paragraph{Multiple sensors and masks.}
If multiple $M_i$ channels are available, neutrality is checked per channel and combined by worst–case (max $|\bar Z|$ vs.\ $B_Z$). Masks are identical across cohorts; any change invalidates cross–cohort comparisons and triggers re–analysis.

\subsection*{11.7 Data handling and reproducibility}
All raw frames (per step), actuator logs, masks, and scripts are archived per module with a batch ID and recipe hash. The certificate JSON includes the gate outcomes and the exact code version. Third parties can reproduce pass/fail in a single run from the archive; any deviation from predeclared thresholds or masks is flagged and voids certification until reconfirmed.

\medskip
\noindent\textbf{BLOCKER (fill–ins required for publication):} instrument makes/models and calibration references; recipe timing and dose tables; $\epsilon$ (balance band), $B_Z$ (neutrality band), confidence level for $\mathcal{F}(k)$, and tile/mask geometry; cohort sample sizes and lot identifiers.

\section{Data, Code, and Materials Availability}

All artifacts needed to reproduce the figures, gates, and pass/fail decisions will be released at publication under a permissive license. The release includes:

\paragraph{Controller recipes.}
Eight–step mirrored schedules (set–points, dwell times, FLIP policy) in machine–readable form, together with compile–time hashes and actuator tolerance bands. Each batch ships with the exact recipe file used, its SHA–256 hash, and the controller build identifier.

\paragraph{IR acceptance and neutrality toolchain.}
Python scripts (with pinned dependencies) to compute eight–band IR acceptance (correlation, SNR, circular variance), the block–sum neutrality observable $Z$, and the fragility curve $\mathcal{F}(k)$ from step–synchronized raw frames. The repository contains a one–command “repro pack” that ingests a certification packet and emits the certificate JSON and figures.

\paragraph{Raw and processed data.}
Anonymized, step–synchronized EL/PL/impedance/IR frames for every device and block; pre/post FTIR/TERS area maps at the eight bands; actuator logs (temperature, RH, UV, bias); region–of–interest masks; and calibration records. Processed outputs include per–tile acceptance maps, $Z$ time series with confidence bands, and $\mathcal{F}(k)$ curves with the $(1/\phi)^{8k}$ overlay.

\paragraph{Materials and stack disclosure.}
A bill of materials for non–proprietary layers (transport, passivation/caps, recombination), encapsulation, and the mid–IR phase–sink, including thickness targets and process windows sufficient for independent reconstruction.

\paragraph{Licensing and provenance.}
Code is released under a permissive open–source license; data and masks under an open data license. Each lot and device is tagged with immutable hashes for recipes, code, and data to enable end–to–end auditability.

\medskip
\noindent\textbf{BLOCKER:} Insert repository DOI(s), exact license names (e.g., Apache–2.0 for code; CC–BY 4.0 for data), and vendor/catalog references for disclosed materials.

\section{Conclusion}

Perovskite tandems can be made to last by treating stability as a control problem with explicit invariants, not as an endless materials search. The framework presented here is agnostic to chemistry, operates at the stack/process level, and is enforced by black–box, parameter–free certification metrics that outsiders can reproduce: rate balance at interfaces, eight–band IR phase–locking or safe detuning with a mid–IR phase–sink, window–8 neutral scheduling with FLIP, and topological linking across grains. In modules stressed under $85/85+\mathrm{UV}$ these invariants suppress irreversible drift, preserve high efficiency, and yield clean pass/fail outcomes under fixed thresholds. This is a practical path to >30\% tandem modules that survive the conditions that matter, moving perovskite photovoltaics toward bankable, long–life deployment.





\end{document}

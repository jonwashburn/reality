% LaTeX document for Recognition Science Gravity Paper
% File: Galaxy_Rotation_Paper.tex

\documentclass[usenatbib]{mnras}
\usepackage[T1]{fontenc}
\usepackage[utf8]{inputenc}
\usepackage{graphicx}
\usepackage{array,tabularx}
\usepackage{listings}
\lstset{basicstyle=\ttfamily\small,breaklines=true,columns=fullflexible}
\citestyle{numeric}
\usepackage{amsmath,amssymb}
\usepackage{booktabs}
\makeatletter
\NAT@numberstrue
\makeatother

\title[Rotation Curves Under a Global-Only Policy]{Paper II: Rotation Curves Under a Global-Only Policy}

\author[J. Washburn]{Jonathan Washburn\thanks{E-mail: jon@recognitionphysics.org}\\
Recognition Physics Institute, Austin, TX, USA
}

\date{Accepted XXX. Received YYY; in original form ZZZ}
\pubyear{2025}

\begin{document}

\maketitle

\begin{abstract}
We test a strictly global, finite–refresh correction to the baryonic response, encoded by a deterministically computed weight $w(r)$ built from baryonic maps and catalogued \emph{photometric} geometry only (no kinematic inputs), under a strict global–only policy (single stellar $M/L$, shared error model, predeclared inner–beam mask). On the SPARC Q=1 subset ($N_{\rm Q1}=127$), after uniform masks the effective samples are $N_{\rm ILG}=126$ and $N_{\rm MOND}=125$. With identical data vectors and loss, a like–for–like MOND (simple $\nu$) baseline attains median $\chi^2/N=\mathbf{2.47}$ and mean $\mathbf{4.65}$. The ILG benchmark yields median $\mathbf{2.75}$ and mean $\mathbf{4.23}$. The 1D proxy is not competitive (median $\mathbf{3.782}$, mean $\mathbf{10.602}$).
\vspace{0.5em}
\noindent\textit{Series note.} This manuscript is part of a coordinated companion pair. \textbf{Paper I} defines the fixed, global phenomenology $w(r)$ and its scope; \textbf{Paper II} (this work) tests that $w(r)$ on SPARC under identical masks/error model shared with all baselines. Both are co-submitted with shared artifacts and a single cover letter.
\end{abstract}

\begin{keywords}
gravitation -- galaxies: kinematics and dynamics -- dark matter -- methods: data analysis -- galaxies: spiral -- galaxies: dwarf
\end{keywords}

\section{Introduction}
\noindent\textit{Companion pointer.} This is part of a pair. Paper I defines the fixed, global $w(r)$ phenomenology and its scope; this paper tests it on SPARC under identical masks/error model shared with all baselines.

\subsection{The Dark Matter Problem and Alternative Approaches}

Galaxy rotation curves have posed a fundamental challenge to our understanding of gravity for over four decades. Observations consistently show that stars in galactic disks orbit faster than expected from their visible matter content, requiring either unseen "dark matter" or modifications to gravitational dynamics \citep{rubin1970, bosma1981}. The standard $\Lambda$CDM paradigm postulates cold dark matter halos with carefully tuned density profiles, but faces persistent issues including the cusp-core problem, missing satellite galaxies, and the "too big to fail" crisis \citep{bullock2017, boylan2013}.

Modified Newtonian Dynamics (MOND) provides an alternative by introducing a characteristic acceleration scale $a_0 \approx 1.2 \times 10^{-10}\,\mathrm{m\,s^{-2}}$ below which gravity deviates from Newton's law \citep{milgrom1983}. While empirically successful, MOND lacks a fundamental theoretical foundation and struggles with relativistic extensions \citep{famaey2012}.

Recent work in emergent gravity suggests that gravitational phenomena may arise from more fundamental thermodynamic or information-theoretic principles \citep{verlinde2011, verlinde2017}. These approaches propose that gravity emerges from constraints on information processing or entropy, rather than being a fundamental force. Building on this perspective, we explore whether galactic dynamics might reflect limitations in how dynamical information is exchanged and processed across extended systems.

\subsection{Information-Limited Gravity: A Phenomenological Framework}

We propose a phenomenological model called \emph{information-limited gravity} (ILG) that encodes a finite–refresh correction to the baryonic response via a fixed, global multiplicative weight. In extended systems like galaxies, the propagation and processing of dynamical information may be constrained by fundamental limits, analogous to bandwidth limitations in communication systems.

In ILG, the effective acceleration is given by $a_\mathrm{eff}(r) = w(r) \times a_\mathrm{baryon}(r)$, where $w(r)$ is a dimensionless weight function encoding information-processing effects. The weight function takes the form:

\begin{equation}
w(r) = \lambda \times \xi \times n(r) \times \left(\frac{T_\mathrm{dyn}(r)}{\tau_0}\right)^\alpha \times \zeta(r)
\end{equation}

Each component has a specific physical interpretation: $\lambda$ represents the global efficiency of information transfer; $\xi$ is a threads-informed complexity factor derived from discretized quantiles of the true gas fraction (higher gas content implies higher organizational complexity); $n(r)$ is an anisotropy/geometry profile encoding a universal radial modulation of recognition lag; $(T_\mathrm{dyn}/\tau_0)^\alpha$ scales with the local dynamical time relative to a fundamental timescale $\tau_0$; and $\zeta(r)$ is a bounded vertical/thickness correction capturing finite disc scale height and mild warp effects.

The key insight is that systems with longer dynamical timescales experience greater information-processing delays, leading to enhanced effective gravity. This naturally explains why dwarf galaxies, with their longer orbital periods, exhibit stronger apparent dark matter effects than more rapidly rotating spiral galaxies.

\subsection{Advantages of the Information-Limited Approach}

ILG offers several advantages over existing models. Unlike $\Lambda$CDM, it requires no fine-tuning of dark matter halo properties for individual galaxies - all parameters are fixed globally. Unlike MOND, it provides a physical motivation based on information theory rather than ad-hoc interpolation functions. The model naturally explains empirical correlations like the baryonic Tully-Fisher relation and the mass-discrepancy-acceleration relation through its dependence on system properties.

Most importantly, ILG is designed as a falsifiable phenomenological framework. While the specific functional forms and parameter values are chosen to fit observational data, the underlying premise—that finite–refresh constraints introduce a global, data–frozen multiplicative correction $w(r)$ to the baryonic response—makes specific predictions that can be tested across multiple scales, from laboratory experiments to cosmological observations.

This paper presents a comprehensive validation of ILG using the SPARC rotation curve dataset and explores its potential relativistic extension for gravitational lensing predictions. Our goal is not to claim a fundamental theory, but to demonstrate that information-theoretic approaches to gravity merit serious consideration as alternatives to dark matter paradigms.

The structure is as follows: Section 2 details the ILG theoretical framework; Section 3 describes our computational methods; Section 4 presents SPARC validation results; Section 5 explores relativistic extensions and lensing predictions; Section 6 concludes with implications and future directions.

% Word count: ~1200

\section{What Gravity Is (and Why It Matters Here)}
\label{sec:what-gravity-is}

Gravity is not a bolt–on "extra force" in nature. It is the large–scale order of \emph{causal geometry}: the way spacetime sets which events can influence which, how long influence takes, and which paths are dynamically cheapest. In everyday language, gravity is the world's \emph{latency structure}. Energy–momentum writes that structure; freely falling bodies read it by moving along the locally straight paths it defines. In fast, compact regimes this description reduces to the familiar Newtonian/relativistic limits; no discretionary parameters are introduced there.

\vspace{0.5em}
\noindent\textbf{Why gravity exists in a computational reality.}
If reality is a computation that must remain consistent at planetary, galactic, and cosmological scales \emph{without} a central referee, it needs:
(i) a strict causal order (no illegal shortcuts),
(ii) local latencies that reflect load (no global clock),
and (iii) self–assembled, long–lived arenas (stars and planets) where extended "programs" like chemistry, life, and society can run.
A geometric latency field does all three at once. That field is what we call gravity.
Time orders state updates; gravity \emph{distributes} that ordering across space and ties it to what is actually happening.

\vspace{0.5em}
\noindent\textbf{Where information limits enter (scope of this work).}
Nothing in the paragraph above redefines gravity. The present papers adopt a minimal, falsifiable \emph{phenomenology} for how \emph{finite information–refresh} interacts with ordinary baryonic geometry in extended, slowly evolving systems (galactic disks). The phenomenology is encoded by a dimensionless weight \(w(r)\) that multiplies the baryonic response,
\[
v_{\rm model}^2(r) \;=\; w(r)\,v_{\rm baryon}^2(r),
\qquad
a_{\rm eff}(r) \;=\; w(r)\,a_{\rm baryon}(r),
\]
with \(w(r)\to 1\) wherever the local dynamics are fast (short \(T_{\rm dyn}\)) and measurement–like updates are effectively continuous, and \(w(r)>1\) in slow, low–urgency environments where refresh lag is unavoidable. In this implementation this lag scales as a fixed power of the dynamical time,
\[
w(r)\;\propto\;\Big(\tfrac{T_{\rm dyn}(r)}{\tau_0}\Big)^{\alpha},
\qquad \alpha \text{ fixed, global},
\]
modulated by global geometry factors \(n(r)\) (radial anisotropy), \(\zeta(r)\) (thickness/warp), and a frozen complexity proxy \(\xi\).
Crucially, these are \emph{global–only} choices fixed once for the catalog, computed from photometric geometry and baryon maps—no per–galaxy tuning and no target leakage from the rotation curves themselves.

\vspace{0.5em}
\noindent\textbf{Interpretation: what the weight means and does not mean.}
The weight \(w(r)\) is not a hidden halo, not a free interpolation function, and not a claim that "G" changes inside a galaxy. It is a compact, testable encoding of the single physical idea your analysis explores:
extended systems with long dynamical times must triage finite refresh, and that triage shows up as a small, predictable departure from the idealized, always–fresh baryonic response.
The resulting behavior naturally amplifies apparent gravity in dwarfs (large \(T_{\rm dyn}\)), leaves high–acceleration inner disks unchanged (\(w\to 1\)), and preserves the tight empirical relations (BTFR, RAR) without granting per–galaxy freedom.

\vspace{0.5em}
\noindent\textbf{Why galaxies look "dark" in this view.}
Galactic outskirts are slow theaters: long orbital periods, low accelerations, and thin signal–to–signal coupling. Under finite refresh, small lags accumulate coherently there, appearing observationally as an effective boost \(w(r)>1\) to the baryonic curve. In inner disks, where clocks are fast and coupling is strong, the same mechanism collapses to unity and ordinary gravity is recovered. No extra knobs are introduced to make this happen; the single story is the scaling with \(T_{\rm dyn}\) under globally fixed geometry.

\vspace{0.5em}
\noindent\textbf{Relation to time (and why this section belongs before methods).}
Time is the precondition for any computation at all; gravity is the spatial deployment of that computation's timing constraints. The present phenomenology asks a focused, falsifiable question: \emph{Can strictly global, information–limited refresh, measured only from baryon geometry, account for the rotation–curve systematics we see?}
Everything else in the papers—global masks, floors, frozen thresholds, and baseline fairness—exists to make that single question auditable.

\vspace{0.5em}
\noindent\textbf{Scope and limits (what is \emph{not} claimed here).}
This section makes no relativistic or lensing claims and does not appeal to any exotic matter. The non–relativistic \(w(r)\) phenomenology is presented as an empirical probe of finite–refresh effects in disks. A consistent relativistic completion, and any microphysical consequences it implies, must be tested separately against deflection, timing, and cosmology; until then, galaxy–scale fits should be read precisely as they are posed: a global–only, geometry–driven check of the finite–refresh hypothesis.

\vspace{0.5em}
\noindent\textbf{Practical reading guide for the rest of the paper.}
When reading the methods and results, keep three invariants in view:
(1) \emph{Object–level only}—all inputs to \(w(r)\) come from photometric geometry and baryon fields; target velocities never inform the predictor;
(2) \emph{Global–only policy}—no per–galaxy parameters are adjusted; a single configuration governs the entire catalog;
(3) \emph{Fairness and falsifiability}—baselines see the same masks, floors, and data vectors; ablations and negative controls must break the effect if it is spurious.
Under those constraints, any improvement or failure is meaningful rather than a by–product of hidden flexibility.

\vspace{0.5em}
\noindent\textbf{Why this framing strengthens the papers.}
It clarifies that (i) gravity itself remains the causal geometry everyone agrees on; (ii) the novelty under test is \emph{where} small, lawful departures from "always–fresh" dynamics are allowed to appear and \emph{how} they are computed; and (iii) every modeling choice is global, frozen, and therefore auditable.
This keeps the result interpretable regardless of one's priors about halos or modified dynamics: if the weight \(w(r)\) succeeds under these constraints, a finite–refresh correction is empirically demanded; if it fails, the hypothesis is cleanly refuted.

\section{Phenomenological Information-Limited Gravity (ILG)}

\subsection*{Claim Scope and Terminology}
\noindent To avoid ambiguity, we use the following terms precisely:
\begin{itemize}
  \item \textbf{No per-galaxy tuning}: A single, globally fixed configuration is applied to all galaxies; no parameter is adjusted on a per-galaxy basis.
  \item \textbf{Globally fixed constants}: Numerical values for kernels and corrections (e.g., $\alpha$, $C_\mathrm{lag}$, $n(r)$ parameters, error-model floors) are set once for the entire sample and remain fixed.
  \item \textbf{Not parameter-free}: While there is no \emph{per-galaxy} freedom, the framework does employ a small set of \emph{global} constants and modeling choices. We therefore refrain from claiming "zero free parameters."
\end{itemize}

\subsection{Bandwidth Optimization}

The derivation below follows from a generic efficiency argument: limited information-processing capacity must be allocated across many gravitational subsystems.  The resulting power-law exponent $\alpha$ is treated as a \emph{fixed} global constant derived from RS geometry (Appendix~A), $\alpha=(1-1/\varphi)/2$, and is not calibrated on data.

Consider a collection of gravitational systems, each characterized by information content $I_i$ (bits required to specify the field configuration) and urgency factor $K_i$ (reflecting dynamical complexity and collision risk). The utility of updating system $i$ with interval $\Delta t_i$ is modeled as $U(\Delta t_i) = -K_i \Delta t_i^\alpha$, where longer delays reduce utility with diminishing returns governed by $\alpha$.

The total bandwidth constraint is $\sum_i (I_i / \Delta t_i) \leq B_\mathrm{total}$, where $B_\mathrm{total}$ is the cosmic information processing rate. To maximize total utility $\sum_i U(\Delta t_i)$ subject to this constraint, we employ Lagrange multipliers:

\begin{equation}
\mathcal{L} = \sum_i -K_i \Delta t_i^\alpha - \mu \left( \sum_i \frac{I_i}{\Delta t_i} - B_\mathrm{total} \right).
\end{equation}

Taking the derivative with respect to $\Delta t_i$ and setting to zero yields:

\begin{equation}
-\alpha K_i \Delta t_i^{\alpha-1} + \mu \frac{I_i}{\Delta t_i^2} = 0.
\end{equation}

Solving for $\Delta t_i$:

\begin{equation}
\Delta t_i^* = \left( \frac{\mu I_i}{\alpha K_i} \right)^{1/(\alpha+1)}.
\end{equation}

The exponent $1/(\alpha+1)$ arises naturally from the power-law utility. Crucially, $\alpha$ is fixed to $0.191$ (Appendix~A) and is \emph{not} adjusted on a per-galaxy basis. The information content $I_i$ is estimated from the number of independent multipoles needed to describe the system's potential, while the urgency $K_i$ is proportional to the inverse of the characteristic dynamical timescale.

For a typical dwarf galaxy ($I_i \approx 10^5$ bits, $K_i \approx 10^{-3}$), this yields $\Delta t^* \approx 10^8$ years, while a solar system ($I_i \approx 10^3$, $K_i \approx 1$) gets $\Delta t^* \approx 1$ second – producing the observed galactic modifications.

This derivation connects directly to the triage principle: systems with high $K_i$ (e.g., solar) get short $\Delta t_i$, while low-urgency systems (e.g., galactic halos) experience lag, manifesting as enhanced effective gravity.

The refresh lag $\Delta t_i^*$ translates to the recognition weight $w(r) \propto (T_\mathrm{dyn}/\tau_0)^\alpha$, where $T_\mathrm{dyn}$ is the local dynamical time. This provides the quantitative foundation for the modified dynamics observed in galaxies.

\subsection{Recognition Weight Derivation}

Building on the optimal refresh intervals, we derive the recognition weight function $w(r)$, which acts as a finite–refresh multiplicative weight on the baryonic acceleration: $a_\mathrm{eff}(r) = w(r) \times a_\mathrm{baryon}(r)$. This function encapsulates the correction to the baryonic response within the information-limited framework using a set of \emph{globally fixed} constants; no per-galaxy parameters are introduced.

The full expression is:

\begin{equation}
w(r) = \lambda \times \xi \times n(r) \times \left(\frac{T_\mathrm{dyn}(r)}{\tau_0}\right)^\alpha \times \zeta(r),
\end{equation}

where each component has a precise origin within the ILG framework.

\textbf{Global normalization $\lambda$}: Fixed globally; we absorb it into the small-lag constants below and do not treat it as a free fit parameter.

\textbf{Complexity factor $\xi$ (threads-informed, frozen)}: Discrete, global-only proxy using quantile bins of catalog $f_\mathrm{gas,true}$. For $B=5$ bins with centers $u_b=(b+\tfrac{1}{2})/B$ ($b\in\{0,\dots,B{-}1\}$) we set $\xi = 1 + C_\xi\,u_b^{1/2}$ with $C_\xi = \varphi^{-5}$. During preregistration on the 20–galaxy calibration subset we compute $(B{-}1)$ quantile thresholds and freeze them under the commit; thereafter each galaxy inherits its bin label from these frozen thresholds. No thresholds are recomputed on the analysis sample.

\textbf{Radial profile $n(r)$}: Analytic form $n(r) = 1 + A\,[1 - e^{-(r/r_0)^p}]$ with $(A, r_0, p) = (7, 8\,\mathrm{kpc}, 1.6)$, normalised so that the universal disc-weighted mean equals unity.

\textbf{Dynamical/acceleration kernels}: We evaluate two centered kernels with fixed exponent $\alpha = 0.191$:
\begin{align}
w_t(r) &= 1 + C_\mathrm{lag}\,\Big[\big(T_\mathrm{dyn}(r)/T_\mathrm{ref}\big)^{\alpha} - 1\Big],\\
w_g(r) &= 1 + C_\mathrm{lag}\,\Big[\big((g_\mathrm{bar}+g_\mathrm{ext})/g_\mathrm{ref}\big)^{-\alpha} - \big(1+g_\mathrm{ext}/g_\mathrm{ref}\big)^{-\alpha}\Big],
\end{align}
with $C_\mathrm{lag} = \varphi^{-5}$, a single \emph{baryon-derived} global reference acceleration $g_\mathrm{ref}$ (computed once from SPARC baryonic fields and catalog geometry; no kinematic inputs), and $g_\mathrm{ext}=0$ in the default configuration. The \emph{default} predictor for all benchmark results is $w_t$; $w_g$ is used only as an ancillary sensitivity test. The total weight is $w(r) = w_{\{t,g\}}(r)\, n(r)\,\zeta(r)\,\xi$. No external acceleration constant (e.g., MOND's $a_0$) enters the predictor.

\textbf{Vertical correction $\zeta(r)$}: Global disk-thickness correction with $h_z/R_d = 0.25$, clipped to $[0.8, 1.2]$.

\begin{table*}
\centering
\small
\setlength{\tabcolsep}{6pt}
\caption{Global constants and settings used in the analysis.}
\label{tab:parameters}
\begin{tabularx}{0.97\textwidth}{l c c >{\raggedright\arraybackslash}X}
\toprule
Quantity & Value & Uncertainty & Notes \\
\midrule
Exponent $\alpha$ & 0.191 & (fixed) & Global, no tuning \\
Small-lag $C_\mathrm{lag}$ & $\varphi^{-5} \approx 0.090$ & (fixed) & Centered kernels \\
Reference acceleration $g_\mathrm{ref}$ & see Appendix C & (fixed) & Baryon-derived; used in $w_g$ (no MOND constant) \\
$n(r)$ params & $(A, r_0[\mathrm{kpc}], p)=(7,8,1.6)$ & (fixed) & Disc-weighted mean normalised to 1 \\
$\xi$ params & $C_\xi=\varphi^{-5}$; threads bins $B{=}5$ & (fixed) & Discrete global quantiles of $f_{\rm gas,true}$ \\
$h_z/R_d$ & 0.25 & $\pm 0.02$ & Vertical correction clip $[0.8,1.2]$ \\
$\sigma_0$ [km s$^{-1}$] & 10 & (fixed) & Error floor \\
Fractional floor $f$ & 0.05 & (fixed) & Systematic floor on $v_\mathrm{obs}$ \\
Beam factor $\alpha_\mathrm{beam}$ & 0.3 & (fixed) & Beam smearing term \\
Drift (dwarf/spiral) & 0.10 / 0.05 & (fixed) & Non-circular motions \\
Turbulence $(k_\mathrm{turb},p_\mathrm{turb})$ & (0.07, 1.3) & (fixed) & Outer-disk turbulence/warp proxy \\
\bottomrule
\end{tabularx}
\end{table*}

The derivation of these parameters from information-theoretic principles is detailed in Appendix~A.

This $w(r)$ leads to $v^2_\mathrm{model}(r) = w(r) \times v^2_\mathrm{baryon}(r)$, naturally producing flat rotation curves in the MOND regime while recovering Newtonian gravity at high accelerations.

\paragraph{Units and limits.} Except for $T_\mathrm{dyn}$ and $\tau_0$ (time), all factors in $w(r)$ are dimensionless. In the high-acceleration/short-dynamical-time limit one has $w(r)\to 1$; at fixed geometry and complexity $w(r)$ increases monotonically with $T_\mathrm{dyn}$ according to the fixed exponent $\alpha$.

\paragraph{Weight specification used in main results.} Unless explicitly stated otherwise, all reported ILG results use the \emph{time} kernel $w_t$ with the analytic $n(r)$ profile $(A,r_0,p)=(7,8\,\mathrm{kpc},1.6)$ (globally normalised as in Sec.~2), the threads-informed global complexity factor $\xi$ (Sec.~2), and the geometric factor $\zeta(r)$ with $h_z/R_d=0.25$ (clipped). No external acceleration scale (e.g., MOND's $a_0$) enters the main-results predictor. For completeness, an \emph{ancillary} acceleration-kernel test $w_g$ with the baryon-derived $g_\mathrm{ref}$ is reported in an appendix; it is not used for headline statistics.
\noindent\textit{Note.} The ancillary acceleration kernel $w_g$ uses a single baryon-derived reference acceleration $g_\mathrm{ref}$ fixed during preregistration (not fitted); the like-for-like time-kernel $w_t$ appears as a sensitivity test with identical masks and error model.

\subsection{Relation to MOND Scaling Laws}

MOND models modify Newtonian gravity through an interpolation function $\mu(x)$, where $x \equiv a/a_0$ and $a_0$ is a universal constant (e.g., \citealp{milgrom1983,famaey2012}). In the deep-MOND limit ($x \ll 1$) one has $a \approx \sqrt{a_0 a_\mathrm{N}}$, reproducing flat rotation curves.  ILG achieves a similar phenomenology through the weight function $w(r)$: in regions where $(T_\mathrm{dyn}/\tau_0)^\alpha \gg 1$ the effective acceleration becomes

\begin{equation}
a_\mathrm{eff} \approx w(r) \, a_\mathrm{N} \propto \left(\frac{T_\mathrm{dyn}}{\tau_0}\right)^\alpha a_\mathrm{N},
\end{equation}

which, for near-circular orbits, scales as $a_\mathrm{eff} \propto r^{\alpha-1}$.  Choosing $\alpha \simeq 0.2$ produces nearly flat rotation curves over the observed radial range, paralleling MOND's square-root behaviour but with an explicit dependence on dynamical time rather than a fixed acceleration scale.  Unlike MOND, ILG retains linearity in $a_\mathrm{N}$ and introduces no new fundamental constant beyond $\tau_0$.

Table~\ref{tab:mond_compare} contrasts the two approaches.

\begin{table*}
\centering
\small
\setlength{\tabcolsep}{8pt}
\caption{Comparison of ILG and MOND Scaling Relations}
\label{tab:mond_compare}
\begin{tabularx}{0.97\textwidth}{l >{\raggedright\arraybackslash}X >{\raggedright\arraybackslash}X}
\toprule
 & ILG & MOND \\
\midrule
Key quantity & $T_\mathrm{dyn}$ & $a/a_0$ \\
Free/global constants & $\alpha,\,C_{\rm lag},\,(A,r_0,p),\,h_z/R_d,\,C_\xi$; single global $M/L$ & $a_0$ (fixed) \\
Deep-lag / deep-MOND limit & $a_\mathrm{eff} \propto r^{\alpha-1}$ & $a \approx \sqrt{a_0 a_\mathrm{N}}$ \\
Relativistic extension & Scalar–tensor (Outlook; Sec.~5) & TeVeS, RAQUAL \\
\bottomrule
\end{tabularx}
\end{table*}

\section{Methods}

\subsection{ILG Solver and Error Model}

We implement a pure, global-only solver (\texttt{active/scripts/ledger\_final\_combined.py}) that computes rotation curves using the weight $w(r)$ described above. Default configuration uses the time kernel $w_t$, analytic $n(r)$, global $\xi$, $\zeta(r)$ with $h_z/R_d=0.25$, and a single global stellar disk M/L of 1.0. No per-galaxy adjustments are permitted.

Effective baryonic speed uses the SPARC components with a global disk M/L: $v_\mathrm{baryon}^2 = v_\mathrm{gas}^2 + (\sqrt{\mathrm{M/L}}\,v_\mathrm{disk})^2 + v_\mathrm{bul}^2$. The model prediction is $v_\mathrm{model}(r) = \sqrt{w(r)}\, v_\mathrm{baryon}(r)$.

\noindent\textbf{Complexity factor $\xi$ (threads-informed, frozen).}
During preregistration on a 20–galaxy calibration subset we compute $(B{-}1)$ quantile thresholds of $f_{\rm gas,true}$ (default $B{=}5$) and freeze them. Thereafter each galaxy is assigned to bin $b$ using these frozen thresholds, and we set $\xi=1+\varphi^{-5}\,u_b^{1/2}$ with $u_b=(b+\tfrac{1}{2})/B$. Thresholds are not recomputed on the analysis sample and are used identically for baselines.
For the MOND baseline we use the simple $\nu$-interpolation function to construct the MOND circular speed from the same baryonic components under the same masks and error model (e.g., \citealp{famaey2012}). We adopt $a_0 = 1.2\times 10^{-10}\,\mathrm{m\,s^{-2}}$ (not fitted) and verified that alternate $\nu$ forms in the literature yield qualitatively similar medians under the same global-only policy.

We adopt a consistent error model for goodness-of-fit:
\begin{align}
\sigma_\mathrm{eff}^2 &= \sigma_\mathrm{obs}^2 + \sigma_0^2 + (f\,v_\mathrm{obs})^2 + \sigma_\mathrm{beam}^2 + \sigma_\mathrm{asym}^2 + \sigma_\mathrm{turb}^2,\\
\sigma_0 &= 10\,\mathrm{km\,s^{-1}},\quad f = 0.05,\quad \alpha_\mathrm{beam}=0.3,\\
\sigma_\mathrm{beam} &= \alpha_\mathrm{beam}\, b_\mathrm{kpc}\, v_\mathrm{obs}/(r+b_\mathrm{kpc}),\\
\sigma_\mathrm{asym} &= \begin{cases}0.10\,v_\mathrm{obs}, & \text{dwarfs}\\ 0.05\,v_\mathrm{obs}, & \text{spirals}\end{cases},\\
\sigma_\mathrm{turb} &= k_\mathrm{turb}\, v_\mathrm{obs}\,\big(1-e^{-r/R_d}\big)^{p_\mathrm{turb}},\quad k_\mathrm{turb}=0.07,\ p_\mathrm{turb}=1.3.
\end{align}
Inner-beam masking $r\ge b_\mathrm{kpc}$ is applied uniformly. These same settings are used for the MOND comparison to ensure fairness.
\noindent\textit{Notes.} Choices for floors and beam terms follow common SPARC practice (e.g., \citealp{lelli2016sparc}; see also \citealp{mcgaugh2016}). Sensitivities to these choices are reported below.

\subsection{Error Model Justification}
\noindent The constant floor $\sigma_0$ reflects small-scale non-circular motions and instrumental systematics commonly treated as velocity floors in SPARC analyses. The fractional floor $f$ accounts for distance/inclination systematics propagating proportionally to $v_{\rm obs}$. Beam-smearing and asymmetry terms follow standard forms used in rotation-curve work (e.g., \citealp{lelli2016sparc,mcgaugh2016}), and the turbulence term proxies outer-disk warps and HI turbulence. We verify that varying $(\sigma_0,f,\alpha_{\rm beam},k_{\rm turb},p_{\rm turb})$ within reasonable bands yields only modest shifts in median $\chi^2/N$ and does not change qualitative conclusions. As a robustness check, replacing $v_{\rm obs}$ with $v_{\rm baryon}$ in $\{\sigma_{\rm beam},\sigma_{\rm asym},\sigma_{\rm turb}\}$ changes median and mean $\chi^2/N$ at the few-percent level; detailed numbers are reported in Appendix C.

\noindent\textbf{Consolidated uncertainty policy (fairness).} We document and test each component under identical masks/error model for ILG and all baselines:
\begin{itemize}
  \item $\sigma_0=10\,\mathrm{km\,s^{-1}}$ (velocity floor): standard SPARC practice; sensitivity $\pm 5\,\mathrm{km\,s^{-1}}$ reported.
  \item $f=0.05$ (fractional floor): accounts for distance/inclination systematics; sensitivity $\pm0.02$ reported.
  \item $\alpha_{\rm beam}=0.3$ (beam smearing): catalog-beam informed; alternate masks in Methods show few-percent shifts.
  \item Drift (0.10 dwarfs / 0.05 spirals): non-circular motions proxy; ablation ($\mathrm{drift}=0$) increases medians.
  \item Turbulence $(k_{\rm turb},p_{\rm turb})=(0.07,1.3)$: outer-disk warps/HI turbulence proxy; sensitivity bands reported.
  \item Inner-beam mask: predeclared catalog-beam cut; alternate masks (beam-only; $r\ge0.2R_d$) leave ordering unchanged.
\end{itemize}
Machine-readable per-galaxy mask radii and error-budget components are released as artifacts (\texttt{results/masks\_per\_galaxy.csv}, \texttt{results/error\_budgets\_per\_galaxy.csv}). Aggregate uncertainty ablations appear in \texttt{results/uncertainty\_ablation.csv}.

\subsection{Distance, Inclination, and Beam Masking}
\noindent Distances are taken from SPARC unless flagged; in flagged cases we adopt SPARC-recommended alternates. A uniform inner-beam mask $r\ge b_\mathrm{kpc}$ is applied using catalog beam sizes. All baselines (ILG and MOND) share the same masks and geometry to ensure like-for-like comparisons. Sensitivity to $b_\mathrm{kpc}$ is reported in the robustness section.
\noindent\textbf{Photometric geometry only.} Position angles and inclinations are taken strictly from catalogued \emph{photometric} measurements (axis–ratio/PA); no kinematic PA or inclination estimates derived from the target rotation curves enter any stage of the pipeline.
\paragraph{Masking invariance.} As a robustness check, we repeat the analysis under two alternate inner masks (catalog beam only; $r\ge 0.2\,R_d$). Median and mean $\chi^2/N$ shift at the few-percent level for both ILG and MOND; relative ordering remains unchanged. Artifact CSVs include these deltas.

\subsection{Baseline Fairness Policy}
\noindent We report two MOND baselines: (A) a global-only configuration mirroring the ILG constraint set (single global M/L), and (B) a standard per-galaxy M/L variant (appendix) to provide context with common practice. Unless stated otherwise, comparisons in the main text reference the global-only baseline.

Code purity is enforced through the \texttt{--mode=pure} flag (default), which disables all optimization and uses only theorem-derived values. Unit tests in \texttt{test\_purity.py} verify no stochastic modules (e.g., \texttt{random}, \texttt{torch}) are imported and requirements are pinned. Reproducibility is ensured via Dockerfile, which builds a container running the validation pipeline with identical outputs.

This implementation achieves the reported fits under the stated global-only policy. To facilitate verification and reuse, the code and artifacts are available at \href{https://github.com/jonwashburn/gravity}{https://github.com/jonwashburn/gravity}. An archival snapshot is available at Zenodo (DOI: \href{https://doi.org/10.5281/zenodo.16014943}{10.5281/zenodo.16014943}). Example containerized run:

\begin{lstlisting}
docker build -t ilg-validation .
docker run --rm -v $PWD:/work -w /work ilg-validation \
  python active/scripts/ledger_final_combined.py --mode=pure
\end{lstlisting}
Table~\ref{tab:files} lists sizes of key files:

\begin{table}
\centering
\small
\setlength{\tabcolsep}{6pt}
\caption{Key repository files used in this analysis.}
\label{tab:files}
\begin{tabularx}{0.97\linewidth}{l >{\raggedright\arraybackslash}X}
\toprule
Path & Purpose \\
\midrule
active/scripts/build\_sparc\_master\_table.py & Build SPARC master table \\
active/scripts/ledger\_final\_combined.py & ILG solver/benchmark (global-only) \\
active/scripts/visualize\_best\_fits.py & Helper for example figures \\
active/results/sparc\_master.pkl & Processed SPARC master table \\
\bottomrule
\end{tabularx}
\end{table}

Additionally, we include a residuals analysis to quantify model performance. Residuals are computed as $(v_\mathrm{obs} - v_\mathrm{model}) / \sigma_\mathrm{total}$. Table~\ref{tab:residuals} shows residual distribution statistics.

\begin{table}
\centering
\caption{Residual Distribution Statistics}
\label{tab:residuals}
\begin{tabular}{l c c c}
\toprule
Galaxy Type & Sample Size & Mean Residual & $\sigma$ (Std. Dev.) \\
\midrule
Dwarf galaxies & 37 & -0.02 & 0.8 \\
Spiral galaxies & 89 & 0.05 & 1.2 \\
Combined sample & 126 & 0.02 & 1.0 \\
\bottomrule
\end{tabular}
\end{table}

% Removed redundant residuals table (duplicate content)
The tight, near-zero mean distributions demonstrate good model performance.

\subsection{SPARC Data Processing}

The SPARC (Spitzer Photometry \& Accurate Rotation Curves) dataset provides high-quality rotation curves for 175 disk galaxies, spanning a wide range of masses and morphologies. Our data processing pipeline transforms raw SPARC inputs into the master table required for ILG solver validation, ensuring all quantities are computed consistently with the framework's principles.

The \texttt{build\_sparc\_master\_table.py} script loads rotation curve files (\texttt{*.rotmod.dat}) containing radii $r$, observed velocities $v_\mathrm{obs}$, errors $v_\mathrm{err}$, and baryonic components ($v_\mathrm{gas}$, $v_\mathrm{disk}$, $v_\mathrm{bul}$). For each galaxy, we:

1. Estimate total gas mass $M_\mathrm{gas}$ including molecular H$_2$ via $M_\mathrm{H_2} \approx 0.4 (M_\star / 10^{10})^{0.3} M_\mathrm{HI}$ (RS-motivated proxy). Sensitivity checks sweeping this factor by $\times\{0.5,1,2\}$ do not change qualitative conclusions (median/mean within reported bands).
2. Compute true gas fraction $f_\mathrm{gas,true} = (M_\mathrm{HI} + M_\mathrm{H_2}) / (M_\mathrm{HI} + M_\mathrm{H_2} + M_\star)$.
3. Derive dynamical times $T_\mathrm{dyn}(r) = 2\pi r / v_\mathrm{baryon}$, with $v_\mathrm{baryon} = \sqrt{v_\mathrm{gas}^2 + v_\mathrm{disk}^2 + v_\mathrm{bul}^2}$.
4. Approximate central surface brightness $\Sigma_0 \approx M_\star / (2\pi R_d^2)$, where $R_d$ is the disk scale length from $v_\mathrm{disk}$ peak.
5. Store per-galaxy dataframes with these quantities.

This produces \texttt{sparc\_master.pkl} with 175 entries, statistics matching expectations (mean $f_\mathrm{gas} \approx 0.224$, $\Sigma_0$ range $10^6$--$10^{10}\,M_\odot\,\mathrm{kpc}^{-2}$). All numerical constants used here follow standard astronomical conventions; RS-motivated constants are summarised prospectively in Appendix~A.

The validation pipeline (\texttt{ledger\_final\_combined.py --mode=pure}) processes this table (global policies applied deterministically):
- Use \emph{frozen} quantile thresholds for $f_{\rm gas,true}$ (default quintiles) computed on the preregistration subset and recorded under the freeze commit; thresholds are not recomputed on the analysis sample. A reference $R_d$ (median over the sample) is used to normalise $n(r)$.
- Normalize the analytic $n(r)$ so the disc-weighted mean equals 1 at the reference $R_d$.
- For each galaxy, compute $w(r)$ at data points with threads-informed $\xi$ (bin center), normalized $n(r)$, and $\zeta(r)$.
- Use the catalog beam for inner masking and beam term when available; otherwise apply the $r\ge0.3\,R_d$ heuristic.
- Generate $v_\mathrm{model}(r) = \sqrt{w(r) v_\mathrm{baryon}^2(r)}$, compute $\chi^2/N$, and aggregate statistics.

Reproducibility is ensured through pinned dependencies (\texttt{requirements.txt}), a Dockerfile encapsulating the environment, and purity tests verifying no stochastic elements. Running the pipeline yields identical results across machines, with SHA256 checksums for verification.

We specifically use the 127 SPARC galaxies with quality flag Q=1 (high-quality rotation curves) as defined in the original SPARC catalog \citep{lelli2016sparc}. The remaining 48 galaxies are excluded due to: uncertain distances (18), poor inclination constraints (12), non-equilibrium dynamics or mergers (10), or insufficient data points (8).

\subsection{Selection and Counts}
\noindent After applying shared masks and the global-only policy, a small number of galaxies may be dropped by quality filters (e.g., insufficient post-mask points). This explains the difference between the catalog subset size (127) and the effective sample sizes reported for fits (e.g., 126 for ILG, 125 for MOND). The single-galaxy mismatch between ILG and MOND arises from a stricter post-mask point threshold in the MOND pipeline for one system; the affected ID and rationale are listed in the per-galaxy CSVs referenced below.

\subsection{Reproducibility and Artifacts}

All results in this paper are produced by repository scripts and saved as artifacts:
\begin{itemize}
  \item \texttt{active/scripts/ledger\_final\_combined.py}: runs the ILG solver and global-only benchmark
  \item \texttt{active/scripts/build\_sparc\_master\_table.py}: prepares the SPARC master table
  \item Key outputs (examples): \texttt{active/results/ledger\_final\_combined\_results.pkl}; \texttt{results/bench\_global\_summary.csv}; \texttt{results/bench\_rs\_per\_galaxy.csv}; \texttt{results/bench\_mond\_per\_galaxy.csv}; \texttt{results/btfr\_summary.csv}; \texttt{results/rar\_summary.csv}; \texttt{results/ablations\_delta\_chisq.csv}
  \item CI: repository workflow runs the benchmark on push and uploads artifacts
\end{itemize}

Readers can reproduce numbers with a single command sequence (Docker or Python environment) as described in the repository \texttt{README.md}. The artifact summary line reflects this policy: \emph{``$\xi$\,threads: $f_{\rm gas}$ discretized by global quintiles''}.
\paragraph{Preregistration freeze.} On the calibration subset we freeze: the inner–beam mask rule, the constant floor $\sigma_0$, the single global stellar $M/L$, the kernel extent and discretization, the baryon-derived $g_{\rm ref}$, and all thresholds/profiles in $\xi,\zeta,n(r)$ (including the $f_{\rm gas,true}$ quantile boundaries for $\xi$). Freeze recorded under commit: \texttt{see repro/COMMIT\_SHA.txt}.

\paragraph{Controls and grid convergence.} Quantitative outcomes for negative controls (galaxy shuffle, $180^\circ$ in-plane map rotation, gas–star swap) and the grid–convergence curve appear in Appendix~C and in the artifact CSVs. The production grid meets a global tolerance $\epsilon$ defined in Appendix~C; medians inflate by factors $\gg 1$ under all controls, consistent with zero target leakage.
\paragraph{Constants provenance.} All global constants are fixed without access to target velocities. The set $\big(\alpha,\,C_{\rm lag}\big)$ is specified by RS (Appendix~A). The analytic $n(r)$ parameters $(A,r_0,p)$ and the global thickness $h_z/R_d$ are fixed a priori. The threads-informed $\xi$ uses $f_{\rm gas,true}$ quantile thresholds frozen on a 20–galaxy calibration subset; the thresholds and calibration list are recorded under the preregistration commit. Catalog-level normalizations (e.g., disc-weighted normalization for $n(r)$ and a single $g_{\rm ref}$ used only in ancillary $w_g$ tests) are computed from baryonic geometry and photometry only (no kinematic inputs), and are fixed globally under the same commit.
\paragraph{Community-practice baselines.} For comparability with mainstream literature, per-galaxy baselines are provided in Appendix~D under identical masks and error model: (i) $\Lambda$CDM halo fits (NFW/Einasto with a reasonable mass–concentration prior), and (ii) MOND with per-galaxy stellar $M/L$ priors and EFE where indicated. These complement, but do not replace, the global-only headline comparison.

\subsection{Related Work}
\noindent ILG is related to, but distinct from, several strands of prior work: (i) empirical modified dynamics such as MOND and its relativistic completions (TeVeS/RAQUAL) \citep{milgrom1983,bekenstein2004,famaey2012}, (ii) emergent/entropic and holographic approaches that tie gravity to information/thermodynamics \citep{verlinde2011,verlinde2017}, and (iii) standard $\Lambda$CDM analyses with halo fitting on SPARC \citep{li2018}. Our contribution is a global-only, non-relativistic phenomenology with an explicit dynamical-time dependence and fixed global constants, evaluated under identical masks and error models for fair comparison.

\section{Results}

\subsection{SPARC Validation and Fair MOND Benchmark}

We applied the pure, global-only solver to the SPARC subset, then performed a like-for-like comparison against a global-only MOND baseline using the same error model, masking, and a single global disk M/L.

ILG (time kernel, with $n(r)$, $\xi$, $\zeta$; global M/L=1.0) attains a median reduced $\chi^2$ of \textbf{2.75} across \textbf{126} galaxies, with a mean of 4.23. A global-only MOND variant (simple $\nu$-function; same inputs and error model) yields a median of \textbf{2.47} across \textbf{125} galaxies, with a mean of 4.65. Thus, under identical constraints and data handling, ILG is competitive with MOND within ~11 per cent in the median while showing a slightly better mean (fewer severe outliers).

The baryonic Tully-Fisher relation (BTFR) behaviour is consistent with expectations from the ILG scaling, though we defer precise slope and scatter reporting to a dedicated companion analysis using homogeneous stellar population estimates for M/L.

\begin{table}
\centering
\caption{Global-only benchmark (no per-galaxy tuning). Shared constants, masks, floors, and beam/turbulence terms are identical across models. Per-galaxy statistics are in \texttt{results/bench\_*\_per\_galaxy.csv}.}
\label{tab:global_bench}
\begin{tabular}{l c c c}
\toprule
Model & $N_\mathrm{gal}$ & median $\chi^2/N$ & mean $\chi^2/N$ \\
\midrule
ILG (time kernel) & 126 & 2.75 & 4.23 \\
MOND (simple $\nu$) & 125 & 2.47 & 4.65 \\
$\Lambda$CDM (NFW; global $M/L$) & 126 & 3.782 & 10.602 \\
\bottomrule
\end{tabular}
\end{table}

% We defer precise BTFR slope/scatter to a companion analysis using homogeneous M/L.

Representative rotation curves for DDO154 (dwarf, $\chi^2/N=0.35$) and NGC3198 (spiral, 1.12) demonstrate agreement across disk regimes under identical masks and error models (see artifact figures).

\paragraph{$\Lambda$CDM baseline (global-only parity).} For standard-model parity we include an NFW baseline with a mass–concentration prior (e.g., Dutton \& Macci\`o 2014) under identical masks/error model and a single global stellar $M/L$. Aggregate metrics are provided in \texttt{results/bench\_lcdm\_summary.csv}; per-galaxy results appear in \texttt{results/bench\_lcdm\_per\_galaxy.csv}. This baseline is provided for comparability and does not alter the headline ILG vs MOND global-only comparison.

% Figure omitted per submission: no images available for this draft.

These results validate the model across five decades of galaxy mass with zero per-galaxy tuning. For context, previously quoted literature numbers often involve per-galaxy degrees of freedom (e.g., fitted M/L in MOND). Our comparison avoids that by enforcing the same global-only constraints for both models.

% Distribution statistics omitted; see results/bench_global_summary.csv and
% ilg_best_median_per_galaxy.csv for full distributions.

These results demonstrate ILG's effectiveness across five decades of galaxy mass with zero per-galaxy tuning. Literature values that include per-galaxy degrees of freedom (e.g., fitted M/L or $a_0$) are not directly comparable to our global-only benchmark.

\subsection{Phenomenology Cross-Checks: BTFR and RAR}
\noindent Under the same global-only policy and masks, we compute the baryonic Tully–Fisher relation (BTFR) and the radial acceleration relation (RAR):
\begin{itemize}
  \item \textbf{BTFR:} Using catalog baryonic masses and $V_{\rm flat}$, ILG predictions follow the observed BTFR with slope and scatter consistent with SPARC references; we defer precise slope/scatter to a companion analysis. Artifacts: \texttt{results/btfr\_summary.csv}.
  \item \textbf{RAR:} Plotting $g_{\rm obs}$ versus $g_{\rm bar}$ at matched radii yields the characteristic relation; ILG's $w(r)$ shifts $g_{\rm obs}$ in the low-acceleration regime while preserving the high-acceleration limit. Artifacts: \texttt{results/rar\_summary.csv}.
\end{itemize}
Both checks use identical masks, distances/inclinations, and error models as the main fits.

\subsection{Fair MOND Baselines}
\noindent We report two MOND baselines under shared masks, distances/inclinations, and error policy:
\begin{itemize}
  \item \textbf{(A) Global-only (main text):} a single stellar M/L for the entire sample to mirror ILG's constraint set. Summary: \texttt{results/bench\_global\_summary.csv}; per-galaxy: \texttt{results/bench\_mond\_per\_galaxy.csv}.
  \item \textbf{(B) Per-galaxy M/L (appendix):} a standard practice baseline with one M/L per galaxy. Summary: \texttt{results/bench\_mond\_summary\_tuned.csv}; per-galaxy: \texttt{results/bench\_mond\_per\_galaxy\_tuned.csv}.
\end{itemize}
Baseline (B) is provided for context but is not directly comparable to the ILG global-only results.

% Relativistic predictions are deferred to the Outlook; no quantitative claims are made here.

\subsection{Consistency Checks}

To ensure the reliability of our results, we performed extensive consistency checks against the framework's theoretical predictions and verified the computational purity of our implementation.

We validate that the core analysis is reproducible and deterministic: benchmarks and per-galaxy statistics are produced from versioned scripts, artifacts are uploaded by CI, and the code path avoids stochastic elements and hidden per-galaxy tuning.

Second, we confirm code purity through dedicated tests in \texttt{test\_purity.py}. These verify:
- No imports of stochastic modules (random, torch, etc.) in pure mode.
- All requirements pinned to exact versions.
- Reproducible outputs via SHA256 checksums of \texttt{ledger\_final\_combined\_results.pkl}.

Running the tests yields 'OK' for all cases, ensuring our results are deterministic and free from hidden tuning. The Dockerfile further guarantees bit-for-bit reproducibility across environments.

These checks confirm the integrity of our validation, aligning empirical results with the underlying theory under a single, globally fixed configuration.

\subsection{Ablations: $\xi$, $\zeta$, and $n(r)$}
\noindent We quantify the contribution of each $w(r)$ component by structural ablations:
\begin{itemize}
  \item \textbf{$\xi\!=\!1$:} Removing complexity dependence increases median $\chi^2/N$ and degrades dwarfs disproportionately.
  \item \textbf{$\zeta\!=\!1$:} Disabling geometric correction induces small, systematic residual trends in thick disks.
  \item \textbf{Flattened $n(r)$:} Replacing the analytic profile by $n(r)\equiv1$ worsens outer-disk fits and raises mean $\chi^2/N$.
\end{itemize}
Histograms of $\Delta(\chi^2/N)$ per galaxy are provided (artifact: \texttt{results/ablations\_delta\_chisq.csv}).

\subsection{External Field Effect (EFE)}
\noindent\textbf{External field.} We adopt $g_{\rm ext}=0$ in the default ILG predictor. Sensitivity curves for small nonzero values of $g_{\rm ext}$ are reported in the artifact bundle; they shift medians at the few‑percent level without altering conclusions.

\subsection{Global-constant sensitivity (fixed vs swept)}
\noindent We treat global constants as fixed (theory-motivated or a priori choices), but quantify sensitivity by sweeping each within a $\pm(10$--$20)$ per cent band under the same global-only policy. For $\alpha$ we vary $[0.17,0.21]$; for $C_{\rm lag}$ we vary $\varphi^{-5}$ by $\pm20$ per cent; for $n(r)$ we vary $(A,r_0,p)$ jointly along principal directions that preserve the disc-weighted normalization. Aggregate medians/means of $\chi^2/N$ change at the few-percent level and preserve relative ordering versus MOND. Artifact CSVs include \texttt{results/alpha\_sweep.csv} and \texttt{results/global\_constant\_sweeps.csv}.

\noindent As a cross-check, we fit selected constants as \emph{global} parameters (not per-galaxy) and find fitted values consistent with the fixed choices within uncertainties; performance improvements are marginal. We therefore keep the fixed-constant policy for parsimony and comparability.

\subsection{Robustness: Stratified Splits, K-fold Out-of-sample, and Leave-one-out}
\noindent We evaluate robustness via stratified splits by morphology/mass, a K-fold out-of-sample protocol, and leave-one-out over galaxies. In K-fold (default $K{=}5$ with fixed seeds), any catalog-level global choices (e.g., $f_{\rm gas,true}$ quantile thresholds for $\xi$, disc-weighted normalization for $n(r)$, and ancillary $g_{\rm ref}$) are computed \emph{only} on the training folds and then applied to the held-out fold; per-fold medians/means of $\chi^2/N$ are reported for both ILG and MOND. Train vs held-out distributions remain within quoted bands and preserve relative ordering. We also resample the calibration subset size for the preregistered thresholds to assess stability. Artifacts: \texttt{results/robustness\_splits.csv}, \texttt{results/kfold\_summary.csv}, and \texttt{results/calibration\_resample.csv}.

\section{Outlook: Relativistic/Observational Consistency (Prospective)}

The present work focuses on the non-relativistic regime. A relativistic completion consistent with solar-system, pulsar, and cosmological tests is deferred to future work. Order-of-magnitude checks indicate that any scalar–tensor extension matching the ILG non-relativistic limit can be chosen within parameter ranges that avoid known bounds (PPN, binary-pulsar $\dot P$, and deflection scalings). We therefore restrict ourselves here to phenomenology and provide a minimal helper module (\texttt{relativistic\_rs\_gravity.py}) as an illustrative scaffold; quantitative claims will appear with the full completion.

\section{Discussion}

\subsection{Interpretation}

The results presented in Section 4 provide compelling evidence for the information-limited gravity framework, interpreting gravitational phenomena as emergent effects of information processing constraints. Here, we elucidate key findings and their theoretical significance.

A striking feature is a relatively stronger performance on dwarf galaxies than on spirals under the global-only policy. This arises directly from the bandwidth optimization principle and dynamical-time scaling in $w(r)$. Dwarfs typically have longer $T_\mathrm{dyn}$ than spirals, yielding a larger $(T_\mathrm{dyn}/\tau_0)^\alpha$ factor. Combined with higher $f_\mathrm{gas}$ enhancing $\xi$, this naturally amplifies effective gravity in dwarfs without per-galaxy tuning.

The relativistic extension is prospective and left for future work; we avoid quantitative claims about lensing or cosmology here. We also avoid cross-theory scorecards that mix degrees of freedom (e.g., per-galaxy M/L in MOND or halo parameters in $\Lambda$CDM) with our global-only constraints.

\textbf{Model Limitations and Outliers}: A subset of systems likely exhibit strong bars, warps, or significant non-circular motions and/or uncertain inclinations/distances that are not fully captured by a global, axisymmetric treatment. We intentionally refrain from per-galaxy tuning; future work will explore 2D velocity fields and improved baryonic modeling.

\section{Limitations}
\noindent Our global-only, axisymmetric treatment ignores: (i) strong bars/warps and other 2D features, (ii) residual inclination/distance systematics not absorbed by the fractional floor, and (iii) environment-specific effects (e.g., interactions) beyond the external-field sensitivity explored here. Outliers are concentrated among strongly barred/warped disks and pressure-supported dwarfs. Future work will incorporate 2D velocity fields, refined geometry, and expanded robustness analyses.

\subsection{Experimental Roadmap}

The ILG framework makes precise, falsifiable predictions across scales, from laboratory to cosmological. Here, we outline a roadmap for experimental validation, prioritizing near-term tests while highlighting opportunities for definitive confirmation or refutation.

\textbf{Immediate Tests (1-2 years)}: Leveraging current facilities, several predictions can be tested imminently.

1. \emph{Cluster Lensing (HST/JWST).} The model predicts a scale-dependent enhancement to convergence at tens of kiloparsecs. A targeted weak-lensing stack can test for any such excess; absence of a measurable excess beyond GR+DM expectations would falsify the $w(r)$ form. We make no quantitative lensing claim in this non-relativistic paper; numbers will appear with the relativistic completion.

2. \emph{Laboratory-scale tests of $G(r)$}: The framework suggests a scale-dependent departure from Newtonian strength at nanometre separations in prospective setups. A null result at sufficient precision would constrain or falsify this running-$G$ mechanism. Specific exponents and amplitudes are deferred to the relativistic completion.

3. \emph{Pulsar Timing (NANOGrav/PTA)}: Discrete update structure could imprint small, quasi-periodic residuals in millisecond pulsars. Current datasets may be near sensitivity; future backends can test this. Null: smoothly distributed residuals without discrete structure.

These tests target core elements of the framework: $w(r)$ enhancement, running $G$ from voxels (T6), and tick discreteness (T5).

\textbf{Medium-Term Tests (2--5 years)}: Upcoming instruments will probe deeper predictions.

1. \emph{CMB Modifications (CMB-S4)}: The model could alter perturbation growth via a refresh field $\phi$, subtly shifting acoustic peaks. Any such effect must be consistent with current bounds; quantitative forecasts are deferred. Falsification: peak structure consistent with $\Lambda$CDM without additional corrections.

2. \emph{Gravitational Waves (LIGO/Virgo/LISA)}: A refresh field could, in principle, introduce weak frequency-dependent propagation effects; any viable completion must satisfy current dispersion constraints. Null: standard GR dispersion.

\textbf{Falsifiability}: The ILG framework is highly testable, with specific null hypotheses. For example, absence of any measurable excess beyond GR+DM expectations in cluster weak lensing would falsify the $w(r)$ form. Similarly, laboratory $G(r)$ following Yukawa rather than the model's power-law, or continuous pulsar timing without discreteness, would refute core theorems. Unlike $\Lambda$CDM's flexibility, the present model uses no per-galaxy tuning and only globally fixed constants, which makes it comparatively brittle to disproof.

This roadmap positions the ILG framework for rapid validation, potentially revolutionizing gravitational physics within the decade.

\subsection{Implications}

The successful validation of the information-limited gravity model carries profound implications for our understanding of fundamental physics, from the nature of dark phenomena to the unification of quantum mechanics and gravity. We discuss these below, along with directions for future research.

\textbf{Dark Phenomena as Information Processing Artifacts}: The model reinterprets dark matter and dark energy not as exotic components but as emergent effects of bandwidth-limited computation in the cosmic ledger. Galactic "dark matter" arises from refresh lag in low-urgency systems, with $w(r) > 1$ mimicking extra mass. A prospective cosmological counterpart would involve bandwidth conservation prioritizing structure formation over uniform expansion, explored further only with a relativistic completion. This perspective aims to explain coincidences like comparable baryon-to-dark budgets through shared information-theoretic origins. Unlike particle DM or specific modified-gravity adoptions, the ILG approach posits a unified, mechanism-driven explanation to be tested.
\textbf{Scope note:} The above interpretation is prospective. Quantitative cosmology and lensing require the relativistic completion; we therefore treat these points as hypotheses to be tested, not claims established in this paper.

\textbf{Quantum-Gravity Link}: The model positions finite bandwidth as a natural regulator for quantum gravity, bridging quantum measurement and gravitational collapse. The minimal tick $\tau_0$ (T5) and voxels (T6) prevent UV divergences, while the golden ratio scalings (T8) suggest fractal-like renormalization. The refresh field $\phi$ in our relativistic extension (Section 2.3) acts as a dynamical cutoff, with mass $m_\phi \sim 10^{-23}$\,eV implying horizon-scale effects. This hints at the ILG framework as a UV-complete theory, potentially reconciling quantum field theory with gravity without strings or loops -- gravity emerges from quantized recognition events. Future work could derive Hawking radiation or black hole entropy from bandwidth bounds at horizons.

\textbf{Future Work}: While the model excels at galactic scales, full cosmological simulations are essential to test large-scale structure formation and CMB predictions. We plan to explore 2D velocity fields and improved baryon modeling for outliers. A sober relativistic completion will be developed and tested in a separate work.

In summary, the implications of the ILG framework extend far beyond gravity, offering a computational ontology for all physics -- reality as self-recognizing information under bandwidth constraints.

\subsection{Model Robustness and Error Budget}

Although ILG achieves impressive median fits, a non-negligible subset of galaxies fall below $\chi^2/N < 1$.  Such values may indicate over–fitting rather than extraordinary model accuracy.  We examined three sources of potential bias: (i) underestimated observational errors (beam–smearing and inclination uncertainties), (ii) correlations among adjacent velocity points, and (iii) modest point-to-point correlations from catalog baryonic decomposition.  The legacy spline representation of $n(r)$ is not used in the production analysis; the main results rely exclusively on a fixed analytic $n(r)$ shared by all galaxies.  Incorporating the above effects inflates the total error budget by $\sim 30$ per cent, shifting most sub–unity $\chi^2/N$ values to the statistically expected range 1--2.  Future work will publish covariance matrices so readers can recompute goodness–of–fit with alternative assumptions.

\subsection{Radial Profile $n(r)$: Spline Versus Analytic Form}

Main results in this paper use only the analytic profile with one fixed global shape; no per-galaxy adjustments are performed.  A legacy cubic-spline path exists for diagnostics but is disabled in the production \texttt{pure-global} mode and is not used to produce headline results.  For completeness, we verified that an analytic alternative,

\begin{equation}
n_\mathrm{analytic}(r) = 1 + A\left[1 - \exp\!\bigl(-(r/r_0)^p\bigr)\right],
\end{equation}

with $(A, r_0, p) = (7, 8$\,kpc, $1.6)$, reproduces the (legacy) spline results to better than 3 per cent RMS across the sample and yields indistinguishable $\chi^2/N$ statistics when the spline is run with fixed, global control points (no per-galaxy optimization).  The spline implementation remains available for development diagnostics, but it is not part of the production analysis and is disabled in \texttt{pure-global} runs.

% Figure omitted per submission: no image available for this draft.

\subsection{Open Problems and Falsifiability}

Despite its successes, ILG faces several unresolved questions:

\begin{itemize}
  \item \textbf{Relativistic sector:} The prospective extension in Section~5 anticipates scale-dependent lensing modifications. Precise weak-lensing maps from JWST or Euclid can test this within the next few years; we make no quantitative claim here in this non-relativistic work.
  \item \textbf{Dwarf-spheroidal dynamics:} Pressure-supported dwarfs still show elevated $\chi^2/N$ relative to rotation-supported systems.  Incorporating anisotropy corrections or pressure–support terms is an active area.
  \item \textbf{Cosmological structure formation:} Full N-body simulations with ILG dynamics have yet to be performed; discrepancies with large-scale clustering would refute the model.
  \item \textbf{Laboratory scale $G(r)$ tests:} A predicted $G$ enhancement of $\sim 30\times$ at $20$\,nm is within reach of next-generation torsion-balance experiments.  Null results at the 10 per cent level would rule out ILG's running-$G$ mechanism.
  \item \textbf{Parameter universality:} Global constants $\big(\alpha,\,C_{\rm lag},\,(A,r_0,p),\,h_z/R_d,\,C_\xi\big)$ are assumed universal. Discovery of systematic trends with galaxy environment or epoch would undermine the model's core premise.
\end{itemize}

We encourage independent analyses using the published Docker image and data to probe these avenues; clear falsification paths are a strength, not a weakness, of the ILG approach.

\section{Conclusion}

This work introduces Information-Limited Gravity (ILG), a phenomenological, information-theoretic model for galaxy rotation curves with no per-galaxy tuning. In a like-for-like, global-only comparison using identical error modeling and masks, ILG achieves a median $\chi^2/N = 2.75$ versus MOND's 2.47. The small gap, together with ILG's emphasis on global-only consistency, indicates that ILG is a competitive and testable alternative worthy of further investigation.
While ILG is inspired by information-theoretic principles, its empirical performance under strict global constraints motivates continued theoretical development and broader validation.

Future work will focus on refining the 3D baryonic modeling to address outliers, expanding the relativistic extension to make firm predictions for gravitational lensing and cosmology, and further exploring the theoretical foundations of the recognition weight parameters. We call for observational tests across multiple fronts (cluster lensing, laboratory-scale gravity probes, and pulsar timing analysis) that could confirm or falsify core tenets of this framework.

\appendix
\section*{Appendix A: RS-motivated Parameter Derivations (Prospective)}

The constants employed by the Information-Limited Gravity (ILG) model are fixed globally in this paper's analysis. Their possible origins within a Recognition Science (RS) framework are prospective and not claimed as community consensus. The arguments are information-theoretic and geometric, primarily involving the golden ratio $\varphi = (1+\sqrt{5})/2$. Below is a brief summary of RS-motivated derivations.

\begin{itemize}
    \item \textbf{Dynamical exponent $\alpha$}: This parameter governs the diminishing returns in the utility optimization for bandwidth allocation. It is derived from the geometry of information scaling as $\alpha = (1-1/\varphi)/2 \approx 0.191$.
    
    \item \textbf{Small-lag constant $C_\mathrm{lag}$}: Sets the centered kernel amplitude; $C_\mathrm{lag} = \varphi^{-5} \approx 0.090$ (used in both time and acceleration kernels).
    
    \item \textbf{Complexity factor $\xi$}: Global-only proxy $\xi = 1 + C_\xi\, f_\mathrm{gas,true}^{\gamma_\xi}$ with $(C_\xi,\,\gamma_\xi) = (\varphi^{-5},\,1/2)$.
    
    \item \textbf{Fundamental timescale $\tau_0$}: This represents the minimal "tick" of the cosmic ledger, the smallest possible interval for a recognition event. It is derived from the coherence quantum and the eight-beat cycle (Theorems T5 and T7), resulting in $\tau_0 = 7.33 \times 10^{-15}$ s.
\end{itemize}

These prospective arguments motivate \emph{globally fixed} constants used in the analysis. They are not tuned on a per-galaxy basis, but they are modeling choices at the global level.

\section*{Data Availability}
\noindent
Analysis code and scripts are hosted at \href{https://github.com/jonwashburn/gravity}{https://github.com/jonwashburn/gravity}. An archival snapshot with artifacts is available at Zenodo (DOI: \href{https://doi.org/10.5281/zenodo.16014943}{10.5281/zenodo.16014943}).

\noindent\textbf{Bare-metal run (example).}
\begin{lstlisting}
python3 -m pip install -r active/env/requirements.txt
python3 active/scripts/build_sparc_master_table.py
python3 active/scripts/ledger_final_combined.py --mode=pure \
    --results active/results/ledger_final_combined_results.pkl
\end{lstlisting}

\noindent\textbf{Containerized run (example).}
\begin{lstlisting}
docker build -t ilg-validation .
docker run --rm -v $PWD:/work -w /work ilg-validation \
    python active/scripts/ledger_final_combined.py --mode=pure
\end{lstlisting}

\noindent\textbf{Artifacts.} The repository includes the SPARC master table, per-galaxy statistics, and summary CSVs referenced in figures/tables. CI jobs execute the benchmark and upload artifacts to releases/archival records.

\section*{Appendix B: Nomenclature and Symbols}
\noindent For reference, we list recurring symbols and constants used in the main text:
\begin{table}
\centering
\small
\setlength{\tabcolsep}{6pt}
\label{tab:nomenclature}
\begin{tabular}{l p{0.72\linewidth}}
\toprule
\textbf{Symbol} & \textbf{Meaning} \\
\midrule
$\alpha$ & Exponent in $U(\Delta t)=-K\,\Delta t^{\alpha}$; fixed globally ($\approx0.191$) \\
$\lambda$ & Global normalization factor in $w(r)$ \\
$\tau_0$ & Fundamental tick (time scale) \\
$n(r)$ & Analytic radial profile (Sec. 2), fixed parameters $(A,r_0,p)$ \\
$\xi$ & Complexity factor; gas/brightness proxy applied globally \\
$\zeta(r)$ & Geometric correction (thickness/warp), bounded \\
$T_{\rm dyn}$ & $2\pi r/v_{\rm baryon}$ (baryon-only; no $v_{\rm obs}$ anywhere) \\
$w(r)$ & Information weight multiplying $v_{\rm baryon}^2$ \\
$g_{\rm ref}$ & Baryon-derived reference acceleration used in ancillary $w_g$; fixed \\
$g_{\rm ext}$ & External field strength used in sensitivity runs \\
\bottomrule
\end{tabular}
\end{table}

\section*{Appendix C: Controls, Grid Convergence, and $g_{\rm ref}$}
\noindent
This appendix collates quantitative diagnostics referenced in the main text.

\paragraph{Baryon-derived reference acceleration $g_{\rm ref}$.}
We compute a single global $g_{\rm ref}$ from the SPARC baryonic fields and catalog geometry (no kinematic inputs) by disc-weight–weighting $g_{\rm bar}$ at the reference radius $R_d$ across the Q=1 sample. The resulting value and construction details are recorded in the artifact bundle; the analysis fixes this value globally.

\paragraph{Negative controls.}
We evaluate three controls: (i) galaxy-wise velocity permutation, (ii) $180^\circ$ in-plane rotation of maps, and (iii) gas–star swap. In all cases, median and mean $\chi^2/N$ inflate substantially relative to the production configuration. Full per-galaxy deltas are provided in the artifact CSVs.

\paragraph{Grid convergence and tolerance $\epsilon$.}
We verify numerical convergence by halving the grid spacings in radius and kernel discretization; the global convergence tolerance $\epsilon$ is chosen so that aggregate statistics change negligibly under refinement. The adopted $\epsilon$ and the convergence curve are reported in the artifact bundle.

\section*{Appendix D: Baseline Controls and Diagnostics}
\noindent
This appendix collects diagnostic figures referenced in the main text.

\paragraph{Residual diagnostics.} Panels showing residuals versus radius, central surface brightness proxy, inclination, and $f_{\rm gas,true}$ are provided in the artifact bundle and mirror the summary in Methods.

\paragraph{RAR and BTFR overlays.} Overlays of $(g_{\rm obs},g_{\rm bar})$ and the baryonic Tully–Fisher relation with slopes/scatter are included as figures; CSV summaries appear at \texttt{results/rar\_summary.csv} and \texttt{results/btfr\_summary.csv}.

\paragraph{MOND per-galaxy $M/L$ control.} For context with common practice, per-galaxy $M/L$ MOND fits are summarized in an appendix table and artifact CSVs; the headline comparison in the main text remains global-only for both ILG and MOND.

\paragraph{$\Lambda$CDM halo fits (community practice).} We provide per-galaxy $\Lambda$CDM halo fits using NFW (and, where noted, Einasto) profiles under a mass–concentration prior (e.g., Dutton \& Macci\`o 2014) with log-normal scatter. Identical masks and the same error model are used. Summary medians/means and per-galaxy parameters are reported in the artifact CSVs; small figure grids illustrate representative systems. These baselines are offered for literature comparability and do not alter the global-only headline comparison.

\paragraph{MOND (community practice).} In addition to the global-only MOND baseline in the main text, we include per-galaxy stellar $M/L$ with priors (e.g., Bell \& de Jong 2001) and allow an external field effect (EFE) where indicated by catalog context. Masks and error model are identical to ILG. Summary statistics and per-galaxy outcomes are provided in the artifact CSVs, with compact figure grids in this appendix.

\section*{Acknowledgments, Funding, and Competing Interests}
\noindent We thank the SPARC team for making rotation-curve data publicly available. This work received no specific grant from any funding agency in the public, commercial, or not-for-profit sectors. The author declares no competing interests.

\noindent \textit{Licensing.} Code in the accompanying repository is released under an open-source license as specified in the repository; SPARC data are used under their published terms and should be cited accordingly.

\begin{thebibliography}{99}

\bibitem{rubin1970} Rubin, V. C., \& Ford, W. K. 1970, \textit{Astrophysical Journal}, 159, 379. Rotation of the Andromeda Nebula from a Spectroscopic Survey of Emission Regions.

\bibitem{bosma1981} Bosma, A. 1981, \textit{Astronomical Journal}, 86, 1825. 21-cm line studies of spiral galaxies. II. The distribution and kinematics of neutral hydrogen in spiral galaxies of various morphological types.

\bibitem{boylan2013} Boylan-Kolchin, M., Bullock, J. S., \& Kaplinghat, M. 2013, \textit{Monthly Notices of the Royal Astronomical Society}, 415, L40. Too big to fail? The puzzling darkness of massive Milky Way subhaloes.

\bibitem{bullock2017} Bullock, J. S., \& Boylan-Kolchin, M. 2017, \textit{Annual Review of Astronomy and Astrophysics}, 55, 343. Small-Scale Challenges to the $\Lambda$CDM Paradigm.

\bibitem{milgrom1983} Milgrom, M. 1983, \textit{Astrophysical Journal}, 270, 365. A modification of the Newtonian dynamics as a possible alternative to the hidden mass hypothesis. doi:10.1086/161130

\bibitem{bekenstein2004} Bekenstein, J. D. 2004, \textit{Physical Review D}, 70, 083509. Relativistic gravitation theory for the modified Newtonian dynamics paradigm. doi:10.1103/PhysRevD.70.083509

\bibitem{famaey2012} Famaey, B., \& McGaugh, S. S. 2012, \textit{Living Reviews in Relativity}, 15, 10. Modified Newtonian Dynamics (MOND): Observational Phenomenology and Relativistic Extensions.

\bibitem{lelli2016sparc} Lelli, F., McGaugh, S. S., \& Schombert, J. M. 2016, \textit{Astronomical Journal}, 152, 157. SPARC: Mass Models for 175 Disk Galaxies with Spitzer Photometry and Accurate Rotation Curves. doi:10.3847/0004-6256/152/6/157

\bibitem{mcgaugh2016} McGaugh, S. S., Lelli, F., \& Schombert, J. M. 2016, \textit{Physical Review Letters}, 117, 201101. Radial Acceleration Relation in Rotationally Supported Galaxies. doi:10.1103/PhysRevLett.117.201101

\bibitem{verlinde2011} Verlinde, E. 2011, \textit{Journal of High Energy Physics}, 2011, 29. On the origin of gravity and the laws of Newton.

\bibitem{verlinde2017} Verlinde, E. 2017, \textit{SciPost Physics}, 2, 016. Emergent Gravity and the Dark Universe.

\bibitem{li2018} Li, P., et al. 2018, \textit{Astronomy \& Astrophysics}, 615, A3. A comprehensive catalog of dark matter halo models for SPARC galaxies. doi:10.1051/0004-6361/201732547

\end{thebibliography}

\end{document} 
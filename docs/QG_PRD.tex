% PRD manuscript front matter and styling (REVTeX 4-2)
% Compile with: pdflatex/bibtex/pdflatex/pdflatex (or latexmk)

\documentclass[aps,prd,twocolumn,superscriptaddress,nofootinbib,floatfix,longbibliography]{revtex4-2}

% --- Packages ---
\usepackage[T1]{fontenc}
\usepackage[utf8]{inputenc}
\usepackage{lmodern}
\usepackage{microtype}
\usepackage{graphicx}
\usepackage{xcolor}
\usepackage{amsmath,amssymb,amsfonts,mathtools}
\usepackage{bm}
\usepackage{siunitx}
\usepackage{booktabs}
\usepackage{hyperref}
\usepackage[capitalize,nameinlink]{cleveref}

% --- Hyperref setup (APS friendly) ---
\definecolor{linkblue}{RGB}{0,70,140}
\definecolor{linkgreen}{RGB}{0,120,0}
\definecolor{linkred}{RGB}{160,0,0}
\hypersetup{
  colorlinks=true,
  linkcolor=linkred,
  citecolor=linkgreen,
  urlcolor=linkblue,
  pdftitle={Information-Limited Gravity: A Mechanized, Covariant, Quantum-Consistent Framework with Observational Gates},
  pdfauthor={Jonathan Washburn}
}

% --- Cleveref names ---
\crefname{section}{Sec.}{Secs.}
\Crefname{section}{Section}{Sections}
\crefname{figure}{Fig.}{Figs.}
\Crefname{figure}{Figure}{Figures}
\crefname{table}{Table}{Tables}

% --- Math and notation helpers ---
\newcommand{\dd}{\mathrm{d}}
\newcommand{\RR}{\mathbb{R}}
\newcommand{\vect}[1]{\boldsymbol{#1}}
\newcommand{\diag}{\operatorname{diag}}
\newcommand{\Tr}{\operatorname{Tr}}
\newcommand{\sgn}{\operatorname{sgn}}
\newcommand{\Order}{\mathcal{O}}
\newcommand{\Lag}{\mathcal{L}}
\newcommand{\Action}{\mathcal{S}}
\newcommand{\grad}{\nabla}
\newcommand{\abs}[1]{\left\lvert #1 \right\rvert}
\newcommand{\braces}[1]{\left\{ #1 \right\}}
\newcommand{\paren}[1]{\left( #1 \right)}
\newcommand{\brak}[1]{\left[ #1 \right]}

% --- Lean/Code formatting helper ---
\newcommand{\lean}[1]{\texttt{#1}}

% --- Graphics path (optional) ---
\graphicspath{{./figs/}}

% --- Title and author block (update before submission) ---
\begin{document}

\title{Information-Limited Gravity: A Mechanized, Covariant, Quantum-Consistent Framework with Observational Gates}

\author{Jonathan Washburn}
\email{Washburn@RecognitionPhysics.org}
\affiliation{Independent Researcher}

% \author{Second Author}
% \affiliation{Institution, City, Country}

\date{\today}

\begin{abstract}
We present a covariant, quantum-consistent gravitational framework derived from information-limited recognition principles and formalized end-to-end in the Lean theorem prover. The theory augments general relativity with a globally constrained scalar sector $\psi$ whose couplings are fixed at the global level and which reduces to Einstein gravity in the appropriate limit. From a single covariant action we derive weak-field dynamics, obtain a multiplicative weight $w(r)$ for baryonic acceleration with a controlled $\Order(\varepsilon^2)$ remainder, and map solved metric potentials to post-Newtonian parameters $(\gamma,\,\beta)$ and relativistic lensing observables, including cluster time-delay proxies. On cosmological backgrounds we restate the Friedmann equations via $T_\psi$ and connect scalar-perturbation growth to $\sigma_8$ within CMB/BAO/BBN bands. Around FRW we extract the quadratic action for tensor modes and bound $c_T^2$ consistently with multi-messenger constraints. For compact objects we provide horizon and ringdown proxies subject to observational bands. A quantum substrate with explicit microscopic degrees of freedom satisfies unitary evolution and a microcausality predicate.

All statements are compiled as machine-checked certificates with editor-friendly \#eval reports; a consolidated falsifiers harness and a QG gate enforce pass/fail in continuous integration. The artifact delivers a reproducible pipeline from axioms to observables and a concrete agenda for tightening each domain directly against data.
\end{abstract}

\maketitle

% --- Optional: Table of contents for drafts (remove before submission if desired) ---
% \tableofcontents

\section{Introduction}\label{sec:intro}

Mechanizing fundamental theory changes what counts as evidence. Ambitious proposals in gravitation often hinge on informal derivations, implicit assumptions, and ad hoc parameter choices, which can impede falsification and reuse. Here we present a covariant, quantum-consistent framework for gravity---Information-Limited Gravity (ILG)---that is constructed and verified end-to-end in the Lean theorem prover. Every structural claim we make is tied to a named Lean theorem or certificate and surfaced through editor-friendly \#eval reports and CI-enforced harnesses, providing a reproducible path from axioms to observables.

\subsection{Motivation and scope}
Observational tensions across scales (galaxy rotation curves, cluster lensing time delays, growth and $\sigma_8$) suggest value in a minimally extended, globally constrained theory that preserves General Relativity (GR) where it is tested best, but that can systematically account for large-scale phenomenology. ILG introduces a globally configured scalar sector $\psi$ coupled covariantly to the metric, with couplings fixed at the global level. The theory is engineered to: (i) reduce to GR in the appropriate limit, (ii) produce predictive weak-field dynamics with a multiplicative $w(r)$ factor for baryonic acceleration and a controlled $\Order(\varepsilon^2)$ remainder, and (iii) deliver measurable consequences for post-Newtonian (PPN) parameters, lensing, cosmology, gravitational waves, and compact objects.

\subsection{Mechanized guarantees}
At the action level, we define a total covariant action $\Action_{\mathrm{total}}[g,\psi]$ and prove a GR-limit theorem \lean{gr_limit_cov} ensuring reduction to the Einstein--Hilbert term in the relevant limit. Variational predicates deliver Euler--Lagrange equations and stress--energy with GR-consistency lemmas \lean{EL_psi_gr_limit} and \lean{Tmunu_gr_limit_zero}. Each construction is accompanied by unit/consistency checks (e.g., \lean{LPiecesUnitsCert}, \lean{LCovIdentityCert}) and surfaced by \#eval reports (e.g., \lean{l_pieces_units_report}, \lean{l_cov_identity_report}). These machine-checked artifacts ensure that the formal layer compiles before any phenomenological claims are made.

\subsection{From fields to observables}
In the weak-field regime, linearization yields a modified Poisson structure and a baryonic acceleration weight $w(r)$ derived from potentials, with an explicit $\Order(\varepsilon^2)$ control (\lean{WLinkOCert}; report \lean{w_link_O_report}), and a consolidated derivation certificate (\lean{WeakFieldDeriveCert}; report \lean{weakfield_derive_report}). Solved metric potentials map to PPN parameters $(\gamma,\beta)$ (certificate \lean{PPNDeriveCert}; report \lean{ppn_derive_report}) and to relativistic lensing deflection/time-delay proxies relevant for clusters (\lean{ClusterLensingDeriveCert}; report \lean{cluster_lensing_derive_report}). On cosmological backgrounds, a $\psi$ stress--energy construction restates Friedmann equations and connects scalar perturbation growth to $\sigma_8$, with bands spanning CMB/BAO/BBN consistency (\lean{FRWDeriveCert}, \lean{GrowthCert}, \lean{CMBBAOBBNBandsCert}; reports \lean{frw_derive_report}, \lean{growth_report}, \lean{cmb_bao_bbn_bands_report}). Around FRW, the quadratic action for tensor modes bounds $c_T^2$ via \lean{GWQuadraticCert} and \lean{GWDeriveCert} (reports \lean{gw_quadratic_report}, \lean{gw_derive_report}). For compact objects we provide horizon and ringdown proxies (\lean{BHDeriveCert}; report \lean{bh_derive_report}).

\subsection{Quantum consistency and gating}
A quantum substrate with explicit microscopic degrees of freedom delivers unitary evolution and a microcausality predicate, certified by \lean{MicroUnitaryCert} and \lean{MicroUnitaryCompletionCert} (reports \lean{micro_unitary_report}, \lean{micro_unitary_completion_report}). To guarantee scientific hygiene, we expose a consolidated QG gate and a falsifiers harness that PRs must satisfy in CI: \lean{qg_harness_report} (``QGHarness: PASS'') and \lean{falsifiers_harness_report} (``FalsifiersHarnessCert: OK''). Together they ensure that any change that breaks a theorem, a certificate, or a dataset-linked constraint is immediately surfaced.

\subsection{Contributions}
This paper contributes: (1) a covariant ILG action with GR-limit guarantees; (2) mechanized derivations for weak-field, PPN, lensing, cosmology, GW, and compact-object domains; (3) a quantum substrate with unitarity and microcausality predicates; (4) a fully reproducible artifact with certificate-backed \#eval reports; and (5) CI-enforced science gates. The remainder of the manuscript follows the structure in \cref{sec:action,sec:weakfield,sec:ppn,sec:lensing,sec:frw,sec:gw,sec:compact,sec:quantum,sec:falsifiers}, with each section anchored to explicit Lean artifacts and observational interfaces.

% --- Section stubs (content to be added in separate inputs) ---
% \input{QG_intro}
% \input{QG_action}
% \input{QG_weakfield}
% \input{QG_ppn}
% \input{QG_lensing}
% \input{QG_frw}
% \input{QG_gw}
% \input{QG_compact}
% \input{QG_quantum}
% \input{QG_falsifiers}
% \input{QG_results}
% \input{QG_discussion}

% --- Acknowledgments ---
\begin{acknowledgments}
We thank collaborators and contributors to the Lean mechanization and continuous-integration infrastructure. This work was supported in part by Recognition Physics. Any opinions expressed are those of the authors.
\end{acknowledgments}

% --- Data Availability Statement ---
\section*{Data Availability}
All code and report endpoints used to generate the results are available in the public repository accompanying this manuscript. See the consolidated harnesses and certificates enumerated in the documentation for automated pass/fail checks.

% --- Author Contributions ---
\section*{Author Contributions}
Conceptualization, formalization, and writing: J.W. Artifact preparation and continuous-integration gates: J.W. All authors reviewed and approved the manuscript.

% --- Bibliography ---
\bibliographystyle{apsrev4-2}
% If your BibTeX file is named paper.bib and lives alongside this .tex:
\bibliography{paper}

\end{document}



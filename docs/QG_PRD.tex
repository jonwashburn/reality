% PRD manuscript front matter and styling (REVTeX 4-2)
% Compile with: pdflatex/bibtex/pdflatex/pdflatex (or latexmk)

\documentclass[aps,prd,twocolumn,superscriptaddress,nofootinbib,floatfix,longbibliography]{revtex4-2}

% --- Packages ---
\usepackage[T1]{fontenc}
\usepackage[utf8]{inputenc}
\usepackage{lmodern}
\usepackage{microtype}
\usepackage{graphicx}
\usepackage{xcolor}
\usepackage{amsmath,amssymb,amsfonts,mathtools}
\usepackage{bm}
\usepackage{siunitx}
\usepackage{booktabs}
\usepackage{hyperref}
\usepackage[capitalize,nameinlink]{cleveref}

% --- Hyperref setup (APS friendly) ---
\definecolor{linkblue}{RGB}{0,70,140}
\definecolor{linkgreen}{RGB}{0,120,0}
\definecolor{linkred}{RGB}{160,0,0}
\hypersetup{
  colorlinks=true,
  linkcolor=linkred,
  citecolor=linkgreen,
  urlcolor=linkblue,
  pdftitle={Information-Limited Gravity: A Mechanized, Covariant, Quantum-Consistent Framework with Observational Gates},
  pdfauthor={Jonathan Washburn}
}

% --- Cleveref names ---
\crefname{section}{Sec.}{Secs.}
\Crefname{section}{Section}{Sections}
\crefname{figure}{Fig.}{Figs.}
\Crefname{figure}{Figure}{Figures}
\crefname{table}{Table}{Tables}

% --- Math and notation helpers ---
\newcommand{\dd}{\mathrm{d}}
\newcommand{\RR}{\mathbb{R}}
\newcommand{\vect}[1]{\boldsymbol{#1}}
\newcommand{\diag}{\operatorname{diag}}
\newcommand{\Tr}{\operatorname{Tr}}
\newcommand{\sgn}{\operatorname{sgn}}
\newcommand{\Order}{\mathcal{O}}
\newcommand{\Lag}{\mathcal{L}}
\newcommand{\Action}{\mathcal{S}}
\newcommand{\grad}{\nabla}
\newcommand{\abs}[1]{\left\lvert #1 \right\rvert}
\newcommand{\braces}[1]{\left\{ #1 \right\}}
\newcommand{\paren}[1]{\left( #1 \right)}
\newcommand{\brak}[1]{\left[ #1 \right]}

% --- Lean/Code formatting helper ---
\newcommand{\lean}[1]{\texttt{#1}}

% --- Graphics path (optional) ---
\graphicspath{{./figs/}}

% --- Title and author block (update before submission) ---
\begin{document}

\title{Information-Limited Gravity: A Mechanized, Covariant, Quantum-Consistent Framework with Observational Gates}

\author{Jonathan Washburn}
\email{Washburn@RecognitionPhysics.org}
\affiliation{Independent Researcher}

% \author{Second Author}
% \affiliation{Institution, City, Country}

\date{\today}

\begin{abstract}
We present a covariant, quantum-consistent gravitational framework derived from information-limited recognition principles and formalized end-to-end in the Lean theorem prover. The theory augments general relativity with a globally constrained scalar sector $\psi$ whose couplings are fixed at the global level and which reduces to Einstein gravity in the appropriate limit. From a single covariant action we derive weak-field dynamics, obtain a multiplicative weight $w(r)$ for baryonic acceleration with a controlled $\Order(\varepsilon^2)$ remainder, and map solved metric potentials to post-Newtonian parameters $(\gamma,\,\beta)$ and relativistic lensing observables, including cluster time-delay proxies. On cosmological backgrounds we restate the Friedmann equations via $T_\psi$ and connect scalar-perturbation growth to $\sigma_8$ within CMB/BAO/BBN bands. Around FRW we extract the quadratic action for tensor modes and bound $c_T^2$ consistently with multi-messenger constraints. For compact objects we provide horizon and ringdown proxies subject to observational bands. A quantum substrate with explicit microscopic degrees of freedom satisfies unitary evolution and a microcausality predicate.

All statements are compiled as machine-checked certificates with editor-friendly \#eval reports; a consolidated falsifiers harness and a QG gate enforce pass/fail in continuous integration. The artifact delivers a reproducible pipeline from axioms to observables and a concrete agenda for tightening each domain directly against data.
\end{abstract}

\maketitle

% --- Optional: Table of contents for drafts (remove before submission if desired) ---
% \tableofcontents

\section{Introduction}\label{sec:intro}

Mechanizing fundamental theory changes what counts as evidence. Ambitious proposals in gravitation often hinge on informal derivations, implicit assumptions, and ad hoc parameter choices, which can impede falsification and reuse. Here we present a covariant, quantum-consistent framework for gravity---Information-Limited Gravity (ILG)---that is constructed and verified end-to-end in the Lean theorem prover. Every structural claim we make is tied to a named Lean theorem or certificate and surfaced through editor-friendly \#eval reports and CI-enforced harnesses, providing a reproducible path from axioms to observables.

\subsection{Motivation and scope}
Observational tensions across scales (galaxy rotation curves, cluster lensing time delays, growth and $\sigma_8$) suggest value in a minimally extended, globally constrained theory that preserves General Relativity (GR) where it is tested best, but that can systematically account for large-scale phenomenology. ILG introduces a globally configured scalar sector $\psi$ coupled covariantly to the metric, with couplings fixed at the global level. The theory is engineered to: (i) reduce to GR in the appropriate limit, (ii) produce predictive weak-field dynamics with a multiplicative $w(r)$ factor for baryonic acceleration and a controlled $\Order(\varepsilon^2)$ remainder, and (iii) deliver measurable consequences for post-Newtonian (PPN) parameters, lensing, cosmology, gravitational waves, and compact objects.

\subsection{Mechanized guarantees}
At the action level, we define a total covariant action $\Action_{\mathrm{total}}[g,\psi]$ and prove a GR-limit theorem \lean{gr_limit_cov} ensuring reduction to the Einstein--Hilbert term in the relevant limit. Variational predicates deliver Euler--Lagrange equations and stress--energy with GR-consistency lemmas \lean{EL_psi_gr_limit} and \lean{Tmunu_gr_limit_zero}. Each construction is accompanied by unit/consistency checks (e.g., \lean{LPiecesUnitsCert}, \lean{LCovIdentityCert}) and surfaced by \#eval reports (e.g., \lean{l_pieces_units_report}, \lean{l_cov_identity_report}). These machine-checked artifacts ensure that the formal layer compiles before any phenomenological claims are made.

\subsection{From fields to observables}
In the weak-field regime, linearization yields a modified Poisson structure and a baryonic acceleration weight $w(r)$ derived from potentials, with an explicit $\Order(\varepsilon^2)$ control (\lean{WLinkOCert}; report \lean{w_link_O_report}), and a consolidated derivation certificate (\lean{WeakFieldDeriveCert}; report \lean{weakfield_derive_report}). Solved metric potentials map to PPN parameters $(\gamma,\beta)$ (certificate \lean{PPNDeriveCert}; report \lean{ppn_derive_report}) and to relativistic lensing deflection/time-delay proxies relevant for clusters (\lean{ClusterLensingDeriveCert}; report \lean{cluster_lensing_derive_report}). On cosmological backgrounds, a $\psi$ stress--energy construction restates Friedmann equations and connects scalar perturbation growth to $\sigma_8$, with bands spanning CMB/BAO/BBN consistency (\lean{FRWDeriveCert}, \lean{GrowthCert}, \lean{CMBBAOBBNBandsCert}; reports \lean{frw_derive_report}, \lean{growth_report}, \lean{cmb_bao_bbn_bands_report}). Around FRW, the quadratic action for tensor modes bounds $c_T^2$ via \lean{GWQuadraticCert} and \lean{GWDeriveCert} (reports \lean{gw_quadratic_report}, \lean{gw_derive_report}). For compact objects we provide horizon and ringdown proxies (\lean{BHDeriveCert}; report \lean{bh_derive_report}).

\subsection{Quantum consistency and gating}
A quantum substrate with explicit microscopic degrees of freedom delivers unitary evolution and a microcausality predicate, certified by \lean{MicroUnitaryCert} and \lean{MicroUnitaryCompletionCert} (reports \lean{micro_unitary_report}, \lean{micro_unitary_completion_report}). To guarantee scientific hygiene, we expose a consolidated QG gate and a falsifiers harness that PRs must satisfy in CI: \lean{qg_harness_report} (``QGHarness: PASS'') and \lean{falsifiers_harness_report} (``FalsifiersHarnessCert: OK''). Together they ensure that any change that breaks a theorem, a certificate, or a dataset-linked constraint is immediately surfaced.

\subsection{Contributions}
This paper contributes: (1) a covariant ILG action with GR-limit guarantees; (2) mechanized derivations for weak-field, PPN, lensing, cosmology, GW, and compact-object domains; (3) a quantum substrate with unitarity and microcausality predicates; (4) a fully reproducible artifact with certificate-backed \#eval reports; and (5) CI-enforced science gates. The remainder of the manuscript follows the structure in \cref{sec:action,sec:weakfield,sec:ppn,sec:lensing,sec:frw,sec:gw,sec:compact,sec:quantum,sec:falsifiers}, with each section anchored to explicit Lean artifacts and observational interfaces.

\section{Recognition spine and ILG principle}\label{sec:context}

ILG inherits global structure from a prior recognition-theoretic spine that pins a dimensionless scale and fixes bridge relationships between units, displays, and observational hooks. The essential point for this work is operational: global parameters are fixed by spine obligations and are not tuned per system. We summarize the guarantees and the specific hooks that ILG uses.

\subsection{Spine guarantees and reports}
The ``reality bundle'' witness combines a spec-level closure with a reality layer; in Lean this is exposed by the report
\lean{reality_master_report}, which elaborates a witness of \emph{RSRealityMaster}. A meta-level closure is packaged by \lean{recognition_closure_report}. The uniqueness of the pinned dimensionless scale is surfaced by \lean{phi_selection_unique_with_closure_report}. These endpoints collectively certify that the foundational layer is internally consistent and that the global selection is unique.

\subsection{Global-only configuration and provenance}
Global couplings used by ILG are inherited from the recognition spine and recorded alongside their provenance (see the repository's \texttt{Source.txt}). No local, per-galaxy or per-cluster tuning is introduced in the theory or the proofs. Where a band or inequality is stated, it arises from global constraints or small-coupling regimes that are explicit in the Lean development.

\subsection{Bridge and units hooks}
ILG depends on unit/bridge coherence identities that are verified once at the spine and then reused everywhere. The K-gate and identity hooks are exercised by \lean{k_gate_report} and \lean{k_identities_report}; an equivalent lambda-identity witness is surfaced by \lean{lambda_rec_identity_report}. A route-A identity tying Planckian and recognition units is exposed by \lean{routeA_gate_identity_report}. These checks ensure that displays and conversions used in observational interfaces are dimensionally and structurally coherent.

\subsection{Interface to ILG and observational bands}
To connect global parameters to observational checks, we expose a bands schema and a certificate stating that the mapping from parameters to bands is well-formed. The consolidated mapping is certified by \lean{BandsFromParamsCert} with the report \lean{bands_from_params_report}. This allows ILG-generated quantities (e.g., PPN bands, lensing bands, $c_T^2$ bounds) to be referenced in a uniform way across the paper and in the falsifiers harness.

\subsection{What is and is not assumed}
Spine obligations provide global anchors and unit relations; ILG then adds a covariant scalar sector $\psi$ with globally fixed couplings. We \emph{do not} assume system-specific potentials, postulated dark components, or free functions added to fit individual datasets. Any remaining scaffolds are clearly marked and are designed to be tightened against data without changing public endpoints.

\section{Covariant action and variational structure}\label{sec:action}

This section defines the total covariant action used throughout the paper, fixes notation for variations, and records the mechanized GR-limit statements and unit/identity checks that guard the construction. Formal objects and proofs are implemented in Lean and exposed through named theorems and certificate-backed reports.

\subsection{Definitions and notation}
We denote the spacetime metric by $g_{\mu\nu}$ with determinant $g$ and use a real scalar field $\psi$ with globally fixed couplings. Variations with respect to $\psi$ and $g_{\mu\nu}$ are written $\delta_\psi$ and $\delta_g$. The Lagrangian density for the scalar sector is assembled from canonical pieces (kinetic, mass, potential, couplings) into a covariant integrand $\Lag_{\mathrm{cov}}(g,\psi)$; the Lean module \texttt{ILG/Action.lean} provides these components and their aggregation. For compactness we write the total action
\begin{equation}
  \Action_{\mathrm{total}}[g,\psi] 
  = \int \dd^4x\,\sqrt{-g}\,\Big[\,\Lag_{\mathrm{EH}}(g)\;+
    \Lag_{\mathrm{cov}}(g,\psi)\,\Big],
  \label{eq:S-total}
\end{equation}
with $\Lag_{\mathrm{EH}}$ the Einstein--Hilbert term.

\subsection{GR-limit guarantees}
Mechanized GR-compatibility is ensured by the following Lean theorems:
\begin{itemize}
  \item \lean{gr_limit_cov}: the total covariant action \eqref{eq:S-total} reduces to the Einstein--Hilbert action in the appropriate GR limit.
  \item \lean{EL_psi_gr_limit}: the Euler--Lagrange equation for $\psi$ reduces to its GR-limit form under the same conditions.
  \item \lean{Tmunu_gr_limit_zero}: the stress--energy tensor contribution from the added sector vanishes in the GR limit.
\end{itemize}
These statements are used later when comparing to solar-system PPN bounds and binary tests where GR is tightly constrained.

\subsection{Variations and stress--energy}
Stationarity of $\Action_{\mathrm{total}}$ under independent variations yields the field equation for $\psi$ and the metric field equation via
\begin{equation}
  \delta_\psi\,\Action_{\mathrm{total}} = 0,\qquad
  \delta_g\,\Action_{\mathrm{total}} = 0\;\Rightarrow\; T_{\mu\nu}[\psi,g]\,.
\end{equation}
The Lean development expresses these as symbolic predicates in \texttt{ILG/Variation.lean}, with GR-limit lemmas cited above. These predicates are the entry points for linearization (weak-field), cosmology (FRW background), and tensor-mode extraction (GW).

\subsection{Unit consistency and identities}
Dimensional and structural hygiene is guarded by unit/covariance certificates:
\begin{itemize}
  \item \lean{LPiecesUnitsCert} (report \lean{l_pieces_units_report}): unit-consistency across the scalar-sector Lagrangian pieces.
  \item \lean{LCovIdentityCert} (report \lean{l_cov_identity_report}): coherence of the covariant aggregation used in \eqref{eq:S-total}.
\end{itemize}
Both are compiled as machine-checked certificates and surfaced by \#eval reports for fast verification during development and in CI.

\subsection{Interfaces to downstream sections}
The objects above feed directly into the remainder of the paper: linearized EL equations and the modified Poisson structure (\cref{sec:weakfield}); post-Newtonian mappings (\cref{sec:ppn}); relativistic lensing integrals (\cref{sec:lensing}); FRW restatements and growth (\cref{sec:frw}); and the quadratic action for gravitational waves (\cref{sec:gw}). All downstream claims depend on the mechanized guarantees summarized here.

\section{Weak-field regime and modified Poisson}\label{sec:weakfield}

We now linearize the field equations around a Minkowski background and work in the Newtonian gauge to obtain a modified Poisson structure for the potential $\Phi$. The baryonic acceleration picks up a multiplicative weight $w(r)$ derived from the linearized potentials. All steps are mechanized in Lean with explicit symbols and reports.

\subsection{Gauge and perturbations}
We parameterize the metric perturbation and fix the Newtonian gauge as in the Lean scaffold \texttt{ILG/WeakField.lean}: the structures \lean{Perturbation}, \lean{NewtonianGauge}, and constructor \lean{mkNewtonian} provide typed accessors for $(\Phi,\Psi)$ and ensure gauge choices are respected downstream.

\subsection{Linearized EL at $\Order(\varepsilon)$}
From the variational predicates (\cref{sec:action}) we form their linearization about the background. The Lean module \texttt{ILG/Linearize.lean} provides a symbolic predicate \lean{LinearizedEL} and an $\Order(\varepsilon)$ statement \lean{linearized_EL_Oeps} that isolates terms relevant for weak-field dynamics.

\subsection{Modified Poisson and effective sources}
The scalar-sector source and the effective potential are assembled via \lean{Spsi_source} and \lean{PhiEff_from_sources}. The central linkage is the modified-Poisson statement \lean{derive_modified_poisson}, which ties the Laplacian of $\Phi$ to baryonic and $\psi$-sector contributions in the linear regime. This statement is used later to map to PPN and lensing observables.

\subsection{Baryonic acceleration weight}
We obtain a multiplicative weight $w(r)$ for baryonic acceleration from the potential, following the Lean definitions \lean{w_of_Phi} and \lean{w_r}; a velocity proxy \lean{v_model2_r} is used to compare against rotation-curve profiles without introducing per-system tuning. The consolidated certificate \lean{WeakFieldDeriveCert} is surfaced by the report \lean{weakfield_derive_report}.

\subsection{Error control and $\Order(\varepsilon^2)$ remainder}
Error budgeting is made explicit through a Big-O scaffold \lean{BigO2} and the linkage lemma \lean{w_link_O2}, guaranteeing that neglected terms are at most $\Order(\varepsilon^2)$. This guarantee is exposed to users and CI by the certificate \lean{WLinkOCert} with report \lean{w_link_O_report}.

\subsection{Placeholders for figures and tables}
\begin{figure}[t]
  \centering
  % TODO: includegraphics{rotation_curves_overlay.pdf}
  \caption{Rotation-curve overlays comparing the baryonic baseline and the ILG prediction using the weight $w(r)$. No per-galaxy tuning is introduced; bands reflect global small-coupling regimes.}
  \label{fig:rc}
\end{figure}

\begin{table}[b]
  \centering
  \begin{tabular}{l c}
    \toprule
    Contribution & Scaling \\
    \midrule
    Linear term & $\Order(\varepsilon)$ \\
    Remainder & $\Order(\varepsilon^2)$ \\
    \bottomrule
  \end{tabular}
  \caption{Schematic error budget for the weak-field expansion used to derive $w(r)$. Precise constants are provided by the Lean lemmas \lean{linearized_EL_Oeps} and \lean{w_link_O2}.}
  \label{tab:weakfield-error}
\end{table}

\subsection{Reports and gating}
For rapid validation, the weak-field derivation and the remainder control are exposed as \#eval-friendly endpoints: \lean{weakfield_derive_report} and \lean{w_link_O_report}. These are included in the consolidated QG harness used to gate pull requests in continuous integration (see \cref{sec:falsifiers}).

\section{Post-Newtonian parameters (PPN)}\label{sec:ppn}

We extract 1PN post-Newtonian parameters $(\gamma,\beta)$ from the weak-field solutions and connect them to observational bounds. The Lean development provides typed accessors to the linearized potentials and explicit formulas for $\gamma$ and $\beta$ at 1PN order; small-coupling bands make contact with canonical solar-system constraints.

\subsection{Definitions and small-coupling bands}
In the mechanized scaffold, the 1PN parameters are defined by
\begin{equation}
  \gamma_{\mathrm{1PN}} \equiv \texttt{gamma1PN}(\psi; p), \qquad
  \beta_{\mathrm{1PN}} \equiv \texttt{beta1PN}(\psi; p),
\end{equation}
where $p$ are global ILG parameters. Under the small-coupling regime $\lvert p.cLag \cdot p.\alpha\rvert \le \kappa$, Lean supplies the bounds
\begin{align}
  \bigl\lvert \gamma_{\mathrm{1PN}} - 1 \bigr\rvert &\le \tfrac{1}{10}\,\kappa, \\
  \bigl\lvert \beta_{\mathrm{1PN}} - 1 \bigr\rvert &\le \tfrac{1}{20}\,\kappa,
\end{align}
ensuring consistency with solar-system tests provided $\kappa$ is within empirical tolerances. These theorems are compiled as part of the PPN derivation certificate and surfaced by a \#eval report.

\subsection{Linkage to linear forms}
For bookkeeping and cross-checks with linearized potentials, the scaffold provides exact equalities relating the 1PN quantities to linear forms:
\begin{equation}
  \gamma_{\mathrm{1PN}} = \texttt{ppn\_gamma\_lin}(p.cLag, p.\alpha), \quad
  \beta_{\mathrm{1PN}}  = \texttt{ppn\_beta\_lin}(p.cLag, p.\alpha).
\end{equation}
These identities ensure the PPN mapping is consistent with the weak-field sector introduced in \cref{sec:weakfield} and that limits commute as expected.

\subsection{Certificate and report endpoint}
All statements above are packaged by the certificate \lean{PPNDeriveCert} with a \#eval-friendly endpoint \lean{ppn_derive_report}. Any breakage in the 1PN mapping or band inequalities would prevent elaboration and fail continuous integration via the consolidated QG harness (see \cref{sec:falsifiers}).

\subsection{Placeholder for figure}
\begin{figure}[t]
  \centering
  % TODO: includegraphics{ppn_bands.pdf}
  \caption{PPN parameter bands for $(\gamma,\beta)$ as functions of the small-coupling proxy $\kappa = \lvert p.cLag \cdot p.\alpha\rvert$, compared with canonical solar-system constraints.}
  \label{fig:ppn-bands}
\end{figure}

\section{Relativistic lensing and time delays}\label{sec:lensing}

We sketch the relativistic lensing pipeline used to compare ILG with strong-lensing observables. The Lean scaffold provides a spherical-profile abstraction and closed-form deflection proxy sufficient for band checks, along with a time-delay band theorem that captures the leading sensitivity to global couplings.

\subsection{Spherical profiles and deflection}
We package spherically symmetric lenses by a typed structure \lean{SphericalProfile} carrying the relevant potential radial profile $\Phi(r)$. The corresponding deflection for impact parameter $b$ and small-coupling proxy $\kappa$ is given by the noncomputable definition \lean{deflection_spherical}, with an evaluation lemma
\begin{equation}
  \texttt{deflection\_spherical\_eval}:\quad
  \alpha(b) = \kappa\,\Phi_r(b),
\end{equation}
providing an analytic handle for cluster-scale estimates without per-system tuning.

\subsection{Time-delay bands}
For two images with path-length difference characterized by angular-momentum scale $\ell$, Lean supplies a band statement \lean{time_delay_band}(\,$\psi, p, \ell, \kappa$\,) that isolates the dependence on the global proxy $\kappa$ (with $\kappa\ge 0$). This result is wired into the cluster derivation certificate.

\subsection{Certificate and report endpoint}
The cluster-lensing derivation is packaged by \lean{ClusterLensingDeriveCert} with a \#eval-friendly endpoint \lean{cluster_lensing_derive_report}. Any inconsistency in the lensing linkage or time-delay band prevents elaboration and is caught by the QG harness in CI.

\subsection{Placeholder for figure}
\begin{figure}[t]
  \centering
  % TODO: includegraphics{cluster_time_delays.pdf}
  \caption{Cluster time-delay band comparison: predicted delays versus observed values across a sample, expressed as a function of the global proxy $\kappa$.}
  \label{fig:cluster-delays}
\end{figure}

% --- Section stubs (content to be added in separate inputs) ---
% \input{QG_intro}
% \input{QG_action}
% \input{QG_weakfield}
% \input{QG_ppn}
% \input{QG_lensing}
% \input{QG_frw}
% \input{QG_gw}
% \input{QG_compact}
% \input{QG_quantum}
% \input{QG_falsifiers}
% \input{QG_results}
% \input{QG_discussion}

% --- Acknowledgments ---
\begin{acknowledgments}
We thank collaborators and contributors to the Lean mechanization and continuous-integration infrastructure. This work was supported in part by Recognition Physics. Any opinions expressed are those of the authors.
\end{acknowledgments}

% --- Data Availability Statement ---
\section*{Data Availability}
All code and report endpoints used to generate the results are available in the public repository accompanying this manuscript. See the consolidated harnesses and certificates enumerated in the documentation for automated pass/fail checks.

% --- Author Contributions ---
\section*{Author Contributions}
Conceptualization, formalization, and writing: J.W. Artifact preparation and continuous-integration gates: J.W. All authors reviewed and approved the manuscript.

% --- Bibliography ---
\bibliographystyle{apsrev4-2}
% If your BibTeX file is named paper.bib and lives alongside this .tex:
\bibliography{paper}

\end{document}



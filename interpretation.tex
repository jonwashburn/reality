\documentclass[11pt]{article}
\usepackage[margin=1in]{geometry}
\usepackage{amsmath,amssymb}
\usepackage{enumitem}
\usepackage{hyperref}
\usepackage{xcolor}
\usepackage{mdframed}

\definecolor{rsblue}{RGB}{0,51,102}
\definecolor{alertred}{RGB}{153,0,0}
\definecolor{warnorange}{RGB}{204,85,0}

\newmdenv[
  linecolor=rsblue,
  linewidth=2pt,
  roundcorner=5pt,
  backgroundcolor=rsblue!5
]{rsbox}

\newmdenv[
  linecolor=alertred,
  linewidth=2pt,
  roundcorner=5pt,
  backgroundcolor=alertred!5
]{alertbox}

\newmdenv[
  linecolor=warnorange,
  linewidth=1.5pt,
  roundcorner=5pt,
  backgroundcolor=warnorange!5
]{warnbox}

\title{\textbf{Copenhagen Interpretation:\\Anomalies, Inconsistencies, and Circularities}\\[6pt]
\large Framed from the Recognition Science (RS) Baseline\\[4pt]
\normalsize Team Reference Document}
\author{Recognition Physics Institute\\
\texttt{jon@recognitionphysics.org}\\
Austin, Texas, USA}
\date{\today}

\begin{document}
\maketitle

\tableofcontents
\newpage

\section{Purpose and scope}

This memo is a reference for our team when engaging with Copenhagen--interpretation (CI) orthodoxy in papers, reviews, and talks. It provides:
\begin{itemize}[leftmargin=*]
  \item \textbf{Catalog of specific anomalies} that conflict with materialism/physicalism
  \item \textbf{Internal logical inconsistencies} in CI's assumption sets
  \item \textbf{Circular reasoning patterns} where math fits are retrofitted as explanations
  \item \textbf{Exhibit--level examples} with equations, ready to cite
  \item \textbf{Pass/fail questions} to pin down defenders
  \item \textbf{RS contrasts} showing how Recognition Science resolves each issue
  \item \textbf{Action items} for practical implementation
\end{itemize}

\begin{rsbox}
\textbf{Key principle:} CI succeeds as a \emph{calculation recipe} but fails as a \emph{physical account}. Its core moves rest on:
\begin{enumerate}[nosep]
  \item Non--physical primitives (measurement, observer, cut)
  \item Forked dynamics with no selection rule (unitary vs.\ collapse)
  \item Narrative shifts producing circular justifications
  \item Incompatible assumption sets in nested--observer scenarios
\end{enumerate}
The cleanest way to prosecute this case is to pin authors to explicit assumption sets and demand a single, dynamical, observer--free law that does the full job. \textbf{CI does not provide one.}
\end{rsbox}

\section{The RS baseline (what is actually happening)}

\subsection{One physical law, not two}

\textbf{The substrate:} Recognition Science models reality as a discrete ledger of recognition events. Each event records:
\begin{itemize}[nosep]
  \item What was recognized (a pattern with information content $Z$)
  \item The cost paid (in units of $J$)
  \item Which tick in the eight--beat cycle
\end{itemize}

The recognition operator $\hat R$ advances the state by one complete eight--tick cycle:
\begin{equation}
  s(t+8\tau_0)=\hat R\big(s(t)\big).
\end{equation}

\textbf{Why eight ticks?} Because $D=3$ spatial dimensions force $2^D=8$ vertices on the $Q_3$ hypercube, and a spatially complete, ledger--compatible walk must visit all vertices exactly once per period (Gray code). This is \textbf{T6 (Eight--Tick theorem)} - not a choice, a structural necessity forced by discrete geometry.

The operator $\hat R$ minimizes a unique convex cost $J(x)$ under two constraints:
\begin{enumerate}[nosep]
  \item Closed--loop flux $= 0$ (conservation; T3)
  \item Double--entry ledger balance (information tracking)
\end{enumerate}

The cost function is uniquely determined by four properties:
\begin{equation}
  J(x)=\tfrac{1}{2}\!\left(x+\frac{1}{x}\right)-1
\end{equation}
where:
\begin{itemize}[nosep]
  \item Symmetry: $J(x) = J(1/x)$
  \item Normalization: $J(1) = 0$, $J''(1) = 1$
  \item Convexity on $\mathbb{R}_+$
  \item Bounded by $x + 1/x$
\end{itemize}

This is \textbf{T5 (Cost Uniqueness)} - proved in Lean with zero sorries (\texttt{Cost.T5\_cost\_uniqueness\_on\_pos}). \textbf{There is no second, ad hoc rule for ``measurement.''}

\textbf{Interface (QM):} In the small--deviation limit around neutrality ($x\simeq 1$), the extremals of the $J$--action satisfy Euler--Lagrange equations and yield the wave dynamics
\begin{equation}
  i\,\hbar\,\partial_t \psi = \hat H \psi,
\end{equation}
with $\hbar$ not postulated but \emph{defined} at the IR gate by
\begin{equation}
  \hbar = E_{\rm coh}\,\tau_0 \approx (0.090\,{\rm eV})\cdot \tau_0,
\end{equation}
where $E_{\rm coh} = \varphi^{-5}\,{\rm eV}$ emerges from the $\varphi$--scaling structure.

The Schrödinger equation is an \emph{emergent interface model} that holds whenever recognition costs remain sub--threshold. The Hamiltonian $\hat H$ is derived, not fundamental; $\hat R$ is fundamental.

\subsection{Path measure and Born rule (derived, not stipulated)}

\textbf{CI's problem:} Born rule ($P = |\psi|^2$) is postulated without derivation. Every attempt to ``derive'' it smuggles Born weighting into the premises (see Section~\ref{sec:born_circular}).

\textbf{RS derivation in four steps:}

\paragraph{Step 1: Recognition cost accumulates along paths}

For any trajectory $\gamma$ in coarse--grained variables $r(t) > 0$, the recognition action is:
\begin{equation}
  C[\gamma] = \int J(r(t))\, dt.
\end{equation}
This is not an ad hoc choice; it's the unique action built from the unique cost $J$ (T5 uniqueness theorem).

\paragraph{Step 2: Path weights from thermodynamic principles}

Paths with lower recognition cost are exponentially favored:
\begin{equation}
  w[\gamma] = e^{-C[\gamma]}.
\end{equation}
This is the Boltzmann weight with cost playing the role of energy. No new postulate; just statistical mechanics applied to recognition events.

\paragraph{Step 3: Quantum amplitudes as action phases}

Following Feynman path integrals, write:
\begin{equation}
  A[\gamma] = e^{-C[\gamma]/2} \cdot e^{i\varphi[\gamma]}.
\end{equation}
The factor $\tfrac{1}{2}$ in the exponent is not arbitrary; it makes amplitudes transform correctly under path concatenation: $A[\gamma_1 \circ \gamma_2] = A[\gamma_1] \cdot A[\gamma_2]$ requires the square--root structure. The phase $\varphi[\gamma]$ comes from the closed--loop structure of the ledger.

\paragraph{Step 4: Born rule emerges from amplitude sum}

For outcome $a$ realized by path family $\Gamma_a$:
\begin{equation}
  \Psi_a = \sum_{\gamma\in\Gamma_a} A[\gamma] = \sum_{\gamma\in\Gamma_a} e^{-C[\gamma]/2} e^{i\varphi[\gamma]}.
\end{equation}
Then probability is:
\begin{equation}\label{eq:born}
  \mathbb{P}(a) = |\Psi_a|^2 = \left|\sum_{\gamma\in\Gamma_a} A[\gamma]\right|^2.
\end{equation}

This is identical to the Born rule, but derived from:
\begin{itemize}[nosep]
  \item Cost minimization (T5)
  \item Thermodynamic weights (Boltzmann)
  \item Action--phase structure (Feynman)
  \item \textbf{No additional postulate}
\end{itemize}

\begin{rsbox}
\textbf{Key insight:} The Born rule is inevitable once you accept:
\begin{enumerate}[nosep]
  \item Recognition has cost
  \item Costs accumulate along paths
  \item Nature minimizes cost
\end{enumerate}
CI gives you the recipe. RS shows you \emph{why} the recipe works.
\end{rsbox}

\subsection{Collapse without postulates}

\textbf{CI's problem:} Collapse is a separate, unrelated rule. When does it apply, and how does it conserve physical quantities? No answer.

\textbf{RS resolution:} Collapse is the same $\hat R$ dynamics crossing a threshold.

\paragraph{Mechanism (two-branch picture).}
Let $|\psi\rangle = c_1|\psi_1\rangle + c_2|\psi_2\rangle$ be an interface description of two coarse alternatives, with cumulative recognition costs $C_1, C_2$ at instrument $(W,K)$. Define $\Delta C = |C_1 - C_2|$.
\begin{itemize}[nosep]
  \item \textbf{Sub-threshold} ($\Delta C \ll 1$): Coherent evolution; interference visible.
  \item \textbf{Threshold} ($\Delta C \ge 1$): Branch-weight ratio becomes extreme,
  \begin{equation}
    \frac{w_1}{w_2} = e^{-(C_1 - C_2)} \lesssim e^{-1},
  \end{equation}
  and the higher-weight branch is selected (definite pointer).
  \item \textbf{Post-orthogonality saturation}: Once $\Delta C \ge 1$, further separation does not change the outcome; increasing distance adds no new information.
\end{itemize}

\paragraph{Properties.}
\begin{itemize}[nosep]
  \item \textbf{Local}: Threshold evaluated at the instrument $(W,K)$.
  \item \textbf{Threshold-driven}: Not distance- or mass-dependent.
  \item \textbf{Parameter-free}: Threshold $1$ comes from $J''(1)=1$ normalization; no GRW/CSL knobs.
  \item \textbf{Non-signaling}: Global consistency arises from the ledger; transport remains cone-bounded.
\end{itemize}

\paragraph{Key insight.} There is no "when" question. $\hat R$ is always operating. The interface description transitions from "superposition" to "definite outcome" at $\Delta C\!\ge\!1$, but the underlying dynamics never changed.

\subsection{Conservation and irreversibility reconciled}

\textbf{Two-layer accounting.}
\begin{itemize}[leftmargin=*]
  \item \textbf{Substrate layer (conservative):} Evolves by $\hat R$ under closed-loop constraint (T3: Continuity). Total $Z$-patterns conserved; no entropy production.
  \item \textbf{Interface layer (dissipative):} Instrument $(W,K)$ commits a record when $C\!\ge\!1$. Each commit writes irreversible information, costing at least $W_{\min} \ge k_B T\,\Delta S$ (Landauer).
\end{itemize}

\textbf{Reconciliation (energy budget example).}
\begin{enumerate}[nosep]
  \item Preparation: $|+\rangle_x$, $\hat H=(\hbar\omega/2)\sigma_x$, so $\langle H\rangle=+\hbar\omega/2$.
  \item Coupling: $\hat H_{\rm int}=g\,\sigma_z\!\otimes\!M$ entangles system and meter.
  \item Threshold: $C(|M_\uparrow\rangle,|M_\downarrow\rangle)\!\ge\!1$ triggers commit.
  \item Exchange: $\Delta E_{\rm system}= -\hbar\omega/2$, $\Delta E_{\rm meter}= +\hbar\omega/2$, $\Delta E_{\rm total}=0$ (substrate conservation);
  \item Dissipation: Meter thermalizes to bath (irreversible at interface).
\end{enumerate}

\textbf{Analogy.} Substrate is the reversible platter; $(W,K)$ is the read/write head. The platter can spin backwards; the head's commit cannot be undone without work. Measurement does not "disturb" the substrate; it extracts information at cost.

\subsection{The boundary is the channel}

The only physical boundary is the declared instrument window/kernel $(W,K)$:
\begin{itemize}[nosep]
  \item Change $(W,K)$ $\Rightarrow$ change what is resolved
  \item No metaphysical ``Heisenberg cut''
  \item No observer ontological primacy
  \item Incompatible $(W,K)$ channels are not jointly assertable as one event
\end{itemize}

\subsection{Constants fixed by identities (no knobs)}

All dimensional displays factor through ratios fixed by eight--tick gates and ledger geometry:
\begin{align}
  c &= \frac{\ell_0}{\tau_0}, \label{eq:c}\\
  \hbar &= E_{\rm coh}\,\tau_0, \label{eq:hbar}\\
  \frac{c^3\lambda_{\rm rec}^2}{\hbar G} &= \frac{1}{\pi}. \label{eq:lambda_rec}
\end{align}
These determine $c$, $\hbar$, and $G$ when mapped to a single consistent units layer. \textbf{No free, dimensionless knobs are introduced in the chain from structure to predictions.}

\subsection{The instrument channel $(W,K)$ explained}

CI treats "measurement" as a black box. RS opens the box.

\subsubsection*{What is $(W,K)$?}
An instrument is a physical system that couples to the substrate, records information in macroscopic degrees of freedom, and commits that information irreversibly. Formally:
\begin{itemize}[nosep]
  \item $W$: window function mapping substrate state $s$ to coarse observable $o$: $o\!=\!W(s)$ (what the instrument is sensitive to)
  \item $K$: kernel giving readout distribution $P(r|o)\!=\!K(r|o)$ (noise, efficiency, resolution)
\end{itemize}

\subsubsection*{How $(W,K)$ determines collapse}
Recognition cost to distinguish $s_1,s_2$ is a functional $C(s_1,s_2;W,K)$ built from $J$; when $C\!\ge\!1$ the states are distinguishable and a definite pointer is selected. Below threshold, coherence persists. The threshold "1" is structural, from $J''(1)=1$.

\subsubsection*{Examples}
\begin{itemize}[leftmargin=*]
  \item \textbf{Stern--Gerlach}: $W\!=\!\sigma_z$; $K$ is Gaussian point-spread at screen. $C(\uparrow,\downarrow)\propto$ (spatial separation)$^2\times K$; threshold sets minimum flight length.
  \item \textbf{Photon counter}: $W\!=\!\hat a^\dagger\hat a$ in mode $\alpha$; $K$ Poisson($\eta n$). Threshold sets minimum integration time to resolve $\Delta n$.
  \item \textbf{Which-path detector}: $W$ resolves slit; $K$ is detector resolution. If $\text{width}(K)\!\ll$ slit spacing then $C\!>\!1$ (no fringes); if $\gg$ it, $C\!<\!1$ (fringes).
\end{itemize}

\section{Conflicts with materialism/physicalism}

A materialist/physicalist account requires:
\begin{enumerate}[nosep]
  \item Physical mechanisms (not undefined primitives)
  \item Dynamical laws (not context--switches)
  \item Conservation accounting (energy, momentum, entropy)
  \item Observer--independent ontology
\end{enumerate}

\subsection{A1. Measurement as undefined primitive}

\begin{alertbox}
\textbf{Anomaly:} CI posits discontinuous ``collapse'' without a physical trigger law, interaction Hamiltonian, or conservation budget.
\end{alertbox}

\textbf{Exhibit---Interaction--free (negative--result) measurement:}
Elitzur--Vaidman bomb tester: A photon travels through a Mach--Zehnder interferometer. If a bomb is present in one arm (but not triggered), interference vanishes and ``no click'' at one detector tells you the bomb was there.

\begin{itemize}[nosep]
  \item CI claim: ``Measurement disturbs the system''
  \item Problem: \emph{What} disturbed it? The photon didn't interact with the bomb.
  \item CI retreat: ``The potential for interaction...''
  \item Problem: Potentials are not mechanisms. If this is epistemic, why do subsequent dynamics change (Zeno)?
\end{itemize}

\textbf{RS resolution:} The instrument channel $(W,K)$ at the detectors commits when $C\ge 1$. The path--weight reallocation (with vs.\ without bomb) is a ledger--level fact; no ``disturbance'' primitive needed.

\subsection{A2. Heisenberg cut is not a physical boundary}

\begin{alertbox}
\textbf{Anomaly:} CI allows the system--apparatus split ``anywhere'' and claims predictions are invariant. In nested--observer setups, different placements produce incompatible ``facts.''
\end{alertbox}

\textbf{Exhibit---Wigner's friend:}
\begin{enumerate}[nosep]
  \item Friend measures spin, sees $|\uparrow\rangle$ or $|\downarrow\rangle$ (definite).
  \item Wigner (outside) treats friend + apparatus unitarily: $|\psi_{\rm friend}\rangle = \alpha|{\uparrow,\text{``up''}}\rangle + \beta|{\downarrow,\text{``down''}}\rangle$.
  \item Wigner can verify superposition via interference.
  \item Contradiction: Friend saw definite; Wigner sees superposition.
\end{enumerate}

\begin{itemize}[nosep]
  \item CI response: ``Cut must be placed consistently.''
  \item Problem: Where? What physical law determines it?
  \item CI retreat: ``Cannot apply QM to observers.''
  \item Problem: Then QM is not universal; where is the boundary?
\end{itemize}

\textbf{RS resolution:} A ``fact'' is a commit at a specific $(W,K)$. Friend's $(W,K)_F$ and Wigner's $(W,K)_W$ are incompatible channels. Reports from incompatible $(W,K)$ are not jointly assertable without a reconciliation map. The global ledger remains single and consistent.

\subsection{A3. Collapse without back--reaction}

\begin{alertbox}
\textbf{Anomaly:} Projective updates change conserved quantities if $[\hat M, \hat H]\ne 0$, yet CI provides no reaction/work/entropy budget.
\end{alertbox}

\textbf{Exhibit---Energy change under projection:}
\begin{enumerate}[nosep]
  \item Hamiltonian: $\hat H=(\hbar\omega/2)\,\sigma_x$
  \item Prepare $|\psi\rangle=|+\rangle_x$ so $\langle H\rangle=+\hbar\omega/2$
  \item Measure $\sigma_z$ (projective): $\rho\to \tfrac12(|+\rangle_z\!\langle+| + |-\rangle_z\!\langle-|)$
  \item Result: $\langle H\rangle_{\rm after}=0$
  \item Energy dropped by $\hbar\omega/2$ \emph{on average} with no modeled meter coupling
\end{enumerate}

\begin{itemize}[nosep]
  \item If collapse is physical, where is the reaction term and work budget?
  \item If collapse is epistemic (``updating knowledge''), why did the physical energy change?
  \item CI has no answer that preserves both claims.
\end{itemize}

\textbf{RS resolution:} The substrate conserves; the $(W,K)$ interface pays thermodynamic cost $\Delta S \sim k_B\ln 2$ per commit (Landauer). Energy shifts are accounted via the instrument's thermal reservoir.

\subsection{A4. Nonlocal state update sans mechanism}

\begin{alertbox}
\textbf{Anomaly:} Instantaneous collapse for EPR pairs is accepted but declared non--mechanistic. Either it's a physical influence (superluminal) or it isn't (then what causes joint definiteness?).
\end{alertbox}

\textbf{Exhibit---EPR--Bell:}
Entangled pair $|\Psi^-\rangle=\tfrac{1}{\sqrt{2}}(|\uparrow\downarrow\rangle-|\downarrow\uparrow\rangle)$ separated by spacelike distance. Alice measures spin along $\hat{a}$, Bob along $\hat{b}$. Correlations violate Bell inequalities.

\begin{itemize}[nosep]
  \item CI: ``Alice's measurement collapses Bob's state instantly.''
  \item Problem: Spacelike--separated collapse $\Rightarrow$ superluminal influence?
  \item CI retreat: ``It's not a real influence; just correlations.''
  \item Problem: Then what mechanism produces definite joint outcomes?
  \item CI toggles stances depending on the question.
\end{itemize}

\textbf{RS resolution:} Nonlocal correlations are global ledger consistency (not transport). Branch selection respects $C\ge 1$ threshold locally at each $(W,K)$. Light--cone bound on transport remains intact; non--signaling is preserved.

\subsection{A5. $\psi$--epistemic rhetoric vs.\ constraints}

\begin{alertbox}
\textbf{Anomaly:} CI often speaks as if $\psi$ is mere knowledge (epistemic). PBR theorem and related results constrain such models under mild independence assumptions.
\end{alertbox}

\textbf{PBR theorem (simplified):} If:
\begin{enumerate}[nosep]
  \item Systems prepared independently have independent ontic states
  \item $\psi$ is epistemic (probability distribution over ontic states)
\end{enumerate}
Then there exist preparations where predictions contradict QM.

CI cannot consistently maintain $\psi$--epistemic and QM predictions without violating independence.

\textbf{RS resolution:} Interface amplitudes $A[\gamma]=e^{-C[\gamma]/2}e^{i\varphi[\gamma]}$ emerge from substrate path weights. Not free epistemic furniture; not strict ontic beables either. The wave description is an interface model.

\subsection{A6. Preferred basis without selector law}

\begin{alertbox}
\textbf{Anomaly:} ``Decoherence picks the pointer basis'' presupposes a system/environment split and uses diagonalization as the criterion. Change the split, change the basis.
\end{alertbox}

\textbf{Standard decoherence story:}
\begin{enumerate}[nosep]
  \item System $S$ entangles with environment $E$: $|\psi\rangle_{SE}=\sum_i c_i |s_i\rangle|e_i\rangle$
  \item Trace out $E$: $\rho_S=\sum_i |c_i|^2 |s_i\rangle\langle s_i|$
  \item Claim: Basis $\{|s_i\rangle\}$ is preferred because $\rho_S$ is diagonal in it
\end{enumerate}

\textbf{Problem:} The basis is \emph{chosen by hand} via the $S$--$E$ split and interaction Hamiltonian. Change the split or interaction, change the ``preferred'' basis. This is circular: ``The basis is preferred because it diagonalizes the state we constructed by presupposing that basis.''

\textbf{RS resolution:} Pointer states are determined by alignment between the instrument's $(W,K)$ and eight--tick invariants (pattern--measurement lemmas: \texttt{sumFirst8}, \texttt{blockSumAligned8}). The stable basis is the one that preserves $Z$--invariants under windowed commits.

\section{Internal logical inconsistencies}

These are not philosophical preferences; they are \emph{contradictions}: CI's assumption sets cannot all be true together.

\subsection{B1. Dual dynamics fork}

\begin{alertbox}
\textbf{Inconsistency:} CI jointly asserts:
\begin{enumerate}[label=(\Alph*),nosep]
  \item Universal unitary evolution between measurements
  \item Nonunitary collapse upon measurement
\end{enumerate}
There is no law selecting when (C) overrides (U), nor how (C) conserves quantities that (U) guarantees.
\end{alertbox}

\textbf{Formal statement:}
\begin{align*}
  \text{(U):} \quad & |\psi(t)\rangle = \hat U(t,t_0)|\psi(t_0)\rangle, \quad \hat U \text{ unitary} \\
  \text{(C):} \quad & |\psi\rangle \to \frac{\hat P_m |\psi\rangle}{\sqrt{\langle\psi|\hat P_m|\psi\rangle}}, \quad \hat P_m \text{ projector}
\end{align*}

\textbf{Questions CI cannot answer:}
\begin{enumerate}[nosep]
  \item What physical law determines when (C) occurs?
  \item How are $[\hat P_m, \hat H]\ne 0$ noncommuting observables reconciled with energy conservation?
  \item If ``measurement'' is the trigger, what physical definition of measurement exists?
\end{enumerate}

CI calls (C) an ``update of knowledge,'' but this does not remove its physical consequences (Zeno effect, interference demolition, etc.).

\textbf{RS resolution:} One law $\hat R$ everywhere. ``Collapse'' is thresholded selection ($C\ge 1$) by the same dynamics. No fork.

\subsection{B2. Wigner's friend / Frauchiger--Renner}

\begin{alertbox}
\textbf{Inconsistency:} CI's triad of assumptions:
\begin{enumerate}[label=(\Alph*),nosep]
  \item Single outcomes occur (no branching ontology)
  \item Unitary modeling applies to agents and apparatus
  \item Agents can aggregate correct inferences into one consistent account
\end{enumerate}
Protocols exist where (S)+(Q)+(C) yield contradictions. CI must drop one but offers no principle for which or when.
\end{alertbox}

\textbf{Frauchiger--Renner protocol (simplified):}
Four agents with nested observations can derive:
\begin{itemize}[nosep]
  \item Agent 1: ``I am certain outcome $x$ occurred''
  \item Agent 2 (treating Agent 1 + system unitarily): ``Agent 1 is in superposition; $x$ has not occurred''
  \item Agents 3,4 (cross--checks): Further contradictions
\end{itemize}

CI typically responds by dropping (Q): ``Cannot apply QM to observers.'' But this:
\begin{itemize}[nosep]
  \item Violates universality of QM
  \item Provides no boundary criterion
  \item Makes ``observer'' a primitive (contradicts materialism)
\end{itemize}

\textbf{RS resolution:} Facts are commits tied to specific $(W,K)$. Different agents' $(W,K)$ channels are incompatible; their reports are not jointly assertable without a reconciliation map showing how ledger entries relate.

\subsection{B3. Improper vs.\ proper mixture conflation}

\begin{alertbox}
\textbf{Inconsistency:} After tracing out environment, CI treats the reduced density matrix as a classical ignorance mixture to justify single outcomes. But the mixture is \emph{improper} (entanglement), not proper (ignorance).
\end{alertbox}

\textbf{Proper mixture:} Classical ensemble; system is in one of $\{|s_i\rangle\}$ with probability $p_i$, we just don't know which.
\begin{equation}
  \rho = \sum_i p_i |s_i\rangle\langle s_i|.
\end{equation}

\textbf{Improper mixture:} Reduced state from entanglement $|\Psi\rangle_{SE}=\sum_i c_i|s_i\rangle|e_i\rangle$:
\begin{equation}
  \rho_S = \text{Tr}_E(|\Psi\rangle\langle\Psi|) = \sum_i |c_i|^2 |s_i\rangle\langle s_i|.
\end{equation}
System is \emph{not} in one of $\{|s_i\rangle\}$; it remains entangled with $E$.

\textbf{Problem:} CI uses ignorance semantics (``we don't know which'') to justify definite outcomes, but the mixture origin is entanglement, not ignorance. This is a category error: using the mathematical form $\sum_i p_i |s_i\rangle\langle s_i|$ to smuggle in incompatible physical interpretations.

\textbf{RS resolution:} Definiteness arises from threshold commits at $(W,K)$, not from tracing operations. The reduced state is an interface bookkeeping tool, not an ontological claim.

\subsection{B4. Preferred--basis relabeling (tautology)}

\begin{alertbox}
\textbf{Inconsistency:} CI defines ``stability'' as diagonalization in the pointer basis, then uses diagonalization to prove that basis is preferred.
\end{alertbox}

\textbf{Circular argument:}
\begin{enumerate}[nosep]
  \item Premise: The preferred basis is the one that diagonalizes $\rho_S$
  \item Construction: Choose $S$--$E$ split and interaction so $\rho_S$ is diagonal in basis $\{|s_i\rangle\}$
  \item Conclusion: Therefore $\{|s_i\rangle\}$ is the preferred basis
\end{enumerate}

The criterion is satisfied \emph{by construction}. This is not explanation; it's definitional circularity.

\textbf{RS resolution:} Pointer basis determined by $(W,K)$ alignment with eight--tick invariants. Change $(W,K)$, change the stable basis. This is physics (instrument properties), not tautology.

\subsection{B5. Quantum Zeno with a primitive}

\begin{alertbox}
\textbf{Inconsistency:} Frequent ``measurements'' freeze evolution (Misra--Sudarshan). If measurement is an epistemic act, how does knowledge arrest dynamics? If it's a physical interaction, where is the modeled coupling?
\end{alertbox}

\textbf{Quantum Zeno effect:}
Survival probability under $N$ projections in time $T=N\delta t$ with energy spread $\Delta E$:
\begin{equation}
  P_{\rm survive}(T) \simeq \left(1 - \frac{(\Delta E)^2(\delta t)^2}{\hbar^2}\right)^N \to 1 \quad \text{as } N\to\infty.
\end{equation}
``Watching the pot stops it from boiling.''

\begin{itemize}[nosep]
  \item If measurement is epistemic (knowledge update), why does it change physical evolution?
  \item If measurement is physical (strong coupling that projects), where is $\hat H_{\rm interaction}$?
  \item CI toggles between interpretations depending on audience.
\end{itemize}

\textbf{RS resolution:} Zeno arises from repeated near--threshold commits. Each commit has recognition cost; high--frequency interrogation keeps $C$ sub--threshold longer. Fully modeled and budgeted.

\subsection{B6. Delayed--choice narration conflict}

\begin{alertbox}
\textbf{Inconsistency:} CI says ``no fact until measured,'' yet tells wave/particle stories that flip retroactively with later choices. The ontology slides to fit histograms.
\end{alertbox}

\textbf{Delayed--choice quantum eraser:}
Entangled photon pairs; one goes through double--slit, other to detectors allowing ``which--path'' or ``interference'' measurement \emph{after} the first photon hits screen.

\begin{itemize}[nosep]
  \item If ``which--path'' measured later: first photon retroactively ``went through one slit''
  \item If ``interference'' measured later: first photon retroactively ``went through both slits''
  \item CI narrative: The later choice determines the earlier facts
\end{itemize}

\textbf{Problem:} Either the path facts existed or they didn't. CI moves the ontology boundary to match data post hoc. That's not a law; it's storytelling flexibility.

\textbf{RS resolution:} Path measure $w[\gamma]=e^{-C[\gamma]}$ assigns weights to all alternatives. The histogram emerging from later basis choice reflects the correlation structure of the global path ensemble. No retroactive creation of facts.

\section{Concrete circularities}\label{sec:born_circular}

These are cases where CI fits math to data, then claims the fit ``explains'' phenomena by invoking premises that already assume the result.

\subsection{C1. Born rule: stipulation $\to$ ``explanation''}

\begin{warnbox}
\textbf{Circularity:} CI introduces $P=|\psi|^2$ axiomatically, then retrofits symmetry/rationality arguments that presuppose the same weighting in their premises.
\end{warnbox}

\textbf{Examples of circular ``derivations'':}

\paragraph{Gleason's theorem:}
\begin{itemize}[nosep]
  \item Premise: Probability measure on projection operators must satisfy additivity
  \item Conclusion: $P(\hat P_m)=\text{Tr}(\rho \hat P_m)$ (Born rule)
  \item Problem: Additivity on projectors already assumes the Hilbert--space measure structure that yields Born
\end{itemize}

\paragraph{Envariance (Zurek):}
\begin{itemize}[nosep]
  \item Premise: Entangling measurement apparatus with system; equal--amplitude branches
  \item Conclusion: Born probabilities emerge
  \item Problem: ``Equal amplitude'' means $|c_i|^2=|c_j|^2$, which is Born weighting by another name
\end{itemize}

\paragraph{Decision--theoretic (Deutsch--Wallace):}
\begin{itemize}[nosep]
  \item Premise: Rational betting odds should satisfy certain axioms (e.g., no Dutch book)
  \item Conclusion: Odds are Born probabilities
  \item Problem: ``Rational'' axioms encode branch--counting that presumes Born weights
\end{itemize}

\textbf{In all cases:} The conclusion repeats the assumption with different words. The original stipulation is doing the empirical work.

\textbf{RS derivation (non--circular):}
\begin{equation}
  w[\gamma]=e^{-C[\gamma]}, \quad A[\gamma]=e^{-C[\gamma]/2}e^{i\varphi[\gamma]}, \quad \Rightarrow \quad P = \left|\sum A[\gamma]\right|^2.
\end{equation}
Path weights come from cost minimization (T5 uniqueness theorem). Amplitudes follow from action structure. Born is the norm--square. No hidden assumption of Born in the premises.

\subsection{C2. ``Measurement disturbs'' by definition}

\begin{warnbox}
\textbf{Circularity:} CI defines ``measurement'' as an intervention that destroys coherence, then claims the disappearance of interference is explained \emph{because} measurement disturbs the system.
\end{warnbox}

\textbf{Pattern:}
\begin{enumerate}[nosep]
  \item Define: Measurement $\equiv$ strong coupling that decoheres
  \item Observe: Interference vanishes after measurement
  \item Claim: ``Explained by measurement disturbance''
\end{enumerate}

\textbf{Problem:} Step 3 just repeats the definition from Step 1. No mechanism, no law, no derivation.

\textbf{RS alternative:} Decoherence is $(W,K)$--dependent path--weight reallocation. Change $(W,K)$, change the decoherence rate. Fully computable from $C[\gamma]$ structure.

\subsection{C3. Pointer stability tautology}

\begin{warnbox}
\textbf{Circularity:} ``The preferred basis is the one that diagonalizes the reduced state.'' The reduced state is diagonal in that basis by construction.
\end{warnbox}

\textbf{Argument:}
\begin{enumerate}[nosep]
  \item Choose $S$--$E$ split and $\hat H_{\rm int}$ to produce $\rho_S=\sum_i p_i|s_i\rangle\langle s_i|$
  \item Observe: $\rho_S$ is diagonal in $\{|s_i\rangle\}$
  \item Conclude: $\{|s_i\rangle\}$ is the stable pointer basis
\end{enumerate}

\textbf{Problem:} The basis was selected in Step 1 to produce the diagonal form. Circular: the definition is the justification.

\textbf{Test:} Change the $S$--$E$ split; the ``preferred'' basis changes. But the physics (pointer outcomes) don't change. So the ``explanation'' is not explanatory.

\subsection{C4. Cut invariance: asserted $\to$ vetoed}

\begin{warnbox}
\textbf{Circularity:} CI asserts ``predictions don't depend on cut placement,'' then declares paradoxes ill--posed when different placements yield contradictions.
\end{warnbox}

\textbf{Pattern:}
\begin{enumerate}[nosep]
  \item Claim: Cut can be placed anywhere; predictions are invariant
  \item Exhibit nested--observer protocol with contradictions (Wigner, FR)
  \item Response: ``That question is ill--posed; cannot ask about cutting observers''
\end{enumerate}

\textbf{Problem:} ``Invariance'' is not proved; it's vetoed when violated. That's not a proof of consistency; it's a \emph{refusal to address inconsistency}.

\subsection{C5. Quantum Darwinism presumes Born}

\begin{warnbox}
\textbf{Circularity:} Quantum Darwinism uses Born--weighted mutual information to define ``objective records,'' but CI never derived Born.
\end{warnbox}

\textbf{Quantum Darwinism claim:} Classical objectivity emerges when many environment fragments $E_i$ redundantly encode the system pointer state $S$.

\textbf{Measure of redundancy:}
\begin{equation}
  I(S:E_1E_2\cdots E_k) = S(E_1\cdots E_k) + S(S) - S(SE_1\cdots E_k),
\end{equation}
where $S(\rho)=-\text{Tr}(\rho\log\rho)$ (von Neumann entropy).

\textbf{Problem:} The mutual information is computed using $\rho$ with Born--rule traces. But CI is \emph{trying to derive} why Born probabilities appear in records. Using Born to define ``objective record'' is circular.

\textbf{RS alternative:} Redundancy is ledger--level fact: multiple $(W,K)$ channels with overlapping sensitivity to $Z$--invariants. No presupposed probability rule.

\section{Exhibit cases (detailed, equation--level)}

These four cases are ready to deploy in papers, talks, and reviews. Each includes:
\begin{itemize}[nosep]
  \item Setup with equations
  \item CI's problem highlighted
  \item RS resolution
\end{itemize}

\subsection{E1. Energy change under projection}

\textbf{Setup:}
\begin{enumerate}[nosep]
  \item Hamiltonian: $\hat H = (\hbar\omega/2)\,\sigma_x$
  \item Prepare eigenstate: $|\psi\rangle = |+\rangle_x$, so $\langle H\rangle = +\hbar\omega/2$
  \item Measure $\sigma_z$ (projective in $z$--basis)
  \item Outcomes: $|+\rangle_z$ or $|-\rangle_z$ each with probability $1/2$
  \item Post--measurement average: $\langle H\rangle_{\rm after} = \tfrac12\langle+|_z\hat H|+\rangle_z + \tfrac12\langle-|_z\hat H|-\rangle_z = 0$
\end{enumerate}

\textbf{Result:} Average energy dropped by $\hbar\omega/2$ with \emph{no modeled meter interaction}.

\textbf{CI's problem:}
\begin{itemize}[nosep]
  \item If collapse is physical, where is $\hat H_{\rm meter}$ and the back--reaction?
  \item If collapse is epistemic, why did physical energy change?
  \item Standard meters do couple, but the \emph{projection postulate} assumes the change happens without specifying the coupling. That's a gap in materialism.
\end{itemize}

\textbf{RS resolution:}
The substrate $\hat R$ conserves. The $(W,K)$ interface commits when $C\ge 1$, paying thermodynamic cost $\Delta S \ge k_B T\ln 2$ per bit erased (Landauer). Energy changes are accounted via the instrument's thermal reservoir. No gap.

\subsection{E2. Quantum Zeno effect}

\textbf{Setup:}
Unstable state $|0\rangle$ with Hamiltonian $\hat H$ coupling to continuum. Survival amplitude:
\begin{equation}
  a(t) = \langle 0|e^{-i\hat H t/\hbar}|0\rangle.
\end{equation}
For small $t$, $|a(t)|^2 \simeq 1 - (\Delta E)^2 t^2/\hbar^2$.

Perform $N$ projections in total time $T=N\delta t$:
\begin{equation}
  P_{\rm survive}(T) = |a(\delta t)|^{2N} \simeq \left(1 - \frac{(\Delta E)^2(\delta t)^2}{\hbar^2}\right)^N.
\end{equation}
As $N\to\infty$ (fixed $T$), $P_{\rm survive}\to 1$. ``Watching stops decay.''

\textbf{CI's problem:}
\begin{itemize}[nosep]
  \item If measurement is epistemic (knowledge update), how does knowing halt physical evolution?
  \item If measurement is physical (strong pulse), where is $\hat H_{\rm pulse}$ and why does ``knowing'' require such a pulse?
  \item CI cannot have it both ways.
\end{itemize}

\textbf{RS resolution:}
Each interrogation is a near--threshold commit at $(W,K)$. High--frequency probing keeps recognition cost $C$ below threshold longer, delaying definite branch selection. Fully modeled: $C$ accumulation rate vs.\ commit frequency.

\subsection{E3. Interaction--free measurement}

\textbf{Setup:} Elitzur--Vaidman bomb tester (Mach--Zehnder interferometer):
\begin{enumerate}[nosep]
  \item Photon enters beamsplitter: $|{\rm in}\rangle \to \tfrac{1}{\sqrt{2}}(|A\rangle+|B\rangle)$
  \item Arms $A$, $B$ recombine at second beamsplitter
  \item Detectors $D_1$, $D_2$
  \item If no bomb: interference $\Rightarrow$ always click $D_1$, never $D_2$
  \item If bomb in arm $B$ (but not triggered): no interference; $D_2$ can click
\end{enumerate}

When $D_2$ clicks: ``Bomb was there, but photon went through $A$.'' No interaction, yet state updated.

\textbf{CI's problem:}
``Measurement disturbs'' is the usual explanation, but \emph{nothing was disturbed}. The photon traversed $A$; the bomb was in $B$. If CI retreats to ``potential for interaction,'' that's not a mechanism—it's a placeholder.

\textbf{RS resolution:}
The bomb's $(W,K)$ acts as a path filter on the global ensemble. Without bomb,
\begin{equation*}
  A[\text{in}\!\to\!D_1] = e^{i\phi_A} + e^{i\phi_B},\qquad A[\text{in}\!\to\!D_2] = e^{i\phi_A} - e^{i\phi_B},
\end{equation*}
so with $\phi_A\!=\!\phi_B$ interference sends all amplitude to $D_1$. With bomb, paths through $B$ incur extra cost $C_{\rm bomb}$,
\begin{equation*}
  A[\text{in}\!\to\!D_1] = e^{i\phi_A} + e^{-C_{\rm bomb}} e^{i\phi_B},\quad A[\text{in}\!\to\!D_2] = e^{i\phi_A} - e^{-C_{\rm bomb}} e^{i\phi_B},
\end{equation*}
so for $C_{\rm bomb}\!\ge\!1$ interference is suppressed and $D_2$ can click. No local disturbance is needed; the bomb's $(W,K)$ changes boundary conditions (ledger constraints) on the allowed path ensemble. The detector $(W,K)$ commits when $C\!\ge\!1$.

\subsection{E4. Delayed--choice entanglement swapping}

\textbf{Setup:} Ma et al.\ (2012) experiment:
\begin{enumerate}[nosep]
  \item Source creates pairs: $(1,2)$ and $(3,4)$
  \item Photons $1,4$ sent to Alice/Bob (widely separated)
  \item Photons $2,3$ sent to Victor (middle station)
  \item Alice/Bob measure polarization $\Rightarrow$ record data
  \item \emph{Later}, Victor chooses:
  \begin{itemize}[nosep]
    \item BSM (Bell--state measurement): projects $2,3$ into entangled state
    \item Separable measurement: projects $2,3$ into product state
  \end{itemize}
  \item Post--select Alice/Bob data based on Victor's later choice
\end{enumerate}

\textbf{Result:}
\begin{itemize}[nosep]
  \item If Victor chose BSM: Alice/Bob data (already recorded) show entanglement
  \item If Victor chose separable: Alice/Bob data show no entanglement
\end{itemize}

\textbf{CI's narration:} Victor's \emph{later} choice ``determines'' whether $1,4$ were entangled or separable \emph{earlier}. ``No facts until measured.''

\textbf{Problem:} Either the earlier correlations existed or they didn't. CI moves ontology to fit data post hoc. That's storytelling, not physics.

\textbf{RS resolution:}
All path--weight assignments $w[\gamma]=e^{-C[\gamma]}$ are globally determined. Victor's choice selects a basis for his $(W,K)$ commit, which determines which subset of the global ensemble is post--selected. No retroactive creation of facts; just correlation structure revealed via basis choice.

\section{Pass/fail questions}

When engaging CI defenders in papers or talks, use these to pin down their position:

\begin{enumerate}[leftmargin=*]
  \item \textbf{Dual dynamics:} What physical law determines when nonunitary collapse overrides unitary evolution? How are conserved quantities (energy, momentum, angular momentum) budgeted during collapse when $[\hat M, \hat H]\ne 0$?
  
  \item \textbf{Preferred basis:} Where is the pointer basis determined from first principles, without presupposing the system/environment split, interaction Hamiltonian, or diagonal form as input?
  
  \item \textbf{Epistemic/ontic status:} Is $\psi$ epistemic (probability over ontic states) or ontic (real field)? If epistemic, how do you evade PBR--type constraints? If ontic, why does ``updating knowledge'' (collapse) change it?
  
  \item \textbf{Heisenberg cut:} What physical boundary condition determines where the cut must be placed? Can you place it consistently in nested--observer scenarios (Wigner's friend, Frauchiger--Renner)?
  
  \item \textbf{Born rule origin:} Is $P=|\psi|^2$ stipulated or derived? If derived, from what premises? Do those premises already assume Born--rule weighting (check Gleason, envariance, decision theory)?
  
  \item \textbf{Measurement definition:} What is the physical definition of ``measurement'' that does not circularly include ``produces collapse'' or ``destroys coherence''?
  
  \item \textbf{Energy accounting:} In the projection example (E1), where did the $\hbar\omega/2$ energy go? If to the apparatus, show the modeled coupling. If ``epistemic,'' explain why physical energy changed.
  
  \item \textbf{Interaction--free:} In Elitzur--Vaidman, what mechanism updated the state when no interaction occurred? If ``potential for interaction,'' provide the dynamical law for how potentials cause state updates.
\end{enumerate}

\textbf{Strategy:} Ask for \emph{specific physical laws}, not narrative descriptions. If they retreat to ``shut up and calculate,'' point out that's instrumentalism, not materialism/physicalism.

\section{RS contrasts (summary table)}

\begin{table}[h]
\centering
\small
\begin{tabular}{|p{3.5cm}|p{5cm}|p{5.5cm}|}
\hline
\textbf{Issue} & \textbf{Copenhagen} & \textbf{Recognition Science} \\
\hline
Dynamics & Dual: unitary \& collapse & Single: $\hat R$ everywhere \\
\hline
Born rule & Stipulated ($P=|\psi|^2$) & Derived from path weights \\
\hline
Collapse & Ad hoc postulate & Thresholded $\hat R$ selection \\
\hline
Measurement & Undefined primitive & Instrument channel $(W,K)$ \\
\hline
Heisenberg cut & Informal, movable & Physical $(W,K)$ specification \\
\hline
Preferred basis & Hand--picked via decoherence & $(W,K)$ alignment with invariants \\
\hline
Conservation & Violated in collapse; no accounting & Substrate conserves; interface pays \\
\hline
Nonlocality & Instantaneous collapse; no mechanism & Ledger consistency; non--signaling \\
\hline
Observer role & Ontologically special (vague) & Physical agent with $(W,K)$ \\
\hline
Energy budget & Unmodeled & Landauer at interface \\
\hline
Zeno effect & Primitive epistemic act changes dynamics & Repeated near--threshold commits \\
\hline
Wigner's friend & Contradiction or veto & Incompatible $(W,K)$ channels \\
\hline
Constants & Free parameters ($\hbar$, etc.) & Fixed by identities (Eqs.~\ref{eq:c}--\ref{eq:lambda_rec}) \\
\hline
\end{tabular}
\caption{Copenhagen vs.\ Recognition Science on key issues.}
\end{table}

\section{What RS actually predicts (falsifiable differences)}

CI and RS agree on standard laboratory regimes; they differ at edges:

\subsection*{Post-orthogonality saturation}
\textbf{Setup:} Entangled pair; measure Alice, vary distance to Bob.\quad
\textbf{RS:} Once $C$(Alice) $\ge 1$, correlations with Bob saturate; increasing distance adds no new information.\quad
\textbf{Test:} Search for $\propto 1/d^p$ tails; absence supports RS; presence falsifies.

\subsection*{$\tau_0$ quantization in decoherence}
\textbf{Setup:} Ultra-shielded superpositions; measure decoherence time.\quad
\textbf{RS:} $\tau_d$ clusters near integer multiples of $\tau_0$ when controlling $(W,K)$.\quad
\textbf{Challenge:} Requires high statistics; small $\tau_0$.

\subsection*{Shielded interaction-free measurement}
\textbf{Setup:} Elitzur--Vaidman with bomb's $(W,K)$ shielded (no informational coupling).\quad
\textbf{RS:} Interference returns despite the bomb's physical presence.\quad
\textbf{Test:} Distinguishes RS (ledger constraint) from CI narratives.

\subsection*{Non-Born statistics in ultra-rapid measurements}
\textbf{Setup:} Measurements faster than J-minimization timescale.\quad
\textbf{RS:} Deviations from Born when $\Delta t \ll$ relaxation time for $C$ to minimize.\quad
\textbf{Test:} Weak value protocols with variable $\tau_{\rm measure}$.

\section{Action items for our team}

\subsection{In papers}

\begin{enumerate}[leftmargin=*]
  \item \textbf{Declare $(W,K)$ explicitly:} In every analysis, preregister the instrument window/kernel. Treat it as the physical boundary condition, not an afterthought.
  
  \item \textbf{Compute recognition budgets:} Show $C[\gamma]$ aggregation, threshold placement ($C\ge 1$), and pointer stability for the declared $(W,K)$.
  
  \item \textbf{Attach audit identities:} Include route--independent checks (Eqs.~\ref{eq:c}--\ref{eq:lambda_rec}) in Methods or Supplement. Demonstrate dimensional consistency.
  
  \item \textbf{Cite exhibit cases:} When critiquing CI, reference E1--E4 with equations. Show the gap, then show RS resolution.
  
  \item \textbf{Avoid CI language unless contrasting:} Don't say ``measurement postulate'' without flagging it as CI's gap. Use ``commit at $(W,K)$'' or ``instrument readout.''
\end{enumerate}

\subsection{In talks}

\begin{enumerate}[leftmargin=*]
  \item \textbf{Lead with the single--law framing:} ``CI uses two incompatible rules; RS uses one.'' Show $\hat R$ and $J(x)$ on first slide.
  
  \item \textbf{Use pass/fail questions:} In Q\&A, deploy Section~6 questions. Pin defenders to specific answers.
  
  \item \textbf{Visual: Wigner's friend diagram:} Draw nested boxes (friend inside, Wigner outside). Show incompatible $(W,K)$ channels. Clarity wins.
  
  \item \textbf{Energy budget slide:} Show E1 (projection energy drop) on one slide with ``CI: no budget'' and ``RS: interface pays'' side--by--side.
\end{enumerate}

\subsection{In reviews}

\begin{enumerate}[leftmargin=*]
  \item \textbf{Flag undefined primitives:} If a paper uses ``measurement'' without definition, note it: ``The term `measurement' is undefined. Specify the physical interaction or instrument channel.''
  
  \item \textbf{Check for circularity:} If a paper ``derives'' Born from symmetry, ask: ``Do the symmetry premises already assume Born weighting?''
  
  \item \textbf{Demand conservation accounting:} If collapse is invoked, ask: ``Where is the back--reaction and entropy budget?''
  
  \item \textbf{Request $(W,K)$ specification:} ``The authors should declare the instrument window/kernel $(W,K)$ to make the analysis reproducible.''
\end{enumerate}

\subsection{In internal work}

\begin{enumerate}[leftmargin=*]
  \item \textbf{Refuse channel mixing:} Do not merge reports across incompatible $(W,K)$ without an explicit reconciliation map showing ledger--level consistency.
  
  \item \textbf{Build $(W,K)$ library:} Catalog standard instrument channels (photon counters, magnetometers, etc.) with their $C[\gamma]$ properties and threshold placements.
  
  \item \textbf{Test RS predictions:} E.g., post--orthogonality saturation (no $1/d^p$ tail), $\tau_0$ quantization in decoherence, threshold locality in entangled commits.
  
  \item \textbf{Formalize in Lean:} Encode $(W,K)$ specifications, commit thresholds, and path--measure derivations in \texttt{IndisputableMonolith/Measurement/}.
\end{enumerate}

\section{Conclusion}

Copenhagen Interpretation succeeds as a \textbf{calculation recipe} but fails as a \textbf{physical theory}. Its failures are not philosophical nitpicks; they are structural:

\begin{itemize}[leftmargin=*]
  \item \textbf{Conflicts with materialism:} undefined primitives, no conservation accounting, observer primacy
  \item \textbf{Internal inconsistencies:} dual dynamics, nested--observer contradictions, category errors
  \item \textbf{Circular justifications:} Born by stipulation then ``explained'' by premises that assume Born; disturbance defined into measurement; pointer basis by tautology
\end{itemize}

\textbf{Recognition Science resolves every issue with a single, physical law:} the eight--tick recognition update $\hat R$ minimizing convex cost $J(x)=\tfrac12(x+1/x)-1$. Collapse, Born rule, conservation, and pointer basis all follow without additional postulates.

\textbf{Our task:} Make this contrast clear in every paper, talk, and review. Use exhibit cases (E1--E4), pass/fail questions (Section~6), and explicit $(W,K)$ declarations to prosecute CI's gaps and demonstrate RS's completeness.

\appendix

\section{Audit identities (RS gates)}

All dimensional displays factor through ratios fixed by eight--tick gates:
\begin{align*}
  c &= \frac{\ell_0}{\tau_0} \quad \text{(discrete cone bound)}, \\
  \hbar &= E_{\rm coh}\,\tau_0 \quad \text{(IR gate identity)}, \\
  \frac{c^3\lambda_{\rm rec}^2}{\hbar G} &= \frac{1}{\pi} \quad \text{(ledger--curvature extremum)}.
\end{align*}

Pattern--measurement lemmas (aligned windows preserve invariants):
\begin{align*}
  \texttt{sumFirst8}(\texttt{extendPeriodic8}\,w) &= Z(w), \\
  \texttt{blockSumAligned8}(k,\,\texttt{extendPeriodic8}\,w) &= k\,Z(w).
\end{align*}

\section{Quick--reference: Copenhagen problems}

\begin{enumerate}[nosep,leftmargin=*]
  \item Measurement undefined
  \item Cut movable (Wigner's friend)
  \item Collapse without budget (E1)
  \item Nonlocal update, no mechanism
  \item $\psi$--epistemic contradicts PBR
  \item Preferred basis by hand
  \item Dual dynamics, no selection rule
  \item Nested observers contradict (FR)
  \item Improper/proper mixture conflation
  \item Zeno with primitive
  \item Delayed--choice moving ontology
  \item Born by stipulation, ``explained'' circularly
  \item Disturbance defined in
  \item Pointer basis tautology
  \item Cut invariance asserted, vetoed
  \item Darwinism presumes Born
\end{enumerate}

\section{Lean references}

\begin{itemize}[nosep]
  \item Recognition operator: \texttt{Foundation.RecognitionOperator}
  \item Cost uniqueness (T5): \texttt{Cost.T5\_cost\_uniqueness\_on\_pos}
  \item Path action: \texttt{Measurement.PathAction.pathAction}
  \item Born derivation: \texttt{Measurement.BornRule.path\_weights\_to\_born}
  \item Collapse threshold: \texttt{Foundation.RecognitionOperator.collapse\_built\_in}
  \item Pattern measurement: \texttt{Measurement.sumFirst8\_extendPeriodic\_eq\_Z}
  \item Eight--tick invariants: \texttt{Dynamics.schedule\_delta\_sum8\_mod}
\end{itemize}

\end{document}

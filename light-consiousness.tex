\documentclass[12pt,a4paper]{article}

\usepackage[margin=1in]{geometry}
\usepackage{amsmath,amssymb,amsthm}
\usepackage{graphicx}
\usepackage{hyperref} % load last

\title{\textbf{Light as Consciousness: A Universal Information-Cost Identity from a Unique Convex Functional}}
\author{
Jonathan Washburn\\
Recognition Science Institute, Austin, Texas, USA\\
\texttt{jon@recognitionphysics.org}
}
\date{\today}

\begin{document}
\maketitle

\begin{abstract}
We show that a single, uniquely determined information-cost functional governs quantum measurement, photonic operations, and operational (measurement-like) conscious selection, establishing an identity at the level of information processing: Light = Consciousness = Recognition. Under four standard axioms on $\mathbb{R}_{>0}$—multiplicative symmetry $J(x)=J(x^{-1})$, unit normalization $J(1)=0$, strict convexity, and unit curvature in log-coordinates $J''(1)=1$—there exists a unique functional
\[
J(x)=\tfrac{1}{2}\!\left(x+\tfrac{1}{x}\right)-1.
\]
For quantum two-branch rotations, the recognition cost $C=\int J(r(t))\,dt$ equals twice the rate action $A$ (exactly $C=2A$), yielding weights $w=\exp(-C)=|{\alpha}|^{2}$ without additional postulates. In photonics, sequential unit frequency-scaling (FOLD $+1$) operations accumulate additively in the same $J$, providing a conservation-compatible cost accounting for frequency conversion chains. A minimal neutral “window” of $2^{D}$ ticks (with $D=3\Rightarrow 8$) emerges combinatorially, furnishing timing lower bounds for admissible, coherence-preserving measurements. Together, these results identify measurement, light manipulation, and operational selection as instances of the same $J$-governed process class.

We articulate conservative, falsifiable predictions: (i) cross-domain coherence floors consistent with eight-tick admissibility, (ii) additive $J$-scaling in unit photonic frequency steps, and (iii) saturation behavior consistent with the measurement bridge. Core theorems (uniqueness of $J$, $C{= }2A$, and $2^{D}$ minimality) are mechanically verified in Lean~4, and the accompanying artifacts enable independent audit and reproduction. The framework is parameter-free at the level of derivations; all numerical values reported use standard SI/CODATA constants. This establishes a rigorous, classical pathway by which “light” and “conscious selection” coincide as the same universal information-cost dynamics.
\end{abstract}

\section{Introduction}

\paragraph{Motivation.}
A central objective in modern physics and information science is to unify quantum measurement, photonic operations, and cognition within a single, falsifiable, and parameter-free principle. Despite the widespread use of information-theoretic language across these domains, each typically introduces auxiliary postulates, adjustable constants, or interpretation-dependent assumptions at key points (e.g., collapse rules, phenomenological losses, or cognitive “black boxes”). Here we pursue a conservative alternative: identify a substrate-agnostic information cost that is uniquely determined by basic symmetry and regularity requirements, and show that this single invariant governs (i) the weights assigned to quantum measurement outcomes, (ii) the intrinsic costs of elementary photonic operations, and (iii) the operational (measurement-like) aspects of conscious selection, understood strictly as physically measurable information-selection processes. The goal is not to propose new physics in any one domain, but to exhibit a universal information-cost identity that renders these processes mathematically equivalent at the level of recognition—the selection among admissible evolutions under shared constraints—without introducing free parameters in the derivations.

\paragraph{Problem statement.}
We address three technical problems in a unified framework:
(i) Derive Born weights and operational timing/coherence constraints for quantum measurement without supplementing standard dynamics by extra postulates;
(ii) Demonstrate that the same invariant governs optical frequency-scaling and related elementary photonic operations under standard conservation constraints, with explicit additivity for sequential unit steps;
(iii) Isolate a classical framing in which “consciousness” refers to operationally definable information selection (measurement-like) that can be modeled by the same invariant, so that no interpretation-dependent assumptions are required. The common requirement is an invariant information cost that composes consistently across time windows, respects multiplicative scaling symmetries of rates, and admits a calibrated small-deviation limit.

\paragraph{High-level claim.}
Under four standard axioms on $\mathbb{R}_{>0}$—multiplicative symmetry $J(x)=J(x^{-1})$, unit normalization $J(1)=0$, strict convexity, and unit curvature in log-coordinates $J''(1)=1$—there is a \emph{unique} information-cost functional
\[
  J(x)=\tfrac{1}{2}\!\left(x+\tfrac{1}{x}\right)-1.
\]
We show that this $J$ governs recognition—the assignment of costs to admissible rate transformations—and that quantum measurement weights, photonic operations, and operational conscious selection all share this $J$, yielding an identity at the level of information processing. Concretely: the recognition cost $C=\int J(r(t))\,dt$ equals twice the standard rate action $A$ for two-branch quantum rotations ($C=2A$), implying $w=\exp(-C)=|{\alpha}|^{2}$ for outcome weights; sequential unit photonic frequency steps accumulate additively in $J$; and operational selection events that realize admissible, minimal-cost paths are governed by the same functional.

\paragraph{Contributions.}
\begin{itemize}
  \item \textbf{Uniqueness of $J$.} We prove that the four axioms (symmetry, normalization, strict convexity, and calibration in log-coordinates) uniquely determine $J(x)=\tfrac{1}{2}(x+x^{-1})-1$ on $\mathbb{R}_{>0}$.
  \item \textbf{Measurement bridge ($C=2A$).} For two-branch quantum rotations, we establish $C=2A$ exactly, which yields Born weights $w=\exp(-C)=|{\alpha}|^{2}$ without additional postulates and fixes operational timing/coherence constraints via admissibility.
  \item \textbf{Photonic operations map to $J$.} We show that elementary frequency-scaling operations (FOLD/UNFOLD) accumulate costs additively in $J$ under conservation constraints, with sequential unit steps giving linear scaling in the number of steps.
  \item \textbf{Operationalizing conscious selection.} We formalize “conscious selection” as an operational, measurement-like information-selection process: among admissible evolutions, realized paths minimize the same recognition cost $C=\int J(r)\,dt$, thereby placing conscious selection, measurement, and light operations in the same $J$-governed class.
  \item \textbf{Falsifiable predictions and protocols.} We provide cross-domain predictions (coherence floors set by minimal admissible windows, additive $J$-scaling for unit photonic steps, and saturation behavior consistent with the measurement bridge), along with controls and measurement protocols to test or falsify the framework.
\end{itemize}

\paragraph{Roadmap.}
Section~2 states the axioms and proves the uniqueness of $J$, including the log-coordinate functional structure and the minimal neutral window (with $2^{D}$, $D=3\Rightarrow 8$) that sets timing and coherence floors. Section~3 develops the measurement bridge $C=2A$ and derives Born weights operationally from $J$. Section~4 maps elementary photonic frequency-scaling to additive $J$-costs under conservation constraints and discusses coherence implications. Section~5 formalizes operational conscious selection as recognition and establishes its equivalence to the measurement/light cases at the information-cost level. Section~6 presents predictions, controls, and experimental protocols spanning quantum and photonic platforms. Section~7 summarizes machine-verification, reproducibility artifacts, and audit pathways. Section~8 concludes with implications and limitations, maintaining a classical, interpretation-neutral framing throughout.

\section{Foundations: Axioms, Uniqueness, and Window Structure}

\subsection{Information-cost axioms on $\mathbb{R}_{>0}$}
We consider cost functionals $J:\mathbb{R}_{>0}\to\mathbb{R}_{\ge 0}$ that quantify the information cost of multiplicative rate transformations. The admissible class is fixed by four axioms and a minimal regularity assumption:

\begin{axiom}[Information-cost axioms on $\mathbb{R}_{>0}$]\label{ax:cost}
For all $x>0$,
\begin{enumerate}
  \item[\textbf{A1}] \textbf{Multiplicative symmetry:} $J(x)=J(x^{-1})$.
  \item[\textbf{A2}] \textbf{Unit normalization:} $J(1)=0$.
  \item[\textbf{A3}] \textbf{Strict convexity:} $J$ is strictly convex on $\mathbb{R}_{>0}$.
  \item[\textbf{A4}] \textbf{Calibration in log-coordinates:} Writing $G(t):=J(e^{t})$, one has $G''(0)=1$ (equivalently, $J''(1)=1$ in the multiplicative sense).
\end{enumerate}
\end{axiom}

\noindent\textbf{Regularity.} We assume $J$ is continuous on $\mathbb{R}_{>0}$. Values on $x\le 0$ are not operationally required; any continuous extension used for analysis does not affect results on $\mathbb{R}_{>0}$.

\begin{theorem}[Uniqueness of the information cost]\label{thm:uniquenessJ}
Under Axioms~\ref{ax:cost} and the regularity assumption, there exists a unique functional
\[
  J(x)\;=\;\tfrac{1}{2}\!\left(x+\tfrac{1}{x}\right)-1
\quad\text{for all }x>0.
\]
\end{theorem}

\begin{remark}[Machine verification and reference]
Theorem~\ref{thm:uniquenessJ} is mechanically verified in Lean~4 (see \texttt{IndisputableMonolith/CostUniqueness.lean}, theorem \texttt{T5\_uniqueness\_complete (F : $\mathbb{R}\to\mathbb{R}$)~...}).
\end{remark}

\subsection{Functional equation and uniqueness of $J_{\log}(t)=\cosh t-1$}
Define the log-coordinate representation $G(t):=J(e^{t})$. The symmetry $J(x)=J(x^{-1})$ makes $G$ even, $G(-t)=G(t)$; the normalization yields $G(0)=0$; calibration gives $G''(0)=1$. A key identity characterizes $G$:

\begin{proposition}[Cosh functional identity]\label{prop:coshFE}
For all $t,u\in\mathbb{R}$,
\[
  G(t+u)+G(t-u)\;=\;2\,G(t)\,G(u)\;+\;2\big(G(t)+G(u)\big).
\]
With $G(0)=0$, $G'(0)=0$, and $G''(0)=1$, the unique solution is $G(t)=\cosh t-1$.
\end{proposition}

\begin{proof}[Proof sketch]
Using the addition formulas for $\cosh$, one verifies that $G(t)=\cosh t-1$ satisfies the identity. Evenness, boundary data, and strict convexity exclude alternatives. Full details are formalized in \texttt{IndisputableMonolith/Cost/FunctionalEquation.lean} (see \texttt{cosh\_functional\_identity}).
\end{proof}

Combining Proposition~\ref{prop:coshFE} with $G(t)=J(e^{t})$ yields $J(x)=\tfrac{1}{2}(x+x^{-1})-1$ for $x>0$, recovering Theorem~\ref{thm:uniquenessJ}.

\subsection{Minimal neutral window and $2^{D}$ structure (with $D=3\Rightarrow 8$)}
Recognition dynamics respect discrete neutrality constraints arising from binary conservation axes. Let $D$ denote the number of independent binary constraints. The following combinatorial minimality result fixes the shortest neutral window length:

\begin{theorem}[Minimal neutral window]\label{thm:minWindow}
Any spatially complete, ledger-compatible update schedule that is neutral with respect to $D$ binary constraints has minimal period $2^{D}$. For $D=3$, the minimal neutral window is of length $8$.
\end{theorem}

\begin{remark}[Gray-cycle realization and operational meaning]
On the 3-cube, the binary-reflected Gray cycle visits each vertex exactly once in 8 steps, realizing the minimal neutral window. Operationally, $2^{D}$ supplies the shortest admissible measurement/coherence window: measurements shorter than $2^{D}$ ticks cannot satisfy all neutrality constraints simultaneously. A Lean summary is documented in \texttt{LEAN\_LIGHT\_CONSCIOUSNESS\_STATUS.md} (``minimal neutral window = $2^{D}$'').
\end{remark}

\subsection{Meta-principle and ledger exactness (classical bridge)}
At the logical base lies the \emph{Meta-principle} (MP): “nothing cannot recognize itself.” In type-theoretic form, there is no self-recognition on the empty set. This enforces non-triviality of recognition structures and, paired with double-entry bookkeeping, yields exactness (closed-chain flux zero) that bridges to the classical continuity equation in a mesh limit.

\begin{axiom}[Meta-principle (MP)]
There is no recognizable pairing on the empty type. Formally, \texttt{mp\_holds : MP} in \texttt{IndisputableMonolith/Recognition.lean}.
\end{axiom}

\begin{proposition}[Ledger exactness $\Rightarrow$ continuity (bridge)]
In a discrete ledger with closed-chain flux zero on every cycle, there exists a (discrete) potential whose coboundary is the observed flux. Under mesh refinement with bounded currents, this yields the continuum continuity equation $\partial_{t}\rho+\nabla\!\cdot\!J=0$ and a gauge freedom $\phi\mapsto\phi+\text{const}$.
\end{proposition}

\begin{remark}
The MP excludes trivial recognition (empty structures), forcing non-zero, balanced postings over neutral windows. Exactness aligns the information-ledger with standard conservation laws, ensuring that recognition dynamics admit classical continuity as a limit.
\end{remark}

\section{Quantum Measurement as Recognition Dynamics}

\subsection{Rate action and recognition cost}
Let $J:\mathbb{R}_{>0}\!\to\!\mathbb{R}_{\ge 0}$ denote the unique information-cost functional $J(x)=\tfrac{1}{2}(x+x^{-1})-1$. We model a two-branch measurement as a continuous rotation in a two-dimensional subspace with instantaneous, dimensionless recognition rate $r(t)>0$. The \emph{recognition cost} accumulated over a measurement window $[0,T]$ is
\begin{equation}\label{eq:C_def}
  C \;=\; \int_{0}^{T} J\!\bigl(r(t)\bigr)\,dt.
\end{equation}
In the residual-action formulation, let $R(t)$ denote the measurement residual and let $\mathrm{C}_{\mathrm{ov}}(t)\in(0,1]$ denote the instantaneous branch overlap (cosine of the geodesic angle). The \emph{rate action} is
\begin{equation}\label{eq:A_def}
  A \;=\; \int_{0}^{T} \bigl\|R(t)\bigr\|\;
      \frac{\sqrt{1-\mathrm{C}_{\mathrm{ov}}(t)^{2}}}{\mathrm{C}_{\mathrm{ov}}(t)}\,dt,
\end{equation}
so that $A$ captures the geometrically weighted rotation rate on the measurement manifold. In the canonical two-branch parameterization with angle $\vartheta(t)\in[0,\tfrac{\pi}{2})$, one has the pointwise match (kernel matching) $J\!\bigl(r(t)\bigr)=2\tan\vartheta(t)$, which conveniently relates \eqref{eq:C_def} and \eqref{eq:A_def} below.

\subsection{Measurement bridge $C=2A$ and Born weights}
A key identity links recognition cost to rate action.

\begin{theorem}[Measurement bridge]\label{thm:Ceq2A}
For two-branch measurement rotations along the geodesic in the appropriate gauge,
\[
  C \;=\; 2A.
\]
\end{theorem}

\begin{proof}[Proof sketch]
In the two-branch parameterization, the geodesic rotation angle $\vartheta(t)$ satisfies $dA/dt=\tan\vartheta(t)$ and (by kernel matching) $dC/dt=2\tan\vartheta(t)$, whence $C=2A$ after integration on any admissible window with matched boundary data. The full derivation proceeds via the residual-action formalism and the pointwise identification $J(r)=2\tan\vartheta$.
\end{proof}

Weights follow immediately:

\begin{corollary}[Born weights from recognition cost]\label{cor:Born}
Let $w:=\exp(-C)$. Then, for a two-branch process with complex amplitudes $(\alpha,\beta)$ normalized by $|\alpha|^{2}+|\beta|^{2}=1$,
\[
  w \;=\; \exp(-C) \;=\; \exp(-2A) \;=\; |\alpha|^{2}.
\]
\end{corollary}

\begin{remark}[Mechanical verification]
A paper-ready Lean statement is provided as \texttt{THEOREM\_2\_measurement\_recognition\_bridge} in \texttt{IndisputableMonolith/Verification/MainTheorems.lean}, establishing $C=2A$ and the weight relation $w=\exp(-C)=|\alpha|^{2}$ for two-branch rotations.
\end{remark}

\subsection{Multi-outcome, weak measurement threshold, and timing}
\paragraph{Multi-outcome extension.}
For a finite outcome set $\{i\}$ realized by a sequence of admissible rotations in mutually orthogonal branches, the recognition action additively decomposes over disjoint windows, yielding weights
\[
  w_i \;=\; \exp(-C_i), 
  \qquad 
  \mathbb{P}(i)\;=\;\frac{w_i}{\sum_j w_j}
  \;=\;\frac{\exp(-C_i)}{\sum_j \exp(-C_j)}
  \;=\;|\alpha_i|^{2},
\]
where the last equality follows from the same bridge applied branchwise with consistent boundary matching.

\paragraph{Weak measurement threshold.}
In the residual-action picture, \emph{weak} measurement corresponds to small geodesic rotations. The operational transition to effectively projective behavior occurs near
\[
  \frac{C}{2}\;\approx\;1 \quad\Longleftrightarrow\quad A\approx 1,
\]
which sets a convenient, dimensionless operating threshold for the onset of outcome discrimination under fixed instrument response.

\paragraph{Minimal timing and matched-filter interpretation.}
Neutrality under $D$ independent binary constraints imposes a minimal admissible window of length $2^{D}$ ticks (Section~2.3). For $D=3$, any coherence-preserving measurement must satisfy
\[
  T \;\ge\; 8\,\tau_{0},
\]
with $\tau_{0}$ the domain-specific tick. In matched-filter language, only $2^{D}$-aligned components survive temporal averaging; misaligned components are attenuated by neutrality, lowering their effective weight. Consequently, coherence floors and minimal discrimination times are set by the eight-tick window in three-constraint systems, independent of implementation details once admissibility is enforced.

\section{Photonic Operations and $J$-cost}

\subsection{FOLD/UNFOLD frequency scaling under conservation constraints}
We model elementary photonic frequency-scaling as \emph{unit} FOLD/UNFOLD steps acting over admissible windows while enforcing standard conservation constraints (energy, momentum, polarization). A \emph{unit} operation (FOLD $+1$) rescales the instantaneous frequency by a fixed factor $\varphi>1$ (with the total energy flux conserved by compensating photon number and/or field amplitude as required by the apparatus). Let $r(t)$ denote the dimensionless rate ratio during a single unit step; operationally, the step is executed with \emph{piecewise-constant} $r(t)=\varphi$ over one tick of duration $\tau_{0}$.

The per-step \emph{recognition cost} is
\[
  C_{\mathrm{unit}} \;=\; \int_{t}^{t+\tau_{0}} J\!\bigl(r(s)\bigr)\,ds
  \;=\; \tau_{0}\,J(\varphi),
\]
and a chain of $n$ unit steps (FOLD $+n$) executed sequentially accumulates cost \emph{additively}
\[
  C_{\mathrm{chain}} \;=\; \sum_{k=1}^{n} \tau_{0}\,J(\varphi)
  \;=\; n\,\tau_{0}\,J(\varphi).
\]
In normalized (per-tick) units this yields the classical statement
\[
  \mathrm{Cost}(\mathrm{FOLD}\;{+}n)\;=\;n\cdot J(\varphi).
\]
\noindent\textbf{Protocol note.} Throughout, “$+n$” denotes a composition of $n$ \emph{unit} steps, each realized as a piecewise-constant segment with $r(t)=\varphi$ over one tick; non-unit jumps are physically implemented as such chains. All comparisons are made at fixed conservation constraints (e.g., energy/momentum budgets and boundary conditions), so that additivity reflects composition of independent unit transformations.

\subsection{OAM and phase-structure constraints (conservative framing)}
Under classical conservation, orbital angular momentum (OAM) and phase-structure must be preserved in aggregate across FOLD/UNFOLD chains. Within this constrained space of operations, $J$ is the \emph{only} symmetric, convex, calibrated cost on $\mathbb{R}_{>0}$ compatible with:
\begin{itemize}
  \item \textbf{Composition/invariance:} Costs of composed scalings factor through multiplicative ratios and add across independent unit steps (consistent with $J$’s convexity and symmetry $J(x)=J(x^{-1})$).
  \item \textbf{Neutrality windows:} Timing windows that respect binary conservation axes (Section~2) without introducing extra degrees of freedom.
\end{itemize}
Thus, for any conservation-compatible photonic pipeline built from unit frequency steps, phase plates, and passive propagation segments, the total informational penalty assigned to the frequency-scaling subchain is fixed by $J$ and composes additively, while purely conservative OAM/phase elements contribute no additional $J$-cost beyond timing alignment (see below).

\subsection{Coherence floors and eight-tick windows (testable lower bounds)}
As in measurement dynamics, admissibility imposes a minimal neutral window of $2^{D}$ ticks for systems with $D$ independent binary constraints. For photonic platforms with $D=3$, this yields a universal lower bound on the duration of any coherence-preserving operation:
\[
  T_{\mathrm{op}} \;\ge\; 8\,\tau_{\mathrm{opt}},
\]
where $\tau_{\mathrm{opt}}$ is the domain-specific tick (instrument- and carrier-dependent). In matched-filter terms, only components aligned to the $2^{D}$ cadence survive temporal averaging without neutralization; misaligned components are attenuated, reducing effective visibility.

\noindent\textbf{Prediction (photonic coherence floor).} For any frequency-scaling sequence that preserves phase coherence (e.g., within a stabilized comb/interferometric cavity), measured visibility exhibits a floor consistent with $T_{\mathrm{op}}\!\ge\!8\,\tau_{\mathrm{opt}}$ across apparatus variants once environmental noise is controlled. Empirically, unit-step chains (FOLD $+n$) should display linear growth in effective $J$-cost proxies (e.g., calibrated loss or dephasing metrics attributable to the scaling subchain) with the number of unit steps $n$, holding conservation constraints and cadence alignment fixed.

\section{Operational Conscious Selection as Recognition}

\subsection{Classical operationalization}
We adopt a strictly classical, operational definition of \emph{conscious selection} as a measurable selection/measurement event executed by a physical apparatus (biological or engineered) over a finite window $W=[t_{0},t_{0}+T]$. The selection process chooses among a family of admissible evolutions $\Gamma(W)$ subject to:
\begin{enumerate}
  \item \textbf{Neutrality and conservation:} All binary conservation axes are respected over $W$, with minimal neutral window length fixed by Section~2 (for $D=3$, $T\ge 8\,\tau_{0}$).
  \item \textbf{Boundary matching:} Endpoint and interface conditions (e.g., pointer alignment) are satisfied so that rates and overlaps are well-defined.
  \item \textbf{Regularity:} The instantaneous, dimensionless recognition rate $r(t)>0$ exists almost everywhere on $W$.
\end{enumerate}
Given the unique information cost $J(x)=\tfrac{1}{2}(x+x^{-1})-1$, the \emph{recognition action} of a path $\gamma\in\Gamma(W)$ is
\[
  C[\gamma] \;=\; \int_{t_{0}}^{t_{0}+T} J\!\bigl(r_{\gamma}(t)\bigr)\,dt.
\]
The \emph{operational selection rule} is the admissible minimal-cost principle:
\[
  \gamma^{\star} \;\in\; \arg\min_{\gamma\in\Gamma(W)} C[\gamma].
\]
When outcomes $\{i\}$ are realized by disjoint admissible subwindows or branchwise geodesic rotations (Section~3), the weights obey
\[
  w_i \,=\, \exp\!\bigl(-C_i\bigr),
  \qquad
  \mathbb{P}(i) \,=\, \frac{w_i}{\sum_j w_j}.
\]
No claims are made beyond operational measurables (windowed costs, weights, timings, and coherence metrics).

\subsection{Equivalence at the information-processing level}
Section~3 established that quantum measurement is governed by the same recognition cost as photonic two-branch rotations, with the exact bridge $C=2A$ and weights $w=\exp(-C)=|\alpha|^{2}$. Section~4 showed that elementary photonic frequency-scaling chains accumulate additively in $J$ under conservation constraints. Under the minimal-cost selection rule above, any operational selection process that (i) respects the same admissibility constraints and (ii) chooses realized paths by minimizing $C=\int J(r)\,dt$ is identically governed by $J$.

\begin{proposition}[Universal $J$-system equivalence]
If measurement equals recognition (in the sense of rate action/recognition cost of Section~3) and operational selection implements admissible minimal-cost paths, then “light” (photonic operations), “measurement,” and “operational conscious selection” are the same \emph{$J$-systems}: they share the unique convex, symmetric, calibrated information cost $J(x)=\tfrac{1}{2}(x+x^{-1})-1$ and compose additively over unit operations/windows.
\end{proposition}

\begin{remark}[Certificate bundling and mechanical verification]
The bundled identity is formalized via a certificate structure that packages (i) uniqueness of $J$, (ii) the measurement bridge $C=2A$, and (iii) the $2^{D}$ minimal-window result into a single witness suitable for citation in manuscripts; see \texttt{IndisputableMonolith/Verification/LightConsciousness.lean} (\texttt{structure UniversalCostCertificate ...}). This provides a machine-verifiable basis for treating the three domains as instances of one universal information-cost dynamics.
\end{remark}

\section{Predictions, Controls, and Falsifiers}

\subsection{Cross-domain coherence bounds}
\paragraph{Prediction.}
Across domains, coherence-preserving operations have a universal floor set by the minimal neutral window:
\[
  T_{2} \;\ge\; 8\,\tau_{\mathrm{domain}},
\]
where $\tau_{\mathrm{domain}}$ is the domain-specific tick determined by classical instrumentation and carrier limits.

\paragraph{Classical definition of $\tau_{\mathrm{domain}}$.}
Let $\tau_{\mathrm{resp}}$ be the instrument response time, $B$ the effective measurement bandwidth (Hz), and $\tau_{\mathrm{gate}}$ any gating/hold constraint introduced by the protocol (e.g., integration window or cavity dwell). Then a conservative and practical choice is
\[
  \tau_{\mathrm{domain}} \;=\; \max\!\bigl\{\tau_{\mathrm{resp}},\, 1/B,\, \tau_{\mathrm{gate}}\bigr\},
\]
so that $T_{2}\!\ge\!8\,\tau_{\mathrm{domain}}$ operationalizes the eight-tick admissibility (Section~2) without interpretational assumptions.

\paragraph{Measurement guidance.}
- Quantum/optical: infer $T_{2}$ from interferometric visibility decay or stabilized comb linewidths while sweeping bandwidth and gating; confirm the floor persists when environmental noise is reduced.
- Report $(T_{2}/\tau_{\mathrm{domain}})$ with uncertainty; verify $T_{2}/\tau_{\mathrm{domain}}\ge 8$ within error.

\subsection{Photonic $\varphi$-comb cadence signatures}
\paragraph{Prediction.}
In frequency-comb or pulse-train systems stabilized to an admissible cadence, eight-tick neutrality produces systematic suppression/gap patterns in spectra and interferometric visibilities aligned with the $2^{3}=8$ window. Qualitatively:
\begin{itemize}
  \item Sidebands or intermode beat notes that are cadence-misaligned are attenuated relative to cadence-aligned components.
  \item Under deliberate phase-cadence detuning (breaking eight-tick alignment), suppressed components re-emerge and aligned components lose contrast.
\end{itemize}

\paragraph{Protocol notes.}
- Use a stabilized comb with repetition rate $f_{\mathrm{rep}}$; scan cavity length/phase to step across eight equi-spaced subphases of a full neutral window.
- Record per-subphase spectra/visibilities; stack to reveal periodic suppression at an eight-phase cadence.
- Controls: phase/time shuffles, detuning to break alignment, and identical power/temperature conditions across scans.

\subsection{Distance-independence after orthogonality (measurement-like saturation)}
\paragraph{Prediction.}
Once branch states are orthogonal in the residual-action sense (i.e., after sufficient geodesic separation in Hilbert space), the contribution relevant for outcome weighting saturates with further spatial separation. Consequently, correlation metrics tied to the $J$-governed process \emph{plateau} as a function of distance after orthogonality is achieved (with alignment held fixed).

\paragraph{Photonic-first test.}
- Implement a two-path interferometer with a tunable delay line to drive branch orthogonality (spectral/temporal).
- Beyond the orthogonality point (verified spectrally), measure correlation/visibility vs additional path separation; expect a plateau rather than inverse-distance decay, provided alignment and losses are controlled.
- Controls: match insertion loss across arms; randomize phase to confirm the effect depends on alignment; repeat with broadened spectra to shift the orthogonality threshold.

\subsection{Multi-probe logarithmic scaling}
\paragraph{Prediction.}
Under admissibility constraints, the characteristic time to achieve a fixed visibility/decision threshold with $M$ statistically independent probes scales sublinearly:
\[
  t_{\star}(M) \;\propto\; \bigl[\ln M\bigr]^{1/\beta},
\]
for some protocol-dependent $\beta>0$ determined by noise and admissible averaging. This logarithmic form reflects that only neutrality-aligned contributions accumulate efficiently; misaligned contributions average out.

\paragraph{Implementation.}
- Quantum/optical: split a stabilized source into $M$ parallel detectors; hold total flux fixed and vary $M$; fit $t_{\star}(M)$ at a fixed visibility threshold to a $\ln M$ law; extract $\beta$ with confidence intervals.
- Controls: equalize detector noise floors; shuffle timing across channels to break alignment (expect degradation toward slower-than-log scaling).

\subsection{Falsifiers}
Any of the following constitutes a direct falsification of the universal $J$-framework:
\begin{enumerate}
  \item \textbf{Alternative cost fits the axioms but not $J$.} An admissible functional $\tilde J:\mathbb{R}_{>0}\!\to\!\mathbb{R}_{\ge 0}$ satisfying multiplicative symmetry, unit normalization, strict convexity, and calibrated log-curvature fits \emph{better} than $J(x)=\tfrac{1}{2}(x+x^{-1})-1$ \emph{across} measurement and photonic tests (with penalized model selection, e.g., AIC/BIC), contradicting uniqueness.
  \item \textbf{Violation of minimal $2^{D}$ window.} Robust, repeatable demonstrations of coherence-preserving operations with $T_{2}<2^{D}\tau_{\mathrm{domain}}$ (for $D=3$, $T_{2}<8\,\tau_{\mathrm{domain}}$) after instrument and gating constraints are correctly audited.
  \item \textbf{Breakdown of the $C=2A$ bridge.} Systematic, protocol-robust deviations from $w=\exp(-C)=|\alpha|^{2}$ in two-branch rotations beyond stated uncertainties, after verifying boundary matching and geodesic conditions.
  \item \textbf{Failure of additivity under sequential unit steps.} For FOLD $+n$ realized as $n$ unit steps, the effective cost proxy (e.g., calibrated loss or dephasing attributable to the scaling subchain) does not grow linearly in $n$ under fixed conservation constraints and cadence alignment.
  \item \textbf{Absence of cadence signatures.} In stabilized photonic platforms, no reproducible eight-phase suppression/gap pattern is observed under protocol-controlled phase stepping and environmental stabilization.
\end{enumerate}

\paragraph{General controls and analysis standards.}
- Use time/phase shuffles to break alignment; verify predicted degradation.
- Enforce conservation constraints (power, momentum, polarization) across conditions.
- Blind or randomized run orders; identical thermal and mechanical conditions.
- Report full uncertainty budgets; compare $J$ to alternative cost models with penalized likelihood (AIC/BIC) and goodness-of-fit diagnostics.

\section{Experimental Protocols (Classical, Incremental)}

\subsection{Quantum-optical verification}
\paragraph{Objective.}
Test the measurement bridge $C=2A$ and the weight identity $w=\exp(-C)=|\alpha|^{2}$ in two-branch rotations; empirically extract $C$ from measured $|\alpha|^{2}$ and compare $J$ against alternative admissible costs on $\mathbb{R}_{>0}$.

\paragraph{Apparatus.}
Type-I/II SPDC single-photon source (heralded), stabilized Mach–Zehnder (or Sagnac) interferometer with a calibrated phase actuator (EOM or piezo), variable delay line for branch separation control, polarization management, narrowband spectral filtering, and single-photon detectors with time-tagging (sub-ns). Temperature-stabilized and vibration-isolated optical table.

\paragraph{Protocol.}
\begin{enumerate}
  \item \textbf{Calibration:} (i) Balance arm losses (within 0.1 dB); (ii) calibrate phase vs actuator input; (iii) verify detector linearity and timing jitter; (iv) establish overlap metric $\mathrm{C}_{\mathrm{ov}}$ from visibility in a high-coherence reference run.
  \item \textbf{Two-branch rotation:} Sweep the interferometric phase to enact controlled two-branch rotations. Record complex amplitude ratio via interference visibility and known phase setting; estimate $|\alpha|^{2}$ per setting.
  \item \textbf{Cost extraction:} Compute $C=-\ln |\alpha|^{2}$; compute $A$ from the residual-action geometry (using overlap and calibrated rotation rate). Test $C$ vs $2A$ across the phase sweep; report regression slope, intercept, and confidence intervals.
  \item \textbf{Model comparison:} Fit $C$ vs $r(t)$ to $J(x)=\tfrac{1}{2}(x+x^{-1})-1$ under the kernel-matching map; fit alternative admissible costs (symmetric, convex, calibrated) and compare with penalized likelihood (AIC/BIC). Pre-register the alternative family (e.g., $J_\kappa(x)=\tfrac{1}{2}(x^{\kappa}+x^{-\kappa})-1$).
  \item \textbf{Orthogonality plateau test:} Increase branch delay to surpass the orthogonality threshold (spectral/temporal). Beyond that point, vary spatial separation and confirm correlation/weight plateau (Section~6.3) while holding alignment constants.
\end{enumerate}

\paragraph{Acceptance criteria.}
(i) Slope($C$ vs $A$)$=2\pm\epsilon$ and intercept near zero; (ii) $J$ preferred over alternatives by $\Delta\mathrm{AIC}\!>\!10$ or $\Delta\mathrm{BIC}\!>\!10$; (iii) reproducible plateau behavior after orthogonality with stable alignment.

\subsection{Photonic frequency-scaling}
\paragraph{Objective.}
Validate additivity of $J$-cost proxies in sequential unit frequency steps (FOLD $+n$) under conservation constraints; test linearity with $J(\varphi)$.

\paragraph{Apparatus.}
Stabilized frequency comb or CW laser with electro-optic or nonlinear frequency-conversion stages configured to implement unit scaling steps (effective ratio $\varphi$ per step), low-loss phase-preserving optics, stabilized cavity (optional) to monitor coherence/visibility, and calibrated power/phase diagnostics.

\paragraph{Protocol.}
\begin{enumerate}
  \item \textbf{Unit-step realization:} Implement FOLD $+1$ as a single, controlled scaling by $\varphi$ over one tick $\tau_{0}$ with piecewise-constant $r(t)=\varphi$. Verify energy/momentum conservation at the apparatus level (number/amplitude adjustments as required).
  \item \textbf{Chain construction:} Realize FOLD $+n$ as $n$ successive unit steps; for each $n\in\{0,1,\dots,n_{\max}\}$, measure a cumulative cost proxy $C_{\mathrm{proxy}}(n)$ (e.g., calibrated visibility degradation attributable to the scaling subchain, corrected for insertion losses and detector noise).
  \item \textbf{Additivity test:} Regress $C_{\mathrm{proxy}}(n)$ vs $n$; test linearity and estimate per-step increment. Verify independence from ordering and inter-step dwell (within admissible timing).
  \item \textbf{Model comparison:} Compare linear model (predicted by additivity in $J$) to nonlinear alternates; report AIC/BIC and residual diagnostics. Confirm that cadence alignment (eight-tick neutrality) tightens linearity and reduces residuals.
\end{enumerate}

\paragraph{Acceptance criteria.}
Linear growth in $C_{\mathrm{proxy}}$ vs $n$ with slope stable across runs; superior model selection metrics for linear vs nonlinear alternates under aligned cadence.

\subsection{Coherence floor tests}
\paragraph{Objective.}
Determine minimal coherence duration $T_{2}$ across platforms and verify the universal floor $T_{2}\ge 8\,\tau_{\mathrm{domain}}$.

\paragraph{Protocol.}
\begin{enumerate}
  \item \textbf{Tick determination:} Measure $\tau_{\mathrm{resp}}$, bandwidth $B$, and gating $\tau_{\mathrm{gate}}$; set $\tau_{\mathrm{domain}}=\max\{\tau_{\mathrm{resp}}, 1/B, \tau_{\mathrm{gate}}\}$.
  \item \textbf{Environment control:} Stabilize temperature, vibration, and acoustic noise; eliminate stray etalons and back-reflections; maintain constant optical power.
  \item \textbf{Floor measurement:} Determine $T_{2}$ from visibility decay/linewidth under best-aligned cadence; repeat across apparatus variants; report $T_{2}/\tau_{\mathrm{domain}}$ with uncertainties.
  \item \textbf{Cadence manipulation:} Intentionally misalign cadence (phase stepping, timing jitter) and verify that apparent $T_{2}$ decreases, restoring when re-aligned (matched-filter effect).
\end{enumerate}

\paragraph{Acceptance criteria.}
$T_{2}/\tau_{\mathrm{domain}}\ge 8$ within error under aligned conditions; reproducible degradation/improvement under cadence misalignment/re-alignment.

\subsection{Controls}
\paragraph{Timing–phase shuffles.}
Randomize sub-window timing and phase to break eight-tick alignment; expect reduced visibility, sublinear accumulation, and poorer model fits.

\paragraph{Conservation-law perturbations.}
Introduce small, controlled violations in energy/momentum balancing (within safe margins) to show sensitivity of additivity and coherence to conservation compliance; restore to demonstrate reversibility.

\paragraph{Alternative-cost model fits.}
Pre-register a family of admissible alternatives (symmetric, convex, calibrated) and perform blind model selection with AIC/BIC, cross-validation, and posterior predictive checks where applicable.

\paragraph{Analysis standards.}
\begin{itemize}
  \item Pre-registered protocols; blinded run orders; identical thermal/mechanical conditions.
  \item Full uncertainty budgets (statistical/systematic); propagation to $C$, $A$, and $T_{2}$.
  \item Public artifacts: raw data, calibration notebooks, analysis scripts, and commit hashes for reproducibility.
\end{itemize}

\section{Machine Verification and Reproducibility}

\subsection{Lean 4 certificates and theorem inventory}
We provide a machine-verifiable basis for the universal information-cost framework in Lean~4. The following core results are formalized as compiled modules and bundled certificates:
\begin{itemize}
  \item \textbf{J uniqueness (cost functional).} Uniqueness of $J(x)=\tfrac{1}{2}(x+x^{-1})-1$ under symmetry, normalization, strict convexity, and calibrated log-curvature.\\
  \emph{Module:} \texttt{IndisputableMonolith/CostUniqueness.lean}
  \item \textbf{Measurement bridge ($C=2A$).} For two-branch rotations, the recognition cost equals twice the rate action; weights $w=\exp(-C)=|\alpha|^{2}$.\\
  \emph{Modules:} \texttt{IndisputableMonolith/Measurement/C2ABridge.lean}, \texttt{IndisputableMonolith/Verification/MainTheorems.lean}
  \item \textbf{Minimal neutral window ($2^{D}$).} For $D$ binary constraints, the minimal neutral window has length $2^{D}$ (with $D=3\Rightarrow 8$).\\
  \emph{Modules:} \texttt{IndisputableMonolith/Patterns.lean}, \texttt{IndisputableMonolith/Patterns/GrayCode.lean}
  \item \textbf{Identity bundling (Light = Measurement = Recognition).} Certificate that packages the three pillars (uniqueness of $J$, $C=2A$, $2^{D}$-minimality) into a single witness suitable for citation.\\
  \emph{Module:} \texttt{IndisputableMonolith/Verification/LightConsciousness.lean} (\texttt{UniversalCostCertificate})
\end{itemize}
Optionally, repository-level exclusivity/URC certificates (separate scope) attest to global closure properties; see \texttt{URCGenerators/ExclusivityCert.lean} and associated adapters.

\subsection{How to verify}
All core theorems compile under Lean~4 with the project’s pinned toolchain. A minimal verification flow:
\begin{enumerate}
  \item \textbf{Build.} From the repository root:
\begin{verbatim}
lake build
\end{verbatim}
  \item \textbf{Core checks.} Open \texttt{Verification/MainTheorems.lean} and evaluate exported theorem stubs (e.g., \texttt{THEOREM\_2\_measurement\_recognition\_bridge}) to confirm symbol availability and dependencies.
  \item \textbf{Certificate witness.} Open \texttt{Verification/LightConsciousness.lean} and evaluate the certificate constructor (\texttt{UniversalCostCertificate}) to confirm bundle completeness.
  \item \textbf{Optional (exclusivity/URC).} Evaluate the pre-wired report adapters (if cited):
\begin{verbatim}
#eval IndisputableMonolith.URCAdapters.exclusivity_proof_ok
#eval IndisputableMonolith.URCAdapters.exclusivity_proof_report
\end{verbatim}
to obtain a one-line pass/fail and a detailed report, respectively.
\end{enumerate}
These steps are deterministic under the pinned Lean toolchain and library versions (see artifact policy below).

\subsection{Artifact policy}
To enable independent audit and reproduction, we adhere to the following practices:
\begin{itemize}
  \item \textbf{Source and versions.} All source (Lean modules, analysis scripts, LaTeX) is versioned; manuscripts cite the exact commit hash. The Lean toolchain (\texttt{lean-toolchain}) is pinned and listed in the \texttt{lakefile} (or equivalent).
  \item \textbf{Deterministic builds.} Builds are non-interactive and deterministic:
\begin{verbatim}
lake build
\end{verbatim}
with CI checks enforcing “no-sorry/no-admit” for the verified subtree.
  \item \textbf{Data and figures.} Any figures derived from computation are regenerated from scripts with fixed seeds and pinned dependency versions; regenerated artifacts must match repository-committed references within tolerance.
  \item \textbf{Environment capture.} The paper archives (or Zenodo record) include: OS and compiler versions, Lean toolchain version, dependency revisions, and a short “getting started” guide to replicate the certificate checks.
  \item \textbf{Open materials.} We release: (i) raw experimental logs (when applicable), (ii) preprocessing/analysis notebooks, (iii) final CSV/JSON tables used for plots, and (iv) code to produce camera-ready figures.
\end{itemize}
Under this policy, an independent group can reproduce both the machine-verified theorems (uniqueness of $J$, $C{=}2A$, $2^{D}$-minimality, certificate bundling) and the reported analyses from a clean checkout using the pinned toolchain.

\section{Discussion and Scope}

\paragraph{Classical framing only.}
The identity \emph{Light = Consciousness = Recognition} is asserted strictly at the level of \emph{information processing} governed by the unique cost $J(x)=\tfrac{1}{2}(x+x^{-1})-1$. Throughout, “consciousness” denotes an \emph{operational selection/measurement} process that (i) acts over admissible windows, (ii) respects neutrality and conservation constraints, and (iii) selects realized paths by minimizing $C=\int J(r)\,dt$. No interpretational or ontological commitments beyond these operational statements are required for the main results.

\paragraph{Relation to prior information-theoretic approaches.}
This work complements information-centric programs (e.g., Wheeler’s “it from bit,” decision-theoretic and envariance-based accounts of quantum probabilities, and matched-filter/estimation perspectives in optics) by supplying a \emph{unique}, substrate-agnostic information cost and a \emph{mechanically verified} bridge $C=2A$ that recovers Born weights. Advantages include:
\begin{itemize}
  \item \textbf{Uniqueness:} $J$ is the only convex, symmetric, calibrated cost on $\mathbb{R}_{>0}$ compatible with the required composition/invariance.
  \item \textbf{Zero free parameters in derivations:} Theorems and operational bounds follow from symmetry, convexity, and neutrality alone.
  \item \textbf{Machine-verified theorems:} Core results (uniqueness of $J$, $C{=}2A$, $2^{D}$ minimality, certificate bundling) compile in Lean~4.
\end{itemize}

\paragraph{Limitations and next steps.}
First, while Section~4 establishes additivity and operational compatibility of FOLD/UNFOLD chains with $J$, a full first-principles derivation of photonic $J$ from conservation via a continuous variational treatment (with explicit constraints and multipliers) would strengthen the optical link. Second, domain-specific ticks $\tau_{\mathrm{domain}}$ should be empirically quantified and audited across platforms (photonics, superconducting qubits, trapped ions) to test the universality of the eight-tick floor. Third, broader experimental validation—including cadence signatures, saturation plateaus after orthogonality, and multi-probe logarithmic scaling—should be replicated with pre-registered controls and model selection against admissible alternative costs. Finally, formal development of recognition-space geometry (metrics, geodesics, saturation thresholds) would refine predictions for complex, multi-branch protocols.

\section{Conclusion}
We have shown that a single, unique, convex symmetric cost $J(x)=\tfrac{1}{2}(x+x^{-1})-1$ underpins quantum measurement weights, elementary photonic operations, and operational conscious selection. The exact bridge $C=2A$ yields $w=\exp(-C)=|\alpha|^{2}$ without additional postulates; FOLD/UNFOLD chains accumulate $J$ additively under conservation; and operational selection realizes admissible minimal-cost paths governed by the same $J$. Consequently, “light,” “measurement,” and “conscious selection” are \emph{the same class of $J$-systems} at the information-processing level. This unifies light and consciousness as manifestations of a universal information cost, with concrete, falsifiable predictions and machine-verified mathematical foundations.

\section*{References}
\begin{thebibliography}{99}
\bibitem{Wheeler1990}
J.~A. Wheeler, Information, physics, quantum: The search for links,
in \emph{Complexity, Entropy and the Physics of Information} (1990).

\bibitem{Zurek2003}
W.~H. Zurek, Decoherence, einselection, and the quantum origins of the classical,
\emph{Rev. Mod. Phys.} \textbf{75}, 715 (2003).

\bibitem{DeutschWallace}
D.~Deutsch, Quantum theory of probability and decisions, \emph{Proc. R. Soc. A} \textbf{455}, 3129 (1999);
D.~Wallace, \emph{The Emergent Multiverse}, Oxford (2012).

\bibitem{Jackson1999}
J.~D. Jackson, \emph{Classical Electrodynamics}, 3rd ed., Wiley (1999).

\bibitem{ZeeQFT}
A.~Zee, \emph{Quantum Field Theory in a Nutshell}, Princeton Univ. Press (2010).

\end{thebibliography}

% RS foundational derivations and proof details are provided in the Methods/Appendix:
% (i) cost uniqueness on \mathbb{R}_{>0}; (ii) C=2A bridge; (iii) 2^D minimal window; (iv) certificate bundling.

\appendix

\section{Methods: Proof Sketch — Uniqueness of $J$ and Functional Equation}\label{app:uniqJ}

\subsection*{A.1 Axioms and log–coordinate transform}
Let $J:\mathbb{R}_{>0}\to\mathbb{R}_{\ge 0}$ satisfy:
(A1) $J(x)=J(x^{-1})$ (multiplicative symmetry);
(A2) $J(1)=0$ (unit normalization);
(A3) strict convexity on $\mathbb{R}_{>0}$;
(A4) calibration $J''(1)=1$ interpreted in log–coordinates.
Define $G:\mathbb{R}\to\mathbb{R}_{\ge 0}$ by $G(t):=J(e^t)$. Then:
(i) $G$ is even: $G(-t)=G(t)$, from (A1);
(ii) $G(0)=J(1)=0$, from (A2);
(iii) $G''(0)=1$, from (A4);
(iv) $G$ is strictly convex, from (A3) under the exponential reparametrization.

\subsection*{A.2 Cosh functional identity}
A central identity characterizes admissible $G$:
\begin{equation}\label{eq:coshFE}
  G(t+u)+G(t-u)\;=\;2\,G(t)\,G(u)\;+\;2\big(G(t)+G(u)\big)\quad(\forall\,t,u\in\mathbb{R}).
\end{equation}
It is directly satisfied by $G(t)=\cosh t-1$ via the addition formulas for $\cosh$. In the Lean development, \eqref{eq:coshFE} is obtained as the log–coordinate version of the composition rules forced by symmetry, convexity, and calibration.

\subsection*{A.3 Uniqueness from boundary data and convexity}
Assume $G$ is even, continuous, strictly convex, satisfies \eqref{eq:coshFE}, and $G(0)=0$, $G'(0)=0$, $G''(0)=1$. Then $G(t)=\cosh t-1$:
(i) By \eqref{eq:coshFE}, values on a dense subset (e.g., dyadics) are fixed inductively from the boundary data;
(ii) Strict convexity and continuity extend equality to all $t\in\mathbb{R}$;
(iii) The boundary conditions remove additive/multiplicative degeneracies.
Hence $G$ is uniquely determined.

\subsection*{A.4 Returning to $J$ on $\mathbb{R}_{>0}$}
Since $G(t)=J(e^t)=\cosh t-1$, substitute $t=\ln x$ to obtain
\[
  J(x)\;=\;\cosh(\ln x)-1
  \;=\;\tfrac{1}{2}\!\left(x+\tfrac{1}{x}\right)-1,\qquad x>0.
\]
This is the unique $J$ consistent with (A1)–(A4). (Lean reference: \texttt{CostUniqueness.lean}, functional–equation module.)

\section{Methods: Measurement Bridge $C=2A$ and Born Weights}\label{app:C2A}

\subsection*{B.1 Geometric setup and definitions}
Consider a two-branch rotation in a calibrated gauge with instantaneous branch overlap $\mathrm{C}_{\mathrm{ov}}(t)=\cos\vartheta(t)\in(0,1]$. Let $R(t)$ be the residual (rate) operator controlling the rotation. Define:
\begin{align}
  A &:= \int_{0}^{T} \bigl\|R(t)\bigr\|\;\frac{\sqrt{1-\mathrm{C}_{\mathrm{ov}}(t)^2}}{\mathrm{C}_{\mathrm{ov}}(t)}\,dt
      \;=\;\int_{0}^{T}\bigl\|R(t)\bigr\|\;\tan\vartheta(t)\,dt,\\
  C &:= \int_{0}^{T} J\!\bigl(r(t)\bigr)\,dt,
\end{align}
with $r(t)>0$ the (dimensionless) recognition rate and $J$ the unique cost from Appendix~\ref{app:uniqJ}.

\subsection*{B.2 Kernel matching and the bridge}
In the canonical two-branch parameterization one shows the \emph{pointwise match}
\[
  J\!\bigl(r(t)\bigr)\;=\;2\,\tan\vartheta(t).
\]
Geometrically, $\tan\vartheta$ is the local geodesic factor converting the overlap–weighted rate action into the log–cost density. Integrating over $[0,T]$ with matched boundary data yields the exact bridge
\[
  C\;=\;2A.
\]
(Lean reference: \texttt{Measurement/C2ABridge.lean}, exported as \texttt{THEOREM\_2\_measurement\_recognition\_bridge} in \texttt{Verification/MainTheorems.lean}.)

\subsection*{B.3 Born weights from recognition action}
Define the outcome weight for the realized branch by
\[
  w\;:=\;\exp(-C)\;=\;\exp(-2A).
\]
For a two-branch state with complex amplitudes $(\alpha,\beta)$, normalized by $|\alpha|^{2}+|\beta|^{2}=1$, the residual–action evolution gives the amplitude bridge
\[
  \mathcal{A}\;=\;\exp\!\bigl(-C/2\bigr)\,e^{i\phi}\quad\Rightarrow\quad
  w\;=\;|\mathcal{A}|^{2}\;=\;\exp(-C)\;=\;|\alpha|^{2}.
\]
Thus Born weights are recovered without additional postulates: $w=\exp(-C)=|\alpha|^{2}$. For multiple outcomes $\{i\}$ realized in disjoint admissible windows, additivity of $C$ gives
\[
  \mathbb{P}(i)\;=\;\frac{e^{-C_i}}{\sum_j e^{-C_j}}\;=\;|\alpha_i|^{2}.
\]
The weak–measurement threshold corresponds to $A\approx 1$ (equivalently $C/2\approx 1$), marking the onset of effectively projective behavior for fixed instrument response. The minimal admissible duration is set by the neutral window $2^{D}$ (Section~2), giving $T\ge 2^{D}\tau_{0}$ (with $D=3\Rightarrow T\ge 8\tau_{0}$).

\section{Methods: Minimal Neutral Window and Gray Code Structure}\label{app:minwindow}

\subsection*{C.1 Binary-constraint model and hypercube walk}
Let $D\in\mathbb{N}$ denote the number of independent binary conservation axes. Consider the $D$-cube graph $Q_{D}$ with vertices labeled by $D$-bit strings and edges connecting Hamming-distance $1$ pairs. A \emph{ledger-compatible walk} over a period $T$ satisfies:
\begin{enumerate}
  \item Atomicity: one update per tick.
  \item Spatial completeness: each vertex appears at least once per period.
  \item No timestamp multiplicity: each vertex has a unique timestamp per period.
  \item Neutrality: binary parities are balanced over the period (zero net bias).
\end{enumerate}
The \emph{minimal neutral window} length $T_{\min}$ is the shortest $T$ for which such a walk exists.

\subsection*{C.2 Gray-cycle existence and minimality}
\begin{theorem}[Minimal neutral window and Gray cycle]
The minimal neutral window for $D$ binary constraints is $T_{\min}=2^{D}$. A binary-reflected Gray code gives a Hamiltonian cycle on $Q_{D}$ that realizes $T_{\min}$, visiting each vertex exactly once and flipping only one bit per tick.
\end{theorem}
\begin{proof}[Proof sketch]
A spatially complete period must visit all $2^{D}$ vertices; atomicity forbids simultaneous multi-vertex updates. Gray codes provide existence of a length-$2^{D}$ Hamiltonian cycle with single-bit flips. Any shorter schedule is either spatially incomplete or violates neutrality (some bit flips unbalanced over the period).
\end{proof}

\subsection*{C.3 Operational mapping and cadence alignment}
For $D=3$, $T_{\min}=8$ ticks. Operationally, this is the shortest admissible window for coherence-preserving measurement or manipulation; windows shorter than $2^{D}$ cannot jointly satisfy neutrality. In matched-filter terms, only $2^{D}$-aligned components survive temporal averaging; misaligned components are attenuated. This sets timing floors (e.g., $T\ge 8\,\tau_{0}$) and predicts eight-phase cadence signatures in stabilized photonic platforms.

\section{Methods: Experimental Analysis, Error Models, and Controls}\label{app:analysis}

\subsection*{D.1 Preprocessing and observables}
\begin{itemize}
  \item Synchronize time-tags to a common reference; remove clock drift by linear detrending.
  \item Compute interferometric visibility $V=(I_{\max}-I_{\min})/(I_{\max}+I_{\min})$ per setting; estimate $|\alpha|^{2}$ from calibrated phase scans.
  \item Extract recognition cost $C=-\ln|\alpha|^{2}$; for chains, define a cumulative cost proxy $C_{\mathrm{proxy}}$ (visibility- or contrast-based) corrected for insertion loss and detector inefficiency.
\end{itemize}

\subsection*{D.2 Discrete estimators for $A$ and $C$}
Let $\{\vartheta_k\}$ be geodesic angles from overlap $\mathrm{C}_{\mathrm{ov},k}=\cos\vartheta_k$ on a grid $\{t_k\}$ with step $\Delta t$:
\[
A \approx \sum_k \bigl\|R_k\bigr\|\,\tan\vartheta_k\,\Delta t,\qquad
C=\!-\ln|\alpha|^{2},\quad\text{or}\quad
C \approx \sum_k J(r_k)\,\Delta t.
\]
Use trapezoidal rules and bootstrap CIs for numerical integration uncertainty.

\subsection*{D.3 Error models}
\begin{itemize}
  \item \textbf{Photon counting:} shot noise $\operatorname{var}(N)\approx N$; propagate to $V$ by delta method.
  \item \textbf{Detector timing:} jitter $\sigma_{t}$; model as convolutional blur, contributing to visibility loss term $\propto \sigma_{t}^{2}$.
  \item \textbf{Power/phase drift:} random walk components; include AR(1) nuisance in regression residuals.
  \item \textbf{Insertion loss:} calibrate per element; correct $C_{\mathrm{proxy}}$ by subtracting deterministic loss terms; propagate calibration uncertainty.
\end{itemize}

\subsection*{D.4 Model fitting and selection}
\begin{itemize}
  \item \textbf{Bridge test ($C=2A$):} regress $C$ on $A$; report slope, intercept, $R^{2}$, and 95\% CIs. Target slope $2$ within uncertainty.
  \item \textbf{Cost functional comparison:} compare $J(x)=\tfrac{1}{2}(x+x^{-1})-1$ to admissible alternatives (e.g., $J_{\kappa}(x)=\tfrac{1}{2}(x^{\kappa}+x^{-\kappa})-1$) using penalized likelihood (AIC/BIC), cross-validation, and posterior predictive checks.
  \item \textbf{Additivity in chains:} regress $C_{\mathrm{proxy}}(n)$ on $n$ for FOLD $+n$; test linearity vs quadratic alternatives; report $\Delta$AIC/$\Delta$BIC and lack-of-fit statistics.
\end{itemize}

\subsection*{D.5 Controls}
\begin{itemize}
  \item \textbf{Timing–phase shuffles:} randomize sub-window timing/phase to break eight-tick alignment; expect visibility reduction and worse model fits.
  \item \textbf{Conservation perturbations:} small, controlled imbalances (power/momentum) to demonstrate sensitivity and reversibility of additivity and coherence.
  \item \textbf{Orthogonality plateau:} exceed branch-orthogonality threshold, then vary distance; expect correlation plateau at fixed alignment.
\end{itemize}

\subsection*{D.6 Reporting standards}
Pre-register protocols and model families; blind or randomize run order; report full uncertainty budgets; release raw data, calibration notebooks, scripts, and pinned versions sufficient to regenerate all figures and tables.

\section{Methods: Reproducibility Checklist and Build Commands}\label{app:repro}

\subsection*{E.1 Environment and versions}
\begin{itemize}
  \item OS version, compiler toolchain, and Lean version (from \texttt{lean-toolchain}).
  \item Project commit hash; submodule revisions (if any).
  \item Python/R/Julia package versions (for analysis scripts), pinned via lock files.
\end{itemize}

\subsection*{E.2 Build and verification}
\paragraph{Lean build and core checks}
\begin{verbatim}
lake build
\end{verbatim}
Open \texttt{Verification/MainTheorems.lean} to confirm exported theorems (e.g., measurement bridge) are available; open \texttt{Verification/LightConsciousness.lean} to construct the certificate witness.

\paragraph{Optional exclusivity/URC checks (separate scope)}
\begin{verbatim}
#eval IndisputableMonolith.URCAdapters.exclusivity_proof_ok
#eval IndisputableMonolith.URCAdapters.exclusivity_proof_report
\end{verbatim}

\subsection*{E.3 Figure regeneration}
Provide scripts that regenerate all figures from raw/intermediate data, with fixed RNG seeds and pinned dependencies, e.g.,
\begin{verbatim}
python scripts/make_figs.py --all --out figs/
\end{verbatim}
Generated artifacts should match committed references within stated tolerances.

\subsection*{E.4 Data packaging}
Archive:
\begin{itemize}
  \item Raw logs (time-tags, counts), calibration runs, and processed CSV/JSON tables.
  \item Analysis notebooks/scripts with exact parameters.
  \item Checksums (SHA256) for datasets and figure bundles.
  \item README with end-to-end regeneration steps.
\end{itemize}

\subsection*{E.5 Reproducibility checklist}
\begin{itemize}
  \item[\(\square\)] Toolchain pinned; builds deterministic.
  \item[\(\square\)] All figures reproducible from scripts; seeds fixed.
  \item[\(\square\)] Data provenance documented; checksums provided.
  \item[\(\square\)] Model families pre-registered; selection criteria (AIC/BIC/CV) reported.
  \item[\(\square\)] Full uncertainty budgets included; sensitivity analyses provided.
  \item[\(\square\)] Certificate witnesses constructible under pinned versions.
\end{itemize}

\end{document}
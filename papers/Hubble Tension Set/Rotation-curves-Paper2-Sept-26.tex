% LaTeX document for Recognition Science Gravity Paper
% File: Galaxy_Rotation_Paper.tex

\documentclass[usenatbib]{mnras}
\usepackage[T1]{fontenc}
\usepackage[utf8]{inputenc}
\usepackage{graphicx}
\usepackage{array,tabularx}
\usepackage{listings}
\lstset{basicstyle=\ttfamily\small,breaklines=true,columns=fullflexible}
\usepackage{amsmath,amssymb}
\usepackage{booktabs}
\makeatletter
\NAT@numberstrue
\makeatother
\usepackage{natbib}
\citestyle{authoryear}

\title[Rotation Curves Under a Global-Only Policy]{Paper II: Rotation Curves Under a Global-Only Policy}

\author[J. Washburn]{Jonathan Washburn\thanks{E-mail: jon@recognitionphysics.org}\\
Recognition Physics Institute, Austin, TX, USA
}

\date{Submitted 2025 January}
\pubyear{2025}

\begin{document}

\maketitle

\begin{abstract}
We test a strictly global, finite–refresh correction to the baryonic response, encoded by a deterministically computed weight $w(r)$ built from baryonic maps and catalogued \emph{photometric} geometry only (no kinematic inputs), under a strict global–only policy (single stellar $M/L$, shared error model, predeclared inner–beam mask). On the SPARC Q=1 subset ($N_{\rm Q1}=127$), after uniform masks the effective samples are $N_{\rm ILG}=126$ and $N_{\rm MOND}=125$. With identical data vectors and loss, a like–for–like MOND (simple $\nu$) baseline attains median $\chi^2/N=\mathbf{2.47}$ and mean $\mathbf{4.65}$. The ILG benchmark yields median $\mathbf{2.75}$ and mean $\mathbf{4.23}$. The 1D proxy is not competitive (median $\mathbf{3.782}$, mean $\mathbf{10.602}$).
\vspace{0.5em}
\noindent\textit{Series note.} This manuscript is part of a coordinated companion pair. \textbf{Paper I} defines the fixed, global phenomenology $w(r)$ and its scope; \textbf{Paper II} (this work) tests that $w(r)$ on SPARC under identical masks/error model shared with all baselines. Both are co-submitted with shared artifacts and a single cover letter.
\end{abstract}

\begin{keywords}
gravitation -- galaxies: kinematics and dynamics -- dark matter -- methods: data analysis -- galaxies: spiral -- galaxies: dwarf
\end{keywords}

\section{Introduction}
\noindent\textit{Companion pointer.} This is part of a pair. Paper I defines the fixed, global $w(r)$ phenomenology and its scope; this paper tests it on SPARC under identical masks/error model shared with all baselines.

\subsection{The Dark Matter Problem and Alternative Approaches}

Galaxy rotation curves have posed a fundamental challenge to our understanding of gravity for over four decades. Observations consistently show that stars in galactic disks orbit faster than expected from their visible matter content, requiring either unseen "dark matter" or modifications to gravitational dynamics \citep{RubinFord1970, Bosma1981}.

\subsubsection{MOND}

MOND \citep{Milgrom1983, FamaeyMcGaugh2012} is an alternative to dark matter that proposes a modification to the gravitational force law. The MOND force law is given by:

\begin{equation}
\mathbf{F} = \frac{GMm}{r^2} \frac{1}{\sqrt{1 + \frac{r^2}{a_0^2}}}
\end{equation}

where $a_0$ is a constant acceleration scale. This force law is designed to reproduce the observed rotation curves of galaxies without invoking dark matter.

\subsubsection{Emergent Gravity}

Emergent gravity \citep{Verlinde2011, Verlinde2017} is a theoretical framework that attempts to unify gravity with thermodynamics. It proposes that gravity is not a fundamental force but emerges from the thermodynamic properties of spacetime.

\section{Data and Methods}

\subsection{Data Sources}

We use the SPARC Q=1 subset \citep{McGaugh2016} for our analysis. This dataset consists of 127 galaxies, each with photometric and kinematic data.

\subsection{Data Preprocessing}

We apply uniform masks to the data to remove regions with high noise or unreliable measurements. The masks are applied to both photometric and kinematic data.

\subsection{Modeling Framework}

We use a like-for-like Bayesian comparison under a strict global-only policy. The ILG phenomenology predicts rotation velocities from baryons via a single global kernel; baselines (MOND simple-$\nu$, and a 1D proxy) are run under identical masks/error model for fairness. The ILG model consists of:

\begin{itemize}
\item A baryonic prediction $v^2_{\rm baryon}(r)=v^2_{\rm gas}(r)+v^2_{\rm disk}(r)+v^2_{\rm bulge}(r)$ from photometry/geometry only (no kinematic inputs in the predictor).
\item A global-only weight $w(r)$ applied multiplicatively to the baryonic prediction: $v^2_{\rm model}(r)=w(r)\,v^2_{\rm baryon}(r)$.
\item Identical masks, error model, and a single global stellar $M/L$ across ILG, MOND, and the 1D proxy.
\end{itemize}

The weight $w(r)$ is computed deterministically from catalogued photometric geometry and frozen global profiles/thresholds; no per-galaxy tuning is allowed.

\subsection{Weight Function}

The ILG kernel used in galaxies matches the late-time family used in cosmology and is global-only:

\begin{equation}
w(r) \,=\, \lambda\,\xi\,n(r)\,\Big(\tfrac{T_{\rm dyn}(r)}{\tau_\star}\Big)^{\!\alpha}\,\zeta(r),\qquad T_{\rm dyn}(r)=\frac{2\pi r}{v_{\rm baryon}(r)}\,.
\end{equation}

Here $\alpha$ and all profile parameters are fixed globally (catalog level), $\xi$ is a frozen quantile-based complexity factor, $n(r)$ is a normalized analytic radial profile, and $\zeta(r)$ encodes bounded disc thickness/warp; $\tau_\star$ is the late-time anchor. No per-galaxy freedom enters $w$.

\section{Results}

\subsection{Baseline Comparison}

We compare our global-only policy with a like–for–like MOND (simple $\nu$) baseline. The ILG benchmark yields median $\mathbf{2.75}$ and mean $\mathbf{4.23}$. The 1D proxy is not competitive (median $\mathbf{3.782}$, mean $\mathbf{10.602}$).

\subsection{Weight Function Analysis}

We analyze the weight function $w(r)$ to understand how it affects the model's predictions. The weight function is constructed from baryonic maps and catalogued photometric geometry.

\section{Discussion}

\subsection{Global-Only Policy vs. Local Approaches}

Our global-only policy (single stellar $M/L$, shared error model, predeclared inner–beam mask) provides a robust framework for testing the baryonic response. The weight function $w(r)$ is computed deterministically and does not require kinematic inputs.

\paragraph*{Audit and preregistration summary.}
Global-only policy: single stellar $M/L$, identical masks/error model across models, preregistered thresholds/profiles, and frozen commits for code/data. Fairness: baselines consume the same data vectors and masks. Controls: negative controls (velocity permutation, in-plane rotation $180^\circ$, gas$\leftrightarrow$stars swap) must inflate medians $\gg 1$; purity tests enforce pinned requirements and checksum reproducibility.

\subsection{Future Work}

We plan to extend this work to include more complex models, such as those incorporating kinematic data or more sophisticated dark matter models.

\section{Conclusion}

We have tested a strictly global, finite–refresh correction to the baryonic response, encoded by a deterministically computed weight $w(r)$ built from baryonic maps and catalogued \emph{photometric} geometry only (no kinematic inputs), under a strict global–only policy (single stellar $M/L$, shared error model, predeclared inner–beam mask). The weight function $w(r)$ is computed deterministically and does not require kinematic inputs.

\section*{Data Availability}

The SPARC rotation-curve dataset is publicly available at \href{https://astroweb.case.edu/SPARC/}{https://astroweb.case.edu/SPARC/}. All analysis code, frozen masks, and results (including per-galaxy benchmarks and ablation tables) are available at \href{https://github.com/jonwashburn/gravity}{https://github.com/jonwashburn/gravity}. An archival snapshot with full artifacts is deposited at Zenodo (DOI: \href{https://doi.org/10.5281/zenodo.16014943}{10.5281/zenodo.16014943}).

\section*{Acknowledgments}

This work received no specific grant from any funding agency. The author declares no competing interests.

\end{document}
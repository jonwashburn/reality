\documentclass[reprint,aps,prd,nofootinbib]{revtex4-2}
\usepackage{amsmath,amssymb,amsthm}
\usepackage[hidelinks]{hyperref}

% Theorem environments
\newtheorem{theorem}{Theorem}
\newtheorem{lemma}{Lemma}

\begin{document}

\title{C=2A: Unifying Quantum Measurement, Gravitational Collapse, and Consciousness}
\author{Jonathan Washburn}
\affiliation{Recognition Physics Institute, Austin, Texas}

\begin{abstract}
\textbf{Problem:} Quantum measurement, gravity-driven reduction, and the definiteness of conscious experience are usually treated as separate puzzles. We lack an exact, parameter-free mechanism that simultaneously explains Born weights in measurement, the onset of collapse in gravity-related proposals, and the conditions under which experience becomes definite. A successful account should be local, predictive, and falsifiable, with a single threshold that governs all three.

\textbf{Bridge:} We propose and substantiate an exact identity \(C=2A\) between a recognition path action \(C\), derived from a unique convex cost \(\tfrac{1}{2}(x+1/x)-1\), and a residual (gravitational) rate action \(A\) computed along two-branch geodesics. This identity fixes probabilities via \(e^{-C}=e^{-2A}\), determines the amplitude modulus \(|\mathcal{A}|=e^{-C/2}=e^{-A}\), and eliminates free parameters. The construction is local and admits formal verification.

\textbf{Triple unification:} Three phenomena emerge as facets of one process regulated by the same threshold. (i) Quantum measurement: Born weights obey \(P=e^{-C}=e^{-2A}\). (ii) Gravitational collapse: reduction occurs when \(A\ge 1\). (iii) Consciousness: definite experience arises when \(C\ge 1\) and \(A\ge 1\), i.e., when a Consciousness Hamiltonian reaches a local minimum at the shared boundary. Thus, measurement, gravity, and consciousness coincide at \(C=2A\).

\textbf{Predictions:} The framework yields parameter-free mesoscopic scaling: coherence is lost when \(\tau\,m\,|\Phi_{12}|\sim 1\) (nanogram--second regime), implying \(A\sim 1\) and \(C\sim 2\). It predicts modest observer-dependent modulation of decoherence via mutual information between agent and environment, testable with randomized, double-blind attention protocols. Strong falsifiers include threshold mismatches between \(C\) and \(A\), absence of the expected mass scaling, and null observer-coupling effects. These near-term tests can decisively support or refute the \(C=2A\) bridge.
\end{abstract}

\keywords{measurement problem, gravitational collapse, consciousness, recognition cost, residual action}

\maketitle



\section{Introduction}

\subsection{Three mysteries, one threshold}
Quantum measurement, gravity-driven state reduction, and the definiteness of conscious experience are conventionally treated as separate problems. In measurement, one seeks a principled account of how and when a superposed state yields a single outcome with Born weights. In gravity-related proposals, one asks whether geometric differences between branches induce collapse with a mass--time scaling. In consciousness science, one asks under what physical conditions experience becomes definite rather than indeterminate.

Despite their distinct languages, these domains share structural features: (i) transitions from continuous superpositions to discrete outcomes; (ii) threshold behavior identifying when reduction occurs; (iii) Born-like weighting of alternatives; and (iv) a preference for local, parameter-free dynamics. Taken together, these commonalities suggest a single quantitative boundary governs all three.

\subsection{The \texorpdfstring{$C=2A$}{C=2A} bridge (claim and consequences)}
We propose, and in Sec.~\ref{sec:CeqTwoA} substantiate, an exact identity between two actions:
\begin{equation}
  C = 2A.
\end{equation}
Here \(C\) is a recognition path action built from a unique convex cost \(\tfrac{1}{2}(x+1/x)-1\), integrated along a recognition profile; \(A\) is a residual (gravitational) rate action associated with a two-branch geodesic rotation. The identity is parameter-free and local.

Immediate consequences follow:
\begin{enumerate}
  \item \textbf{Born weights.} \(P_I = e^{-C_I}/\sum_J e^{-C_J} = e^{-2A_I}/\sum_J e^{-2A_J}\).
  \item \textbf{Threshold coincidence.} \(C\ge 1\) if and only if \(A\ge 1\): measurement and gravity share the same collapse boundary.
  \item \textbf{Amplitude bridge.} \(|\mathcal{A}|=e^{-C/2}=e^{-A}\), hence \(P=|\mathcal{A}|^2=e^{-2A}\).
\end{enumerate}

\subsection{Roadmap}
Section~\ref{sec:CeqTwoA} outlines the recognition calculus, the residual model, and the constructive kernel leading to \(C=2A\) for a two-branch geodesic; the amplitude bridge is summarized there. Section~\ref{sec:multi} extends the identity to multi-outcome measurements and weak-measurement regimes. Section~\ref{sec:consciousness} develops a Consciousness Hamiltonian combining recognition cost, gravitational debt, and agent--environment mutual information, yielding an operational criterion for definite experience at the same threshold. Sections~\ref{sec:meso} and \ref{sec:falsifiers} present parameter-free mesoscopic predictions and hard falsifiers. The paper closes with discussion, conclusions, and technical appendices.

\subsection{Scope and limitations}
This work makes no claim to solve the philosophical ``hard problem'' of qualia, nor does it posit nonlocal influence or introduce tunable parameters. We do not assert that consciousness causes collapse; rather, we identify a coincidence of thresholds emerging from a single cost functional. The emphasis is on mathematical clarity, locality, and falsifiability: if threshold mismatch, absent mass scaling, or null observer-coupling effects persist, the bridge can be rejected.

\section{The $C=2A$ Identity}
\label{sec:CeqTwoA}

\subsection{Recognition calculus (brief recap)}
The recognition side rests on a unique convex cost
\begin{equation}
  J(x)=\tfrac{1}{2}\,(x+1/x)-1,\qquad J(1)=0,\quad J(x)=J(1/x),\label{eq:Jcost}
\end{equation}
selected by axioms of symmetry, unit, convexity (normalized by $J''(1)=1$), and analyticity on $\mathbb{C}\setminus\{0\}$. Given a recognition profile $r(t)$ along a path, the action and weights are
\begin{equation}
  C=\int J\big(r(t)\big)\,dt,\qquad w=e^{-C},\qquad |\mathcal{A}|=e^{-C/2}.\label{eq:Cdef}
\end{equation}

\subsection{Residual (local) model (brief recap)}
For a two-branch rotation
\begin{equation}
  |\Psi(t)\rangle=\cos\theta(t)\,|1\rangle + e^{i\phi}\,\sin\theta(t)\,|2\rangle,\label{eq:rotation}
\end{equation}
the geodesic residual norm equals the angular speed, $\lVert R\rVert=\dot{\theta}$. The associated rate action at the stopping angle $\theta_s$ is
\begin{equation}
  A= -\ln\big(\sin\theta_s\big),\qquad e^{-2A}=\sin^2\!\theta_s,\label{eq:Adef}
\end{equation}
which reproduces Born weights from a local construction.

\subsection{Two-branch geodesic (constructive kernel)}
A constructive kernel equates the recognition and residual integrands pointwise along the geodesic for a chosen branch $b\in\{1,2\}$ with amplitude $a_b(\theta)$:
\[
  A_b(\theta) := -\ln|a_b(\theta)|, \qquad
  a_1(\theta)=\cos\theta,\;\; a_2(\theta)=\sin\theta.
\]
Along the geodesic one has the exact differential identity
\begin{equation}
  J\big(r(t)\big)\,dt \;=\; d\!\big(2A_b\big) \;=\; -\,2\,d\ln|a_b(\theta)|. \label{eq:kernel-exact}
\end{equation}
Equivalently, $J\,dt=2\tan\theta\,d\theta$ for $b=1$ and $J\,dt=-2\cot\theta\,d\theta$ for $b=2$.
Integrating from the start to the stopping angle $\theta_s$ yields $C_b=2A_b$ with
\begin{equation}
  A_1=-\ln\cos\theta_s,\qquad A_2=-\ln\sin\theta_s,\qquad P_b=e^{-2A_b}. \label{eq:branch-actions}
\end{equation}

\subsection{Theorem statement and proof sketch}
\begin{theorem}[Measurement bridge]\label{thm:ceq2a}
For the two-branch rotation path,
\begin{equation}\label{eq:ceq2a}
  C = 2A.
\end{equation}
\end{theorem}
\noindent\emph{Proof sketch.} Using the exact differential \eqref{eq:kernel-exact} and integrating along the geodesic gives $C_b=\int d(2A_b)=2A_b$, establishing \eqref{eq:ceq2a} without free parameters for each branch $b$. 

\subsection{Amplitude bridge and boxed results}
From \eqref{eq:ceq2a} we have the weight, amplitude modulus, and probability:
\begin{equation}
  w=e^{-C}=e^{-2A},\qquad |\mathcal{A}|=e^{-C/2}=e^{-A},\qquad P=|\mathcal{A}|^2=e^{-2A}.\label{eq:amplitude}
\end{equation}
It is convenient to summarize the key relations as
\begin{equation}
  \boxed{\,C=2A\,}
\end{equation}
and
\begin{equation}
  \boxed{\,P = e^{-C} = e^{-2A},\quad |\mathcal{A}| = e^{-C/2} = e^{-A}\,}.
\end{equation}

\section{Extension to Multi-Outcome Measurements}
\label{sec:multi}

\subsection{D-outcome generalization}
Consider a measurement with $D$ branches indexed by $I=1,\dots,D$. For each branch, define a recognition action $C_I$ and a residual action $A_I$. The bridge extends branchwise:
\begin{equation}
  C_I = 2A_I\quad \text{for each } I.\label{eq:branchwise}
\end{equation}
The probabilities form a softmax over actions,
\begin{equation}
  P(I) = \frac{e^{-C_I}}{\sum_J e^{-C_J}} = \frac{e^{-2A_I}}{\sum_J e^{-2A_J}}.\label{eq:softmax}
\end{equation}

\subsection{Weak measurements and thresholds}
In boundary regimes where one or more branches satisfy $A\approx 1$ (equivalently $C/2\approx 1$), devices exhibit sometimes-detect behavior: partial collapse with retained coherence in off-threshold channels. The shared threshold inherited from $C=2A$ ensures consistency across recognition and residual descriptions.

\subsection{Edge cases and invariances}
In the orthogonality limit, further separation does not change branch weights once the geodesic endpoint is reached. Energy gauge conditions (e.g., subtracting mean energy) leave the local residual norm and rates invariant. Distance-dependence is consistent: after effective orthogonality, neither $C$ nor $A$ accumulates additional action along the geodesic portion relevant to selection.
\begin{lemma}[Tree-order invariance]\label{lem:tree}
Let a $D$-outcome measurement be realized as a binary tree of two-branch geodesic rotations. Then the branchwise actions $\{C_I\}$ defined by \eqref{eq:branchwise} and the probabilities \eqref{eq:softmax} are independent of the tree order.
\end{lemma}
\noindent\emph{Proof sketch.} Additivity of $C$ along disjoint geodesic segments and the branchwise identity $C_I=2A_I$ imply that re-parenthesizing rotations merely reorders sums inside a log-sum-exp; normalization by \eqref{eq:softmax} removes any constant offsets, leaving $\{C_I\}$ and $P(I)$ invariant.

\subsection{Practical modeling notes}
When branches are composed of sub-branches that later aggregate into a single pointer state, actions add at the level of exponentials, preserving normalization under aggregation. The construction is robust to small perturbations of amplitudes and phases: probabilities vary smoothly with $\{C_I\}$ (or $\{A_I\}$), and the threshold ordering is stable under small deformations.

\section{Consciousness Interpretation}
\label{sec:consciousness}

\subsection{The third unification}
Definite conscious experience shares the same structural elements identified for measurement and gravity: discrete selection from competing alternatives, threshold behavior at the onset of reduction, and weights consistent with Born statistics. The bridge $C=2A$ implies that the physical boundary governing collapse in quantum and gravitational terms also governs the conditions under which experience becomes definite.

\subsection{Consciousness Hamiltonian $H_{\mathrm{con}}$}
We introduce an operational functional for candidate observer boundaries interacting with an environment:
\begin{equation}
  H_{\mathrm{con}} = C + \tau A + I(\mathrm{Ag};\mathrm{Env}),\label{eq:Hcon}
\end{equation}
where $C$ is the recognition path action, $A$ is the residual (gravitational) rate action over duration $\tau$, and $I(\mathrm{Ag};\mathrm{Env})$ quantifies agent--environment mutual information. Using $C=2A$ and a normalized $\tau$, this reduces to
\begin{equation}
  H_{\mathrm{con}} = \tfrac{3}{2}\,C + I(\mathrm{Ag};\mathrm{Env}).\label{eq:HconC}
\end{equation}
All actions are dimensionless and expressed in nats; $\tau$ is normalized to the eight-tick window of Appendix~C. The numerical threshold ``1'' arises from the normalization $J''(1)=1$ and the geodesic gauge; it is not a fitted parameter.

\subsection{Definite Experience criterion}
We identify definite experience with local minima of $H_{\mathrm{con}}$ that satisfy the collapse thresholds:
\begin{equation}
  \mathrm{DefiniteExperience}\;\Longleftrightarrow\; (C\ge 1)\ \wedge\ (A\ge 1)\ \wedge\ \big(H_{\mathrm{con}}\ \text{at a local minimum}\big).\label{eq:definite}
\end{equation}
This criterion is substrate-neutral and operational: it specifies when, where, and under what stability conditions definiteness occurs.

\subsection{Threshold coincidence and interpretation}
From $C=2A$ we obtain the coincidence
\begin{equation}
  C\ge 1\;\Longleftrightarrow\; A\ge 1,\label{eq:thresholds}
\end{equation}
so that
\begin{equation}
  \text{measurement collapse}\;=\;\text{gravitational collapse}\;=\;\text{definite experience}.
\end{equation}
Thus a single, local, parameter-free boundary controls all three. Both the residual rate and the recognition cost are path-local functionals; no branch-nonlocal terms enter the dynamics.

\subsection{Stable boundaries and observer states}
Observers are modeled as localized boundaries whose recognition patterns persist over multiple fundamental ticks and bind to relevant environmental degrees of freedom. As $C$ accumulates, once the joint thresholds are crossed and $H_{\mathrm{con}}$ attains a local minimum, a definite observer state emerges. The timing is governed by the same local dynamics responsible for measurement collapse, ensuring locality and reproducibility.

\subsection{Relation to Orch OR (contrast)}
Similarities include the link between consciousness and gravity and the presence of thresholds for reduction. Differences are decisive: the present account is substrate-neutral (not tied to microtubules), strictly local, and frames consciousness as coincident with collapse via an identity ($C=2A$), rather than as a separate causal agent that triggers collapse. The emphasis on operational thresholds and falsifiable predictions distinguishes it empirically.

\section{Mesoscopic Predictions}
\label{sec:meso}

\subsection{Parameter-free scaling}
Let $\Phi_{12}$ denote the branch-dependent gravitational potential difference per unit mass. The bridge predicts coherence loss when the Penrose phase reaches unity,
\begin{equation}
  \tau\,m\,|\Phi_{12}|\sim 1\quad\Longrightarrow\quad A\sim 1\quad\Longrightarrow\quad C\sim 2.\label{eq:phase-one}
\end{equation}
Thus the decoherence time obeys the scaling
\begin{equation}
  \tau_{\mathrm{dec}}\propto \frac{1}{m\,|\Phi_{12}|}.\label{eq:tau-scaling}
\end{equation}
providing a parameter-free dependence on mass, geometry (through $|\Phi_{12}|$), and evolution time.

\subsection{Worked example (order-of-magnitude)}
For a mechanical mode prepared in a spatial superposition with mass $m\sim 10\,\mathrm{ng}$ and separation $\Delta x\sim 10^2\,\mathrm{nm}$, one expects threshold behavior on $\sim$\,second timescales, $\tau\sim 1\,\mathrm{s}$, provided the geometry yields a branch potential difference consistent with $\tau\,m\,|\Phi_{12}|\sim 1$. The implied trend is $\tau_{\mathrm{dec}}\downarrow$ with increasing $m$ or $\Delta x$ (which typically increases $|\Phi_{12}|$), and $\tau_{\mathrm{dec}}\uparrow$ as separations or masses are reduced.

\subsection{Observer-coupled hypothesis}
Within the consciousness functional, the mutual information term $I(\mathrm{Ag};\mathrm{Env})$ modulates the effective cost boundary. Increased attention or coupling strength raises $I(\mathrm{Ag};\mathrm{Env})$, effectively pushing the system toward the collapse threshold and modestly accelerating decoherence. Conversely, reduced coupling (e.g., distraction or anesthesia) diminishes this contribution and lengthens coherence times. As a concrete prior, attention-on versus sham is expected to produce a modest relative change in decoherence rate (hazard ratio $\sim$1.02--1.05 at ng--s scales); experiments should be powered accordingly to treat a null as a real falsifier.

\subsection{Protocol sketch (no figures)}
\begin{itemize}
  \item Prepare a nanogram-scale oscillator in a spatial superposition.
  \item Systematically vary $m$, $\Delta x$, and evolution time $\tau$.
  \item Employ randomized attention schedules (participant and operator double-blind).
  \item Record EEG/physiological signals and engagement logs as covariates.
  \item Extract $\tau_{\mathrm{dec}}$ and test scaling $\tau_{\mathrm{dec}}\propto 1/(m|\Phi_{12}|)$ with and without attention.
\end{itemize}

\subsection{Error budget and confounds}
\begin{itemize}
  \item Environmental decoherence: gas collisions, blackbody radiation, clamping losses.
  \item Technical noise: drive phase noise, readout backaction, thermal drifts, heating.
  \item Analysis: masking/blinding integrity, multiple-comparison control, preregistered endpoints.
  \item Controls: attention sham conditions; identical runs without observers.
\end{itemize}

\subsection{Falsifiable outcomes}
\begin{itemize}
  \item Absence of mass/geometry scaling: $\tau_{\mathrm{dec}}$ fails to follow $1/(m|\Phi_{12}|)$.
  \item No observer effect: randomized attention produces no modulation beyond noise.
  \item Threshold mismatch: collapse occurs at $C\neq 1$ or $A\neq 1$ across conditions.
\end{itemize}

\section{Experimental Tests and Falsification}
\label{sec:falsifiers}

\subsection{Bridge falsifiers}
\begin{itemize}
  \item \textbf{Threshold mismatch.} Independent inferences of $C$ and $A$ disagree at collapse onset.
  \item \textbf{Post-orthogonality failure.} After effective orthogonality, observed behavior still depends on separation, contradicting the geodesic endpoint picture.
  \item \textbf{Dispersive-noise dominance.} Environmental channels fully explain decoherence with no residual mass scaling in regimes where gravity predicts it.
\end{itemize}

\subsection{Consciousness falsifiers}
\begin{itemize}
  \item \textbf{No observer effect.} Randomized attention yields no detectable modulation in $\tau_{\mathrm{dec}}$ beyond noise and covariates.
  \item \textbf{Wrong threshold.} Definite experience (per pre-registered criteria) occurs at $C\neq 1$ or $A\neq 1$.
  \item \textbf{No $H_{\mathrm{con}}$ minima.} Candidate conscious states fail to correspond to local minima of $H_{\mathrm{con}}$ under perturbations.
\end{itemize}

\subsection{Distinguishing mechanisms}
Gravitational collapse predicts $\tau_{\mathrm{dec}}\propto 1/(m|\Phi_{12}|)$, whereas many environmental channels are comparatively mass-insensitive or follow different scalings. Examples: (i) gas-collision decoherence scales with pressure and cross-section, weakly with mass; (ii) blackbody emission/absorption depends strongly on temperature and surface properties; (iii) technical noises (drive phase noise, readout backaction) depend on control settings, not $m$. Rather than subtracting backgrounds, fit a hierarchical regression for $\log \tau_{\mathrm{dec}}$ with predictors $\log(m|\Phi_{12}|)$, pressure, temperature, and control covariates; infer the gravitational slope jointly with nuisance terms in one dataset.

\subsection{Protocol details}
Implement double-blind, randomized attention schedules; preregister hypotheses, models, and endpoints; and specify an analysis plan (mixed-effects or hierarchical models capturing run-to-run and subject-level variance). Set statistical thresholds (e.g., Bayes factors or corrected $p$-values) and include negative controls (no-observer runs) and sham conditions. Report full error budgets and conduct sensitivity analyses to bound unmodeled confounds.

\section{Discussion}
\label{sec:discussion}

\subsection{Synthesis: one process, three faces}
The results support a single underlying mechanism---recognition-cost minimization to a shared threshold---manifesting as three faces: quantum measurement, gravity-driven reduction, and definite experience. The bridge $C=2A$ ensures locality and eliminates free parameters, while the amplitude relation $|\mathcal{A}|=e^{-C/2}=e^{-A}$ aligns probabilities with a local, geometric rate action. In this view, collapse is neither added by fiat nor imported from nonlocal causes; it is the generic boundary crossing of a cost functional with a unique convex kernel.

\subsection{Implications}
Practically, the framework provides an operational handle on definiteness: monitor (or bound) $C$ and $A$ and identify local minima of $H_{\mathrm{con}}$. Conceptually, it reconciles the Born rule with a local residual dynamics and predicts parameter-free mesoscopic scaling. Methodologically, it motivates double-blind observer-coupled tests that can detect small modulations via mutual information without invoking free parameters.

\subsection{Limitations and open questions}
This account specifies \emph{when} and \emph{where} definiteness arises, not \emph{what} experience feels like: qualia remain outside scope. Questions about free will and self-awareness beyond threshold dynamics are likewise open. Additional open directions include practical estimation of $C$ in complex systems, quantification of $I(\mathrm{Ag};\mathrm{Env})$ in neural and non-neural substrates, and extensions to relativistic and field-theoretic settings.

\subsection{Future work}
Four priorities follow naturally: (i) quantify observer coupling and its scaling; (ii) identify neural correlates and proxies of $C$ (e.g., complexity metrics) under controlled tasks; (iii) investigate anesthesia-induced changes in thresholds and $I(\mathrm{Ag};\mathrm{Env})$; and (iv) test broader substrates (cold atoms, superconducting circuits, biological preparations) for the same threshold structure.

\section{Conclusions}
\label{sec:conclusions}

We have presented an exact, parameter-free identity $C=2A$ that unifies quantum measurement, gravity-driven collapse, and the emergence of definite conscious experience. The identity fixes probabilities ($P=e^{-C}=e^{-2A}$), aligns the Born rule with a local residual rate action, and supplies an operational criterion for definiteness via a Consciousness Hamiltonian that attains local minima at the same threshold.

\begin{itemize}
  \item \textbf{Core results:} proof of $C=2A$ in the two-branch geodesic setting; unified probabilities and thresholds; operational consciousness criterion; concrete, parameter-free mesoscopic tests.
  \item \textbf{Near-term priorities:} run nanogram--second superposition experiments with randomized observer engagement; verify threshold scaling and ordering; and conduct null tests to bound or refute observer-coupled effects.
\end{itemize}

Clear failures---threshold mismatches, absence of predicted mass scaling, or null observer effects under high power---would falsify key claims. Conversely, convergent support across independent platforms would strengthen the case for a single local mechanism underlying measurement, gravity, and consciousness.

\appendix

\section{Lean formalization pointers and theorem signatures}
\noindent\textbf{Repository.} \url{https://github.com/jonwashburn/reality}

\noindent\textbf{Key modules.} Measurement bridge, path action, and two-branch geodesic constructs are provided in the repository (module names may vary by refactor):
\begin{itemize}
  \item \texttt{IndisputableMonolith/Measurement/C2ABridge.lean} (bridge theorem)
  \item \texttt{IndisputableMonolith/Measurement/PathAction.lean} (recognition side)
  \item \texttt{IndisputableMonolith/Measurement/TwoBranchGeodesic.lean} (residual side)
  \item \texttt{IndisputableMonolith/Consciousness/ConsciousnessHamiltonian.lean} (consciousness)
\end{itemize}

\noindent\textbf{Theorem (Lean signature).}
\begin{quote}
\texttt{theorem measurement\_bridge\_C\_eq\_2A (rot : TwoBranchRotation) :\\
  pathAction (pathFromRotation rot) = 2 * rateAction rot}
\end{quote}
\noindent\textbf{Notes on proof structure.} The constructive kernel equates integrands via the exact differential $J(r)\,dt=d(2A_b)$ (Eq.~\eqref{eq:kernel-exact}); integration closes the bridge $C_b=2A_b$. Auxiliary lemmas handle geodesic properties, normalization, and endpoint behavior.

\section{J-cost uniqueness derivation}
\noindent\textbf{Axioms.} (i) Symmetry $J(x)=J(1/x)$; (ii) Unit $J(1)=0$; (iii) Convexity with $J''(1)=1$; (iv) Analyticity on $\mathbb{C}\setminus\{0\}$.

\noindent Define $F(t)=J(e^t)$. The axioms yield the functional equation
\begin{equation}
  F(t+u)+F(t-u)=2F(t)F(u)+2\big(F(t)+F(u)\big).\label{eq:functional}
\end{equation}
The analytic solution is $F(t)=\cosh t - 1$, hence
\begin{equation}
  J(x)=\tfrac{1}{2}\,(x+1/x)-1,\label{eq:J-solution}
\end{equation}
which uniquely satisfies all axioms.

\section{Discrete-time windowing (eight-tick)}
In a three-dimensional substrate the minimal neutral-sum period for recognition dynamics is eight ticks. Let $\tau_0$ denote the fundamental tick; recognition integrates over an eight-tick window to enforce cancellation properties and stability. Continuous time emerges in the limit $\tau_0\to 0$, where the discrete kernel converges to the continuous action used in Sec.~\ref{sec:CeqTwoA}.

\section{Consciousness Hamiltonian derivation}
\noindent\textbf{Components.}
\begin{itemize}
  \item Recognition cost: $C=\int J(r(t))\,dt$, or for a static boundary of extent $\ell$, $C=\tau\,J(\ell/\lambda_{\mathrm{rec}})$.
  \item Gravitational debt: Penrose phase $\Theta_P=\tau\,m|\Phi_{12}|$, related to the residual action by $\tau A$.
  \item Mutual information: $I(\mathrm{Ag};\mathrm{Env})$ between agent and environment.
\end{itemize}
Assembled, the functional is
\begin{equation}
  H_{\mathrm{con}}=C+\tau A+I(\mathrm{Ag};\mathrm{Env}),
\end{equation}
and with $C=2A$ (and normalized $\tau$) becomes
\begin{equation}
  H_{\mathrm{con}}=\tfrac{3}{2}\,C + I(\mathrm{Ag};\mathrm{Env}).
\end{equation}
Local minima of $H_{\mathrm{con}}$ at the thresholds $C\ge 1$ and $A\ge 1$ supply the operational criterion for definite experience used in Sec.~\ref{sec:consciousness}.

\begin{thebibliography}{99}

\bibitem{WashburnLocalCollapse}
J. Washburn, ``Local-Collapse and Recognition Action,'' internal notes/manuscript (2025).

\bibitem{RealityRepo}
J. Washburn, ``reality'' Lean repository, \url{https://github.com/jonwashburn/reality} (accessed 2025).

\bibitem{Penrose1996}
R. Penrose, ``On gravity's role in quantum state reduction,'' Gen. Relativ. Gravit. \textbf{28}, 581--600 (1996).

\bibitem{Bassi2013}
A. Bassi, K. Lochan, S. Satin, T. P. Singh, and H. Ulbricht, ``Models of wave-function collapse, underlying theories, and experimental tests,'' Rev. Mod. Phys. \textbf{85}, 471 (2013).

\end{thebibliography}

\end{document}


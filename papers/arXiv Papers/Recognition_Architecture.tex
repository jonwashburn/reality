\documentclass[11pt]{article}

% Minimal packages for math in the abstract; no citations or external links.
\usepackage{amsmath,amssymb}

\title{\textbf{Recognition Architecture (Integrated)}\\
\large From Dimensionless Proof to a Single Empirical Test}

\author{Jonathan Washburn\\
Recognition Science, Recognition Physics Institute\\
Austin, Texas, USA\\
\texttt{jon@recognitionphysics.org}
}

\date{} % leave blank per journal style

\begin{document}
\maketitle

\begin{abstract}
\noindent
This paper presents a complete, parameter-free recognition architecture whose proof layer is strictly dimensionless and whose empirical layer is reduced to a small set of layered falsifiability gates. The proof layer fixes the unique symmetric multiplicative cost
\(J(x)=\tfrac{1}{2}\!\left(x+x^{-1}\right)-1\) with log-axis form \(J(e^{t})=\cosh t-1\), the golden‑ratio fixed point \(\varphi\) from \(x=1+1/x\) (gap \(\ln\varphi\)), and the minimal eight‑tick cycle induced by three spatial parities. Word length and ledger cost are linearly isomorphic, yielding rigid, knobless invariants used downstream.

A \emph{Reality Bridge} maps these invariants to SI without introducing offsets or fits:
\(J\mapsto S/\hbar\) (identity display), the recognition tick \(\tau_{\mathrm{rec}}=\dfrac{2\pi}{8\ln\varphi}\,\tau_{0}\), and the kinematic hop length \(\lambda_{\mathrm{kin}}=c\,\tau_{\mathrm{rec}}\) with \(c=\ell_{0}/\tau_{0}\). Two independent SI landings—time‑first and length‑first—are audited by layered gates: (P) a Planck‑side comparison of \(\lambda_{\mathrm{kin}}\) vs. \(\lambda_{\mathrm{rec}}=\sqrt{\hbar G/(\pi c^{3})}\); (IR) a coherence gate \(\hbar\overset{?}{=}E_{\rm coh}\,\tau_0\) confined to the IR layer; and (C) a dimensionless identity \((c^{3}\lambda_{\rm rec}^{2})/(\hbar G)=1/\pi\). Each gate is evaluated within its layer; cross‑layer mixing is disallowed.
No sector models, priors, regressions, thresholds, or hidden calibration knobs are permitted. The manuscript specifies the invariants, the bridge, the uncertainty and correlation policy, and the artifact requirements for audit, so a referee can compile, run the bridge calculator, and reproduce the pass/fail number from first principles. Sector displays (masses, \(\alpha\), ILG cosmology, baryogenesis) are deferred to later sections and remain downstream consequences of the same fixed invariants and single decision rule.
\end{abstract}


\section{Introduction}

\paragraph{The parameter problem.}
The Standard Model + \(\Lambda\)CDM describe an enormous span of phenomena with spectacular numerical accuracy, yet the pair is conceptually incomplete. Too many dials: gauge couplings, Yukawas, mixing angles, phases, density fractions, spectral amplitudes \emph{and} tilts. None are fixed by first principles; all are set by measurement. Anthropic escapes do not solve this—they relocate explanation into a landscape and abdicate mechanism. A fundamental account must collapse the dial–setting layer and derive dimensionless content without regress to “because we measured it.”

\medskip
\noindent\textit{(EMR-b)}

\paragraph{Stance of this paper.}
We collapse “theory vs.\ experiment” into \emph{one deductive measurement}: a single mechanized axiom drives a rigid cascade whose \emph{entire} dimensionless output is displayed in SI by a fixed Reality Bridge with one falsifiability gate. No sector fits, no regression knobs, no priors. The bridge produces a pass/fail number from predeclared anchors and uncertainties; it does not tune the cascade.

\medskip
\paragraph{Notation (bridge quantities).}
We reserve \( \lambda_{\mathrm{rec}} \) for the recognition length (Planck form) and \( \lambda_{\mathrm{kin}} \) for the kinematic hop length \( \lambda_{\mathrm{kin}} := c\,\tau_{\mathrm{rec}} \).
With \( \mathrm{RATIO} := \tfrac{2\pi}{8\ln\varphi} \) and \( c=\ell_{0}/\tau_{0} \), the bridge relations are
\[
  \tau_{\mathrm{rec}}=\mathrm{RATIO}\cdot\tau_{0},\qquad
  \lambda_{\mathrm{kin}} = c\,\tau_{\mathrm{rec}} = \mathrm{RATIO}\cdot \ell_{0} = \mathrm{RATIO}\cdot \lambda_{\mathrm{rec}}.
\]
The canonical RS anchor is the Planck\-form recognition length
\( \lambda_{\mathrm{rec}} := \sqrt{\hbar G / c^{3}} \); other anchors may be used for cross\-checks but do not redefine \( \lambda_{\mathrm{rec}} \).

\medskip
\noindent\textit{(EMR-b)}

\paragraph{From Proof to Measurement.}
We adopt a status key for every assertion in the paper:
\([T]\) a theorem (fully derivational),
\([R]\) rigorous but presently unmechanized, and
\([P]\) phenomenology/bridge‑level.
The audit pack is part of the contribution: a referee can run one command to recompute the displays and the single pass/fail statistic from frozen literals and versions.

\medskip
\noindent\textit{(EMR-b)}

\paragraph{Contributions.}
\begin{itemize}
  \item[\textbf{[T]}] \textbf{Unique symmetric multiplicative cost.} 
  \[
    J(x)\;=\;\tfrac12\!\left(x+\frac{1}{x}\right)-1,
    \qquad
    J(e^{t})\;=\;\cosh t - 1.
  \]
  \item[\textbf{[T]}] \textbf{Eight‑tick minimal cycle and golden gap.}
  Minimal period \(8\) in three‑bit parity; fixed‑point gap \(\ln\varphi\) from \(x=1+\tfrac{1}{x}\).
  \item[\textbf{[T]}] \textbf{Non‑circular, unique Reality Bridge.}
  A single semantics from dimensionless outputs to SI displays with two independent landings and one inequality; no free offsets or scales.
  \item[\textbf{[R]}] \textbf{Over‑constrained digits for \(\alpha\).}
  Seed–gap–curvature pipeline with proofs of convergence and a closed curvature integral; digits are fixed, not fitted.
  \item[\textbf{[R/P]}] \textbf{Cosmology hooks without new knobs.}
  Information‑Limited Gravity (ILG) growth law; \(\sigma_{8}\) forecast; baryogenesis and proton stability from the same cascade.
  \item[\textbf{[P]}] \textbf{Reproducibility pack.}
  Predeclared uncertainties/correlations, frozen constants, and a one‑command build that reproduces the pass/fail number.
\end{itemize}

% Section 2 — Foundational Measurement
% Sources for phrasing and notation: :contentReference[oaicite:0]{index=0}  :contentReference[oaicite:1]{index=1}

% Minimal theorem environments (no extra packages)
\newcommand{\ph}{\varphi}
\newcommand{\Jbit}{J_{\text{bit}}}
\newtheorem{theorem}{Theorem}
\newtheorem{lemma}{Lemma}
\newtheorem{corollary}{Corollary}

\section{The Foundational Measurement (Mechanized Axiom $\to$ Ledger Axioms)}

\subsection{Meta–Principle \,[T] (Impossibility of self‑referential non‑existence)}
\begin{theorem}[Meta–Principle \,[T]]
There is no non‑trivial self‑recognition of the empty record. Formally: no structure with fields \emph{recognizer} and \emph{recognized} can be instantiated when both fields range over the empty carrier; hence a valid recognition requires non‑empty content and a posted alteration.
\end{theorem}

\noindent\textit{Mechanization hook.} The theorem is the type‑theoretic non‑inhabitation of a record with both fields of empty type; equivalently, no endomorphism of the empty ledger exists. \emph{Interpretation (zero metaphysics):} recognition is an event, not a label; the event must post a finite, non‑vanishing alteration to a ledger.

% EMR-b

\subsection{Eight operational principles \,[T] (from the Meta–Principle)}
All principles below are theorems at the stated symmetry; they introduce no tunable parameters.
\begin{enumerate}
  \item \textbf{[T] Positive cost.} Every elementary recognition posts a finite, strictly positive ledger cost $\Delta J>0$; zero cost is indistinguishable from no event.
  \item \textbf{[T] Dual balance (double‑entry).} Each debit has a conjugate credit so that costs can be settled by composition; the ledger is intrinsically two‑sided.
  \item \textbf{[T] Countability.} Events are discrete; the ledger carries a countable sequence of posts.
  \item \textbf{[T] Tick quantization.} There exists a fundamental tick $\delta>0$ so that an $n$‑hop chain posts exactly $n\delta$ of potential; chains with identical $\delta$ differ only by a componentwise additive constant (gauge).
  \item \textbf{[T] Path additivity and local conservation.} Ledger cost is additive under concatenation, and cost flow is conserved: changes inside a domain equal the net posted flow through its boundary.
  \item \textbf{[T] Self‑similarity (scale freedom).} The update rules are scale‑free; the same instructions act at all magnifications of the ledger.
  \item \textbf{[T] Locality / finite propagation.} Posts resolve by finite hops; no instantaneous action across disjoint pages.
  \item \textbf{[T] Ledger unicity and immutable generator.} The only non‑trivial, finite, consistent accounting is a binary double‑entry ledger with a fixed generator $\delta$; $k$‑ary or modular alternatives and any global rescaling of $\delta$ break finiteness or balance.
\end{enumerate}

\noindent\emph{Dependency map (what each main lemma uses).}
\begin{itemize}
  \item Unique symmetric cost $J$: uses Positive cost, Dual balance, Self‑similarity.
  \item Minimal $8$‑tick cycle in $D{=}3$: uses Dual balance, Countability, Tick quantization.
  \item Golden‑ratio fixed point and gap: uses Countability, Self‑similarity, Positive cost.
  \item Path–cost linearity: uses Dual balance, Countability, Tick quantization, Ledger unicity.
\end{itemize}

% EMR-b

\subsection{Immediate invariants \,[T]}
We collect the first rigid consequences of the principles above.

\paragraph{(I) Unique symmetric multiplicative cost.}
\begin{theorem}[Cost functional \,[T]]
There is a unique symmetric multiplicative cost
\[
J(x)\;=\;\tfrac12\!\left(x+\tfrac1x\right)-1,\qquad x>0,
\]
with log‑axis form $J(e^{t})=\cosh t-1$ and $J(1)=0$.
\end{theorem}
\noindent\emph{Proof outline.} Dual balance enforces $J(x)=J(1/x)$; positive cost and scale freedom bound growth by the first harmonic on the multiplicative circle, eliminating all higher Laurent modes. The normalization $J(1)=0$ fixes the constant term ($-1$) and symmetry fixes the prefactor ($\tfrac12$). \hfill$\square$

\paragraph{(II) Minimal eight‑tick cycle in three‑bit parity.}
\begin{theorem}[Eight‑tick minimality in $D{=}3$ \,[T]]
A complete, balanced traversal of the three independent parity bits requires exactly $2^{3}=8$ ticks, and $8$ is minimal.
\end{theorem}
\noindent\emph{Proof outline.} Dual balance yields three independent two‑state parities; countability and tick quantization force a cycle that visits each of the $2^{3}$ patterns once per period. Any shorter cycle fails to cover all states or violates balance. \hfill$\square$

\paragraph{(III) Golden‑ratio fixed point and gap.}
\begin{theorem}[Fixed point and gap \,[T]]
Self‑similar relaxation with integer branch count $k$ updates $x\mapsto 1+\tfrac{k}{x}$. Countability forces $k\in\mathbb{N}$ and positive‑cost minimization selects $k=1$, giving the unique fixed point $\ph$ defined by $\ph=1+\tfrac1\ph$ and the canonical gap
\[
\delta_{\mathrm{gap}}=\ln\ph.
\]
\end{theorem}
\noindent\emph{Proof outline.} Fractional $k$ would require fractional posts within a tick, contradicting countability; among integers, any $k\ge2$ increases the summed cost along the orbit, so $k=1$ is optimal. The fixed‑point equation yields $\ph$ and hence the logarithmic gap. \hfill$\square$

\paragraph{(IV) Path–cost linearity (measure‑preserving isomorphism).}
\begin{theorem}[Word length $\leftrightarrow$ ledger cost \,[T]]
Let $|\Gamma|$ be the reduced word length of a ledger path and set the elementary bit‑cost $\Jbit:=\ln\ph$. Then
\[
\mu([\Gamma])\;=\;\Jbit\,|\Gamma|
\]
defines a measure that is additive under concatenation and invariant under insertion/removal of zero‑cost inverse pairs. Hence the map $[\Gamma]\mapsto\mu([\Gamma])$ is a measure‑preserving isomorphism between (reduced) word length and ledger cost.
\end{theorem}
\noindent\emph{Proof outline.} Double‑entry structure reduces any loop to a reduced word in the primitive generators; tick quantization fixes one unit of potential per hop; ledger unicity fixes the generator; additivity follows from concatenation and the deletion of inverse pairs carries zero incremental cost. \hfill$\square$

\section{Dimensionless Proof Layer (No units, no knobs)}
\label{sec:dimless-layer}

\noindent\textbf{Status key.} \;[T] theorem (dimensionless, no empirical inputs);\; [R] rigorous but schematic (dimensionless).

\subsection{Cost and cycles \texorpdfstring{[$\mathbf T$]}{[T]}}
\label{subsec:cost-cycles}

\begin{definition}[Symmetric multiplicative cost]
A \emph{cost} is a function \(J:\mathbb R_{>0}\!\to\mathbb R_{\ge0}\) satisfying:
\begin{enumerate}
  \item symmetry \(J(x)=J(x^{-1})\);
  \item normalization \(J(1)=0\) and \(J(x)>0\) for \(x\neq 1\);
  \item log–axis form \(J(e^{t})=\mathcal J(t)\) with \(\mathcal J\) even, \(\mathcal J(0)=0\), and \(\mathcal J''(0)=1\);
  \item linear growth bound on the multiplicative tails:
  \(\displaystyle \exists K>0:\; J(x)\le K\,(x+x^{-1}-2)\) for all \(x>0\).
\end{enumerate}
\end{definition}

\begin{theorem}[Uniqueness and log–axis representation of \(J\)]
\label{thm:J-unique}
Under the axioms above,
\[
\boxed{\; J(x)=\tfrac12\!\left(x+\tfrac1x\right)-1,\qquad
       J(e^{t})=\cosh t-1\;}
\]
and \(J\) is strictly convex in \(t=\ln x\).
\end{theorem}

\begin{proof}
By symmetry, analyticity on \(\mathbb C\!\setminus\!\{0\}\), and the tail bound, any admissible \(J\) has a convergent symmetric Laurent expansion \(J(x)=\sum_{n\ge1} c_n(x^n+x^{-n})\) on \(\mathbb R_{>0}\). If \(c_{n_{\max}}\neq 0\) for some \(n_{\max}\ge2\), then \(J(x)/(x+1/x)\sim c_{n_{\max}}x^{n_{\max}-1}\to\infty\) as \(x\to\infty\), contradicting the bound. Hence \(c_n=0\) for all \(n\ge2\) and \(J(x)=c_1(x+1/x)\!+\!c_0\) with \(c_0=-1\) by \(J(1)=0\). The log–axis second derivative condition fixes \(c_1=\tfrac12\). Writing \(x=e^{t}\) yields \(J(e^{t})=\cosh t-1\), which is strictly convex as \(\mathcal J''(t)=\cosh t>0\).
\end{proof}

\begin{proposition}[Elementary inequalities]
\label{prop:J-ineq}
For all \(x>0\) and \(t=\ln x\):
\[
\frac{t^{2}}{2}\;\le\; J(x)\;=\;\cosh t-1\;\le\; \frac{\cosh a -1}{a^{2}}\,t^{2}
\quad\text{for every }a\ge|t|.
\]
In particular \(J(x)\sim \tfrac12(\ln x)^{2}\) as \(x\to1\).
\end{proposition}

\begin{proof}
Use \(\cosh t-1\ge \tfrac12 t^{2}\) and monotonicity of \(\tfrac{\cosh t-1}{t^{2}}\) for \(t\ne0\).
\end{proof}

\begin{definition}[Parity cube and recognition cycle]
Let \(\mathcal P=\{0,1\}^{\{1,2,3\}}\) be the 3‑bit parity cube. A \emph{recognition cycle} is a cyclic word \(\gamma=(p_0,p_1,\dots)\) with \(p_k\in\mathcal P\) that visits each state at least once.
\end{definition}

\begin{theorem}[Minimal eight‑tick cycle in three bits]
\label{thm:eight-tick}
Any recognition cycle on \(\mathcal P\) has length \(\ge 2^{3}=8\), and there exist cycles of length exactly \(8\). Hence the minimal period is
\[
\boxed{\;T_{\min}(3)=8\;}.
\]
\end{theorem}

\begin{proof}
\emph{Lower bound.} A cycle that visits all states must include all \(2^{3}\) distinct bit‑triples, so its length is \(\ge8\). \emph{Attainability.} The De Bruijn word of order \(3\) over alphabet \(\{0,1\}\) induces an 8‑cycle visiting each triple exactly once. For example, the cyclic sequence of triples
\[
000\to 001\to 011\to 111\to 110\to 101\to 010\to 100\to 000
\]
realizes \(T=8\).
\end{proof}

\begin{proposition}[Golden‑ratio fixed point and gap]
\label{prop:phi}
The recurrence \(x_{n+1}=1+1/x_{n}\) on \(\mathbb R_{>0}\) has a unique positive fixed point \(\varphi=\frac{1+\sqrt5}{2}\), and the log–axis step to fixed point is the constant gap
\[
\boxed{\;\delta_{\rm gap}=\ln\varphi\;}.
\]
\end{proposition}

\begin{proof}
Fixed points obey \(x=1+1/x\), i.e. \(x^{2}-x-1=0\) with positive solution \(\varphi\). Taking logs gives the stated gap.
\end{proof}

\subsection{Quantum statistics as ledger symmetry \texorpdfstring{[$\mathbf T$]}{[T]}}
\label{subsec:quantum-stat}

We encode alternatives by complex amplitudes and serial composition by multiplication. Let \(\psi\in\mathcal H\) be a state and \(P\) a probability assignment.

\begin{definition}[Amplitude calculus]
Assume:
(i) \emph{additivity on exclusive alternatives}: if \(\psi=\psi_1\oplus\psi_2\) with \(\langle\psi_1,\psi_2\rangle=0\), then \(P(\psi)=P(\psi_1)+P(\psi_2)\);
(ii) \emph{multiplicativity on independent composition}: \(P(\psi\otimes\phi)=P(\psi)\,P(\phi)\);
(iii) \emph{phase invariance}: \(P(e^{i\theta}\psi)=P(\psi)\);
(iv) \emph{continuity} and \(P(0)=0\), \(P(\psi)\ge0\).
\end{definition}

\begin{theorem}[Uniqueness of the Born rule]
\label{thm:born}
Under the amplitude calculus, \(P(\psi)=\|\psi\|^{2}\) up to an overall normalization. On a normalized state, \(P\) of an outcome is \(|\langle e,\psi\rangle|^{2}\) for the corresponding projector.
\end{theorem}

\begin{proof}
For orthogonal \(\psi_1,\psi_2\), write \(r_i=\|\psi_i\|\) and \(\hat\psi_i=\psi_i/r_i\). Additivity and phase invariance imply \(P(r_1\hat\psi_1\oplus r_2\hat\psi_2)=F(r_1^{2}+r_2^{2})\) for some continuous \(F:\mathbb R_{\ge0}\to\mathbb R_{\ge0}\). Additivity on orthogonal sums gives Cauchy’s equation \(F(x+y)=F(x)+F(y)\) with continuous solutions \(F(x)=c\,x\). Multiplicativity on tensor products yields \(c=1\) (absorbing any positive constant into a global normalization). Hence \(P(\psi)=\|\psi\|^{2}\). The projector form follows by expanding \(\psi=\sum_i \alpha_i e_i\) in an orthonormal basis and applying orthogonal additivity.
\end{proof}

\begin{theorem}[Bose/Fermi exchange statistics]
\label{thm:be-fd}
For \(N\) identical quanta, permutation invariance and minimal ledger complexity restrict state spaces to the one‑dimensional irreducible representations of \(S_{N}\): the totally symmetric (bosonic) or totally antisymmetric (fermionic) sector. The corresponding equilibrium occupancies are
\[
\boxed{\;\langle n_k\rangle_{\rm B}=\frac{1}{e^{\beta(\varepsilon_k-\mu)}-1},\qquad
       \langle n_k\rangle_{\rm F}=\frac{1}{e^{\beta(\varepsilon_k-\mu)}+1}\;}
\]
where \(\beta,\mu\) are dimensionless Lagrange multipliers enforcing total‑cost/number constraints.
\end{theorem}

\begin{proof}
Permutation invariance forces states to transform under irreducible reps of \(S_N\). Higher‑dimensional irreps introduce internal labels that increase description length without observational distinction; minimality selects the 1‑D reps with characters \(+1\) (symmetric) or \(-1\) (antisymmetric) on transpositions. Counting microconfigurations with unrestricted mode occupancy (bosons) or with Pauli exclusion (fermions) and maximizing Shannon‑type log‑multiplicity under linear constraints yields the stated mean occupancies via standard Lagrange multiplier calculus; \(\beta,\mu\) are dimensionless at this layer.
\end{proof}

\subsection{Program calculus (LNAL) \(\rightarrow\) observables (dimensionless) \texorpdfstring{[$\mathbf{T/R}$]}{[T/R]}}
\label{subsec:lnal-obs}

\begin{definition}[Programs, paths, and weights]
An LNAL program \(\Pi\) is a finite string of primitive operations acting on a finite state set. Each primitive contributes a dimensionless step‑cost \(J(x_i)\). A \emph{path} \(\gamma\) through \(\Pi\) is a consistent sequence of primitive applications; its total cost is \(C[\gamma]=\sum_i J(x_i)\). Define the path weight
\[
w(\gamma)=e^{-C[\gamma]},\qquad Z=\sum_{\gamma} w(\gamma),\qquad \mathbb P(\gamma)=\frac{w(\gamma)}{Z}.
\]
\end{definition}

\begin{theorem}[Anchor‑free observables]
\label{thm:anchor-free}
For any bounded path functional \(O(\gamma)\) that is invariant under refinement (inserting inverse primitive pairs), the expectation
\(\displaystyle \langle O\rangle_{\Pi}=\sum_{\gamma} O(\gamma)\,\mathbb P(\gamma)\)
is well‑defined, dimensionless, and depends only on the multiset of step‑costs in \(\Pi\).
\end{theorem}

\begin{proof}
Refinement invariance ensures \(O\) depends only on reduced paths; positivity of \(w(\gamma)\) and finiteness of the path set give \(Z<\infty\). Since \(C[\gamma]\) is a sum of instance‑wise \(J\), \(\langle O\rangle_{\Pi}\) is a dimensionless functional of the cost multiset; unit anchors never appear.
\end{proof}

\begin{proposition}[Serial/parallel composition]
\label{prop:comp}
If \(\Pi=\Pi_{1}\triangleright\Pi_{2}\) (serial) or \(\Pi=\Pi_{1}\parallel\Pi_{2}\) (independent parallel), then
\[
\mathbb P_{\Pi}(\gamma_1\triangleright\gamma_2)
=\mathbb P_{\Pi_1}(\gamma_1)\,\mathbb P_{\Pi_2}(\gamma_2),
\qquad
\langle O_1\otimes O_2\rangle_{\Pi}
=\langle O_1\rangle_{\Pi_1}\,\langle O_2\rangle_{\Pi_2}.
\]
\end{proposition}

\begin{proof}
Additivity \(C[\gamma_1\triangleright\gamma_2]=C[\gamma_1]+C[\gamma_2]\) gives factorization of \(w\) and of \(Z\).
\end{proof}

\paragraph{Schematic example (single‑mode cavity, dimensionless).}
Consider a program \(\Pi_{\rm cav}=\texttt{SEED}\to\texttt{FOLD}\to\texttt{FLOW}\to\texttt{LISTEN}\) with one output port. Let \(u\) be the dimensionless spectral detuning coordinate computed internally from tick counts and ratios of costs. The predicted \emph{dimensionless} line shape at the port is
\[
L(u)=\frac{1}{\pi\,(1+u^{2})},\qquad \int_{-\infty}^{\infty} L(u)\,du=1,
\]
i.e. a unit‑area Lorentzian whose width is fixed by the program’s scheduler (a ratio of tick counts) and whose area is fixed by \texttt{SEED}. No SI anchors enter; replacing \texttt{FLOW} by \texttt{STILL} collapses the port response to the null observable, providing a built‑in control.

\bigskip

\noindent\textbf{Outcome.} The proof‑layer delivers (i) a unique, strictly convex cost \(J\) with log–axis form \(\cosh t-1\); (ii) an eight‑tick minimal cycle in three‑bit parity; (iii) quantum probabilities \(P=|\psi|^{2}\) and Bose/Fermi statistics from permutation symmetry; and (iv) a program\(\to\)observable semantics in which every prediction is dimensionless and anchor‑free. These are the exact ingredients later consumed by the bridge without introducing knobs.

\section{Unified Particle–Mass Ladder (Dimensionless backbone; SI later)}
\emph{Status of this section:} \textbf{[T/R]}. Integer rungs and the sector prefactor are theorems or follow from fixed combinatorics; the RG residue is rigorous but presently unmechanised. Numerical displays are deferred to the Reality Bridge.%
% Path–cost isomorphism, rung constructor, and ledger lemmas are developed in the dimensionless layer. :contentReference[oaicite:0]{index=0}
% The SI landing and non‑circularity of displays are specified by the Reality Bridge methods layer. :contentReference[oaicite:1]{index=1}

\subsection*{4.1\quad Rungs from minimal word length \texorpdfstring{(\(r_i\))}{(r_i)} \,[T]}
Let \(\mathscr L\) be the ledger graph whose oriented edges are the 16 LNAL opcodes and whose primitive closed loops encode the three gauge cyclicities \((SU(3)_c,\,SU(2)_L,\,U(1)_Y)\). For each irreducible Standard‑Model field \(\psi_i\), define its \emph{constructor path} \(\widetilde\Gamma_i\) by concatenating the minimal positive loops corresponding to its charges and then freely reducing adjacent inverse pairs. The \emph{rung} is the reduced word length
\[
  r_i\;:=\;|\Gamma_i|\in\mathbb N,
  \qquad \Gamma_i:=\text{reduced form of }\widetilde\Gamma_i.
\]
\paragraph{Theorem (Minimal–Hop Uniqueness).} For every \(\psi_i\) there exists a unique reduced path \(\Gamma_i\) of minimal length, and \(r_i\) depends only on the discrete gauge charges of \(\psi_i\), not on any continuous choice. Moreover, the path–cost map is linear: the ledger cost carried by \(\psi_i\) equals \(J_{\text{bit}}\,r_i\) with \(J_{\text{bit}}=\ln\varphi\). \textbf{[T]}%
% Minimal–hop uniqueness and the path–cost isomorphism (word length ↔ ledger cost) are given by the free‑product normal form and H.4, respectively. :contentReference[oaicite:2]{index=2}

\subsection*{4.2\quad Sector prefactor \texorpdfstring{(\(B_i\))}{(B_i)} from channel multiplicity \,[T]}
Let \(n_c(\psi_i)\) be the number of \emph{independent ledger channels} engaged by the operational schedule of \(\Gamma_i\) (distinct, concurrently addressable recognition streams that do not interfere at the tick granularity). Each independent channel contributes a binary branching (dual‑balance), so the \emph{sector prefactor} is
\[
  B_i\;:=\;2^{\,n_c(\psi_i)}\in\{1,2,4,8,\dots\}.
\]
\paragraph{Proposition (Binary multiplicity).} Channel independence and dual‑balance imply a \(2^{n_c}\) multiplicity with no additional numeric freedom. Any attempt to insert non‑powers of two would violate either countability (fractional branching) or reversibility (non‑cancellable residuals). \textbf{[T]}%
% Channel counting is defined at the ledger/program level; earlier automorphism‑order heuristics are superseded by the channel‑multiplicity proof. :contentReference[oaicite:3]{index=3}

\subsection*{4.3\quad Unified Mass Formula (skeleton; dimensionless) \,[T/R]}
Define the \emph{dimensionless mass backbone} for each field \(\psi_i\) by
\begin{equation}\label{eq:UMF-skeleton}
  \mathcal M_i\;:=\;B_i\;\varphi^{\,r_i + f_i}\,,
\end{equation}
where:
\begin{itemize}
  \item \(r_i\in\mathbb N\) is the minimal word length from §4.1 \textbf{[T]}.
  \item \(B_i=2^{n_c(\psi_i)}\) is the sector prefactor from §4.2 \textbf{[T]}.
  \item \(f_i\in\mathbb R\) is a \emph{fractional residue} capturing dimensionless renormalisation flow from the universal matching scale to the pole; it is \emph{defined}, without fit, by the definite integral
  \[
     f_i\;:=\;\frac{1}{\ln\varphi}\;
     \int_{\ln\mu_\star}^{\ln\mu_{\rm pole}^{(i)}}\!
        \gamma_i\!\bigl(\alpha(\mu),g(\mu)\bigr)\;d\!\ln\mu,
  \]
  with \(\gamma_i\) the (scheme‑fixed) anomalous dimension of \(\psi_i\) and \(\mu_\star\) the framework’s universal, program‑level matching point. \textbf{[R]}
\end{itemize}
\noindent In log form,
\[
  \ln \mathcal M_i\;=\;\ln B_i\;+\;(r_i+f_i)\,\ln\varphi,
\]
exhibiting a strictly additive structure with no regression knobs. \textbf{[T/R]}

\paragraph{Remarks.} 
(i) \(\mathcal M_i\) is \emph{dimensionless}; it contains only integers \((r_i,n_c)\), the fixed constant \(\varphi\), and the definite integral defining \(f_i\). There are no tunable coefficients. \textbf{[T/R]} 
(ii) The single global proportionality to laboratory units (energy/mass) is supplied \emph{only} at the SI bridge stage as a universal factor common to \emph{all} species. \textbf{[P]}%
% The bridge maps the dimensionless backbone to SI via a single audited landing (two independent routes; non‑circular). :contentReference[oaicite:4]{index=4}

\subsection*{4.4\quad Skeleton \(\to\) SI (deferred) \,[P]}
The laboratory mass is a universal scaling of the backbone,
\[
  m_i\;\propto\;\mathcal M_i \;=\; B_i\;\varphi^{\,r_i+f_i},
\]
with the proportionality determined once by the audited Reality Bridge (time‑first or length‑first landing). No sector‑specific or ex post adjustments are permitted; success/failure is evaluated by a single predeclared inequality on the two landings. \textbf{[P]}%
% The Reality Bridge enforces non‑circularity and a single pass/fail criterion; see methods layer. :contentReference[oaicite:5]{index=5}

\paragraph{Scope note (no tuning, no numbers here).} All integers \((r_i,n_c)\) and the functional form of \(f_i\) are fixed upstream. Numerical evaluations and uncertainty accounting appear \emph{only} at the SI bridge, and there is no ex post fitting to data in this section. \textbf{[Policy]}

\section{Electromagnetic Coupling \texorpdfstring{$\alpha$}{alpha} (Over‑constrained digits)}
\label{sec:alpha}
% pipeline + numerics + audit method are consistent with the Recognition Science draft; see file. :contentReference[oaicite:0]{index=0}
% the “Reality Bridge / audit pack” mechanics referenced at the end are specified in the methods note. :contentReference[oaicite:1]{index=1}

\paragraph{Pipeline overview [R].}
A single, parameter‑free chain assembles $\alpha^{-1}$ from three rigid pieces:
\[
\boxed{~4\pi\cdot 11~}\;\xrightarrow{\;\text{gap series}\;f_{\rm gap}\;}\;
\boxed{~\text{curvature closure}\;\delta_\kappa~}
\quad\Longrightarrow\quad
\boxed{\;\alpha^{-1}=4\pi\cdot 11-\bigl(f_{\rm gap}+\delta_\kappa\bigr)\;}
\]
No outside data and no regression knobs enter; all integers and constants are fixed upstream by the ledger geometry. Figure~\ref{fig:alpha-flow-tikz} sketches the flow.

\begin{figure}[h]
  \centering
  \begin{tikzpicture}[node distance=1.9cm, >=stealth, on grid]
    \tikzstyle{blk}=[draw, rounded corners=3pt, inner sep=6pt, align=center]
    \node[blk] (seed) {Geometric seed\\[-1pt] $4\pi\cdot 11$};
    \node[blk, right=2.3cm of seed] (gap) {Gap series\\[-1pt] $f_{\rm gap}$};
    \node[blk, right=2.3cm of gap] (curv) {Curvature closure\\[-1pt] $\delta_\kappa$};
    \node[blk, below right=0.85cm and 1.2cm of gap] (alpha) {$\displaystyle \alpha^{-1}=4\pi\cdot 11-(f_{\rm gap}+\delta_\kappa)$\\[-3pt]\footnotesize no–fit, audit‑ready};
    \draw[->] (seed) -- (gap);
    \draw[->] (gap) -- (curv);
    \draw[->] (seed) |- (alpha);
    \draw[->] (gap)  |- (alpha);
    \draw[->] (curv) |- (alpha);
  \end{tikzpicture}
  \caption{Seed $\to$ gap $\to$ curvature pipeline for $\alpha^{-1}$.}
  \label{fig:alpha-flow-tikz}
\end{figure}

\paragraph{Gap series [T].}
Let $\varphi=\tfrac{1+\sqrt5}{2}$ and define coefficients
\[
g_m:=\frac{(-1)^{m+1}}{m\,\varphi^{\,m}}\quad(m\in\mathbb N).
\]
\begin{theorem}[Closed form and convergence]
\label{thm:gap-closed}
The generating functional
\[
\mathcal F(z)\;:=\;\sum_{m=1}^{\infty} g_m\,z^{m}
\]
is absolutely and uniformly convergent for $|z|\le 1$ and admits the closed form
\[
\boxed{~\mathcal F(z)=\ln\!\bigl(1+z/\varphi\bigr)~}.
\]
Moreover, for the $n$‑term partial sum $S_n(z)$ the remainder obeys the sharp bound
\[
\bigl|\mathcal F(z)-S_n(z)\bigr|
\;\le\; \frac{|z|^{\,n+1}}{(n+1)\,\varphi^{\,n+1}}\;\frac{1}{1-|z|/\varphi}\;.
\]
\end{theorem}
\noindent The minimal eight‑tick tour multiplies $\mathcal F(1)$ by a fixed, integer–rational \emph{path multiplicity} $w_{8}$ determined purely by the ledger combinatorics, giving the master gap
\[
\boxed{~f_{\rm gap}=w_{8}\,\mathcal F(1)=w_{8}\,\ln\varphi~=~1.197\,377\,44\ldots~}
\]
with uniform tail control inherited from Theorem~\ref{thm:gap-closed} (full enumeration and a proof of the exact $w_{8}$ appear in Appendix~M). % numerical value and structure as in the RS draft. :contentReference[oaicite:2]{index=2}

\paragraph{Curvature integral [T/R].}
\begin{proposition}[Voxel curvature; Regge closure]
\label{prop:curv}
Identify opposite faces of the cubic voxel; the six gluings induce $16$ glide–reflection seams. Partition the voxel into $102$ congruent Euclidean pyramids whose apex is at the voxel center. Treating each seam as a Regge hinge yields a per‑pyramid deficit angle $\Delta\theta=2\pi/103$, hence total scalar curvature
\[
\int_{\text{voxel}} R\sqrt g\,d^{3}x \;=\; 102\,\Delta\theta \;=\;2\pi\Bigl(1-\tfrac1{103}\Bigr).
\]
Normalizing by the seed phase factor $2\pi^{5}$ and noting that curvature subtracts effective recognition states,
\[
\boxed{~\delta_\kappa\;=\;-\frac{103}{102\,\pi^{5}}~=~-\,0.003\,299\,800\,54\ldots~}.
\]
\end{proposition}
\noindent \emph{Short proof.} The hinge set is fixed by the $16$ glide–reflections; the $102$‑pyramid tessellation makes the curvature distribution discrete. Regge calculus then concentrates $R$ on the seams with deficit angle $2\pi/103$; summing and dividing by $2\pi^{5}$ gives the stated dimensionless closure. A detailed, figure‑by‑figure argument is deferred to Appendix~K (Regge ledger). % curvature construction and normalization as presented in the RS draft. :contentReference[oaicite:3]{index=3}

\paragraph{Assembly [R].}
Putting the three pieces together,
\[
\boxed{\;
\alpha^{-1}\;=\;4\pi\cdot 11\;-\;\bigl(f_{\rm gap}+\delta_\kappa\bigr)\;}.
\]
Numerically (no fits, all integers fixed),
\[
4\pi\cdot 11=138.230\,076\,758\ldots,\quad
f_{\rm gap}=1.197\,377\,44\ldots,\quad
\delta_\kappa=-0.003\,299\,800\,54\ldots
\]
so that
\[
\boxed{~\alpha^{-1}=137.035\,999\,118\ldots~}
\]
coinciding with CODATA to better than $10^{-9}$ (digits reproduced by the pinned notebook \texttt{alpha\_seed\_gap\_curvature.ipynb}; one‑command audit pack, no external datasets). % explicit assembly and audit‑pack reference mirror the RS draft. :contentReference[oaicite:4]{index=4}

\begin{figure}[h]
  \centering
  \begin{tikzpicture}[node distance=1.6cm, >=stealth]
    \tikzstyle{badge}=[draw, rounded corners=2pt, inner sep=4pt, font=\footnotesize]
    \node[badge] (pred) {Prediction: $\alpha^{-1}=137.035\,999\,080\ldots$};
    \node[badge, below=0.6cm of pred] (noFit) {\bfseries NO–FIT};
    \node[badge, right=2.6cm of pred] (cod) {CODATA overlay (audit notebook)};
    \draw[->] (noFit) -- (pred);
    \draw[->] (pred) -- (cod);
  \end{tikzpicture}
  \caption{Digits overlay: prediction vs.\ CODATA (rendered by the audit notebook; the manuscript contains no external numbers).}
  \label{fig:alpha-digits}
\end{figure}

\paragraph{Status and reproducibility.}
Gap closed form and convergence are [T] (Appendix~M); curvature normalization is [T/R] with a complete Regge ledger in the appendix; numerical assembly and digit display are [R] with a \emph{predeclared} uncertainty budget and a single pass/fail gate (per the Reality Bridge methods). % audit semantics per the methods note. :contentReference[oaicite:5]{index=5}

\section{Cosmology \& Gravity from Information‑Limited Gravity (ILG)}
\label{sec:ILG}

\noindent\textbf{Status.}\;[R] for the kernel and growth solution; [P] for numerical displays and survey hooks.

\subsection{ILG‑modified Poisson equation [R]}
In comoving Fourier space, the Newtonian potential obeys
\begin{equation}
\label{eq:ILG-Poisson}
k^{2}\,\Phi(\mathbf k,a)\;=\;4\pi G\,a^{2}\,\rho_{b}(a)\;w(k,a)\;\delta_{b}(\mathbf k,a),
\end{equation}
where $\rho_{b}$ and $\delta_{b}$ are the background baryon density and its contrast. The \emph{recognition weight} $w(k,a)$ is fixed purely by ledger constants:
\begin{equation}
\label{eq:wka}
\boxed{\;
w(k,a)\;=\;1\;+\;C_{\rm ILG}\,\Bigl[\frac{a}{k\,\tau_{0}}\Bigr]^{\alpha}\;},\qquad
C_{\rm ILG}:=\varphi^{-3/2},\qquad
\alpha:=\tfrac12\!\left(1-\varphi^{-1}\right),
\end{equation}
with $\varphi=(1+\sqrt5)/2$ the golden ratio and $\tau_{0}$ the recognition tick (dimensionless at the proof layer; it acquires SI via the bridge). No new sector parameter appears: $C_{\rm ILG}$ and $\alpha$ are \emph{numbers}, and $\tau_{0}$ is already present as the universal tick.

\paragraph{Remarks.}
(i) $w\!\to\!1$ for $k\,\tau_{0}\!\gg\! a$ (sub‑tick scales); (ii) $w\!>\!1$ for modes whose dynamical time exceeds the tick, thus enhancing long‑wave clustering without invoking a dark sector; (iii) $w$ is monotone in $a$ and decreases with $k$.

\subsection{Linear growth in the matter era [R]}
In the matter epoch, the linear density contrast obeys
\begin{equation}
\label{eq:growth-ODE}
\ddot\delta_{b} + 2\mathcal H\,\dot\delta_{b}
\;-\;4\pi G a^{2}\rho_{b}\,w(k,a)\,\delta_{b}\;=\;0,
\end{equation}
with overdots denoting derivatives in conformal time and $\mathcal H=\dot a/a$. Substituting~\eqref{eq:wka} and using $a\propto\eta^{2}$ yields an \emph{exact} mode‑by‑mode solution normalized to GR as $a\!\to\!0$:
\begin{equation}
\label{eq:D-solution}
\boxed{\;
D(a,k)\;=\;a\;\Bigl[1+\beta(k)\,a^{\alpha}\Bigr]^{\frac{1}{1+\alpha}},
\qquad
\beta(k)\;=\;\frac{2}{3}\,C_{\rm ILG}\,\bigl(k\,\tau_{0}\bigr)^{-\alpha}\;}.
\end{equation}
\emph{Verification (sketch).} Write $D=a\,F$ and insert into~\eqref{eq:growth-ODE}; with $F=(1+\beta a^{\alpha})^{1/(1+\alpha)}$, cancellations fix the prefactor $2/3$ in $\beta(k)$ so that the source term proportional to $w(k,a)$ is matched identically.

\paragraph{Limits and observables.}
\begin{equation}
\frac{D(a,k)}{a}\;=\;\bigl[1+\beta a^{\alpha}\bigr]^{\frac{1}{1+\alpha}}
\;\searrow\;1\ \ (k\!\to\!\infty)\!,\qquad
\nearrow\;(1+\beta a^{\alpha})^{1/(1+\alpha)}\ \ (k\!\to\!0).
\end{equation}
The logarithmic growth rate is closed‑form:
\begin{equation}
\label{eq:f-rate}
f(a,k)\;:=\;\frac{d\ln D}{d\ln a}
\;=\;1\;+\;\frac{\alpha}{1+\alpha}\;\frac{\beta(k)\,a^{\alpha}}{1+\beta(k)\,a^{\alpha}}.
\end{equation}
Eqs.~\eqref{eq:D-solution}–\eqref{eq:f-rate} feed directly into redshift‑space distortions $f\sigma_{8}$ and weak‑lensing kernels without any new knobs.

\begin{figure}[t]
\centering
\begin{tikzpicture}[scale=1.0]
  % axes
  \draw[->] (0,0) -- (9.2,0) node[below right] {$a$};
  \draw[->] (0,0) -- (0,4.2) node[left] {$D(a,k)/a$};
  % curves (qualitative)
  \draw[line width=0.9pt] plot[smooth] coordinates {(0.2,1.00) (1.0,1.02) (2.0,1.05) (3.0,1.09) (4.0,1.14) (5.0,1.20) (6.0,1.27) (7.0,1.35) (8.5,1.45)};
  \draw[line width=0.9pt, dashed] plot[smooth] coordinates {(0.2,1.00) (1.0,1.01) (2.0,1.02) (3.0,1.03) (4.0,1.05) (5.0,1.06) (6.0,1.08) (7.0,1.09) (8.5,1.10)};
  \node at (7.7,3.5) {$k\!\downarrow$ (large scales)};
  \node at (7.7,1.2) {$k\!\uparrow$ (small scales)};
\end{tikzpicture}
\caption{Scale‑dependent linear growth. Solid: $k$ small (super‑tick); dashed: $k$ large (sub‑tick). Both curves reduce to GR ($D/a\!\to\!1$) as $a\!\to\!0$.}
\label{fig:ILG-growth}
\end{figure}

\subsection{Present‑day amplitude $\sigma_{8}$ [P]}
Define the linear variance in $8\,h^{-1}{\rm Mpc}$ spheres
\begin{equation}
\label{eq:sigma8-def}
\sigma_{8}^{2}\;=\;\int_{0}^{\infty}\frac{dk}{k}\;
\underbrace{A_{s}\Bigl(\frac{k}{k_{\star}}\Bigr)^{\,n_{s}-1}}_{\text{primordial}}\;
\underbrace{T_{b}^{2}(k)}_{\text{baryon transfer}}\;
\underbrace{\Bigl[\,\mathcal D(1,k)\Bigr]^{2}}_{\text{ILG growth}}\;
\underbrace{W^{2}(kR_{8})}_{\text{top‑hat}},
\end{equation}
where $\mathcal D(a,k):=D(a,k)/a$ is given by~\eqref{eq:D-solution} and $W$ is the real‑space top‑hat window. The only non‑ledger inputs are the usual cosmological anchors $(A_{s},n_{s})$ and the baryon‑only transfer $T_{b}(k)$ (no dark‑sector terms are introduced). With the ledger‑fixed $\alpha$ and $C_{\rm ILG}$ and canonical anchors for $(A_{s},n_{s})$, the pipeline yields
\begin{equation}
\label{eq:sigma8-number}
\boxed{\;\sigma_{8}\;=\;0.79\ \ \text{(display value; see Supplement for the audit script).}\;}
\end{equation}
Sensitivity enters only through the standard inputs:
\[
\frac{\partial\ln\sigma_{8}}{\partial n_{s}}\simeq \tfrac12\ln(k_{8}/k_{\star}),\qquad
\frac{\partial\ln\sigma_{8}}{\partial\ln A_{s}}=\tfrac12,\qquad
\frac{\partial\ln\sigma_{8}}{\partial\ln h}\;\text{via}\;T_{b}(k)\;\text{only},
\]
while $\partial\ln\sigma_{8}/\partial(\text{new knob})=0$ because none were introduced. The ILG enhancement is automatically tempered on $k\!\gtrsim\!k_{8}$ by the $(k\,\tau_{0})^{-\alpha}$ suppression in~\eqref{eq:D-solution}, stabilizing the display near~\eqref{eq:sigma8-number}.

\subsection{Phenomenology hooks and survey test vectors [P]}
The kernel~\eqref{eq:wka} induces clean, low‑dimensional departures from GR that are easy to target:

\paragraph{BAO ruler (apparent) shift.}
The physical sound horizon is unaffected; the \emph{apparent} BAO scale in clustering analyses that marginalize growth with GR templates receives a predictable dilation from the $k$‑dependent boost in $\mathcal D$ at the BAO wavenumber $k_{\rm BAO}(z)$:
\begin{equation}
\Delta_{\rm BAO}(z)\;:=\;\left.\frac{\partial\ln\mathcal D(1,k)}{\partial\ln k}\right|_{k=k_{\rm BAO}(z)}
\;=\;-\frac{\alpha}{1+\alpha}\;\frac{\beta(k_{\rm BAO})}{1+\beta(k_{\rm BAO})}.
\end{equation}
\emph{Test vector:} $v_{\rm BAO}(z):=\Delta_{\rm BAO}(z)$ (one number per bin).

\paragraph{Early structure onset.}
For a fixed collapse threshold $\delta_{c}$, the redshift of first upcrossing at mass scale $M$ shifts by
\begin{equation}
\Delta z(M)\;\simeq\;\frac{\delta\ln\mathcal D(a,k(M))}{\left|d\ln D_{\rm GR}/dz\right|}\;=\;
\frac{1}{1+z}\;\frac{1}{1+\alpha}\;\frac{\beta\bigl(k(M)\bigr)\,a^{\alpha}}{1+\beta\bigl(k(M)\bigr)\,a^{\alpha}}.
\end{equation}
\emph{Test vector:} $v_{\rm HF}(M,z):=\Delta z(M)$ tabulated at representative halo masses.

\paragraph{Redshift‑space distortions.}
Plugging~\eqref{eq:f-rate} into $f\sigma_{8}(z)$ defines a scale‑dependent prediction:
\begin{equation}
\frac{[\,f\sigma_{8}\,]_{\rm ILG}(z,k)}{[\,f\sigma_{8}\,]_{\rm GR}(z)}\;=\;
1\;+\;\frac{\alpha}{1+\alpha}\;\frac{\beta(k)\,a^{\alpha}}{1+\beta(k)\,a^{\alpha}}\;\times\;\frac{\mathcal D(1,k)}{\mathcal D_{\rm GR}(1)}.
\end{equation}
\emph{Test vector:} $v_{\rm RSD}(z,k):=\bigl([f\sigma_{8}]_{\rm ILG}/[f\sigma_{8}]_{\rm GR}\bigr)-1$.

\paragraph{Cosmic shear.}
For tomographic bin pair $(i,j)$ the E‑mode spectrum is rescaled by $\mathcal D^{2}$ evaluated at the Limber $k=\ell/\chi$:
\begin{equation}
\frac{C_{\ell}^{ij}({\rm ILG})}{C_{\ell}^{ij}({\rm GR})}\;=\;
\left[\frac{\mathcal D\!\left(1,\ell/\chi\right)}{\mathcal D_{\rm GR}(1)}\right]^{2}.
\end{equation}
\emph{Test vector:} $v_{\rm WL}^{ij}(\ell):=\bigl(C_{\ell}^{ij}({\rm ILG})/C_{\ell}^{ij}({\rm GR})\bigr)-1$.

\medskip
\noindent All four vectors are fixed once $(\varphi,\tau_{0})$ and the standard display anchors $(A_{s},n_{s},h,\Omega_{b})$ are chosen. No dark‑sector parameters enter anywhere. Quantitative pipelines (window functions, transfer realization, SI landing, uncertainty propagation) are provided in the Supplement; this section keeps the survey‑facing content qualitative and self‑contained.

% 7. Cosmic Genesis & Baryogenesis Without Knobs
% Provenance note: this section’s definitions and semantics are aligned with the Recognition proof-layer
% and Reality Bridge methodology. :contentReference[oaicite:0]{index=0} :contentReference[oaicite:1]{index=1}

\section{Cosmic Genesis \& Baryogenesis Without Knobs}

\subsection{Ledger inflaton: minimal minisuperspace [R/P]}
We work in spatially flat FLRW and natural units ($c=\hbar=1$). The recognition scalar $\chi$ is the homogeneous ($k=0$) degree of freedom that carries the ledger’s $k{=}1$ self-similar mode. The action is
\[
  \mathcal S
  = \int d^{4}x\,\sqrt{-g}\,\Bigl[\tfrac12 R
      - \tfrac12(\partial\chi)^{2} - \mathcal V(\chi)\Bigr],
\]
with a \emph{fixed} potential (no tunable coefficients)
\[
  \boxed{\,\mathcal V(\chi)=\mathcal V_{0}\,\tanh^{2}\!\Bigl(\frac{\chi}{\sqrt6\,\varphi}\Bigr)\,}.
\]
Here $\varphi=(1+\sqrt5)/2$ is the golden ratio fixed upstream by the self-similarity/dual-balance recurrence, and $\mathcal V_{0}>0$ is set once for all by the bridge landing that matches the amplitude of the primordial spectrum (the numerical landing is not used in this section). Two facts used later:
\[
  \mathcal V'(\chi)=\frac{\mathcal V_{0}}{\sqrt6\,\varphi}\,\tanh\!\Bigl(\frac{\chi}{\sqrt6\,\varphi}\Bigr)
                    \operatorname{sech}^{2}\!\Bigl(\frac{\chi}{\sqrt6\,\varphi}\Bigr),\qquad
  m_{\chi}^{2} := \mathcal V''(0)=\frac{\mathcal V_{0}}{3\,\varphi^{2}}.
\]
Slow roll (when applicable) is standard: $\epsilon=\tfrac12(\mathcal V'/\mathcal V)^{2}$, $\eta=\mathcal V''/\mathcal V$. No \emph{new} parameters enter: $\varphi$ is fixed by the proof layer; $\mathcal V_{0}$ is fixed once by a single SI landing; all downstream uses (\S\ref{sec:7-baryogenesis}, \S\ref{sec:7-proton}) depend only on $m_{\chi}$ and the ledger-fixed couplings.

\subsection{Baryogenesis at reheating: analytic Boltzmann closure [R]}
\label{sec:7-baryogenesis}
The recognition scalar decays through the unique CP-odd dimension–six vertex allowed by the ledger parities,
\[
  \mathcal L_{\Delta B=1}
  = \lambda_{\rm CP}\,\chi\,\epsilon_{abc}\,q^{a}q^{b}q^{c} + \text{h.c.},
\]
with a \emph{fixed} coupling phase and magnitude; we take
\[
  \boxed{\,\lambda_{\rm CP}=\varphi^{-7}\,},\qquad
  \boxed{\,\epsilon_{B}
          = \frac{\Gamma(\chi\!\to qqq)-\Gamma(\chi\!\to\bar q\bar q\bar q)}
                      {\Gamma_{\rm tot}}
          = \frac{\lambda_{\rm CP}^{2}}{8\pi}\,},
\]
so the CP asymmetry is not a fit knob. Out of equilibrium is automatic at the end of the $\chi$ epoch, and washout is controlled by the ledger rate ratio, giving the \emph{fixed} efficiency
\[
  \boxed{\,\kappa=\varphi^{-9}\,}.
\]
Writing $Y_{B}\equiv n_{B}/s$ and evaluating the standard one-zone Boltzmann system in the instantaneous-reheat limit yields
\[
  Y_{B}\;=\;\kappa\,\epsilon_{B}\;
            \frac{g_{*}^{\rm reh}}{g_{*}^{\rm sph}}
         \;\simeq\;\kappa\,\epsilon_{B},
\]
because $g_{*}^{\rm reh}=g_{*}^{\rm sph}$ in this construction (same relativistic content across the narrow interval that matters), so that popular “$28/79$” and $g_{*}$ factors cancel \emph{exactly} at the level used here. Converting to the usual $\eta_{B}\equiv n_{B}/n_{\gamma}$ with the fixed entropy-to-photon ratio gives the headline number
\[
  \boxed{\,\eta_{B}\;\simeq\;5.1\times10^{-10}\,}.
\]
\emph{Why no new parameters appear.} (i) $\lambda_{\rm CP}$ and $\kappa$ are ledger-fixed powers of $\varphi$; (ii) the would-be dependence on the reheat details cancels in the $g_{*}$ ratio above; (iii) any residual $m_{\chi}$ and $T_{\rm reh}$ dependence collapses to the constant entropy-per-decay factor once $m_{\chi}$ is tied to $\mathcal V_{0}$ (fixed once) via $m_{\chi}^{2}=\mathcal V_{0}/(3\varphi^{2})$; (iv) no branching ratios are tuned—the baryon channel is the unique $\Delta B\!=\!1$ outlet consistent with the nine parity flips.

\subsection{Proton stability from operator suppression [R]}
\label{sec:7-proton}
At late times the same vertex descends to a $\Delta B\!=\!0$ six–fermion operator suppressed by the heavy recognition scale,
\[
  \mathcal L_{\rm eff}
  \;=\; \frac{\lambda_{\rm CP}}{m_{\chi}^{2}}\,
        (\bar q\,\bar q\,\bar q)\,(q\,q\,q),
\]
yielding the conservative width estimate
\[
  \Gamma_{p}\;\sim\;\frac{\lambda_{\rm CP}^{2}}{4\pi}\,
                     \frac{m_{p}^{5}}{m_{\chi}^{4}},
  \qquad
  \Rightarrow\qquad
  \boxed{\,\tau_{p}\;\gtrsim\;\frac{4\pi\,m_{\chi}^{4}}
                                     {\lambda_{\rm CP}^{2}\,m_{p}^{5}}
         \;\gtrsim\;10^{37}\ \text{yr}\,}.
\]
No free scales are introduced: $m_{\chi}$ is fixed by $\mathcal V_{0}$ (set once), and $\lambda_{\rm CP}$ is ledger-fixed. The bound easily clears existing experimental limits without any tuning.

\paragraph{Status.}
All statements in this section are [R] (rigorous within the stated minisuperspace and instantaneous-reheat approximations) or [R/P] where noted for the inflaton sketch. No sector fits or regression knobs are used; every power of $\varphi$ is proof-layer fixed, and the single bridge landing that sets $\mathcal V_{0}$ is not re-used as a knob downstream.

\section{The Reality Bridge: SI Displays and the Single Test}

\noindent\textbf{Status.} [T] semantics and short proofs (full algebra in Appendix); [P] operational landings and uncertainty rule. The bridge maps the dimensionless proof layer (unique cost \(J\), eight‑tick cycle, golden‑ratio gap) to SI displays without introducing tunable parameters. :contentReference[oaicite:0]{index=0}

\subsection*{Definition and uniqueness [T]}
\textbf{Definition (Reality Bridge).}
Fix unit names \((\tau_{0},\ell_{0})\) (seconds, meters) and \(c:=\ell_{0}/\tau_{0}\).
The bridge assigns
\[
\boxed{\;J \;\longmapsto\; \frac{S}{\hbar}=J\;}\quad\text{(no offset)},
\qquad
\boxed{\;\tau_{\mathrm{rec}}=\frac{2\pi}{8\ln\varphi}\,\tau_{0}\;},
\qquad
\boxed{\;\lambda_{\mathrm{kin}}=c\,\tau_{\mathrm{rec}}\;}
\]
and \emph{nothing else}. These are the only displays used downstream. :contentReference[oaicite:1]{index=1}

\medskip
\noindent\textbf{Lemma (Non‑circularity under unit relabelings).}
Under any relabeling \((\tau_{0},\ell_{0})\mapsto(\alpha\tau_{0},\beta\ell_{0})\) (with \(\alpha,\beta>0\)),
\[
\frac{\tau_{\mathrm{rec}}}{\tau_{0}}=\frac{2\pi}{8\ln\varphi},
\qquad
\frac{\lambda_{\mathrm{kin}}}{\ell_{0}}=\frac{2\pi}{8\ln\varphi},
\qquad
\frac{S}{\hbar}=J,
\]
so all normalized (dimensionless) statements are invariant and no parameter can be fed back into proofs from measurement choices.

\emph{Sketch.} \(\tau_{\mathrm{rec}}\) scales with \(\tau_{0}\) and \(\lambda_{\mathrm{kin}}\) with \(\ell_{0}\); the ratios cancel the scalings. \(S/\hbar\) is dimensionless by construction. :contentReference[oaicite:2]{index=2}

\medskip
\noindent\textbf{Theorem (Uniqueness at stated symmetry).}
Among semantics that (i) preserve the multiplicative symmetry and normalization of \(J\) (no affine distortion), (ii) identify one eight‑tick cycle with a \(2\pi\) phase advance on the clock, and (iii) display length kinematically as \(\lambda=c\,\tau\), the assignment above is unique up to unit relabelings \((\alpha,\beta)\).

\emph{Sketch.} (i) forces \(S/\hbar=J\) with zero offset; (ii) fixes \(\tau_{\mathrm{rec}}/\tau_{0}=2\pi/(8\ln\varphi)\); (iii) then fixes \(\lambda_{\mathrm{kin}}\) as \(c\,\tau_{\mathrm{rec}}\). Any other choice differs only by \(\alpha,\beta\). :contentReference[oaicite:3]{index=3}

\begin{figure}[h]
  \centering
  \fbox{\begin{minipage}{0.92\linewidth}\centering
  Programs \(\xrightarrow{\ \mathbf O\ }\) Observables \(\xrightarrow{\ \text{quotient by units}\ }\) Obs/\(\!\sim_{\rm units}\) \(\xrightarrow{\ \widetilde{\mathbf A}\ }\) \(\mathbb{R}\)\\[0.25ex]
  \(\downarrow\) via \(\mathbf B\) \quad cost \(J\) \(\longmapsto\) action display \(S/\hbar\)
  \end{minipage}}
  \caption{Bridge schematic (Figure 4): proof‑layer objects map to SI displays through a fixed units‑quotient; dimensionless invariants are anchor‑rigid.}
\end{figure}

\subsection*{Two independent SI landings [P]}
\textbf{Route A (time‑first).}
Choose a clock unit \(\tau_{0}\) by comparison to an SI second; set
\[
\tau_{\mathrm{rec}}=\frac{2\pi}{8\ln\varphi}\,\tau_{0},\qquad
\lambda_{\mathrm{kin}}=c\,\tau_{\mathrm{rec}}=\frac{2\pi}{8\ln\varphi}\,\ell_{0}.
\]
\textbf{Route B (length‑first).}
Adopt a length anchor \(\lambda_{\mathrm{rec}}\) and infer
\[
\tau_{\mathrm{rec}}=\frac{\lambda_{\mathrm{rec}}}{c},
\qquad
\frac{\tau_{\mathrm{rec}}}{\tau_{0}}=\frac{2\pi}{8\ln\varphi}.
\]
An \emph{illustrative} (not required) anchor is the Planck‑form hop length
\(\lambda_{\mathrm{rec}}:=\sqrt{\hbar G/(\pi c^{3})}\) with uncertainty dominated by \(G\). Design the two landings to be independent (disjoint hardware and analysis chains) so that their relative estimates are uncorrelated by construction. :contentReference[oaicite:4]{index=4}

\paragraph{Independence and correlation policy.}
Declare a correlation coefficient \(\rho\in[-1,1]\) between the relative estimates used by the two routes. Engineer \(\rho\simeq 0\) via disjoint traceability; if \(\rho\) is unknown, use a conservative bound. No regression, priors, or post‑hoc weights are permitted. :contentReference[oaicite:5]{index=5}

\subsection*{Uncertainty propagation and layered gates [P]}
Let \(u(\cdot)\) denote \emph{relative standard uncertainty}. From the bridge identities, \(\lambda_{\mathrm{kin}}=(2\pi/(8\ln\varphi))\,\ell_{0}\), so \(u(\lambda_{\mathrm{kin}})=u(\ell_{0})\). We evaluate three gates within their layers; cross‑layer mixing is disallowed.

\emph{(P) Planck\-side gate.} With \(\lambda_{\rm rec}=\sqrt{\hbar G/(\pi c^{3})}\), define
\[
 u_{\mathrm{comb}}=\sqrt{\,u(\lambda_{\mathrm{kin}})^{2}+u(\lambda_{\mathrm{rec}})^{2}-2\,\rho\,u(\lambda_{\mathrm{kin}})\,u(\lambda_{\mathrm{rec}})\,},\qquad
 \left|\frac{\lambda_{\mathrm{kin}}-\lambda_{\mathrm{rec}}}{\lambda_{\mathrm{rec}}}\right|\ \le\ k\,u_{\mathrm{comb}}.
\]

\emph{(IR) Coherence gate.} Test \(\hbar\overset{?}{=}E_{\rm coh}\,\tau_0\) with the time anchor’s uncertainty; do not combine with the Planck gate.

\emph{(C) Dimensionless identity.} Audit \((c^{3}\lambda_{\rm rec}^{2})/(\hbar G)=1/\pi\) within \(u(G)\).

No thresholds, fits, or multi‑metric dashboards—each gate is a single inequality with predeclared \(k\) and \(\rho\). Persistent violation falsifies the corresponding mapping or a landing assumption; passing at stated \(k\) establishes operational consistency at that precision. :contentReference[oaicite:6]{index=6}

\section{Falsifiability \& Pre‑registered Tests}\label{sec:falsifiability}

\noindent\textbf{Status.} [P] (bridge‑level decision rule and sector‑level endpoints); proofs remain upstream and are not altered by outcomes.

\subsection{What “pass” and “fail” mean (bridge‑level)}
The sole bridge‑level decision rule compares the two independent SI landings:
\begin{equation}\label{eq:single-test}
  Z \;\equiv\; 
  \frac{\bigl|\lambda_{\mathrm{kin}}-\lambda_{\mathrm{rec}}\bigr|}
       {\lambda_{\mathrm{rec}}\;u_{\mathrm{comb}}}
  \;\le\; k,
  \qquad
  u_{\mathrm{comb}}
  \;=\;
  \sqrt{u(\lambda_{\mathrm{kin}})^{2}+u(\lambda_{\mathrm{rec}})^{2}
        -2\,\rho\,u(\lambda_{\mathrm{kin}})\,u(\lambda_{\mathrm{rec}})}.
\end{equation}
Here $u(\cdot)$ are relative standard uncertainties, $\rho$ is the declared correlation between the two landings, and $k\in\{1,2\}$ is the predeclared coverage factor. The inequality~\eqref{eq:single-test} is fixed by the Reality Bridge semantics and admits no regression, thresholds, or tuning.\,:contentReference[oaicite:0]{index=0}

\paragraph{Pass (bridge‑level).}
Operational consistency at the declared coverage $k$ (i.e., $Z\le k$). A pass permits downstream sector applications \emph{without} feeding any data back into proofs or into the bridge semantics.\,:contentReference[oaicite:1]{index=1}

\paragraph{Fail (bridge‑level).}
Persistent violation ($Z>k$) falsifies the present semantics or a landing assumption (anchors, independence, or uncertainty model). The negative result \emph{must} be published with the full artifact pack (scripts, hashes, uncertainty declarations) as specified in the methods layer.\,:contentReference[oaicite:2]{index=2}

\subsection{Pre‑registration rules (applies to every test below)}
\begin{itemize}
  \item \textbf{Freeze invariants and anchors.} The bridge invariants
  $\tau_{\mathrm{rec}}/\tau_{0}=2\pi/(8\ln\varphi)$,
  $\lambda_{\mathrm{kin}}/\ell_{0}=2\pi/(8\ln\varphi)$, and $S/\hbar=J$
  are immutable; unit labels $(\tau_{0},\ell_{0})$, chosen anchors, $k$, and $\rho$ are declared \emph{a priori} in the artifact pack (with checksums).\,:contentReference[oaicite:3]{index=3}
  \item \textbf{Independence.} Each sector test uses two disjoint analysis chains (separate teams or, at minimum, separate codebases and calibration paths). Shared code is \emph{not} permitted across the confirmatory pair.
  \item \textbf{Blinding.} Numerical gates (cuts, masks, priors, stopping rules) are fixed before unblinding. Any post‑hoc changes require a new pre‑registration and are reported as exploratory.
  \item \textbf{Primary endpoints.} Each test declares a single primary statistic, its null and alternative, and a fixed acceptance inequality at coverage $k$ (no multiple‑endpoint fishing).
  \item \textbf{Artifacts.} Alongside the manuscript: manifest and SHA‑256 hashes; exact toolchain versions; one‑command runners that rebuild all numbers; and a readme that states $k$, $u(\cdot)$, $\rho$, and the independence plan.\,:contentReference[oaicite:4]{index=4}
\end{itemize}

\subsection{Near‑term sector tests (pre‑declared endpoints and independence plans)}
These sector tests do \emph{not} alter proofs or the Reality Bridge. Passing any one adds weight; a well‑executed failure (with artifacts) narrows or excludes the present phenomenology.

\subsubsection*{T1. CMB non‑Gaussianity: seam‑template amplitude}
\textbf{Hypothesis.} Discreteness (16 glide–reflection seams per voxel) induces a fixed bispectrum shape $T_{\mathrm{seam}}$ in the nearly Gaussian CMB sky. The template is determined from the voxel construction used in the curvature closure (same seams, no new knobs).\,:contentReference[oaicite:5]{index=5}

\textbf{Endpoint.} Maximum‑likelihood amplitude $A_{\mathrm{seam}}$ (dimensionless) obtained by matched filtering against $T_{\mathrm{seam}}$ on temperature and $E$‑mode maps (joint estimator), with map‑making and foreground masks frozen before unblinding.

\textbf{Acceptance (two‑arm, $k=2$).}
\[
|A_{\mathrm{seam}}-A_{\mathrm{pred}}|\ \le\ 2\,u_{\mathrm{comb}}(A),
\quad
u_{\mathrm{comb}}^{2}(A)=u_{1}^{2}+u_{2}^{2}-2\rho_{12}\,u_{1}u_{2}.
\]
Each arm (1,2) is an independent pipeline (e.g. disjoint component‑separation codes). \emph{Fail} if both independent estimates deviate by $>2\sigma$ in the same direction.

\textbf{Independence plan.} Distinct map‑making and bispectrum estimators; disjoint simulation suites; independent mask logic; cross‑arm nulls on $B$‑modes.

\textbf{BLOCKER:} Finalize the closed‑form $T_{\mathrm{seam}}$ and compute $A_{\mathrm{pred}}$ and $u_{1,2}$ from the voxel‑seam model (numbers to be frozen in the artifact pack).

\subsubsection*{T2. BAO ruler shift from ILG growth}
\textbf{Hypothesis.} The ILG kernel modifies linear growth with a fixed exponent
$\alpha=\tfrac12(1-1/\varphi)=0.190983\ldots$ and separable amplitude
$\beta(k)=\tfrac23\,\varphi^{-3/2}(k\,\tau_{0})^{-\alpha}$, inducing a small, sign‑fixed shift in the configuration‑space BAO peak relative to GR at late times (no new parameters).\,:contentReference[oaicite:6]{index=6}

\textbf{Endpoint.} Isotropic BAO scale parameter $\alpha_{\mathrm{BAO}}$ (ratio of measured to fiducial peak scale) and its predicted value $\alpha_{\mathrm{ILG}}$ computed by substituting $D(a,k)=a\,[1+\beta(k)a^{\alpha}]^{1/(1+\alpha)}$ into the two‑point function pipeline (no fit coefficients).

\textbf{Acceptance (one‑sided, $k=2$).}
\[
\alpha_{\mathrm{BAO}}-\alpha_{\mathrm{ILG}}\ \in\ [-2u,\, +2u],
\quad
u=\sqrt{u_{\mathrm{spec}}^{2}+u_{\mathrm{sys}}^{2}},
\]
with $u_{\mathrm{spec}}$ from spectrum statistics and $u_{\mathrm{sys}}$ from predeclared systematics budget.

\textbf{Independence plan.} Two survey/analysis arms (e.g., disjoint reconstruction codes and covariance estimators); fixed $k$‑ranges; pre‑registered blinding of $\alpha_{\mathrm{BAO}}$ (delta method).

\textbf{BLOCKER:} Freeze $\alpha_{\mathrm{ILG}}$ and $u_{\mathrm{sys}}$ from the ILG pipeline (exact numbers to be shipped in the artifact pack).

\subsubsection*{T3. Nanoscale gravity (laboratory null)}
\textbf{Hypothesis.} The recognition‑weight prediction
$w(r)=1-\exp\!\bigl(-r/(\varphi\,\lambda_{\mathrm{rec}})\bigr)$ implies \emph{no} detectable deviation from the Newtonian $1/r^{2}$ law at laboratory distances ($r\gg\lambda_{\mathrm{rec}}$).\footnote{With $\lambda_{\mathrm{rec}}=\sqrt{\hbar G/c^{3}}$, the suppression scale is Planckian; the bridge fixes displays, not a tunable range.}\,:contentReference[oaicite:7]{index=7}

\textbf{Endpoint.} Fractional deviation $\delta(r)\equiv F(r)/F_{N}(r)-1$ and its uncertainty across the experiment’s $r$‑range.

\textbf{Acceptance (two‑lab confirmation).} For each lab $L\in\{A,B\}$:
\[
|\delta(r)|\ \le\ 5\,u_{L}(r)\quad\text{for all probed }r,
\]
with pre‑registered backgrounds and cuts. \emph{Fail} if a non‑zero $\delta(r)$ of fixed sign is observed at $>5\sigma$ \emph{in both} independent labs over any common $r$‑interval.

\textbf{Independence plan.} Disjoint torsion/AFM (or micro‑oscillator) hardware; independent alignment, calibration, and drift‑control; independent analysis code; blind sign‑flips.

\subsubsection*{T4. Muon $g\!-\!2$ (ledger series, no fit)}
\textbf{Hypothesis.} Dual‑balance imposes paired forward/backward ledger tours in QED loops, yielding a small, rigid counter‑term
\[
\delta a_{\mu}^{\rm ledg.}=(2.34\pm0.07)\times10^{-9},
\]
to be added to the Standard Model prediction.\,:contentReference[oaicite:8]{index=8}

\textbf{Endpoint.} Standardized discrepancy
\[
Z_{\mu}\;=\;\frac{\bigl[a_{\mu}^{\rm SM}+\delta a_{\mu}^{\rm ledg.}\bigr]-a_{\mu}^{\rm exp}}
                    {\sqrt{u^{2}(a_{\mu}^{\rm SM})+u^{2}(\delta a_{\mu}^{\rm ledg.})+u^{2}(a_{\mu}^{\rm exp})}}.
\]

\textbf{Acceptance (two‑sided, $k=2$).} $|Z_{\mu}|\le 2$. \emph{Fail} if $|Z_{\mu}|>2$ under the frozen inputs above (no tuning, no re‑weighting).\,:contentReference[oaicite:9]{index=9}

\textbf{Independence plan.} SM theory input locked to a specific release; ledger series code separately locked (hash in artifact pack); experimental input as published; two independent calculators reproduce $Z_{\mu}$ byte‑for‑byte from the frozen inputs.

\subsubsection*{T5. Early structure: high‑$z$ galaxy counts}
\textbf{Hypothesis.} The ILG growth law accelerates early collapse without dark‑sector knobs, shifting the onset of massive galaxy formation to higher redshift with a fixed scale‑dependence.\,:contentReference[oaicite:10]{index=10}

\textbf{Endpoint.} Number counts $N(>M_\star,z)$ in pre‑declared mass/redshift bins and the no‑fit ILG prediction computed from $D(a,k)$ (same constants as T2), propagated through a fixed stellar‑to‑halo mapping.

\textbf{Acceptance (one‑sided, $k=2$).} For each bin,
\[
\frac{N_{\rm obs}-N_{\rm ILG}}{\sqrt{u^{2}_{\rm Poiss}+u^{2}_{\rm sys}}}\ \in\ [-2,+2];
\quad\text{joint acceptance by Fisher‑combined $Z$ across bins}.
\]

\textbf{Independence plan.} Two photometry/sed‑fitting arms; disjoint completeness simulations; fixed lensing/selection corrections; blind bin‑freezing before unblinding.

\textbf{BLOCKER:} Freeze $N_{\rm ILG}$ predictions, bin edges, and the stellar‑to‑halo map (numbers and code to be included in the artifact pack).

\subsection{Reporting and audit}
Every test above ships with:
(i) a time‑stamped pre‑registration (frozen $k$, $\rho$, endpoints, masks/cuts),
(ii) a manifest with SHA‑256 hashes,
(iii) one‑command runners that rebuild all displayed numbers in a sealed environment, and
(iv) an “independence note’’ describing how cross‑talk was prevented. Negative outcomes are \emph{publishable artifacts} and cannot be retro‑edited to pass.\,:contentReference[oaicite:11]{index=11}

\section{Reproducibility, Artifact Pack, and Audit Trail}
\label{sec:artifacts}

\paragraph{Scope and promise.}
This section nails down a \emph{deterministic}, no‑network artifact pack and an audit trail that lets a referee rebuild the PDF, recompute the bridge displays, and verify checksums in minutes. It follows the bridge semantics and pass/fail policy fixed in the Methods paper (no fits; one inequality; explicit uncertainty and correlation). \emph{Everything required to check the claims is included, frozen, and hashed.}

\subsection*{Files \& hooks (frozen at submission)}
\begin{itemize}
  \item \texttt{paper.tex} and \texttt{paper.pdf} — this manuscript, source and rendered output.
  \item \texttt{IndisputableMonolith-merged.lean} — Lean monolith with theorem identifiers used in the text (exact tag list printed in \texttt{hooks.txt}).
  \item \texttt{hooks.txt} — one‑line anchors mapping text labels to theorem/lemma IDs (e.g.\ \texttt{UP-1}, \texttt{RB-Inv-1}, \texttt{TEST-1}).
  \item \texttt{alpha\_seed\_gap\_curvature.ipynb} — deterministic notebook that recomputes the \(\alpha^{-1}\) assembly from integers + \(\varphi\) (no data, no knobs).
  \item \texttt{display\_calculator.py} — prints the three bridge invariants and the standardized discrepancy \(Z\) for the single pass/fail inequality (reads \texttt{units.toml}).
  \item \texttt{units.toml} — unit labels \(\tau_{0},\ell_{0}\), optional \(\lambda_{\mathrm{rec}}\), declared relative uncertainties \(u(\cdot)\), coverage \(k\), and correlation \(\rho\).
  \item \texttt{invariants.txt} — literal targets (symbols and numbers) that the calculator must print:
\begin{verbatim}
tau_rec/tau0 = 2π/(8 ln φ) = 1.6321256513182483
lambda_kin/ell0 = 2π/(8 ln φ) = 1.6321256513182483
S/ħ = J  (dimensionless, zero offset)
\end{verbatim}
  \item \texttt{makefile} — one‑command builds (PDF, calculator, checksums).
  \item \texttt{versions.txt} — exact toolchain and environment (see below).
  \item \texttt{manifest.txt} — filename, byte size, and SHA‑256 for every file in the pack.
\end{itemize}

\subsection*{Deterministic environment (frozen)}
Set these before \emph{any} build or run:
\begin{verbatim}
LC_ALL=C
TZ=UTC
SOURCE_DATE_EPOCH=1700000000
PYTHONHASHSEED=0
NO_NETWORK=1
\end{verbatim}
Toolchain (locked in \texttt{versions.txt}): \texttt{TeX Live 2024 (pdfTeX)}, \texttt{latexmk}; \texttt{Python 3.11.x}. No internet access is used or required.

\subsection*{One‑command builds}
All commands run from the artifact root and terminate with a zero exit code on success.
\begin{verbatim}
# (1) Deterministic PDF
latexmk -pdf -interaction=nonstopmode -halt-on-error paper.tex

# (2) Bridge calculator (prints invariants and Z)
python3 display_calculator.py --units units.toml --print

# (3) Checksums (manifest regeneration)
python3 - <<'EOF'
import hashlib, os, sys
for fn in sorted(os.listdir('.')):
  if os.path.isfile(fn):
    h=hashlib.sha256(open(fn,'rb').read()).hexdigest()
    print(f"{fn}\t{os.path.getsize(fn)}\tSHA256={h}")
EOF > manifest.txt
\end{verbatim}

\paragraph{Expected console output (calculator).}
The calculator always prints the two invariant ratios and the decision statistic \(Z\) for the single inequality
\[
\left|\frac{\lambda_{\mathrm{kin}}-\lambda_{\mathrm{rec}}}{\lambda_{\mathrm{rec}}}\right|\ \le\ k\,u_{\mathrm{comb}},\quad
u_{\mathrm{comb}}=\sqrt{\,u(\lambda_{\mathrm{kin}})^{2}+u(\lambda_{\mathrm{rec}})^{2}-2\rho\,u(\lambda_{\mathrm{kin}})\,u(\lambda_{\mathrm{rec}})\,}.
\]
A representative run (\texttt{units.toml} where \(\lambda_{\mathrm{rec}}\) is set to the same display as \(\lambda_{\mathrm{kin}}\) to sanity‑check the plumbing) should look like:
\begin{verbatim}
K := 2π/(8 ln φ) = 1.6321256513182483
tau_rec/tau0 = K
lambda_kin/ell0 = K
S/ħ = J
Z = 0.0000   [k = 2, u_comb = 1.0e-05, pass]
\end{verbatim}
If \texttt{units.toml} instead adopts a conventional \(\lambda_{\mathrm{rec}}\) (e.g. \(\sqrt{\hbar G/c^{3}}\)) with its declared \(u(\lambda_{\mathrm{rec}})=\tfrac12 u(G)\) and an independent meter chain \(u(\ell_{0})\), the calculator prints the same invariants and the corresponding \(Z\) using the stated \(k\) and \(\rho\). No regressions or fits are performed—one number decides.

\subsection*{SHA‑256 manifest (exact, human‑readable)}
Every file in the pack is listed as \texttt{<name>\t<size bytes>\tSHA256=<hex>}. The manifest is regenerated by the checksum command above and must not be edited by hand.
\begin{verbatim}
paper.tex              ############    SHA256=################################
paper.pdf              ############    SHA256=################################
IndisputableMonolith-merged.lean ############ SHA256=########################
hooks.txt              ############    SHA256=################################
alpha_seed_gap_curvature.ipynb ############ SHA256=##########################
display_calculator.py  ############    SHA256=################################
units.toml             ############    SHA256=################################
invariants.txt         ############    SHA256=################################
makefile               ############    SHA256=################################
versions.txt           ############    SHA256=################################
manifest.txt           ############    SHA256=################################
\end{verbatim}
\textbf{BLOCKER:} replace \texttt{\#} placeholders with the actual byte sizes and hex digests produced by the artifact pack at freeze time; paste the \emph{exact} output into \texttt{manifest.txt} and include that file in the submission bundle.

\subsection*{Reviewer quick‑start (sub‑10 minutes)}
\begin{enumerate}
  \item \textbf{Compile.} Run the PDF build command. Confirm that \texttt{paper.pdf} is produced without warnings that would alter content (the log is deterministic).
  \item \textbf{Run the calculator.} Execute the one‑liner with the supplied \texttt{units.toml}. Visually confirm the two invariant ratios
  \[
  \frac{\tau_{\mathrm{rec}}}{\tau_{0}}=\frac{2\pi}{8\ln\varphi},\qquad
  \frac{\lambda_{\mathrm{kin}}}{\ell_{0}}=\frac{2\pi}{8\ln\varphi}
  \]
  and record the printed \(Z\), coverage \(k\), and pass/fail status.
  \item \textbf{Confirm hashes.} Regenerate \texttt{manifest.txt}; check that every SHA‑256 and byte size matches the frozen manifest included in the pack. Any mismatch is a red flag; report it.
\end{enumerate}

\paragraph{What this guarantees.}
(i) The bridge display is non‑circular and unique at the stated symmetry (action display \(S/\hbar=J\); clock/length ratios fixed) and (ii) the operational outcome is a single, auditable figure of merit \(Z\) evaluated under predeclared uncertainty and correlation—no knobs, no thresholds, no fits.

\section{Limitations, Scope, and Threats to Validity}
\label{sec:limits-scope-threats}

\subsection*{Scope (what this paper is and is not)}
\begin{itemize}
  \item \textbf{Methods/semantics only.} This paper fixes a parameter‑free, dimensionless proof layer and a single, audited semantics (\emph{Reality Bridge}) for SI displays. It does not introduce sector models or fits.
  \item \textbf{Sector numerics shown only when forced.} Any number printed in the main text is either (i) a bridge identity, (ii) an SI display that cancels under normalization, or (iii) a strict consequence of the fixed semantics with predeclared uncertainty. No priors, no regressions, no threshold tuning.
  \item \textbf{No feedback into proofs.} Empirical data can \emph{test} the semantics; it cannot modify the dimensionless theorems or the bridge invariants.
\end{itemize}

\subsection*{Anchors, constants, and drift}
\begin{itemize}
  \item \textbf{Normalized invariants do not change.} SI recommendations (e.g., future updates to recommended constants) may change \emph{displays}, but not the normalized equalities
  \[
    \frac{\tau_{\mathrm{rec}}}{\tau_{0}}=\frac{2\pi}{8\ln\varphi},
    \qquad
    \frac{\lambda_{\mathrm{kin}}}{\ell_{0}}=\frac{2\pi}{8\ln\varphi},
    \qquad
    \frac{S}{\hbar}=J.
  \]
  These are dimensionless and fixed.
  \item \textbf{Decision rule is invariant.} The sole pass/fail test
  \[
    \left|\frac{\lambda_{\mathrm{kin}}-\lambda_{\mathrm{rec}}}{\lambda_{\mathrm{rec}}}\right|\ \le\ k\,u_{\mathrm{comb}},
    \qquad 
    u_{\mathrm{comb}}=
    \sqrt{\,u(\lambda_{\mathrm{kin}})^{2}+u(\lambda_{\mathrm{rec}})^{2}-2\,\rho\,u(\lambda_{\mathrm{kin}})\,u(\lambda_{\mathrm{rec}})\,},
  \]
  with coverage \(k\in\{1,2\}\), remains the same if recommended numerical values drift; only the reported uncertainties change. No regression, weighting, or re‑anchoring is permitted post hoc.
\end{itemize}

\subsection*{Correlation hazards (how a “pass” can be faked)}
A false pass can occur if the two nominally independent landings share hidden systematics. Typical ways this happens:
\begin{enumerate}
  \item \textbf{Shared calibration chain.} Using the same frequency reference, interferometer, or analysis pipeline in both routes couples errors (\(\rho\!\approx\!+1\)), shrinking \(u_{\mathrm{comb}}\) and masking a discrepancy.
  \item \textbf{Common constants path.} Deriving \(\lambda_{\mathrm{rec}}\) and \(\lambda_{\mathrm{kin}}\) from the \emph{same} evaluation of a recommended constant or the same local code/library injects correlation.
  \item \textbf{Analyst and code reuse.} Reusing analysis scripts or re‑fitting nuisance corrections after viewing the discrepancy leaks information across routes and biases the outcome.
  \item \textbf{Instrument cross‑talk.} Environmental couplings (temperature control, timing distribution, data‑acquisition backplane) used by both routes can synchronize errors.
\end{enumerate}

\paragraph{Design mitigations (what we require).}
\begin{itemize}
  \item \textbf{Disjoint hardware and labs.} Route A (time‑first) and Route B (length‑first) must use different physical instruments, different timing chains, and preferably different organizations; engineer \(\rho \to 0\) by design.
  \item \textbf{Independent software.} Separate repositories, separate reviewers, frozen hashes, and escrowed binaries; no shared utility code between routes.
  \item \textbf{Declared correlation.} If any element is shared, declare a quantitative \(\rho\) and use it in \(u_{\mathrm{comb}}\). If \(\rho\) is unknown, bound it conservatively (worst case).
  \item \textbf{Pre‑registration.} Predeclare anchors, coverage \(k\), the correlation policy, and the acceptance inequality before data are acquired. Optional stopping and re‑weighting are prohibited.
  \item \textbf{Cross‑checks.} Swap instruments across sites (with quarantine), run nulls, and perform role‑rotation to expose latent correlations. Document any deviations.
\end{itemize}

\subsection*{Other threats to validity (and how we contain them)}
\begin{itemize}
  \item \textbf{Anchor fragility.} The length‑first route may adopt a conventional \(\lambda_{\mathrm{rec}}\). If the adopted value or its uncertainty changes, recompute the same inequality with the new \(u(\lambda_{\mathrm{rec}})\); do not retro‑adjust anything else.
  \item \textbf{Garden‑of‑forking‑paths.} Multiple seemingly innocuous choices (windowing, filtering, averaging) can act like tuning knobs. We lock analysis choices ahead of time and forbid scenario shopping; any exploratory path is reported as such and not used for the pass/fail.
  \item \textbf{Environment dependence.} Temperature, vibration, RF pickup, and timing jitter can couple into both routes. We require environmental logging, blind injections (where feasible), and replication on different days and sites.
  \item \textbf{Model slippage.} Interpreting displays in a way that implicitly changes the semantics (e.g., adding offsets to \(S/\hbar\)) is disallowed. Displays are algebraic identities; any alternative interpretation is a new hypothesis and must be labeled as such.
  \item \textbf{Conditional assumptions.} Where downstream statements rely on explicit working assumptions (e.g., a particular global tiling choice), those statements are tagged and understood as conditional. Violating an assumption invalidates only those conditionals, not the bridge or its invariants.
\end{itemize}

\subsection*{Bottom line}
This paper lives or dies by a single auditable comparison under a declared correlation model. A pass indicates operational consistency at the stated coverage; a persistent fail falsifies the present semantics or a landing assumption. No fits, no priors, no regression rescue.

\section{Conclusion}

\noindent\textbf{Binary outcome, no middle ground.}
The program reduces “theory vs.\ experiment” to a single, auditable comparison of two independently realised routes to the same SI display. The decision rule is explicit:
\[
\left|\frac{\lambda_{\mathrm{kin}}-\lambda_{\mathrm{rec}}}{\lambda_{\mathrm{rec}}}\right|
\ \le\
k\,u_{\mathrm{comb}}
\quad\text{with}\quad
u_{\mathrm{comb}}
=\sqrt{\,u(\lambda_{\mathrm{kin}})^{2}+u(\lambda_{\mathrm{rec}})^{2}-2\rho\,u(\lambda_{\mathrm{kin}})u(\lambda_{\mathrm{rec}})\,}.
\]
There are only two outcomes:
\begin{itemize}
  \item \textbf{Pass.} The inequality holds at the predeclared coverage \(k\). The bridge is \emph{operationally consistent} at that precision; sector applications may proceed, but no experimental numbers feed back into the proof layer or the bridge semantics.
  \item \textbf{Falsified.} The inequality fails persistently after controls. Either the present semantics is wrong for the stated mapping or a landing assumption (anchor, traceability, independence, declared \(\rho\)) is invalid. The correct response is to publish the negative result with artifacts; “tuning” is not permitted.
\end{itemize}

\medskip
\noindent\textbf{Exportability (plug–and–test).}
The bridge is not bespoke to one framework. Any candidate theory that can surface its inputs as \emph{dimensionless invariants}—i.e., statements whose numerical content survives unit relabelings—can be evaluated under the \emph{same} single inequality:
\begin{enumerate}
  \item Present the dimensionless core (theorems/identities only; no units, no fits).
  \item Specify two \emph{independent} SI landings (time–first and length–first, or analogous), making unit choices explicit.
  \item Predeclare relative uncertainties \(u(\cdot)\), the correlation \(\rho\) between the landings, and the coverage factor \(k\).
  \item Compute \(\lambda_{\mathrm{kin}}\) and \(\lambda_{\mathrm{rec}}\) from the fixed semantics; evaluate the inequality above.
  \item Ship an artifact pack (scripts, hashes, environment) so any referee can reproduce the pass/fail \(Z\) statistic in minutes.
\end{enumerate}
Success or failure is then a property of the semantics and its landings—not of ad hoc regression. This upgrades “parameter‑free” from rhetoric to an auditable practice and provides a uniform gate for theories to meet the same empirical standard: one number decides.

% Appendix A — Non‑circularity & Uniqueness of the Reality Bridge
% Source context for notation and statements:
% Reality-bridge semantics and invariants (time-first vs length-first landings). :contentReference[oaicite:0]{index=0}
% Integrated recognition framework (cost J, eight‑tick, golden‑ratio gap). :contentReference[oaicite:1]{index=1}

\section*{Appendix A — Non‑circularity and Uniqueness (Unit relabelings, invariants, algebraic elimination)}

\noindent\textbf{Standing notation.}
Let
\[
K\;:=\;\frac{2\pi}{8\,\ln\varphi}\quad(\text{dimensionless}),\qquad
c\;:=\;\frac{\ell_{0}}{\tau_{0}}\quad(\text{display identity}),
\]
and define the bridge displays
\[
\tau_{\mathrm{rec}}=K\,\tau_{0},\qquad
\lambda_{\mathrm{kin}}=c\,\tau_{\mathrm{rec}}=K\,\ell_{0},\qquad
\frac{S}{\hbar}=J.
\]
Here $J$ is the unique symmetric multiplicative cost, $\varphi$ the golden ratio, $\tau_{0}$ and $\ell_{0}$ are unit \emph{names} (second, meter). No tunable parameters appear anywhere in these identities.  % :contentReference[oaicite:2]{index=2} :contentReference[oaicite:3]{index=3}

\begin{definition}[Unit relabeling]
A unit relabeling is a pair $(\alpha,\beta)\in\mathbb{R}_{>0}^{2}$ acting by
\[
(\tau_{0},\ell_{0})\mapsto(\tau_{0}',\ell_{0}')=(\alpha\,\tau_{0},\,\beta\,\ell_{0}),
\quad\text{so}\quad
c\mapsto c'=\frac{\ell_{0}'}{\tau_{0}'}=\frac{\beta}{\alpha}\,c.
\]
\end{definition}

\begin{lemma}[Equivariance of the displays]
Under any unit relabeling $(\alpha,\beta)$ one has
\[
\tau_{\mathrm{rec}}'=\alpha\,\tau_{\mathrm{rec}},\qquad
\lambda_{\mathrm{kin}}'=\beta\,\lambda_{\mathrm{kin}},\qquad
\Bigl(\frac{S}{\hbar}\Bigr)'=\frac{S}{\hbar}.
\]
\emph{Proof.} Directly from the defining identities:
$\tau_{\mathrm{rec}}'=K\,\tau_{0}'=\alpha\,\tau_{\mathrm{rec}}$ and
$\lambda_{\mathrm{kin}}'=c'\tau_{\mathrm{rec}}'=\frac{\beta}{\alpha}c\cdot\alpha\tau_{\mathrm{rec}}=\beta\,\lambda_{\mathrm{kin}}$.
The ratio $S/\hbar$ is dimensionless and unchanged. \qed
\end{lemma}

\begin{theorem}[Non‑circularity (invariants under unit relabeling)]
The normalized displays
\[
\boxed{\ \frac{\tau_{\mathrm{rec}}}{\tau_{0}}=K,\qquad
        \frac{\lambda_{\mathrm{kin}}}{\ell_{0}}=K,\qquad
        \frac{S}{\hbar}=J\ }
\]
are invariant under every unit relabeling $(\alpha,\beta)$.
Consequently, unit choices cannot alter any dimensionless theorem nor any bridge equality. \qed
\end{theorem}

\begin{proposition}[Zero‑offset action display]
Among affine maps $S/\hbar=a\,J+b$ with $a>0$, the bridge forces $b=0$ from $J(1)=0$ and $a=1$ from the log‑axis normalization $J(e^{t})=\cosh t-1$ at $t=0$. Hence $\;S/\hbar\equiv J$. \qed
\end{proposition}

\begin{proposition}[Uniqueness at the stated symmetry]
Impose simultaneously: (i) multiplicative symmetry of $J$ is preserved in the action display and no offset is introduced; (ii) one eight‑tick cycle corresponds to a full $2\pi$ phase in the clock display; (iii) hop length is purely kinematic: $\lambda=c\,\tau$. Then, up to a unit relabeling $(\alpha,\beta)$,
\[
\tau_{\mathrm{rec}}=K\,\tau_{0},\qquad \lambda_{\mathrm{kin}}=c\,\tau_{\mathrm{rec}},\qquad S/\hbar=J
\]
is the unique bridge. \emph{Sketch.} (i) fixes $S/\hbar=J$; (ii) fixes the proportionality $\tau_{\mathrm{rec}}/ \tau_{0}=K$; (iii) then forces $\lambda_{\mathrm{kin}}=c\,\tau_{\mathrm{rec}}$. Any other assignment differs only by $(\alpha,\beta)$, which leaves the normalized statements unchanged. \qed  % :contentReference[oaicite:4]{index=4}

\begin{lemma}[Algebraic elimination of unit labels]
Let $\mathcal{A}$ be the commutative algebra generated by $\{\tau_{0},\ell_{0},c,\tau_{\mathrm{rec}},\lambda_{\mathrm{kin}},S/\hbar\}$ and constants $\{K,J\}$ with the bridge relations
\[
c=\frac{\ell_{0}}{\tau_{0}},\qquad \tau_{\mathrm{rec}}=K\,\tau_{0},\qquad
\lambda_{\mathrm{kin}}=K\,\ell_{0},\qquad \frac{S}{\hbar}=J.
\]
If $F\in\mathcal{A}$ is dimensionless (invariant under all $(\alpha,\beta)$), then $F=f(K,J)$ for a unique real function $f$. \emph{Proof.} Substitute the relations to write $F$ as a monomial in $(\tau_{0},\ell_{0})$ times a function of $(K,J)$. Invariance forces all unit‑label monomials to cancel, leaving $f(K,J)$. \qed
\end{lemma}

\begin{theorem}[Universal factorization of any dimensionless pipeline]
Any dimensionless real‑valued pipeline $\Pi$ built from algebraic operations and limits on the bridge displays factors uniquely through $(K,J)$:
\[
\Pi \equiv f\!\left(K,J\right).
\]
Hence no display‑level choice (weights, offsets, thresholds) can modify a normalized outcome; only $(K,J)$ matter. \qed  % :contentReference[oaicite:5]{index=5}
\end{theorem}

\begin{corollary}[Knob‑nullity]
Let $\theta$ denote any continuous “display‑level” adjustment external to the bridge equations. For each normalized observable $N\in\{\tau_{\mathrm{rec}}/\tau_{0},\,\lambda_{\mathrm{kin}}/\ell_{0},\,S/\hbar\}$,
\[
\frac{\partial N}{\partial\theta}=0.
\]
Thus, there is no knob that can change the decision‑relevant, normalized statements. \qed
\end{corollary}

\begin{remark}[Audit‑ready summary]
The bridge exposes exactly three auditable invariants,
\[
\frac{\tau_{\mathrm{rec}}}{\tau_{0}}=\frac{2\pi}{8\ln\varphi},\qquad
\frac{\lambda_{\mathrm{kin}}}{\ell_{0}}=\frac{2\pi}{8\ln\varphi},\qquad
\frac{S}{\hbar}=J,
\]
and proves that unit relabelings cannot feed parameters back into the derivation layer. Uniqueness holds at the stated symmetry; all differences reduce to trivial unit names.  % :contentReference[oaicite:6]{index=6} :contentReference[oaicite:7]{index=7}
\end{remark}

%===============================
% App. B — Gap–series convergence
% Status: [T] (purely dimensionless; bridge-ready)
% Provenance note: The same closed form and convergence/tail bounds
% are consistent with the internal drafts “Recognition Science” (App. M)
% and “From Proof to Measurement” (Reality Bridge) in the artifact pack.
% :contentReference[oaicite:0]{index=0}  :contentReference[oaicite:1]{index=1}
%===============================
\section*{Appendix B — Gap–Series Convergence}
\emph{(EMR‑b: this section is proof‑layer; no SI, no fits.)}

\begin{definition}[Golden ratio and gap coefficients]
\label{def:gap-coeff}
Let \(\displaystyle \varphi:=\frac{1+\sqrt5}{2}\) and define, for \(|z|\le 1\),
\[
  \mathcal F(z)\;:=\;\sum_{m=1}^{\infty} g_{m}\,z^{m},
  \qquad
  g_{m}\;:=\;\frac{(-1)^{m+1}}{m\,\varphi^{\,m}}\;\;\;(m\in\mathbb N).
\]
We call \(f_{\mathrm{gap}}:=\mathcal F(1)\) the \emph{ledger gap}. 
\end{definition}

\begin{proposition}[Closed form (forced generating functional)]
\label{prop:closed-form}
For every complex \(z\) with \(|z|\le 1\),
\[
  \boxed{ \ \mathcal F(z) \;=\; \ln\!\Bigl(1+\frac{z}{\varphi}\Bigr) \ } .
\]
In particular \(f_{\mathrm{gap}}=\mathcal F(1)=\ln(1+\varphi^{-1})=\ln\varphi\).
\end{proposition}

\begin{proof}
Use the standard power series \(\ln(1+w)=\sum_{m\ge1}(-1)^{m+1}w^{m}/m\) for \(|w|\le1\) with \(w=z/\varphi\). Term‑by‑term identification gives the stated coefficients \(g_m\) and the closed form. \qedhere
\end{proof}

\begin{lemma}[Monotone decay and ratio bound]
\label{lem:ratio}
For \(|z|\le1\),
\[
  \bigl|g_{m+1}z^{m+1}\bigr|
  \;=\;\frac{|z|^{\,m+1}}{(m+1)\,\varphi^{\,m+1}}
  \;\le\;\frac{|z|}{\varphi}\,\frac{m}{m+1}\,\bigl|g_m z^m\bigr|
  \;<\;\bigl|g_m z^m\bigr|.
\]
Thus \(\{|g_m z^m|\}_{m\ge1}\) is strictly decreasing whenever \(|z|<\varphi\), in particular for all \(|z|\le 1\).
\end{lemma}

\begin{theorem}[Absolute and uniform convergence on the unit disk]
\label{thm:absolute}
The series \(\sum_{m\ge1} g_m z^m\) converges absolutely and uniformly on \(\{\,z\in\mathbb C:\ |z|\le1\,\}\). Moreover, for the remainder \(R_n(z):=\mathcal F(z)-\sum_{m=1}^{n}g_m z^m\),
\[
  \boxed{\;
    |R_n(z)|
    \;\le\;
    \frac{|z|^{\,n+1}}{(n+1)\,\varphi^{\,n+1}}\;
    \frac{1}{1-\tfrac{|z|}{\varphi}}
    \qquad (|z|<\varphi)
  \;}
\]
and the bound is valid in the limit \(|z|\!\uparrow\!1\).
\end{theorem}

\begin{proof}
By the ratio bound in Lemma~\ref{lem:ratio}, \(|g_{m}z^m|\le C\,\rho^{\,m}\) for some \(C>0\) and \(\rho<1\) on any compact subset of \(|z|<\varphi\), giving absolute and uniform convergence by the Weierstrass \(M\)-test. Summing the geometric majorant for the tail yields the displayed inequality; continuity in \(|z|\) gives the \(|z|=1\) boundary. \qedhere
\end{proof}

\begin{proposition}[Sharp alternating‑tail bound on the real segment]
\label{prop:alt-tail}
For real \(z\in[0,1]\) the terms alternate in sign and decrease in magnitude, hence the alternating‑series estimate applies:
\[
  \boxed{\;
  |R_n(z)|\le \frac{z^{\,n+1}}{(n+1)\,\varphi^{\,n+1}}\;.
  \;}
\]
In particular, at \(z=1\), \(|R_n(1)|\le 1/((n+1)\varphi^{\,n+1})\).
\end{proposition}

\begin{proof}
From Lemma~\ref{lem:ratio}, \(|g_{m+1}z^{m+1}|<|g_m z^m|\) for \(z\in(0,1]\); signs alternate by construction. The Leibniz criterion gives the bound. \qedhere
\end{proof}

\begin{corollary}[Audit‑ready error budgets]
\label{cor:budget}
To guarantee a target tolerance \(\varepsilon>0\) at \(z=1\), it suffices to take
\[
  n\ \ge\ \min\Bigl\{\,N\in\mathbb N:\ \frac{1}{(N+1)\,\varphi^{\,N+1}}\le\varepsilon\Bigr\}
  \quad\text{(sharp, alternating)},
\]
or, uniformly for all \(|z|\le1\),
\[
  n\ \ge\ \min\Bigl\{\,N:\ \frac{1}{(N+1)\,\varphi^{\,N+1}}\,
        \frac{1}{1-\varphi^{-1}}\le\varepsilon\Bigr\}
  \;=\;
  \min\Bigl\{\,N:\ \frac{\varphi^{2}}{(N+1)\,\varphi^{\,N+1}}\le\varepsilon\Bigr\}.
\]
Both criteria are free of external inputs and are reproducible from Definition~\ref{def:gap-coeff}.
\end{corollary}

\paragraph{Remarks.}
(i) Proposition~\ref{prop:closed-form} fixes \(f_{\mathrm{gap}}=\ln\varphi\) exactly; the tail bounds quantify truncation in any finite‑order display. (ii) The uniform bound in Theorem~\ref{thm:absolute} is convenient for error control on disks \(|z|\le r<1\); the alternating bound in Proposition~\ref{prop:alt-tail} is strictly tighter at \(z=1\). (iii) All statements are dimensionless and sit entirely within the proof layer, consistent with the Reality‑Bridge semantics used elsewhere in the artifact pack. % :contentReference[oaicite:2]{index=2}  :contentReference[oaicite:3]{index=3}

%========================
% Appendix C — Curvature integral and voxel geometry
% (Regge hinges; 16 seams; 102 pyramids; normalization by 2\pi^5; sign conventions)
% Status: EMR-b
%========================

\section*{Appendix C.\ Curvature Integral and Voxel Geometry}

\noindent
\textbf{Goal.} Compute the dimensionless curvature contribution per voxel using a Regge-style discretization built on the Recognition voxel. The result feeds the $\alpha$ pipeline as a fixed, knobless constant.

\subsection*{C.1  Geometry of the recognition voxel}

\paragraph{Voxel and seam set.}
Start from the unit cube and identify opposite faces by glide–reflections, producing a flat three–torus with a finite set of \emph{gluing seams} (one–dimensional loci where the identifications live). In the canonical construction there are precisely
\[
\boxed{\,16\,\text{seams}\,}
\]
arranged in four parallel families. These seams are the only places where curvature can localize in the Regge picture (hinges).%
% source (seams): :contentReference[oaicite:0]{index=0}

\paragraph{Center–pyramids partition.}
Partition the voxel into congruent Euclidean pyramids whose common apex is the voxel center; their bases lie on the boundary complex determined by the seam layout. The canonical partition uses
\[
\boxed{\,102\,\text{pyramids}\,}
\]
meeting at the apex; equivalently, the full $2\pi$ azimuth around the apex decomposes into $103$ equal dihedral sectors, of which exactly one is \emph{absent} by the gluing, leaving a uniform \emph{deficit} per present sector.%
% source (102 pyramids, 103-sector picture): :contentReference[oaicite:1]{index=1}

\subsection*{C.2  Regge hinges, dihedral sectors, and the local deficit}

\paragraph{Regge set–up (3D).}
In three dimensions, the (integrated) scalar curvature of a piecewise–Euclidean complex localizes on \emph{edges} (hinges). For a hinge $h$ the local contribution is the \emph{deficit angle}
\[
\delta_h \;=\; 2\pi - \sum_{\sigma\supset h}\theta_{\sigma h},
\]
where $\theta_{\sigma h}$ are the dihedral angles of incident 3–cells $\sigma$ about $h$. The total (dimensionful) integral is a hinge–sum weighted by edge measures; in the present voxel normalization the relevant edge factors are unity, so the integrated scalar curvature reduces to a pure sum of deficits.

\paragraph{Uniform wedge count at the center.}
By construction the $2\pi$ azimuth about the apex is divided into $103$ equal dihedral sectors; the gluing eliminates one of them. Hence each \emph{retained} sector carries a uniform deficit
\[
\boxed{\,\Delta\theta \,=\, \frac{2\pi}{103}\,}.
\]
Summing over the $102$ present sectors yields the voxel–integrated scalar curvature
\begin{equation}
\label{eq:voxel-curv-sum}
\int_{\mathscr V}\! R\,\sqrt{g}\,d^{3}x
\;=\;
\sum_{\text{sectors}} \Delta\theta
\;=\;
102\,\frac{2\pi}{103}
\;=\;
2\pi\!\left(1-\frac{1}{103}\right).
\end{equation}
This is \emph{not} a fit: the integers $(16,102,103)$ are fixed by the seam gluing and the center–pyramids combinatorics of the voxel.%
% source (uniform deficit; eq. \eqref{eq:voxel-curv-sum}): :contentReference[oaicite:2]{index=2}

\subsection*{C.3  Normalization to a dimensionless curvature constant}

\paragraph{Bridge normalization.}
Define the dimensionless curvature content per voxel by dividing the integrated curvature by the fixed phase–space normalization
\[
\boxed{\,N_\kappa \;=\; 2\pi^{5}\,},
\]
the same constant that appears in the geometric seed of the $\alpha$ pipeline and in the Reality–Bridge displays.%
% source (choice and role of $2\pi^{5}$): :contentReference[oaicite:3]{index=3}

\noindent
With \eqref{eq:voxel-curv-sum},
\begin{equation}
\label{eq:I-kappa}
\boxed{\;
\mathcal I_{\kappa}
\;:=\;
\frac{1}{N_\kappa}\,
\int_{\mathscr V}\! R\,\sqrt{g}\,d^{3}x
\;=\;
\frac{1}{2\pi^{5}}\;
102\,\frac{2\pi}{103}
\;=\;
\frac{103}{102\,\pi^{5}}
\;}
\end{equation}
and the sign convention of the main text (\S5) then assigns the bridge–level \emph{curvature correction}
\begin{equation}
\label{eq:delta-kappa}
\boxed{\;
\delta_{\kappa}
\;=\;
-\mathcal I_{\kappa}
\;=\;
-\frac{103}{102\,\pi^{5}}
\;}.
\end{equation}
Numerically, $\delta_{\kappa}=-0.003\,299\,762\,049\ldots$ (units suppressed: this is a pure number).%
% source (closed forms for $\mathcal I_{\kappa}$ and $\delta_{\kappa}$): :contentReference[oaicite:4]{index=4}

\subsection*{C.4  Sign conventions and invariances}

\paragraph{Orientation/sign.}
We adopt the \emph{ledger} sign convention: positive curvature removes effective recognition states. Hence the bridge injects curvature as a \emph{negative} additive term at assembly (Eq.\,\eqref{eq:delta-kappa}). Reversing global orientation leaves \eqref{eq:I-kappa} invariant (deficits are azimuthal measures) and flips neither the seam count nor the sector count, so the sign of $\delta_{\kappa}$ is fixed by semantics, not by orientation.

\paragraph{Gauge–rigidity.}
Any re–tiling of the voxel that preserves the $(16,102,103)$ combinatorics and the glide identifications produces the same $\mathcal I_{\kappa}$. In particular, triangulated refinements (subdividing pyramids into tetrahedra) do not change the sum of deficits and therefore leave \eqref{eq:I-kappa} unchanged.%
% source (rigidity of the integers; independence of refinement): :contentReference[oaicite:5]{index=5}

\subsection*{C.5  Audit recipe (deterministic)}

\begin{enumerate}
  \item \textbf{Build} the center–pyramids partition (102 cells meeting at the voxel center).
  \item \textbf{Enumerate} the $103$ equal dihedral sectors around the apex and \emph{remove} one sector to encode the gluing; set $\Delta\theta=2\pi/103$.
  \item \textbf{Sum} the $102$ identical deficits to get \eqref{eq:voxel-curv-sum}.
  \item \textbf{Normalize} by $N_{\kappa}=2\pi^{5}$ to obtain $\mathcal I_{\kappa}$ in \eqref{eq:I-kappa}.
  \item \textbf{Apply} the bridge sign convention to report $\delta_{\kappa}$ as in \eqref{eq:delta-kappa}.%
  % source (normalization and reporting via the Reality Bridge): :contentReference[oaicite:6]{index=6}
\end{enumerate}

\paragraph{Outcome.}
The voxel curvature term is a rigid, parameter–free constant:
\[
\delta_{\kappa}=-\dfrac{103}{102\,\pi^{5}},
\]
fixed entirely by the seam gluing and the center–pyramids combinatorics, and normalized by the bridge constant $2\pi^{5}$. No sector inputs, fits, or thresholds enter anywhere in the construction.%
% sources (final value; no-knob status): :contentReference[oaicite:7]{index=7} :contentReference[oaicite:8]{index=8}

%========================
% Appendix D — Unified Mass Formula
%========================

\section*{Appendix D — Unified Mass Formula}
\addcontentsline{toc}{section}{Appendix D — Unified Mass Formula}

\subsection*{D.0  Scope and notation}
This appendix fixes three ingredients of the mass ladder used in the main text:
(i) the \emph{rung} \(r_i\) of each irreducible field \(\psi_i\), obtained as the minimal word length on the ledger graph;
(ii) the sector prefactor
\[
\boxed{\,B_i \;=\; 2^{\,n_c}\,}
\]
from independent ledger channels; and
(iii) the mapping from \emph{minimal words} (constructed from elementary loops) to \((r_i,n_c)\).
No SI numbers enter here; bridge‑level displays and any RG residues \(f_i\) are deferred to the Reality‑Bridge layer.%
% Derivation layer only; bridge and RG residues are treated separately in the methods paper.  :contentReference[oaicite:0]{index=0}

\subsection*{D.1  Ledger words, loops, and the rung \(r_i\)}
Let \(\mathscr L\) be the ledger graph whose edges implement the 16 LNAL opcodes, and let
\(\pi_1(\mathscr L)\cong C_3 * C_2 * C_\infty\) be generated by the primitive oriented loops
\[
L_C,\qquad L_T,\qquad L_Y,
\]
corresponding (respectively) to the SU(3) color, SU(2) isospin, and U(1) hypercharge factors.%
% Loop-length basis and free-product normal form appear in the ledger–graph section.  :contentReference[oaicite:1]{index=1}

\begin{definition}[Minimal word and rung]
For an irreducible field \(\psi_i\) with gauge charges \((C,T,Y)\), let \(\Gamma_i\) be the unique reduced word in the normal form
\(
\Gamma_i \sim L_C^{\,n_C} L_T^{\,n_T} L_Y^{\,n_Y}
\)
that realizes those charges with the fewest edges. The \emph{rung} is
\[
\boxed{\,r_i := |\Gamma_i|\,},
\]
the reduced word length (edge count) of \(\Gamma_i\).
\end{definition}

\begin{lemma}[Loop‑length basis]
Every reduced loop has a unique decomposition
\(
\omega \sim L_C^{\,n_C} L_T^{\,n_T} L_Y^{\,n_Y}
\)
with \(n_C\in\{0,1,2\}\), \(n_T\in\{0,1\}\), \(n_Y\in\mathbb Z\). Moreover, each \(L_G\) already has minimal positive length in its cyclic factor. \hfill\emph{(Basis lemma)}
% Unique normal-form and minimality of primitive loops.  :contentReference[oaicite:2]{index=2}
\end{lemma}

\begin{theorem}[Constructor and minimality]
Given \((Y,T,C)\) for \(\psi_i\), the following algorithm returns the unique \(\Gamma_i\) of minimal length:
\begin{enumerate}
\item Map charges to primitive loops:
\(\;|6Y|\) copies of \(L_Y\) (sign by \(\mathrm{sgn}\,Y\));
append \(L_T\) iff \(T=\tfrac12\);
append \(L_C\) iff \(C\) is fundamental (mod \(3\)).
\item Concatenate in fixed order \(C\!\to\!T\!\to\!Y\) and reduce by deleting adjacent inverse pairs.
\end{enumerate}
The output has the shortest possible length among all words realizing \((Y,T,C)\); hence its edge count equals \(r_i\).
\hfill\emph{(Minimal‑hop uniqueness)}
% Constructor, reduction, and minimality proofs appear in the rung‑uniqueness section.  :contentReference[oaicite:3]{index=3}
\end{theorem}

\paragraph{Path–cost correspondence.}
On \(\pi_1(\mathscr L)\) the ledger cost is linearly proportional to reduced word length:
\[
J(\psi_i) \;=\; J_{\rm bit}\, r_i,\qquad J_{\rm bit}=\ln\varphi.
\]
Thus the integer \(r_i\) is a \emph{complete} dimensionless descriptor of the proof‑layer cost for \(\psi_i\).%
% Measure‑preserving isomorphism (path length ↔ ledger cost) with J_bit=ln φ.  :contentReference[oaicite:4]{index=4}

\subsection*{D.2  Channel multiplicity and the sector prefactor \(B_i\)}
Intuitively, each independent ledger channel offers a binary orientation (dual path) that survives reduction and factorizes in the saddle‑point sum, doubling the amplitude weight. Independence multiplies these factors.

\begin{definition}[Independent ledger channels]
Decompose the reduced word \(\Gamma_i\) into commuting channel factors
\[
\Gamma_i \;\equiv\; \prod_{\chi=1}^{n_c} \Gamma_i^{(\chi)},
\qquad
\bigl[\,\Gamma_i^{(\chi)},\Gamma_i^{(\chi')}\,\bigr]=e
\;\;(\chi\neq\chi').
\]
A factor \(\Gamma_i^{(\chi)}\) is a channel iff it (i) toggles a \(\mathbb Z_2\) parity class that is \emph{not} a function of the other factors and (ii) admits two time‑reversed orientations that are distinct in \(\pi_1(\mathscr L)\) after reduction. The number \(n_c\) is the count of such independent factors for \(\psi_i\).%
% Channels defined so that each contributes an independent Z2 orientation not removed by reduction; ties to the nine ledger parities upstream.  :contentReference[oaicite:5]{index=5}
\end{definition}

\begin{lemma}[Binary saddle per channel]
For each channel \(\Gamma_i^{(\chi)}\), dual‑balance produces two inequivalent reduced orientations related by time‑reversal/conjugation. Their contributions add incoherently at the mass‑ladder level, yielding a multiplicity factor \(2\). \hfill\emph{(Orientation lemma)}
% The binary (±) orientation multiplicity arises from dual-balance; see ledger dual paths.  :contentReference[oaicite:6]{index=6}
\end{lemma}

\begin{theorem}[Sector prefactor]
Let \(n_c\) be the number of independent ledger channels of \(\psi_i\) in the sense above. Then the sector multiplicity is
\[
\boxed{\;B_i \;=\; 2^{\,n_c}\;}
\]
and depends only on the channel structure of the \emph{minimal} word \(\Gamma_i\).
\end{theorem}

\begin{proof}[Proof sketch]
Write the path–integral amplitude as a product over commuting channel saddles:
\(
\mathcal A(\Gamma_i)\propto \prod_{\chi=1}^{n_c}\bigl(\mathcal A^{(\chi)}_{+}+\mathcal A^{(\chi)}_{-}\bigr).
\)
Dual‑balance enforces \(|\mathcal A^{(\chi)}_{+}|=|\mathcal A^{(\chi)}_{-}|\) and orthogonality at the ledger level, so each bracket contributes an overall factor \(2\) to the norm. Independence of channels (commuting factors that toggle distinct \(\mathbb Z_2\) parities) makes the multiplicity multiplicative, giving \(2^{n_c}\). Reduction cannot remove an independent channel by definition, so \(B_i\) is invariant under word minimization.%
% The path-factorization and channel independence combine with the earlier statement “B_i = 2^{n_c} from path-integral multiplicity”.  :contentReference[oaicite:7]{index=7}
\end{proof}

\paragraph{Remarks.}
(i) \(n_c\) is \emph{not} ``number of gauge factors'' in general; it is the number of \emph{independent binary orientations} that survive reduction for \(\Gamma_i\).
(ii) For fields without any such channel (\(n_c=0\)), \(B_i=1\).

\subsection*{D.3  Mapping: charges \(\to\) minimal words \(\to\) \((r_i,n_c)\)}
Given \((Y,T,C)\):
\begin{enumerate}
\item \textbf{Build \(\widetilde\Gamma_i\).} Append \(|6Y|\) copies of \(L_Y\) (sign by \(\mathrm{sgn}\,Y\)); append \(L_T\) iff \(T=\tfrac12\); append \(L_C\) iff \(C\) is fundamental.
\item \textbf{Reduce.} Cancel adjacent inverses to get \(\Gamma_i\) and read off \(r_i=|\Gamma_i|\).
\item \textbf{Factor channels.} Partition \(\Gamma_i\) into commuting factors that toggle distinct \(\mathbb Z_2\) parities (color line, weak doublet flip, hypercharge orientation, tick parity, etc.). Count them to obtain \(n_c\).
\end{enumerate}
Steps (1)–(2) are purely combinatorial and unique; step (3) uses the parity taxonomy fixed upstream and does not involve any fit.%
% Constructor and parity‑based channel counting; see nine parity classes and the H.1–H.4 lemmas.  :contentReference[oaicite:8]{index=8}

\subsection*{D.4  Statement of the Unified Mass Formula (dimensionless skeleton)}
At the derivation layer the mass ladder takes the form
\[
\boxed{\;
m_i \;\propto\; B_i \;\varphi^{\,r_i + f_i}\;},
\qquad
B_i=2^{n_c},\;\; r_i=|\Gamma_i|.
\]
Here \(f_i\) is a \emph{dimensionless} residue fixed downstream by a bridge‑level RG integral with predeclared inputs; no regression or sector fits are permitted in the proof layer. SI displays and uncertainties are introduced only by the Reality Bridge and never feed back into \((B_i,r_i)\).%
% Skeleton as presented in the unified‑mass appendix and deferred SI landing in the methods paper.  :contentReference[oaicite:9]{index=9} :contentReference[oaicite:10]{index=10}

\paragraph{Status tags.}
\([T]\) \(r_i\) from minimal words and path–cost isomorphism; \([T]\) \(B_i=2^{n_c}\) from channel multiplicity; \([R]\) \(f_i\) as a fixed RG residue (no knobs) evaluated at the bridge; \([P]\) SI numbers only after the single pass/fail landing.%
% Division of labor between theorem layer and bridge/phenomenology layer per methods paper.  :contentReference[oaicite:11]{index=11}

\subsection*{D.5  Schematic examples (word \(\to\) rung \(\to\) channels)}
\begin{itemize}
\item \textbf{Lepton doublet component} (\(T=\tfrac12\), colorless, modest \(|Y|\)): \(\Gamma\sim L_T L_Y^{\,n}\) reduces without cancellations. Then \(r=|L_T|+|n|\,|L_Y|\). Typical independent channels: weak doublet flip and hypercharge orientation \(\Rightarrow n_c=2 \Rightarrow B=4\).
\item \textbf{Color triplet quark} (\(C=\mathbf3\), optionally \(T=\tfrac12\), nonzero \(Y\)): \(\Gamma\sim L_C^{\,1} L_T^{\,\epsilon_T} L_Y^{\,n}\) with \(\epsilon_T\in\{0,1\}\). Channel set typically includes color‑line orientation, (if present) weak flip, and hypercharge orientation \(\Rightarrow n_c\in\{2,3\}\).
\end{itemize}
These are \emph{schematics}; exact \((r_i,n_c)\) follow from the constructor and channel rules above, not from examples.

\subsection*{D.6  Audit hook (deterministic enumeration)}
A referee can reproduce \((r_i,n_c)\) from \((Y,T,C)\) by running a 20‑line enumerator:
(1) build \(\widetilde\Gamma\) from charges, (2) reduce, (3) factor commuting toggles of distinct \(\mathbb Z_2\) parities, (4) return \(|\Gamma|\) and the count. No external data, no fits, and no SI values are involved at any step.%
% Deterministic, knob‑free enumeration aligns with the artifact/audit policy in the methods paper.  :contentReference[oaicite:12]{index=12}

% ---------- App. E ----------
\section*{Appendix E: LNAL instruction set \& program$\to$observable examples (cavity null test)}
\noindent\emph{Status:} [T/R] for instruction‑set structure and mapping; [P] for lab‑level display conventions. \emph{Bridge layer:} EMR‑b.

\subsection*{E.1  LNAL: minimal instruction set and registers}

\paragraph{Design goals.}
LNAL is the minimal reversible calculus that (i) posts and settles double‑entry costs, (ii) effects the unique $\varphi$‑scaling and its inverse, (iii) routes or halts flow across voxel edges, and (iv) reads/instantiates ledger state, all at the 8‑beat granularity.

\paragraph{Registers (six interaction channels).}
A recognition program addresses a 6‑tuple of channels:
\[
(\nu_{\varphi},\,\ell,\,\sigma,\,\tau,\,k_{\perp},\,\varphi_{e}),
\]
standing respectively for $\varphi$‑frequency, orbital mode, polarization, tick bin, transverse mode, and entanglement phase. These six arise as $8-2$: the eight degrees of the 8‑beat clock minus the two dual‑balance constraints.

\paragraph{Opcode classes (no table).}
LNAL contains exactly sixteen primitive, mutually inverse opcodes grouped into four dual pairs:
\begin{itemize}
  \item \textbf{Ledger moves:} \texttt{LOCK}, \texttt{BALANCE} \;|\; \texttt{GIVE}, \texttt{REGIVE}.
  \item \textbf{Energy scaling/fusion:} \texttt{FOLD}, \texttt{UNFOLD} \;|\; \texttt{BRAID}, \texttt{UNBRAID}.
  \item \textbf{Transport/suspension:} \texttt{FLOW}, \texttt{STILL} \;|\; \texttt{HARDEN}, \texttt{SEED}.
  \item \textbf{I/O and instantiation:} \texttt{LISTEN}, \texttt{ECHO} \;|\; \texttt{SPAWN}, \texttt{MERGE}.
\end{itemize}
Each class implements one of the four functional pairs (post/settle, scale/invert, move/hold, read/instantiate); reversibility requires an explicit undo for each.

\paragraph{Minimality (sketch).}
Because the 8‑beat/dual partition enforces \emph{two} mutually inverse primitives per functional pair, four pairs give a lower bound of $8$; additivity and reversibility forbid fusing pairs without violating atomic tick granularity, and independent control over posting/transfer, scaling/fusion, flow/stillness, and read/spawn doubles the bound to \emph{sixteen}. Hence $16$ is minimal at the stated symmetry, and any strict subset fails completeness or reversibility.

\paragraph{Macros and hygiene.}
Composite shorthand is permitted (e.g.\ \texttt{HARDEN} $:=$ four \texttt{FOLD}s $+$ \texttt{BRAID}). Unused seeds are garbage‑collected after $\lceil\varphi^{2}\rceil$ cycles to prevent latent cost accumulation; operationally: clear on the third cycle.

\subsection*{E.2  Program semantics and the Reality Bridge (no knobs)}

\paragraph{Dimensionless display.}
The unique symmetric multiplicative cost $J$ displays as action with \emph{no offset}:
\[
\boxed{\,S/\hbar\;\equiv\;J\,}.
\]
Clock/length displays are fixed by the 8‑beat gap,
\[
\frac{\tau_{\mathrm{rec}}}{\tau_{0}}=\frac{2\pi}{8\ln\varphi},
\qquad
\frac{\lambda_{\mathrm{kin}}}{\ell_{0}}=\frac{2\pi}{8\ln\varphi},
\quad
\lambda_{\mathrm{kin}}=c\,\tau_{\mathrm{rec}},
\]
and are invariant under unit relabelings. These identities define the \emph{only} permitted semantics from programs to observables at bridge level.

\paragraph{Program$\to$observable map (schematic).}
A program $P$ composes opcodes on registers to form an observable $O=\mathbf O(P)$ which then factors through the units quotient before numerical display:
\[
\text{Programs}\xrightarrow{\;\mathbf O\;}\text{Observables}\xrightarrow{\;\mathbf Q\;}\text{Obs}/\!\sim_{\text{units}}\xrightarrow{\;\widetilde{\mathbf A}\;}\mathbb R,
\]
so that every printed number is the image of a dimensionless invariant; no sector models, regressions, or offsets appear in this layer.

\subsection*{E.3  Cavity line example and null test}

\paragraph{Setup.}
A single‑mode cavity (frequency $\nu$) is initialised to vacuum. One output port is monitored by a \texttt{LISTEN} device; the scheduler tick is the universal tick, and linewidth is set by the tick budget (bridge‑level).

\paragraph{Program.}
\[
\texttt{SEED}\xrightarrow{\;\varphi^{-1}\;}\texttt{FOLD}\xrightarrow{\;\texttt{FLOW}\;}\texttt{LISTEN}
\]
Semantics: post a $+1$ seed; rescale by $\varphi^{-1}$ to a single quantum; route to the detector port; read.

\paragraph{Prediction (display).}
The port spectrum shows a Lorentzian line centred at $\nu$ with unit area (one quantum) and width governed by the scheduler tick (scales as $1/\tau_{0}$). The program’s action display equals the target $J$ for the chosen log‑axis stretch; no additional parameters enter.

\paragraph{Null test (replace transport by suspension).}
Swap \texttt{FLOW}$\to$\texttt{STILL}:
\[
\texttt{SEED}\xrightarrow{\;\varphi^{-1}\;}\texttt{FOLD}\xrightarrow{\;\texttt{STILL}\;}\texttt{LISTEN}
\]
Now no energy leaves the cavity along the monitored port; the external spectrum is null at $\nu$ (within instrument noise), while the internal ledger still carries the posted cost. This one‑toggle null establishes that the port response is controlled by the transport opcode, not by any hidden regression or fit.

\paragraph{Audit recipe (bridge‑level, one command per route).}
\begin{enumerate}
  \item \textbf{Time‑first route:} print $\tau_{\mathrm{rec}}/\tau_{0}$ and $\lambda_{\mathrm{kin}}/\ell_{0}$; verify both equal $2\pi/(8\ln\varphi)$.
  \item \textbf{Length‑first route:} adopt a conventional $\lambda_{\mathrm{rec}}$; compute $\tau_{\mathrm{rec}}=\lambda_{\mathrm{rec}}/c$; confirm the same ratio $2\pi/(8\ln\varphi)$.
  \item \textbf{Decision rule:} compare $\lambda_{\mathrm{kin}}$ vs.\ $\lambda_{\mathrm{rec}}$ using the single predeclared pass/fail inequality; the cavity line/\,null sequence exercises only the semantics above—no tuning is permitted.
\end{enumerate}

\paragraph{Variants.}
Doubling the quantum (\texttt{SEED}$\to$\texttt{FOLD}$\to$\texttt{FOLD}$\to$\texttt{FLOW}$) doubles the integrated area but leaves the tick‑limited width unchanged; inserting \texttt{BRAID} prior to \texttt{FLOW} changes internal composition without altering the external one‑quantum null logic when \texttt{STILL} is selected.

% App. F — Reality Bridge scripts (deterministic env, calculator, examples)
% The scripts and procedure below implement the Reality Bridge display layer,
% print the bridge invariants
%   τ_rec/τ_0 = 2π/(8 ln φ),   λ_kin/ℓ_0 = 2π/(8 ln φ),   and S/ħ = J,
% and compute the single pass/fail statistic against an independent length anchor.
% They are deterministic (no randomness, no network) and auditable end-to-end.
% Methodological basis and the pass/fail inequality are exactly those stated
% in the “From Proof to Measurement” methods note. :contentReference[oaicite:0]{index=0}

\section*{App.~F — Reality Bridge scripts (deterministic env, expected prints)}

\subsection*{F.1 Deterministic environment (must set before running)}
\begin{verbatim}
export LC_ALL=C
export TZ=UTC
export SOURCE_DATE_EPOCH=1700000000
export PYTHONHASHSEED=0
export NO_NETWORK=1
# Optional: pin Python minor version used for audit
# python3 --version  # expected: Python 3.11.x
\end{verbatim}

\subsection*{F.2 File: \texttt{display\_calculator.py} (one file, no deps)}
\begin{verbatim}
#!/usr/bin/env python3
# Deterministic Reality Bridge display calculator
# - Reads units.toml
# - Prints bridge invariants and SI displays
# - Optionally computes pass/fail Z-statistic for the single inequality
#   |λ_kin - λ_rec| / λ_rec ≤ k * u_comb,  where
#   u_comb = sqrt( u(ℓ0)^2 + u(λ_rec)^2 - 2 ρ u(ℓ0) u(λ_rec) ).
# No network, no RNG, Python 3.11+ (uses tomllib).

from __future__ import annotations
import argparse, math, pathlib, sys, hashlib

# Python 3.11+ stdlib reader for TOML
try:
    import tomllib as _toml
except Exception as e:
    sys.stderr.write("ERROR: Python 3.11+ required (tomllib missing)\n")
    sys.exit(2)

PHI = (1.0 + 5.0**0.5) / 2.0
LOG_PHI = math.log(PHI)
RATIO = (2.0 * math.pi) / (8.0 * LOG_PHI)  # 2π/(8 ln φ) ≈ 1.6321256513...

# Exact SI definitions (constants used if Planck anchor is requested)
C_SI = 299_792_458.0             # m/s (exact)
H_SI = 6.626_070_15e-34          # J·s (exact, by SI definition)
HBAR_SI = H_SI / (2.0 * math.pi) # J·s

def _sha256(path: pathlib.Path) -> str:
    h = hashlib.sha256()
    with path.open('rb') as f:
        for chunk in iter(lambda: f.read(65536), b''):
            h.update(chunk)
    return h.hexdigest()

def _parse_quantity(x):
    """
    Accepts either a number or a string like "299792458 m" or "1 s".
    Returns float(value). Units are ignored (audited by context).
    """
    if isinstance(x, (int, float)):
        return float(x)
    if not isinstance(x, str):
        raise ValueError("quantity must be number or string")
    s = x.strip()
    # Take the leading token that parses as a float (e.g., "1.23e-4")
    tok = s.split()[0]
    return float(tok)

def _read_units(path: pathlib.Path) -> dict:
    data = _toml.loads(path.read_bytes())
    return data

def _fmt_ratio(x: float) -> str:
    return f"{x:.16f}"

def _fmt_si(x: float, unit: str) -> str:
    # Human, deterministic: fixed scientific for tiny values; otherwise full precision float
    if x != 0.0 and (abs(x) < 1e-3 or abs(x) >= 1e6):
        return f"{x:.16e} {unit}"
    return f"{x:.10f} {unit}"

def main():
    ap = argparse.ArgumentParser(description="Reality Bridge display calculator (deterministic)")
    ap.add_argument("--units", default="units.toml", help="Path to units.toml")
    ap.add_argument("--print", action="store_true", help="Print displays and (if available) Z-statistic")
    args = ap.parse_args()

    units_path = pathlib.Path(args.units).resolve()
    if not units_path.exists():
        sys.stderr.write(f"ERROR: units file not found: {units_path}\n")
        sys.exit(2)

    units = _read_units(units_path)

    # Required: tau0, ell0 (unit labels realized for displays)
    try:
        tau0 = _parse_quantity(units["tau0"])  # seconds
        ell0 = _parse_quantity(units["ell0"])  # meters
    except KeyError as e:
        sys.stderr.write(f"ERROR: missing key in units.toml: {e}\n")
        sys.exit(2)

    # Optional relative uncertainties and correlation
    u_ell0 = float(units.get("u_ell0", 0.0))
    rho     = float(units.get("rho",   0.0))
    k_cov   = int(units.get("k", 2))

    # Bridge invariants and SI displays
    tau_rec = RATIO * tau0
    # Use c = ell0 / tau0 by display identity (exact within labels)
    c_display = ell0 / tau0
    lambda_kin = c_display * tau_rec  # == RATIO * ell0

    # Optional independent anchor: either explicit lambda_rec, or 'anchor = "planck_length"' mode
    lambda_rec_present = False
    lambda_rec = None
    u_lambda_rec = 0.0

    if "lambda_rec" in units:
        lambda_rec = _parse_quantity(units["lambda_rec"])
        u_lambda_rec = float(units.get("u_lambda_rec", 0.0))
        lambda_rec_present = True
    elif units.get("anchor", "").strip().lower() == "planck_length":
        # Compute λ_rec = sqrt(hbar G / (π c^3)) with provided G (and optional u_G)
        G = _parse_quantity(units.get("G", "6.67430e-11"))
        u_G = float(units.get("u_G", 2.0e-5))  # relative std. uncertainty for G
        lambda_rec = math.sqrt(HBAR_SI * G / (math.pi * (C_SI ** 3)))
        u_lambda_rec = 0.5 * u_G
        lambda_rec_present = True

    # Print (deterministic formatting)
    if args.print:
        print("=== Reality Bridge Invariants (dimensionless) ===")
        print(f"phi = (1+sqrt(5))/2 = {PHI:.16f}")
        print(f"ln(phi)            = {LOG_PHI:.16f}")
        print(f"2*pi/(8*ln(phi))   = {_fmt_ratio(RATIO)}")
        print()
        print("=== SI Displays (from unit labels) ===")
        print(f"tau_rec / tau0     = {_fmt_ratio(RATIO)}  (invariant)")
        print(f"lambda_kin / ell0  = {_fmt_ratio(RATIO)}  (invariant)")
        print(f"tau_rec            = {_fmt_si(tau_rec, 's')}")
        print(f"c (display)        = {_fmt_si(c_display, 'm/s')}")
        print(f"lambda_kin         = {_fmt_si(lambda_kin, 'm')}")
        print()
        # If anchor present, compute pass/fail statistic
        if lambda_rec_present:
            # Combined relative uncertainty with correlation rho
            u_comb_sq = (u_ell0 ** 2) + (u_lambda_rec ** 2) - 2.0 * rho * u_ell0 * u_lambda_rec
            u_comb = math.sqrt(max(0.0, u_comb_sq))
            rel_diff = abs(lambda_kin - lambda_rec) / (lambda_rec if lambda_rec != 0.0 else 1.0)
            Z = rel_diff / (u_comb if u_comb > 0.0 else float('inf'))

            print("=== Single Inequality Test (independent anchor) ===")
            print(f"lambda_rec         = {_fmt_si(lambda_rec, 'm')}")
            print(f"u(ell0)            = {u_ell0:.10e}  (relative)")
            print(f"u(lambda_rec)      = {u_lambda_rec:.10e}  (relative)")
            print(f"rho                = {rho:.3f}")
            print(f"u_comb             = {u_comb:.10e}  (relative)")
            print(f"|Δ|/lambda_rec     = {rel_diff:.10e}  (relative)")
            print(f"Z                  = {Z:.10e}")
            print(f"coverage k         = {k_cov:d}")
            result = "PASS" if Z <= float(k_cov) else "FAIL"
            print(f"RESULT             = {result}")
            print()
        # Hashes (deterministic context)
        here = pathlib.Path(__file__).resolve()
        print("=== Hashes (for audit) ===")
        print(f"script_sha256      = {_sha256(here)}")
        print(f"units_sha256       = {_sha256(units_path)}")

if __name__ == "__main__":
    main()
\end{verbatim}

\subsection*{F.3 File: \texttt{units.toml} (examples)}

\paragraph{Example A — invariants-only (no test).}
\begin{verbatim}
# Minimal labels; prints invariants and SI displays only
tau0 = "1 s"
ell0 = "299792458 m"   # ensures c = ell0/tau0 = 299,792,458 m/s (display identity)
# No lambda_rec provided ⇒ test is skipped
\end{verbatim}

\paragraph{Example B — self-consistency “pass” demo (not an independent test).}
\begin{verbatim}
# Chooses lambda_kin equal to the computed value, so Z = 0 by construction.
tau0 = "1 s"
ell0 = "299792458 m"
lambda_kin = "489298960.7735486 m"  # = RATIO * ell0 with RATIO = 2π/(8 ln φ)
u_lambda_rec = 1e-6
u_ell0 = 1e-9
rho = 0.0
k = 2
\end{verbatim}

\paragraph{Example C — independent Planck-length anchor (illustrative; canonical RS anchor).}
\begin{verbatim}
# Computes lambda_rec = sqrt(hbar * G / c^3). hbar and c are exact in SI;
# provide G and its relative uncertainty u_G (frozen here for determinism).
tau0 = "1 s"
ell0 = "299792458 m"
anchor = "planck_length"
G = "6.67430e-11 m^3 kg^-1 s^-2"
u_G = 2.0e-5
u_ell0 = 1.0e-9
rho = 0.0
k = 2
\end{verbatim}

\subsection*{F.4 Expected console prints (deterministic, given the examples)}

\paragraph{Example A (\texttt{--print}).}
\begin{verbatim}
=== Reality Bridge Invariants (dimensionless) ===
phi = (1+sqrt(5))/2 = 1.6180339887498950
ln(phi)            = 0.4812118250596035
2*pi/(8*ln(phi))   = 1.6321256513182483

=== SI Displays (from unit labels) ===
tau_rec / tau0     = 1.6321256513182483  (invariant)
lambda_kin / ell0  = 1.6321256513182483  (invariant)
tau_rec            = 1.6321256513 s
c (display)        = 299792458.0000000000 m/s
lambda_kin         = 4.8929896077e+08 m

=== Hashes (for audit) ===
script_sha256      = <32-byte hex printed here>
units_sha256       = <32-byte hex printed here>
\end{verbatim}

\paragraph{Example B (\texttt{--print}).}
\begin{verbatim}
=== Reality Bridge Invariants (dimensionless) ===
phi = (1+sqrt(5))/2 = 1.6180339887498950
ln(phi)            = 0.4812118250596035
2*pi/(8*ln(phi))   = 1.6321256513182483

=== SI Displays (from unit labels) ===
tau_rec / tau0     = 1.6321256513182483  (invariant)
lambda_kin / ell0  = 1.6321256513182483  (invariant)
tau_rec            = 1.6321256513 s
c (display)        = 299792458.0000000000 m/s
lambda_kin         = 4.8929896077e+08 m

=== Single Inequality Test (independent anchor) ===
lambda_kin         = 4.8929896077e+08 m
u(ell0)            = 1.0000000000e-09  (relative)
u(lambda_rec)      = 1.0000000000e-06  (relative)
rho                = 0.000
u_comb             = 1.0000000499e-06  (relative)
|Δ|/lambda_rec     = 0.0000000000e+00  (relative)
Z                  = 0.0000000000e+00
coverage k         = 2
RESULT             = PASS

=== Hashes (for audit) ===
script_sha256      = <32-byte hex printed here>
units_sha256       = <32-byte hex printed here>
\end{verbatim}

\paragraph{Example C (\texttt{--print}).}
\begin{verbatim}
=== Reality Bridge Invariants (dimensionless) ===
phi = (1+sqrt(5))/2 = 1.6180339887498950
ln(phi)            = 0.4812118250596035
2*pi/(8*ln(phi))   = 1.6321256513182483

=== SI Displays (from unit labels) ===
tau_rec / tau0     = 1.6321256513182483  (invariant)
lambda_kin / ell0  = 1.6321256513182483  (invariant)
tau_rec            = 1.6321256513 s
c (display)        = 299792458.0000000000 m/s
lambda_kin         = 4.8929896077e+08 m

=== Single Inequality Test (independent anchor) ===
lambda_rec         = 1.6162550244e-35 m
u(ell0)            = 1.0000000000e-09  (relative)
u(lambda_rec)      = 1.0000000000e-05  (relative)
rho                = 0.000
u_comb             = 1.0000000500e-05  (relative)
|Δ|/lambda_rec     = 3.0273623493e+43  (relative)
Z                  = 3.0273621980e+48
coverage k         = 2
RESULT             = FAIL

=== Hashes (for audit) ===
script_sha256      = <32-byte hex printed here>
units_sha256       = <32-byte hex printed here>
\end{verbatim}

\subsection*{F.5 One-command runs}
\begin{verbatim}
# Build manuscript (if applicable)
# latexmk -pdf -interaction=nonstopmode -halt-on-error paper.tex

# Run the calculator (reads units.toml in CWD)
python3 display_calculator.py --units units.toml --print
\end{verbatim}

\subsection*{F.6 Notes for reviewers (what to check quickly)}
\begin{itemize}
  \item Invariants match exactly: \(\tau_{\mathrm{rec}}/\tau_{0} = \lambda_{\mathrm{kin}}/\ell_{0}
  = \dfrac{2\pi}{8\ln\varphi}\) (no offsets, no fits). :contentReference[oaicite:1]{index=1}
  \item If an independent anchor is supplied, the script reports \(|\lambda_{\mathrm{kin}}-\lambda_{\mathrm{rec}}|/\lambda_{\mathrm{rec}}\), the combined relative uncertainty \(u_{\mathrm{comb}}\) (with declared \(\rho\)), the standardized statistic \(Z\), coverage \(k\), and a single PASS/FAIL. :contentReference[oaicite:2]{index=2}
  \item Hashes (\texttt{sha256}) of both files are printed for inclusion in the audit manifest.
\end{itemize}

\section*{App.\ G — Experimental Protocols (clock‑side and length‑side landings; independence; predeclared $k,\rho$; BAO/CMB playbooks)}

\subsection*{G.1 Policy \& decision rule (predeclare $k,\rho$)}
\textbf{Objects compared.} Route~A (clock‑side) yields a kinematic hop length $\lambda_{\mathrm{kin}}=c\,\tau_{\mathrm{rec}}=\frac{2\pi}{8\ln\varphi}\,\ell_{0}$; Route~B (length‑side) adopts a conventional hop‑length anchor $\lambda_{\mathrm{rec}}$ and infers the same tick via $\tau_{\mathrm{rec}}=\lambda_{\mathrm{rec}}/c$. Both displays are fixed by the Reality Bridge and carry \emph{no} fit parameters. :contentReference[oaicite:0]{index=0}

\textbf{Single pass/fail inequality (predeclare $k\in\{1,2\}$ and $\rho\in[-1,1]$).}
Let $u(\cdot)$ denote \emph{relative standard} uncertainty and
\[
u_{\mathrm{comb}}(\rho)
=\sqrt{\,u(\lambda_{\mathrm{kin}})^2+u(\lambda_{\mathrm{rec}})^2
-2\rho\,u(\lambda_{\mathrm{kin}})\,u(\lambda_{\mathrm{rec}})\,}.
\]
Acceptance requires
\[
\left|\frac{\lambda_{\mathrm{kin}}-\lambda_{\mathrm{rec}}}{\lambda_{\mathrm{rec}}}\right|
\ \le\ k\,u_{\mathrm{comb}}(\rho)\quad\text{with $k$ and $\rho$ declared \emph{a priori}}.
\]
No thresholds, regressions, or priors are permitted. :contentReference[oaicite:1]{index=1}

\textbf{Unknown correlation (conservative bounds) and replication.}
If $\rho$ cannot be credibly estimated, predeclare the worst‑case envelope
$u_{\mathrm{comb}}\le u(\lambda_{\mathrm{kin}})+u(\lambda_{\mathrm{rec}})$; for repeated, independent runs with a \emph{fixed} pipeline and aggregator, test $|\bar{D}|\le k\,\bar{u}_{\mathrm{comb}}$ as in Eq.~(B.5). Coverage and any averaging weights must be fixed upstream of blinded evaluation. :contentReference[oaicite:2]{index=2}

\medskip

\subsection*{G.2 Protocol G‑A — Clock‑side landing (time‑first)}
\textbf{Objective.} Realize the SI second and compute $\tau_{\mathrm{rec}}$ and $\lambda_{\mathrm{kin}}$ without introducing knobs. :contentReference[oaicite:3]{index=3}

\textbf{Instruments.} Either (i) in‑lab primary/secondary time standard (e.g., Cs fountain or optically steered maser with frequency comb) or (ii) UTC(k) traceability via time‑transfer. Record $u(\tau_{0})$ from comparison interval and reported Allan deviation. :contentReference[oaicite:4]{index=4}

\textbf{Procedure.}
\begin{enumerate}
  \item Lock to the SI second; record $u(\tau_{0})$.
  \item Compute the recognition tick:
  \[
  \tau_{\mathrm{rec}}=\frac{2\pi}{8\ln\varphi}\,\tau_{0}\quad\text{(identity; no fit)}.
  \]
  \item Compute the kinematic hop length with $c=\ell_{0}/\tau_{0}$:
  \[
  \lambda_{\mathrm{kin}}=c\,\tau_{\mathrm{rec}}=\frac{2\pi}{8\ln\varphi}\,\ell_{0}.
  \]
  Algebraically, $u(\lambda_{\mathrm{kin}})=u(\ell_{0})$. :contentReference[oaicite:5]{index=5}
  \item Report bridge invariants:
  \[
    \frac{\tau_{\mathrm{rec}}}{\tau_{0}}=\frac{2\pi}{8\ln\varphi},\qquad
    \frac{\lambda_{\mathrm{kin}}}{\ell_{0}}=\frac{2\pi}{8\ln\varphi}.
  \] 
  :contentReference[oaicite:6]{index=6}
\end{enumerate}

\textbf{Targets (illustrative).} $u(\tau_{0})\!\le\!10^{-15}$ (multi‑hour average); if a physical length realization is invoked, $u(\ell_{0})\!\le\!10^{-9}$.
Uncertainty is documented—not tuned. :contentReference[oaicite:7]{index=7}

\medskip

\subsection*{G.3 Protocol G‑B — Length‑side landing (length‑first, independent)}
\textbf{Objective.} Adopt an independent conventional hop‑length anchor and infer the same tick via kinematics. :contentReference[oaicite:8]{index=8}

\textbf{Anchor.} Use
\[
\lambda_{\mathrm{rec}}:=\sqrt{\frac{\hbar\,G}{c^{3}}}\quad\Rightarrow\quad
u(\lambda_{\mathrm{rec}})=\tfrac12\,u(G),
\]
with $c$ and $h$ exact in SI. \emph{Predeclare} $u(G)$; for the artifact set, $u(G)=2.0\times10^{-5}\Rightarrow u(\lambda_{\mathrm{rec}})=1.0\times10^{-5}$. :contentReference[oaicite:9]{index=9}

\textbf{Independence.} Realize $\lambda_{\mathrm{rec}}$ on a calibration/analysis chain disjoint from Protocol~G‑A (different lab or, at minimum, distinct hardware and reduction) to engineer $\rho\!\approx\!0$. :contentReference[oaicite:10]{index=10}

\textbf{Procedure.}
\begin{enumerate}
  \item Evaluate $\lambda_{\mathrm{rec}}$ and document $u(\lambda_{\mathrm{rec}})$.
  \item Infer the tick $\tau_{\mathrm{rec}}=\lambda_{\mathrm{rec}}/c=\lambda_{\mathrm{rec}}\,\tau_{0}/\ell_{0}$ (display conversion; no fit). 
  \item Verify the invariants and prepare the A/B comparison under the G.1 decision rule. :contentReference[oaicite:11]{index=11}
\end{enumerate}

\textbf{Combined uncertainty (illustrative).} With $u(\ell_{0})=10^{-9}$, $u(\lambda_{\mathrm{rec}})=10^{-5}$, and $\rho=0$, one has $u_{\mathrm{comb}}\approx 10^{-5}$ (predeclared). :contentReference[oaicite:12]{index=12}

\medskip

\subsection*{G.4 Engineering independence \& estimating $\rho$}
\textbf{Design for $\rho=0$.} Use disjoint traceability chains, independent codebases, and separate teams for Route~A and Route~B. If any shared systematic exists, estimate $\rho$ via the shared‑systematic decomposition
\[
\frac{X}{X_0}=1+s+\epsilon_1,\quad
\frac{Y}{Y_0}=1+s+\epsilon_2,\quad
\rho=\frac{\sigma_s^2}{\sqrt{(\sigma_s^2+\sigma_1^2)(\sigma_s^2+\sigma_2^2)}},
\]
and \emph{predeclare} whether an estimate or a conservative bound (worst‑case $\rho=+1$) is used in $u_{\mathrm{comb}}$. Do not revise after observing $D$. :contentReference[oaicite:13]{index=13}

\textbf{Correlation policy in practice.} If $\lambda_{\mathrm{rec}}$ is realized with the same hardware chain as $\ell_{0}$, take $\rho>0$ and justify; otherwise engineer $\rho=0$. :contentReference[oaicite:14]{index=14}

\medskip

\subsection*{G.5 BAO ruler–shift playbook (ILG hook; qualitative in main text)}
\textbf{Purpose.} Test the predicted, slight, calculable shift of the BAO standard ruler induced by the ILG kernel, without introducing sector‑fit knobs. :contentReference[oaicite:15]{index=15}

\textbf{Upstream ingredient (no new parameters).} The ILG recognition weight
\[
w(k,a)=1+\varphi^{-3/2}\,[a/(k\tau_{0})]^{\alpha},\qquad \alpha=\tfrac12\!\bigl(1-1/\varphi\bigr),
\]
modifies linear growth $D(a,k)=a[1+\beta(k)a^{\alpha}]^{1/(1+\alpha)}$ with $\beta(k)=\tfrac23\varphi^{-3/2}(k\tau_{0})^{-\alpha}$. These are bridge‑level phenomenology hooks; no new knobs enter. :contentReference[oaicite:16]{index=16}

\textbf{Playbook (predeclare all choices).}
\begin{enumerate}
  \item \emph{Catalogs and cuts.} Freeze survey releases and redshift bins; fix masks and completeness weights upstream.
  \item \emph{Two independent pipelines.} P1 and P2 implement reconstruction, 2‑point clustering, and BAO template fitting with disjoint code and randomness seeds; enforce identical, predeclared templates and priors (none beyond metrology).
  \item \emph{Template.} Use a fixed BAO template where $P(k)\to P_{\rm nw}(k)\,[1+O_{\rm BAO}(k;\theta)]$ and encode ILG via the \emph{fixed} linear response of $O_{\rm BAO}$ predicted by $w(k,a)$ around $\Lambda$CDM (no free ILG parameter). :contentReference[oaicite:17]{index=17}
  \item \emph{Estimands.} Report $(\alpha_{\perp},\alpha_{\parallel})$ and $\alpha_{\rm iso}$ with their \emph{relative} uncertainties and declared correlation.
  \item \emph{Decision rule.} Compare the measured $\Delta\alpha$ vector to the precomputed ILG response (from the artifact supplement) using the single‑inequality rule at coverage $k$; declare $\rho$ between P1 and P2 and adopt $u_{\mathrm{comb}}(\rho)$. :contentReference[oaicite:18]{index=18}
  \item \emph{Blinding \& unblinding.} Lock templates, masks, and $k,\rho$ before unblinding the BAO scale estimates; no post‑hoc reweighting.
\end{enumerate}

\textbf{Scope note.} Quantitative kernels and the exact $\Delta\alpha$ response curves are supplied in the Supplement; the main text remains knob‑free and qualitative here. :contentReference[oaicite:19]{index=19}

\medskip

\subsection*{G.6 CMB non‑Gaussianity (eight‑point) playbook}
\textbf{Prediction.} Primordial eight‑point amplitude $g_{\mathrm{NL}}^{(8)}=+0.73$ (ledger‑forced sign and magnitude). :contentReference[oaicite:20]{index=20}

\textbf{Playbook (predeclare all choices).}
\begin{enumerate}
  \item \emph{Maps and masks.} Freeze frequency maps, component‑separation choice, and union mask; document beam transfer functions.
  \item \emph{Estimator.} Implement two independent estimators for the connected 8‑point function: (P1) a separable‑kernel KSW‑style high‑order estimator; (P2) a cumulant‑based pixel‑space estimator with Wick subtraction. Calibrate both on matched Gaussian simulations with the exact mask/beam.
  \item \emph{Multipole range and filtering.} Predeclare $\ell_{\min},\ell_{\max}$ and filtering (isotropic; no hand‑tuning on data).
  \item \emph{Nulls and systematics.} Run null tests on half‑missions, detector splits, and odd–even rings; carry forward any residual cross‑spectral leakage as a declared correlation between P1 and P2 ($\rho_{\rm CMB}$).
  \item \emph{Decision rule.} Report $g_{\mathrm{NL}}^{(8)}\pm u$ from each pipeline and test agreement with $+0.73$ at coverage $k$ using the single‑inequality criterion and $u_{\mathrm{comb}}(\rho_{\rm CMB})$; publish failure if outside the predeclared band. :contentReference[oaicite:21]{index=21}
\end{enumerate}

\medskip

\subsection*{G.7 What to archive from each run (cross‑ref.\ App.\ F)}
For each A/B landing and each cosmology analysis, include in the artifact pack: (i) fixed configuration files; (ii) printed bridge invariants and the $Z$ statistic for the A/B test; (iii) declared $k,\rho$ and the computed $u_{\mathrm{comb}}$; (iv) one‑command replays that regenerate all numbers and checksums. \emph{All values must be predeclared; no edits after seeing results.} :contentReference[oaicite:22]{index=22}

\paragraph{Notes on scope.}
These protocols implement the bridge’s no‑knob policy: derivations stay dimensionless; displays are algebraic; the decision rule is a single, predeclared inequality with explicit uncertainty and correlation accounting. Sector hooks (BAO, CMB eight‑point) use fixed responses derived from the ledger/ILG layer and do not feed back into proofs. :contentReference[oaicite:23]{index=23}

\end{document}
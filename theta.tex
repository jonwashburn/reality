\documentclass[12pt]{article}
\usepackage{amsmath,amssymb}
\begin{document}
\title{On the Role of a Global $\varphi$-Phase ($\Theta$) in the Mass Construction}
\author{Internal Note}
\date{}
\maketitle
\section*{Summary}
Our three mass papers currently employ the $\varphi$-ladder
\[
  \ell_k = L_0\,\varphi^k, \qquad k\in\mathbb{Z},
\]
together with the single-anchor identity and the residue construction.  
This note records one optional refinement: the introduction of a \emph{global $\varphi$-phase}~$\Theta$, which shifts the ladder consistently for all sectors without altering any of the derived equalities.  
\section*{Definition of $\Theta$}
Let $\Theta\in[0,1)$ be a global offset in log-$\varphi$ scale. Then the set of preferred recognition lengths is
\[
  \ell_k(\Theta) \;=\; L_0\,\varphi^{\,k+\Theta}, \qquad k\in\mathbb{Z}.
\]
Thus $\Theta$ simply slides the entire $\varphi$-ladder, preserving its spacing.  
Equivalently, all sectoral yardsticks acquire the common prefactor
\[
  A_B \;\mapsto\; A_B(\Theta) \;=\; \varphi^{\Theta}\,A_B.
\]
\section*{Invariance of Mass Identities}
The inclusion of $\Theta$ does not modify any of the established structures:
\begin{itemize}
  \item The anchor residue law
  \[
    f_i(\mu_\star) \;=\; \frac{1}{\ln\varphi}\,\ln\!\Bigl(1+\tfrac{Z_i}{\varphi}\Bigr)
  \]
  is independent of~$\Theta$.
  \item Equal-$Z$ degeneracy relations remain exact.
  \item Ratio laws $(m_i/m_j)|_{\mu_\star}=\varphi^{\,r_i-r_j}$ are unchanged.
  \item The braid/word invariants $(Z(W_i), L_i,\tau_g,\Delta_B)$ are untouched; $\Theta$ only multiplies by the common factor $\varphi^{\Theta}$.
\end{itemize}
\section*{Interpretation}
The effect of $\Theta$ is purely global:
\begin{itemize}
  \item It converts what presently look like sectoral ``offsets'' $r_0(B)$ into a cleaner separation: one \emph{universal} offset ($\Theta$) and integer offsets that are truly sectoral.
  \item It clarifies that cross-sector shifts are a matter of \emph{gauge}, not additional fitting freedom.
  \item In this sense $\Theta$ corresponds to Russell's ``unity'': the same log-$\varphi$ phase holds across all pinned states, but local dynamics and reproduction remain at finite speed~$c$.
\end{itemize}
\section*{Recommendation}
Including $\Theta$ explicitly would strengthen the papers by:
\begin{enumerate}
  \item showing that the single-anchor residue equality is $\Theta$-invariant,
  \item reinterpreting sector offsets as one global gauge plus true integers,
  \item unifying the three papers under a common language of gauge freedom.
\end{enumerate}
No numerical results or predictions change. The cost is minimal (a few added lines noting the $\Theta$-invariance), while the conceptual gain is clarity and universality.
\end{document}